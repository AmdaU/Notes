\documentclass{article}    
\usepackage[utf8]{inputenc}    
    
\title{Épisode 4}    
\author{Jean-Baptiste Bertrand}    
\date{\today}    
    
\setlength{\parskip}{1em}    
    
\usepackage{physics}    
\usepackage{graphicx}    
\usepackage{svg}    
\usepackage[utf8]{inputenc}    
\usepackage[T1]{fontenc}    
\usepackage[french]{babel}    
\usepackage{fancyhdr}    
\usepackage[total={19cm, 22cm}]{geometry}    
\usepackage{enumerate}    
\usepackage{enumitem}    
\usepackage{stmaryrd}    
\usepackage{mathtools,slashed}
%\usepackage{mathtools}
\usepackage{cancel}
    
\usepackage{pdfpages}
%packages pour faire des math    
%\usepackage{cancel} % hum... pas sur que je vais le garder mais rester que des fois c'est quand même sympatique...
\usepackage{amsmath, amsfonts, amsthm, amssymb}    
\usepackage{esint}  
\usepackage{dsfont}

\usepackage{import}
\usepackage{pdfpages}
\usepackage{transparent}
\usepackage{xcolor}
\usepackage{tcolorbox}

\usepackage{mathrsfs}
\usepackage{tensor}

\usepackage{tikz}
\usetikzlibrary{quantikz}
\usepackage{ upgreek }

\newcommand{\incfig}[2][1]{%
    \def\svgwidth{#1\columnwidth}
    \import{./figures/}{#2.pdf_tex}
}

\newcommand{\cols}[1]{
\begin{pmatrix}
	#1
\end{pmatrix}
}

\newcommand{\avg}[1]{\left\langle #1 \right\rangle}
\newcommand{\lambdabar}{{\mkern0.75mu\mathchar '26\mkern -9.75mu\lambda}}

\pdfsuppresswarningpagegroup=1

\begin{document}

\section*{Gaz d'électrons?}


$$n \sim 10^{22} / \rm{cm}^3$$ 

$$E_f = \frac{\hbar^2k_f^{2}}{2m} =\frac{\hbar^2}{2m} \left( \frac{3\pi^2N}{V}   \right)^{2/3} \sim 1 \rm{eV} $$ 


$$N = \frac{\frac{4\pi{}k_F^{2}}{3}}{\left( \frac{2\pi}{L}  \right)^3}  $$ 


Parenthèse température pièce:
$$T = 300 \rm{K} \to 25 \rm{meV}$$ 

$$1 \rm{eV} \to T_f \sim 5E4 \rm{K}$$ 


Puisque la température pièce est très faible par rapport à $T_F$, la distribution de Fermi ressemble à une fonction de Heavy side.

Densité d'état en 3D:


$$D(E_f) = \eval{\dv{N}{E}}_{E_f} = \frac{3}{2} \frac{N}{E} $$ 

$$\text{car } N \propto A E^{3/2} $$ 

En $2D$

$$D(E) = \dv{{}}{E} (AE) = A$$ 


\section*{Structure de Bande}


$$\psi(x) = u_k(x)e^{ikx} = \sum_{G} c(k-G)e^{-iGx}$$ 

Mais comment obtenir la masse effective??

$$m^* = \frac{\hbar^2}{\dv[2]{E}{k}} $$ 

$$\frac{\hbar^2k^2}{2m} = E$$ 

nombre impaire de $V$ (électrons de valence je suppose) $\implies$ métal    

un nombre pair implique un isolant 



\begin{figure}[ht]
    \centering
    \incfig{semiconducteur}
    \caption{semiconducteur}
    \label{fig:semiconducteur}
\end{figure}


\begin{figure}[ht]
    \centering
    \incfig{un-graphique-vraiment-cool}
    \caption{Un graphique vraiment cool}
    \label{fig:un-graphique-vraiment-cool}
\end{figure}

\begin{figure}[ht]
    \centering
    \incfig{wowowowowowowowo}
    \caption{wowowowowowowowo}
    \label{fig:wowowowowowowowo}
\end{figure}

\section*{Effet Hall (whaoo!)}

$$R_H = \frac{1}{e} = \frac{p\mu_p^{2}-n\mu_n^2}{\left( p\mu_p -n\mu_n \right)^2 } $$ 
\clearpage

\section*{Supraconductivité}

diamgagnétisme parfait et résistivité nulle

Distinction entre type I et type II: Le champ magnétique pénètre pas dans le type I alors que dans le II oui

Flux associé à un vortex:

...

Ça va \textbf{EXTRÊMEMENT} vite mais tout le monde à l'air de trouver que la supra c'est parfaitement trivial faque c'est correct I guess...





\end{document}
