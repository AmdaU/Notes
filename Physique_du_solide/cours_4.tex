\documentclass{article}    
\usepackage[utf8]{inputenc}    
    
\title{Épisode 4}    
\author{Jean-Baptiste Bertrand}    
\date{\today}    
    
\setlength{\parskip}{1em}    
    
\usepackage{physics}    
\usepackage{graphicx}    
\usepackage{svg}    
\usepackage[utf8]{inputenc}    
\usepackage[T1]{fontenc}    
\usepackage[french]{babel}    
\usepackage{fancyhdr}    
\usepackage[total={19cm, 22cm}]{geometry}    
\usepackage{enumerate}    
\usepackage{enumitem}    
\usepackage{stmaryrd}    
    
%packages pour faire des math    
%\usepackage{cancel} % hum... pas sur que je vais le garder mais rester que des fois c'est quand même sympatique...
\usepackage{amsmath, amsfonts, amsthm, amssymb}    
\usepackage{esint}  


\begin{document}
Live on parle de mode optique pis acoutsique avec un dessin de relation de dispertion comme dans le lab de mode de vibration. Je sais pas exactement de quoi on parle mais il a écris la valeur de $\omega$ à plusieurs point du diagramme de phase en fonction des deux masses pis des constante

\begin{figure}[ht]
    \centering
    \incfig{relation}
    \caption{relation}
    \label{fig:relation}
\end{figure}

La il a parlé de branches logitudinal et transverse, je comprends fuckall ce que c'est.

La il parle de passe au quantique.

On passe d'un mode normal à des états propore avec $E_n= \hbar\omega$ 

On définit un phonon comme un quantum de vibration $\hbar\omega.$ 

On fait un problème d'ossillateur harmonique quantique, je pense??

On essais de trouver la valeur moyenne de l'énérgie de quleque chose...

$$<\epsilon> = \frac{\sum_{\rm n}(n + \frac{1}{2}) \hbar \omega e^{-(n+1/2)\hbar\omega/k_{\rm B}T}}{\sum_{\rm n}e^{(-n+1/2)\hbar\omega/k_BT}} = \frac{\sum_{\rm n}n\hbar\omega e^{-n\hbar\omega/k_bT}}{e^{-n\hbar\omega/k_{\rm B}T}} + \frac{\hbar\omega}{2} $$ 

On a une série géométrique, on prend $x = e^{-\hbar\omega/k_{\rm B}T}$ 

Un confusion absoulue à propos de dérivé polynomiale s'en suit.

On conlus éventuellement que $<\epsilon>=\frac{\hbar\omega}{e^{\hbar\omega/k_{\rm B}T}-1} + \frac{\hbar\omega}{2} = \hbar\omega <n> + \frac{1}{2}  $ 

$$U_{\rm tot} = \sum_{k=-\frac{pi}{a}}^{\frac{\pi}{a} }<\epsilon_k>$$ 

$$a_{\rm a}$$ 

Je sais pas pantoute ce qu'on fait, il a écris des trucs au tableau pendant la pause pis je vois pas le lien entre les différents trucs ni avec ce qu'on fesait avant ni ce qu'il dit. mais bon

$$P_{\rm tot} =  m \sum_{S=0}^{N-1} \dv{u_s}{t} = m \dv{t} (u_0e^{i\omega t})\sum_{s=0}^{N-1} e^{iksa} = \sum (e^{ika})^S = \frac{1 - e^{ikNa}}{1-e^{ika}} = 0$$

car $kNa = 2\pi n$ 

On parle de conlision/réfléxion la je pense. Il y a une particule incidente en tout cas.

$$e^{i(\omega_it=k_ix) \to \text{ particule incidente}}$$
$$e^{i(\omega_{\rm i}t-k_{\rm i}tx_{\rm s}+ ?)}$$

C'est pas une particule, c'est de la lumi'res

Il calcul des champ éléctriques (Boogaloo??)

Il a sorti la condition pour avoir un pic de diffraction. On a un photon qui réflètent sur un crystal?? Je sais toujours pas de quoi on parle

Oups, J'ai oublié d'écouter pendant un moment.

Il a pas avancé beauocup, il travaille avec des ondes plannes là. C'est un peu le bordel au tableau.

Antoine viens de lâcher un "Da fuck"

Je pense honnêtement pas que je vais passer le cours. Il y a des gens qui posent des questions... How the fuck qu'il font pour même commencer à comprendre ce qui se passe?

oh! ON viens de changer de sujet! On parle de déformations!

\begin{figure}[ht]
    \centering
    \incfig{déformations}
    \caption{Déformations}
    \label{fig:déformations}
\end{figure}

$$e = \dv{u}{x}O$$ 

Il est arrivé a une équation d'onde, j'ai pas compris comment, je sais pas trop ce que ses variables représentent. 

Oh fuck on va faire des exercices.





\end{document}
