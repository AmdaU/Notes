\documentclass{article}    
\usepackage[utf8]{inputenc}    
    
\title{Épisode 4}    
\author{Jean-Baptiste Bertrand}    
\date{\today}    
    
\setlength{\parskip}{1em}    
    
\usepackage{physics}    
\usepackage{graphicx}    
\usepackage{svg}    
\usepackage[utf8]{inputenc}    
\usepackage[T1]{fontenc}    
\usepackage[french]{babel}    
\usepackage{fancyhdr}    
\usepackage[total={19cm, 22cm}]{geometry}    
\usepackage{enumerate}    
\usepackage{enumitem}    
\usepackage{stmaryrd}    
    
%packages pour faire des math    
%\usepackage{cancel} % hum... pas sur que je vais le garder mais rester que des fois c'est quand même sympatique...
\usepackage{amsmath, amsfonts, amsthm, amssymb}    
\usepackage{esint}  


\begin{document}

On peut faire un développement en série autour de $R = R_0$.

\begin{figure}[ht]
    \centering
    \incfig{potentiel}
    \caption{potentiel}
    \label{fig:potentiel}
\end{figure}

$$U(x) = U(x_0) + (x-x_0)\eval{\dv{U}{x}}_{x_0} + \frac{1}{2} (x-x_0)^2\dv[2]{U}{x}$$ 

$$F_s = \sum_p c_p(u_{s+p}-u_s)$$ 


\begin{figure}[ht]
    \centering
    \incfig{force}
    \caption{force}
    \label{fig:force}
\end{figure}


$$u_s(t) = u_0 e^{-i\omega t}e^{ikx}$$ 

$$m\dv[2]{u}{t} = \sum_p (U_{sp}-U_s)$$ 
$$\dotsb$$ 
$$-m\omega = \sum_{p>0}C_p (e^{ikpa}-1) + \sum_{p<0}C_p(e^{ikpa}-1)$$ 
$$\dotsb$$ 
$$-m\omega^2 = 2\sum_{p>0} c_p(\cos(kpa)-1)$$ 
$$\omega^2 = \frac{2c}{m}(1-\cos(ka))$$ 
$$\omega^2 = \frac{4C}{m}\sin^2\qty(\frac{ka}{2})$$ 

\begin{figure}[ht]
    \centering
    \incfig{relation-de-dispertion}
    \caption{relation de dispertion}
    \label{fig:relation-de-dispertion}
\end{figure}

\textit{Pourquois je suis surpris d'obtenir ce résultat?} - Françis <3

$$v_g \dv{\omega}{k}\quad \vb V_g =\grad \omega(\vb k)$$ 

$$a_{allo}$$ 

$$a_{1,2,3}$$ 

\begin{figure}[ht]
    \centering
    \incfig{banane}
    \caption{banane}
    \label{fig:banane}
\end{figure}

\end{document}
