\documentclass{article} \usepackage[utf8]{inputenc}

\title{Cours A} \author{Jean-Baptiste Bertrand} \date{\today}

%\setlength{\parindent}{4em} \setlength{\parskip}{1em}

\usepackage[utf8]{inputenc} \usepackage[T1]{fontenc} \usepackage[french]{babel}
\usepackage{fancyhdr} \usepackage[total={19cm, 22cm}]{geometry}
\usepackage{enumerate} \usepackage{enumitem} \usepackage{svg}
\usepackage{physics} \usepackage{mathrsfs} \usepackage{tcolorbox}

%packages pour faire des math %\usepackage{cancel} % hum... pas sur que je vais
\usepackage{amsmath, amsfonts, amsthm, amssymb} \usepackage{esint}

\begin{document}

\section*{ÉQUATION FONDAMENTALE DE LA PHYSIQUE DU SOLIDE}


Merde, j'ai pas eu le temps de la notter


on a $\vec r_j (t) = \vec r_j + \vec u(t)$


$$\left<f_i e^{i\vec G \cdot(\vec r_j + \vec u(t))}\right> = f_1e^{-i\vec G\cdot \vec r_j} \left<e^{-i\vec G \cdot \vec u(t)}\right> = e^{-\]}$$

\end{document}