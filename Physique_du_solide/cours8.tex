\documentclass{article}    
\usepackage[utf8]{inputenc}    
    
\title{Épisode 4}    
\author{Jean-Baptiste Bertrand}    
\date{\today}    
    
\setlength{\parskip}{1em}    
    
\usepackage{physics}    
\usepackage{graphicx}    
\usepackage{svg}    
\usepackage[utf8]{inputenc}    
\usepackage[T1]{fontenc}    
\usepackage[french]{babel}    
\usepackage{fancyhdr}    
\usepackage[total={19cm, 22cm}]{geometry}    
\usepackage{enumerate}    
\usepackage{enumitem}    
\usepackage{stmaryrd}    
\usepackage{mathtools,slashed}
%\usepackage{mathtools}
\usepackage{cancel}
    
\usepackage{pdfpages}
%packages pour faire des math    
%\usepackage{cancel} % hum... pas sur que je vais le garder mais rester que des fois c'est quand même sympatique...
\usepackage{amsmath, amsfonts, amsthm, amssymb}    
\usepackage{esint}  
\usepackage{dsfont}

\usepackage{import}
\usepackage{pdfpages}
\usepackage{transparent}
\usepackage{xcolor}
\usepackage{tcolorbox}

\usepackage{mathrsfs}
\usepackage{tensor}

\usepackage{tikz}
\usetikzlibrary{quantikz}
\usepackage{ upgreek }

\newcommand{\incfig}[2][1]{%
    \def\svgwidth{#1\columnwidth}
    \import{./figures/}{#2.pdf_tex}
}

\newcommand{\cols}[1]{
\begin{pmatrix}
	#1
\end{pmatrix}
}

\newcommand{\avg}[1]{\left\langle #1 \right\rangle}
\newcommand{\lambdabar}{{\mkern0.75mu\mathchar '26\mkern -9.75mu\lambda}}

\pdfsuppresswarningpagegroup=1


\begin{document}

\section*{Modèle de Drude}

La probabilité de collision d'un éléctron est donnée par $\frac{1}{\tau}$ (par unité de temps) 

$\mathcal{P}(t)$ est la probailité qu'il n'y ai \underline{pas} de colision entre $0$ et $t$

$$\mathcal{P} (t+dt) = \mathcal{P} (t) \qty(1- \frac{dt}{\tau} )$$ 


$$d \mathcal{P} = \mathcal{P}(t+dt) - \mathcal{P} (t)= -\mathcal{P} (t) \frac{dt}{\tau} $$ 


$$\implies \mathcal{P} (t) = c e^{-t/\tau}$$ 


Calcul du temps moyen entre deux collisions

$$ \avg{t} = \int_{0}^\infty \frac{te^{-t/\tau}}{\tau} dt = \tau \int_{0}^\infty ? - \int_{0}^\infty - e^{-u} du = \tau $$ 

On s'interesse maintenant à la quantitée de mouvement 


$$	\vb{p}(t+dt) = \frac{dt}{\tau} \vb{F} dt + \left( 1-\frac{dt}{\tau}  \right) \left( \vb{p}(t) + \vb{F} dt \right) = \vb{p}(t) - \frac{dt}{\tau} \vb{p}(t) + \vb{F}dt
$$
$$\dv{\vb{p}}{t} = - \frac{\vb{p}(t)}{\tau} + \vb{F}$$ 



La force sur les éléctrons, si on ne considère qu'un champ éléctrique est 

$$\vb{F} = -e\vb{E}$$ 

donc 

$$\frac{d\vb{p}}{t} - \frac{\vb{p}}{\tau} - e\vb{E} = 0$$ 

$$\vb{j} = ne\vb{v}$$ 

$$e\vb{E}=  \frac{m\vb{v}}{\tau} = \frac{m}{\tau} \frac{\vb{j}}{ne} $$ 

$$\vb{E} = \frac{m}{ne^2\tau} \vb{j} = \rho \vb{j}$$ 

$$\vb{j} = \underbrace{\frac{ne^2\tau}{m}}_{\sigma} \vb{E}$$ 


On considère mainteant la force de Lorentz

$$\vb{F} = -e\left( \vb{E} + \vb{v} \times \vb{B} \right) $$ 

On considère que $\vb{v}$ est dans le plan $x,y$ et $\vb{B}$ est en $\vb{z}$ donc

$$\frac{d\vb{p}}{dt} = - \frac{\vb{p}}{\tau} = -e \vb{E}v_{y}B\hat x + ev_{x} B \hat y$$  

\begin{figure}[ht]
    \centering
    \incfig{force-de-lorentz-dans-une-plaque-conductrice}
    \caption{Force de Lorentz dans une plaque conductrice}
    \label{fig:force-de-lorentz-dans-une-plaque-conductrice}
\end{figure}

$$\dotsb$$ 
$$R_{H} = \frac{E_{y}}{j_{x}B} = - \frac{1}{ne}  $$ 

$$\frac{1}{\tau} = \frac{1}{\tau_{\rm déf}} + \frac{1}{\tau_{\rm rés}} $$ 


$$\sigma = \frac{ne^2\tau}{m} = \frac{1}{\rho} $$ 

$$\rho = \rho_{déf} +\rho_{res} + \rho_{e-e}$$ 

\begin{figure}[ht]
    \centering
    \incfig{résistance-en-fonction-de-la-température}
    \caption{résistance en fonction de la température}
    \label{fig:résistance-en-fonction-de-la-température}
\end{figure}

\end{document}
