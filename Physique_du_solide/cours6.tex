\documentclass{article}    
\usepackage[utf8]{inputenc}    
    
\title{Épisode 4}    
\author{Jean-Baptiste Bertrand}    
\date{\today}    
    
\setlength{\parskip}{1em}    
    
\usepackage{physics}    
\usepackage{graphicx}    
\usepackage{svg}    
\usepackage[utf8]{inputenc}    
\usepackage[T1]{fontenc}    
\usepackage[french]{babel}    
\usepackage{fancyhdr}    
\usepackage[total={19cm, 22cm}]{geometry}    
\usepackage{enumerate}    
\usepackage{enumitem}    
\usepackage{stmaryrd}    
    
%packages pour faire des math    
%\usepackage{cancel} % hum... pas sur que je vais le garder mais rester que des fois c'est quand même sympatique...
\usepackage{amsmath, amsfonts, amsthm, amssymb}    
\usepackage{esint}  


\begin{document}

$$g(\omega) = \frac{2N}{\omega} (\omega_{n}^2-\omega^2)^{-\frac{1}{2}}$$ 

$$\omega_{m}^2 = \frac{4c}{m} $$ 


$$U = \int_{0}^\omega \frac{2N}{\pi} (\omega_{n}^2 - \omega^2)^{- \frac{1}{2}} \frac{\hbar\omega}{e^{i\hbar\omega}-1} = \frac{2N}{\pi} k_{B}T\int_{0}^{\omega_{m}} (x^2_m-x^2)^{- \frac{1}{2}} \frac{x}{e^x-1} \dd x$$ 

\begin{figure}[ht]
    \centering
    \incfig{brache-acoustique}
    \caption{brache acoustique}
    \label{fig:brache-acoustique}
\end{figure}

Il a écris 1000000 équation pendant que je fasais le shéma

Wtf il dessine des trucs à des endroits aléatoire au tableau

\section*{Nouvelle section}

Les métaux ont des éléctrons délcalissé. 

Il existe deux approches pour décrire les métaux, l'approche classique et l'approche quantique. La distinction entre les deux traitement viens de la densité d'éléctrons



\underline{Traitement quantique} 

L'énérgie qui est importante est l'énérigie cintétique

$$H = \frac{p^2}{2m} $$ 


\begin{figure}[ht]
    \centering
    \incfig{cristal-unidimentionel}
    \caption{cristal unidimentionel}
    \label{fig:cristal-unidimentionel}
\end{figure}


$$- \frac{\hbar^2}{2m} \dv[2]{x}\Psi(X) = E_{n}\Psi(x)$$ 

$$\boxed{\Psi_{n}(0)=\Psi_{n}(L) = 0;\qquad \Psi_{n}(x) = A\sin(k_nx)}$$ 

$$ \frac{\hbar^2}{2m} A^2 \sin(\frac{nx\pi}{L}) = E_{n}A \sin( \frac{nx\pi}{L})$$ 

$$E_{n}= \frac{\hbar}{2m} \qty(\frac{n\pi}{L} )^2 \to ? \text{(Il viens de l'effacer smh)}$$ 

On a deux spins possibles. Les éléctrons ont autant de chances d'avoir l'un que l'autre

$$N=2n_F$$ 



\headrule

$$f(\epsilon) = \frac{1}{e^{\epsilon-\mu/k_BT}+1} $$ 

$$f(\epsilon) \approx e^{\mu -\epsilon/k_{B}T$$ 

\begin{figure}[ht]
    \centering
    \incfig{niveau-de-fermi}
    \caption{niveau de fermi}
    \label{fig:niveau-de-fermi}
\end{figure}



$$-\frac{\hbar^2}{2m} \grad^2 \Psi_{k}(\vec r) = \epsilon_{k}\Psi_{k}(\vec r)$$ 

$$\Psi_{k}(\vec r) \Psi_{k}(x,y,x) = \Psi_{k}(x+L,y,z)$$ 

$$\Psi_k(\vec r) = \Psi_{0}e^{i\vec k \cdot \vec r} $$ 

$$-\frac{\hbar^2}{2m} \qty(\frac{-k^2}{2m} ) \qquad k^2 = k_{x}^2+k_{y}^2 + k_{z}^2$$ 

$$\epsilon_{f}= \frac{\hbar^2k_{^2_f}}{2m} $$ 

\begin{figure}[ht]
    \centering
    \incfig{sphère-de-fermi}
    \caption{sphère de fermi}
    \label{fig:sphère-de-fermi}
\end{figure}

$$N(\epsilon) = \frac{4{\pi}k_{f}^3}{3} \over \qty(\frac{2\pi}{L} )^3 = \frac{Vk_{f}^3}{3\pi^2}$$ 

$$k_{f}= \qty(\frac{3\pi^2N}{V})^{1/3}$$ 

$$\epsilon_{f}= \frac{\hbar^2}{2m} \qty(\frac{3\pi^2N}{V} )^{2/3}$$ 

$$T_{f}= \frac{\espilon_{f}}{k_{B}} \sim 10^4K$$ 

\headrule

On définit la densité (tout cours?)

$$D(\epsilon) = \dv{N}{\epsilon}$$ 

$$N(\epsilon) = \frac{V}{3\pi^2} \frac{\qty(2m\espilon)^{3/2}}{\hbar^3}$$ 


$$\dv{N}{\epsilon} = \frac{V}{3\pi^2\hbar^3} \frac{3}{2} \qty(2m)^{\frac{3}{2}}\espion^{1/2} = \frac{3N}{2E} $$ 



\begin{figure}[ht]
    \centering
    \incfig{fonction-dentsitée}
    \caption{fonction dentsitée}
    \label{fig:fonction-dentsitée}
\end{figure}


$$\Delta U = U(T) = U(0)\approx \Delta N_{\text{excitées}}k_{B}T$$ 


$$\Delta N_{\rm excités} = D(\epsilon_f)\Delta \epsilon = \frac{3N}{2K_{B}T_{f}} k_{B}T = \frac{3NT}{2T_{f}} $$ 

$$\Delta U = \frac{3NT^2}{2T_{F}} k_B$$ 

$$c = \dv{T}\Delta U = 3NK_{B} \frac{T}{T_F}$$ 

Le facteur $T_{f}$ est important car il change complètement l'ordre de grandeur des prédictions!


$$U = \int_{0}^\infty \dd\epsilon \epsilon D(\epsilon) f(\epsilon)$$ 

Il faut faire des tours de passe passe. Premier tour de passe passe: 

$$\epsilon_{f}N = \int_{0}^\infty \espilon_{f}D(\epsilon)f(\epsilon) \dd \epsilon$$ 

$$\pdv{T}(U-\epsilon_{f}N) = \pdv{U}{T} = C_e$$ 

$$C_{e}= \int_{0}^\infty (\epsilon-\epsilon_{f})D(\espilon) \pdv{f(\epsilon)}{T} \dd \espilon$$ 

$$ \pdv{f}{T} = \frac{(-1)e^{\epsilon-\mu/k_{B} T}}{\qty(E^\dots + 1)^2} \frac{(-1)(\epsilon-\mu)}{k_BT^2}$$ 

La il réecrit l'intégrale avec la dérivée pis il évalue $D$ à $\epsilon_{f} $pour une certaine raison  

Oh, c'est une approximation finalement, la fonction est très piqué alors c'est essentiellement un delta en $\epsilon_f$ 





\end{document}
