\documentclass{article}    
\usepackage[utf8]{inputenc}    
    
\title{Épisode 4}    
\author{Jean-Baptiste Bertrand}    
\date{\today}    
    
\setlength{\parskip}{1em}    
    
\usepackage{physics}    
\usepackage{graphicx}    
\usepackage{svg}    
\usepackage[utf8]{inputenc}    
\usepackage[T1]{fontenc}    
\usepackage[french]{babel}    
\usepackage{fancyhdr}    
\usepackage[total={19cm, 22cm}]{geometry}    
\usepackage{enumerate}    
\usepackage{enumitem}    
\usepackage{stmaryrd}    
    
%packages pour faire des math    
%\usepackage{cancel} % hum... pas sur que je vais le garder mais rester que des fois c'est quand même sympatique...
\usepackage{amsmath, amsfonts, amsthm, amssymb}    
\usepackage{esint}  
G

\begin{document}

Il écris pleins de chiffres 

$$E_{f} = \frac{\hbar^2}{2m}\left( 3\pi^2 n \right)^{3/2} = \dotsb = 0.4 {\rm meV}$$ 

$$n=\frac{\rho}{m} = \dotsb = 1.6E22/{\rm cm^3} $$ 

$$T_{F} = 5 {\rm K}$$ 



\section*{La structure de bande!}

Jusqu'à maintenant on à traité les éléctrons comme un gaz de particules libes, on à jamais vraiment considéré les effet de leur environement. On va maintenant s'intéresser au fait qu'il sont dans un réseau periodique.


\begin{figure}[ht]
    \centering
    \incfig{relation-de-dispersion}
    \caption{relation de dispersion}
    \label{fig:relation-de-dispersion}
\end{figure}

On a levé de dégénéressance!!! ????


$$\Psi_{k}^a = e^{i\vb{k}\cdot r} u_{k}^\alpha(\vb{r})\qquad u_{k}^\alpha(\vb{r}) = u_{k}^\alpha(\vb{r} + \vb{R})$$ 

Le $\alpha$ sert à parler de la bande dans laquelle ont se trouve.

On va s'intéresser à l'effet de la periodicité sur le Hamiltonnien du système

$$H = \underbrace{H_{0}}_{\frac{p^2}{2m} } + V$$ 

La solution à l'hamiltonien non perturbé est:

$$H_{0} \ket{k} = \epilon_{k}^0 \ket{k}\quad \epsilon_{k}^0 = \frac{\hbar^2k^2}{2m} $$ 


$$\bra{k}V\ket{k} \text{n'est pas important apparement}$$ 


Puisuqe le potentiel est periodique, il admet un décomposition de Fourrier et donc:

$$\bra{k'}V\ket{k} = \begin{cases}
	0 & \vb{k}'-\vb{k} \neq \vb{G}\\
	V_{\vb{G}} & \vb{k}' - \vb{k} = \vb{G}
\end{cases}$$ 

$$\Psi_{k} = \sum_{G} A_{\vb{G}+\vb{k}} e^{i(\vb{k} + \vb{G}) \cdot \vb{r}} = e^{i\vb{k} \cdot \vb{r}} \left( \sum_{G} A_{\vb{G} + \vb{k}} e^{i \vb{G}\cdot r} \right) $$ 

Il y a deux conséquences au théorème de Bloch:
\begin{itemize}
	\item Tout les excitation peuvent être décritent dans la première zone de Brilloin
	\item ?????
\end{itemize}

\begin{figure}[ht]
    \centering
    \incfig{zonne-de-brilloin-repliée}
    \caption{Zonne de brilloin repliée}
    \label{fig:zonne-de-brilloin-repliée}
\end{figure}

Correction de l'énérgie 
$$\epsilon_{k} = \epsilon_{k}^0 + \underbrace{\bra{k}V\ket{k}}_{\text{même} V_{0} \forall k}+ \sum_{k\neq k'} \frac{|\bra{k'}V\ket{k}|^2}{\epsilon_{k}^0-\epsilon_{k}^0 }  $$ 

Il y a un dégénéressance du au fait que la relation de dispertion est symétique : $\pm k$ donne la même énérige. On doit donc utiliser la théorie des perturbations dégénéré.

$$\ket{\Psi} = \phi_{k} \ket{k} + \phi_{k+G} \ket{k+G}$$ 

$$\sum_{m} H_{nm} \phi_{m} = E \phi_m$$ 

$$\bra{k}H\ket{k} = E_{k}^0 + V_{0}$$ 

$$\bra{k+G}H \ket{k+g} = E_{k+G}^0 + V_0$$ 

$$\bra{k+G}H\ket{k} = V_{\vb{G}}$$ 

$$\bra{k}H\ket{k+G} = V_{-\vb{G}} = V_{\vb{G}}^*$$ 


La représentation matricielle est alors

$$\cols{E_{k}^0 +V_0 & V_{G}^* \\ V_{G} & E_{k+G} +V_0}$$

Ses valeurs propres sont $$E_{\pm} = E_{k} + V_{0} \pm |V_{\vb{G}}| $$ 

	
\end{document}
