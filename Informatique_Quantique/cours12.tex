\documentclass{article}    
\usepackage[utf8]{inputenc}    
    
\title{Épisode 4}    
\author{Jean-Baptiste Bertrand}    
\date{\today}    
    
\setlength{\parskip}{1em}    
    
\usepackage{physics}    
\usepackage{graphicx}    
\usepackage{svg}    
\usepackage[utf8]{inputenc}    
\usepackage[T1]{fontenc}    
\usepackage[french]{babel}    
\usepackage{fancyhdr}    
\usepackage[total={19cm, 22cm}]{geometry}    
\usepackage{enumerate}    
\usepackage{enumitem}    
\usepackage{stmaryrd}    
    
%packages pour faire des math    
%\usepackage{cancel} % hum... pas sur que je vais le garder mais rester que des fois c'est quand même sympatique...
\usepackage{amsmath, amsfonts, amsthm, amssymb}    
\usepackage{esint}  


\begin{document}
2022-32-01

\setcounter{section}{2}

\section{Correction d'erreur}

\subsection{codes classiques}

Canal symétrique binaire

\begin{cases}
	0 &\to 0 : \text{probabilité }  p-1\\ 
	{}& \to 1: \text{probabilité } p 
\end{cases}

\begin{cases}
	1 &\to 1 : \text{probabilité }  p-1\\ 
	{}& \to 0: \text{probabilité } p 
\end{cases}

Pour corriger les erreur on rajoute de la \underline{redondance}  

\underline{ex: code de répétition} 

\[ 0 \to000 \qquad 1\to 111 \] 

\underline{Code classique (linéaire)} 

\[ \left[ n,k,d  \right]  \] 

\noindent$n$: nombre de bits envoyé (nombres de bits physiques)\\
$k$: nombre de bit d'information\\
$d$: distance du code 

\underline{distance:} nombre de bits qu'il faut inverser pour passer d'un mot code à une autre 

code de répétition: 2 mot codes: $\{ 000, 111 \} $, $\left[ 3,1,3 \right] $ 

code universel: tout les chaînes de bits sont des mots codes 
: $\left[ n,n, 1 \right] $ 


code de parité: 
\begin{matrix}
	00\\01\\10\\11
\end{matrix}\to \begin{matrix}
	000\\011\\101\\110 
\end{matrix}, $\left[ 3,2,2 \right] $ 

Objectifs du design de code  
\begin{itemize}
	\item avec un $n,k$ donné maximiser $d$\\
	\item avec $n,d$ fixé, maximiser k\\
	\item $d,k$ minimiser $n$  
\end{itemize}

Le \underline{taux du code} est \[ R = \frac{k}{n}  \]  

Pour le code de répétition $[n,1,n]$, $n\to \infty \implies R \to 0$  

En général, on peut corriger $\left\lfloor \frac{d-1}{2}  \right\rfloor$ erreurs pour le canal symétrique binaire

Avec le code de répétition 3 bits

la probabilité d'avoir 2 erreur ou plus est de 

\[ 3p^{2}\left( 1-p \right) +p^{3} \sim p^2 <p \] 

En général $p_L \sim p^{\frac{d+1}{2} }$ 


\subsection{Codes quantiques}

Arguments en défaveur de la possibilité de la correction d'erreur quantique:

\begin{enumerate}
	\item Pas de copie possible (thm de non clonage )
	\item Le nombre d'erreurs est \textit{infini}
	\item La mesure en MQ mène à l'effondrement de la fonction d'onde $\implies $ pas de vote pas majorité???????? 
	\item Comment gérer l'intrication avec l'environnement?
\end{enumerate}


code de répétition quantique:

$\ket{0} \to \ket{000}$ 
$\ket{1}\to\ket{111}$ 

\begin{center}
	\begin{quantikz}
		\alpha \ket{0} + \beta \ket{1} & \ctrl{1} &\ctrl{2} &\qw\rstick[wires=3]{$\ket{\psi}$}\\
		\ket{0} &\targ{} &\qw &\qw\\
		\ket{0} & \qw &\targ{} &\qw
	\end{quantikz}
\end{center}


On doit mesurer les \textit{syndromes} de l'état pour savoir s'il y a eu erreur ou non.

On mesure $Z_1 Z_2$ et $Z_2 Z_3$. On peut alors determiner quel erreur est \textit{probablement} arrivée 

on mesure $Z_1 Z_2$ avec

\begin{quantikz}
	\ket{+} & \ctrl{1}&\gate{H} &\meter{}\\
	\ket{\psi} & \gate{U} & \qw &\qw
\end{quantikz} $\to$ \begin{quantikz}
	\ket 0 & \gate{H} &\ctrl{1} & \gate{H} &\gate{H} &\ctrl{2} &\gate{H} &\meter{}\\
	\lstick[wires=3]{$\ket{\psi}$} &\qw & \control{} & \qw & \qw &\qw & \qw &\qw\\
\qw &\qw &\qw &\qw &\qw &\control{} &\qw &\qw\\
\qw &\qw &\qw & \qw &\qw &\qw &\qw & \qw
\end{quantikz} 


\end{document}
