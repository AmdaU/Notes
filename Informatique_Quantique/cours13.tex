\documentclass{article}    
\usepackage[utf8]{inputenc}    
    
\title{Épisode 4}    
\author{Jean-Baptiste Bertrand}    
\date{\today}    
    
\setlength{\parskip}{1em}    
    
\usepackage{physics}    
\usepackage{graphicx}    
\usepackage{svg}    
\usepackage[utf8]{inputenc}    
\usepackage[T1]{fontenc}    
\usepackage[french]{babel}    
\usepackage{fancyhdr}    
\usepackage[total={19cm, 22cm}]{geometry}    
\usepackage{enumerate}    
\usepackage{enumitem}    
\usepackage{stmaryrd}    
\usepackage{mathtools,slashed}
%\usepackage{mathtools}
\usepackage{cancel}
    
\usepackage{pdfpages}
%packages pour faire des math    
%\usepackage{cancel} % hum... pas sur que je vais le garder mais rester que des fois c'est quand même sympatique...
\usepackage{amsmath, amsfonts, amsthm, amssymb}    
\usepackage{esint}  
\usepackage{dsfont}

\usepackage{import}
\usepackage{pdfpages}
\usepackage{transparent}
\usepackage{xcolor}
\usepackage{tcolorbox}

\usepackage{mathrsfs}
\usepackage{tensor}

\usepackage{tikz}
\usetikzlibrary{quantikz}
\usepackage{ upgreek }

\newcommand{\incfig}[2][1]{%
    \def\svgwidth{#1\columnwidth}
    \import{./figures/}{#2.pdf_tex}
}

\newcommand{\cols}[1]{
\begin{pmatrix}
	#1
\end{pmatrix}
}

\newcommand{\avg}[1]{\left\langle #1 \right\rangle}
\newcommand{\lambdabar}{{\mkern0.75mu\mathchar '26\mkern -9.75mu\lambda}}

\pdfsuppresswarningpagegroup=1


\begin{document}
2022-31-02

\section*{Correction d'erreur (suite)}

La barre est souvent utilisée pour désigner un bit \textit{logique} ex:
\[ \ket{\bar 0} = \ket{000} \qquad \ket{\bar1} = \ket{111}  \] 

Circuit pour mesurer les stabilisateurs

\begin{quantikz}
	\ket+ & \ctrl{1} &\meter{}\\
				&\gate U &\qw
\end{quantikz}

Cas où l'erreur commute avec le stabilisateur

\begin{quantikz}
	\ket+ & \qw &\ctrl{1} &\meter{}\\
				\qw&\gate E&\gate U &\qw
\end{quantikz}=\begin{quantikz}
				\ket+ & \ctrl{1} &\qw &\meter{}\\
							&\gate U &\gate E&\qw
\end{quantikz}


\subsection*{opérateurs logiques}

Afin de préserver la \textit{stabilité} \textit{durant} les calculs on veut pouvoir effectuer les opération directement sur les qubits logiques eux-mêmes. On définit donc

\[ \bar X \ket{\bar 0} = \ket{\bar 1} \qquad \bar X = X_1 X_2 X_3 \] 

Il est évidemment essentiel que tout ces opérateurs \textit{logiques} commutent avec les stabilisateurs

\[ \bar Z = Z_1 \] 

\begin{tcolorbox}[title=Rotation logique]
	 \begin{center}
	 	Un rotation logique \textbf{n'est pas} simplement le produit de la rotation sur chaque qubit \[ \bar R(\theta)\neq R_{x_1}  (\theta) R_{x_2} (\theta) R_{x_3} (\theta) \] 
		\[ \bar R_x (\theta) = e^{-i \frac{\theta}{2} \bar X} \] 
	 \end{center}
\end{tcolorbox}


Le code de répétition ne peux pas corriger une erreur $Z$. On aurait pu choisir un code différent tel quel \[ \ket{ \bar 0 } = \ket{+++} \quad \ket{\bar 1} =  \ket{---}\] 

Les stabilisateur sont alors $X_1 X_2$ et $X_2 X_3$. On as alors gagné la capacité de corriger les erreurs en $Z$ mais plus celles en $X$.  







\end{document}
