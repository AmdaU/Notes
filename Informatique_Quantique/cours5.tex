\documentclass{article}    
\usepackage[utf8]{inputenc}    
    
\title{Épisode 4}    
\author{Jean-Baptiste Bertrand}    
\date{\today}    
    
\setlength{\parskip}{1em}    
    
\usepackage{physics}    
\usepackage{graphicx}    
\usepackage{svg}    
\usepackage[utf8]{inputenc}    
\usepackage[T1]{fontenc}    
\usepackage[french]{babel}    
\usepackage{fancyhdr}    
\usepackage[total={19cm, 22cm}]{geometry}    
\usepackage{enumerate}    
\usepackage{enumitem}    
\usepackage{stmaryrd}    
\usepackage{mathtools,slashed}
%\usepackage{mathtools}
\usepackage{cancel}
    
\usepackage{pdfpages}
%packages pour faire des math    
%\usepackage{cancel} % hum... pas sur que je vais le garder mais rester que des fois c'est quand même sympatique...
\usepackage{amsmath, amsfonts, amsthm, amssymb}    
\usepackage{esint}  
\usepackage{dsfont}

\usepackage{import}
\usepackage{pdfpages}
\usepackage{transparent}
\usepackage{xcolor}
\usepackage{tcolorbox}

\usepackage{mathrsfs}
\usepackage{tensor}

\usepackage{tikz}
\usetikzlibrary{quantikz}
\usepackage{ upgreek }

\newcommand{\incfig}[2][1]{%
    \def\svgwidth{#1\columnwidth}
    \import{./figures/}{#2.pdf_tex}
}

\newcommand{\cols}[1]{
\begin{pmatrix}
	#1
\end{pmatrix}
}

\newcommand{\avg}[1]{\left\langle #1 \right\rangle}
\newcommand{\lambdabar}{{\mkern0.75mu\mathchar '26\mkern -9.75mu\lambda}}

\pdfsuppresswarningpagegroup=1

\begin{document}

\setcounter{subsection}{5}
\setcounter{section}{2}

\subsection{Universalité}

Calcul Classique $\{ \text{NAND}, \text{COPY}   \} $ 

Calcul classique réversible $\{ \text{Toffoli}  \} $ 

Calcul quantique: plusieurs choix possibles

\begin{enumerate}
	\item Matrice générique $4x4$ ( $M\in \text{SU}(4) $  )
		\underline{Ex} CR_x($\theta$)  $\quad \frac{ \theta}{\pi}$ irrationnel 
\begin{verbatim}
----.----
    |
-[Rx(z)]-
\end{verbatim}
\item SU(2)+CNOT
	2 générateurs de rotations sont suffisant pour générer SU(2)
	CNOT permet de générer de l'enchevêtrement. Ce n'est pas la seul pour qui permet de faire cela et d'autre porte aurait fait le travail (Cz par exemple)

Cz:
\begin{verbatim}
----.----   ----o----
    |     =     |
--[(+)]--   ----o----
\end{verbatim}

\item Un ensemble discret de porte peut être universel:
	ex: $\{ \text{H}, \text{S}, \text{T}, \text{CNOT}     \} $ 

Ces portes sont importantes dans la théorie du calcul tolérant au faute.

$T^2=S$ , on pourrait donc enlever $S$ de l'ensemble. Par contre faire des portes $T$ est vraiment difficile donc en pratique on aime remplacer $T^{2}$ par $S$      

\begin{tcolorbox}[title=]
	$\{ \text{H} , \text{S} , \text{CNOT}  \} $ n'est pas universel mais génère le groupe Clifford à plusieurs qubits. 
v
\end{tcolorbox}

\begin{tcolorbox}[title=Gotesman-Krill 98]
	 Le calcul quantique avec seulement les portes Clifford peut-être simulé classiquement.
	 L'intrication est nécessaire mais pas suffisant pour avoir un avantage quantique.
	 \underline{Preuve Sketch}: 

	 Suivre l'évolution des opérateurs de Pauli à travers le circuit. 
\end{tcolorbox}
\end{enumerate}

\subsection{L'algorithme de Deustch}

Soit une fonction classique $f: \mathds{Z}  \mapsto \mathds{Z}$.On cherche à savoir si $f$ est balancée ou constante. 

$$f \text{ balancé } f(0)\neq f(1) \qquad f(0) = 0, f(1)=1 \quad \text{ou}\quad f(0)=1, f(1)=0$$ 
$$f \text{ constante } f(0)= f(1)\qquad f(0) = 0, f(1)=0 \quad \text{ou}\quad f(0)=1, f(1)=1 $$ 

Calcul classique, il fait évaluer $f$ deux fois et comparer calcul quantique. On peut évaluer qu'une seule fois ...


Puisque $f$ n'est pas forcément réversible car utilise un registre.

$U_f \ket{xy} \equiv \ket{x, y \oplus f(x) }$ 

Ex: $f$ constant $f(0)=f(1)=1$  
$$U_f \ket{10}= \ket{1, 0 \oplus 1} = \ket{11}$$ 

$$U_f^{2}\ket{xy} = \ket{x,y \oplus 2 f(x)} = \ket{xy}$$ 

$$\implies U_f^{2}= \mathds{1}$$ 

$$U_f = U_f^{\dagger} = U_f^{-1}$$ 

Ce circuit quantique est 
\begin{verbatim}
|0>----[H]----|----|----[H]----[Mes]
              |    |
              | U_f|
              |    |
|0>--[X]--[H]-|----|--------[Poubelle]
\end{verbatim}

$$\ket{\psi_{0}} = \ket{00}$$ 
$$\ket{\psi_{1}} = \left( \frac{\ket{0}+\ket{1}}{\sqrt{2}}  \right) \otimes \left( \frac{\ket{0}-\ket{1}}{\sqrt{2}}  \right) = \ket{+-}$$ 

Appliquons $U_f$ sur $\ket{x} \otimes \left( \frac{\ket{0}-\ket{1}}{\sqrt{2}}  \right) $  

$$U_f \left( \frac{\ket{X0}-\ket{X1}}{\sqrt{2}}  \right) = \frac{1}{\sqrt{2}} \left( \ket{X,f(x)} - \ket{x, 1 \oplus f(x)} \right) $$ 

Si $f(x)=0$ ou $1$, l'état final est le même à un signe près  


$$U_f \ket{x} \otimes \left( \frac{\ket{0}-\ket{1}}{\sqrt{2}}  \right) = (-1)^{f(x)} \ket{x}\ket{1}$$ 

$$\dotsb$$ 


\hrule

\underline{Préparation d'un état de Bell} 


\begin{verbatim}
|0>----[H]----.----
              |    |Phi+>
|0>----------(x)---
\end{verbatim}

$$\ket{\Psi_{1}} = \left( \frac{\ket{0}+\ket{1}}{\sqrt{2}}  \right) \otimes \ket{0} = \frac{1}{\sqrt{2}} \left( \ket{00} + \ket{10} \right)  $$ 

$$\ket{\psi_{2}} =\text{CNOT} \ket{\psi_1} = \frac{1}{\sqrt{2}} \left( \ket{00}+\ket{11} \right) $$ 


\begin{verbatim}
|0>----[H]----.----
              |    |Psi+>
|1>----------(x)---
\end{verbatim}
(Démarche similaire pour le démontrer)


\begin{verbatim}
|0>----[H]-,--.---,--
 !         !  |   !
|0>--------!-(x)--!--
 !         !      !
 !         !      !
 !         !      !
 Z         X      X
 I         I      X
 ____________________
 I         I      Z
 Z         Z      Z
\end{verbatim}

$\ket{00}$ étant propre de $ZI$ et $IZ$   

\underline{SWAP gate} 

\begin{verbatim}
----x----   ----o---(+)---o----
    |     =     |    |    |    
----x----   ---(x)---o---(x)---
\end{verbatim}


$$\ket{\psi_{0}} = \ket{ij}$$ 

$$\ket{\psi_{1}} = \ket{i, j \oplus i}$$ 

$$\ket{\psi_2} =\ket{i \oplus j \oplus i, j \oplus i} = \ket{2 i \oplus j, j \oplus i} = \ket{j, j \oplus i}$$ 

$$\ket{\psi_{3}} = \ket{j, j \oplus i \oplus j} = \ket{ji} = \text{SWAP} \ket{ij}$$ 


\underline{Notation binaire} 

On écrit le nombre binaire en décimale.


\end{document}
