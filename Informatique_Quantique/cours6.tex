\documentclass{article}    
\usepackage[utf8]{inputenc}    
    
\title{Épisode 4}    
\author{Jean-Baptiste Bertrand}    
\date{\today}    
    
\setlength{\parskip}{1em}    
    
\usepackage{physics}    
\usepackage{graphicx}    
\usepackage{svg}    
\usepackage[utf8]{inputenc}    
\usepackage[T1]{fontenc}    
\usepackage[french]{babel}    
\usepackage{fancyhdr}    
\usepackage[total={19cm, 22cm}]{geometry}    
\usepackage{enumerate}    
\usepackage{enumitem}    
\usepackage{stmaryrd}    
\usepackage{mathtools,slashed}
%\usepackage{mathtools}
\usepackage{cancel}
    
\usepackage{pdfpages}
%packages pour faire des math    
%\usepackage{cancel} % hum... pas sur que je vais le garder mais rester que des fois c'est quand même sympatique...
\usepackage{amsmath, amsfonts, amsthm, amssymb}    
\usepackage{esint}  
\usepackage{dsfont}

\usepackage{import}
\usepackage{pdfpages}
\usepackage{transparent}
\usepackage{xcolor}
\usepackage{tcolorbox}

\usepackage{mathrsfs}
\usepackage{tensor}

\usepackage{tikz}
\usetikzlibrary{quantikz}
\usepackage{ upgreek }

\newcommand{\incfig}[2][1]{%
    \def\svgwidth{#1\columnwidth}
    \import{./figures/}{#2.pdf_tex}
}

\newcommand{\cols}[1]{
\begin{pmatrix}
	#1
\end{pmatrix}
}

\newcommand{\avg}[1]{\left\langle #1 \right\rangle}
\newcommand{\lambdabar}{{\mkern0.75mu\mathchar '26\mkern -9.75mu\lambda}}

\pdfsuppresswarningpagegroup=1

\begin{document}

2022-09-20

\section*{Deustch}

\begin{quantikz}
	\ket{0} & & \gate{H}  &\gate[wires=2]{U_f} &\gate{H} &\meter{}
	\\
	\ket{0} & \gate{X} &  \gate{H} & & \trash{\text{trash}}
\end{quantikz}


\setcounter{section}{2}
\setcounter{subsection}{7}

\subsection{Problème de Deutsh-Jozsa}

On cherche à trouver si 
$$f: \mathds{Z}_2^n\mapsto\mathds{Z}_2$$
est balancé (la moitié des résultats est 1 et l'autre 0) ou constant (toujours 0 ou 1). On sait que $f$ est un ou l'autre.

Classiquement, on doit calculer $f$ $2^{n/2}+1$ fois (au pire)


\begin{quantikz}
	\ket{0}^{ \otimes n } & & \gate{H}\qwbundle{n}  &\gate[wires=2]{U_f} &\gate{H} &\meter{}
	\\
	\ket{0} & \gate{X} &  \gate{H} & & \trash{\text{trash}}
\end{quantikz}


\begin{tcolorbox}[title=]
	 On appelle $U_f$ un oracle en informatique quantique 
\end{tcolorbox}

$$\ket{\psi_{0}} = \ket{0}^{ \otimes n+1 }$$ 

$$\ket{\psi_{1}} = \ket{+}^{\otimes n} \ket{-} = \frac{1}{\sqrt{2}^n} \sum_{x=0}^{2^n -1} \ket{x} \otimes \left( \frac{\ket{0} - \ket{1}}{\sqrt{2}} \right) $$ 

$$x = \sum_{k=1}^{n} x_k 2^{n-k}$$ 

On applique maintenant $U_f$. On a un \textit{phase kick-back}

$$\ket{\psi_{2}} = U_f \ket{\psi_{1}} = \sum_x \frac{(-1)^{f(x)}}{2^{n/2}} \ket{x} \otimes \left( \frac{\ket{0}-\ket{1}}{\sqrt{2}}  \right)  $$ 

L'infromation sur $f(x)$ est conteny dans la phase des $n$ premiers qubits. On applique les Hadamard $H^{\otimes n}$ pour faire \textit{tomber} les phases vers des probabilité

$n=1:$ 
$$H \ket{x} = \begin{cases}
	\frac{\ket{0}+ \ket{1}}{\sqrt{2}} & \text{si }x=0\\
	\frac{\ket{0}-\ket{1}}{\sqrt{2}}& \text{si } x=1 
\end{cases} = \sum_{z=0}^{1} \frac{\left( -1 \right) ^{XZ}}{\sqrt{2^n}} \ket{z} $$ 

On généralise à $n \qeq 1$   

$$H^{ \otimes n } \ket{x_1 \, x_2 \, \dotsb \, x_n } = \bigotimes_{j=1}^n \frac{1}{\sqrt{2}} \sum_{z_j=0}^{1} \left( -1 \right)^{x_j z_{j}}\ket{z_{j}} = \frac{1}{\sqrt{2^n}} \sum_{z_1 , \dotsb, z_n =0}^{1} \left( -1 \right) ^{x_{1z_1+\dotsb}} \ket{z_1 \, \dotsb \, z_n } = \frac{1}{\sqrt{2}^n} \sum_{z=0}^{2^n-1} \left( -1 \right)^(x \cdot z) \ket{z}$$ 

Donc 

$$\ket{\psi_{3}} = H^{ \otimes n } \otimes 1 \ket{\psi_{2}} = \sum_x \frac{(-1)^{f(x)}}{2^{n/2}} H^{ \otimes } $$ 







\end{document}
