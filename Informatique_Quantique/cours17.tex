\documentclass{article}    
\usepackage[utf8]{inputenc}    
    
\title{Épisode 4}    
\author{Jean-Baptiste Bertrand}    
\date{\today}    
    
\setlength{\parskip}{1em}    
    
\usepackage{physics}    
\usepackage{graphicx}    
\usepackage{svg}    
\usepackage[utf8]{inputenc}    
\usepackage[T1]{fontenc}    
\usepackage[french]{babel}    
\usepackage{fancyhdr}    
\usepackage[total={19cm, 22cm}]{geometry}    
\usepackage{enumerate}    
\usepackage{enumitem}    
\usepackage{stmaryrd}    
\usepackage{mathtools,slashed}
%\usepackage{mathtools}
\usepackage{cancel}
    
\usepackage{pdfpages}
%packages pour faire des math    
%\usepackage{cancel} % hum... pas sur que je vais le garder mais rester que des fois c'est quand même sympatique...
\usepackage{amsmath, amsfonts, amsthm, amssymb}    
\usepackage{esint}  
\usepackage{dsfont}

\usepackage{import}
\usepackage{pdfpages}
\usepackage{transparent}
\usepackage{xcolor}
\usepackage{tcolorbox}

\usepackage{mathrsfs}
\usepackage{tensor}

\usepackage{tikz}
\usetikzlibrary{quantikz}
\usepackage{ upgreek }

\newcommand{\incfig}[2][1]{%
    \def\svgwidth{#1\columnwidth}
    \import{./figures/}{#2.pdf_tex}
}

\newcommand{\cols}[1]{
\begin{pmatrix}
	#1
\end{pmatrix}
}

\newcommand{\avg}[1]{\left\langle #1 \right\rangle}
\newcommand{\lambdabar}{{\mkern0.75mu\mathchar '26\mkern -9.75mu\lambda}}

\pdfsuppresswarningpagegroup=1


\begin{document}
2022-11-22

\setcounter{section}{4}
\setcounter{subsection}{3}
\setcounter{subsubsection}{3}



\subsubsection{Qubit supraconducteur: transmon (Co-inventé par alexandre Blais)}


On remplace l'inducteur d'un circuit RLC par une indictance non-linéaire nt_\


\begin{figure}[ht]
    \centering
    \incfig{qbit-transoms}
    \caption{qbit transoms}
    \label{fig:qbit-transoms}
\end{figure}


\[ \begin{matrix}
	\text{Inductance} & \text{Jonction Josephson (J)}\\
	I = \frac{\Phi}{L} & I = I_0 \sin(\frac{2\pi}{\Phi_0} \Phi)\\
	\Phi (t) = \int_{-\infty}^{t}\dd t V(t) & \Phi(t) = \int_{- \infty}^{t}\dd t V(t)\\
E = \int \dd t V(t) I(t)= \int \dd t \dv{\Phi}{t} \frac{\Phi}{L} = \frac{\Phi^{2}}{2L} & E = \dotsb = -\underbrace{I_c \frac{\Phi_{0}}{2\pi}}_{E_J}   \cos(\underbrace{\frac{2\pi}{\Phi_0} \Phi}_{\varphi} )
\end{matrix} \]

où $I_c$ est le courant critique, $	\Phi_0$ est le quanta de flux ($\Phi_0 = \frac{h}{2e} $)



Le Hamiltonien de notre nouveau système est \[ H = 4 E_c \hat n^{2}-E_J \cos(\hat \varphi) = E_c \left[ 4 \hat n^{2}- \frac{E_J}{E_c} \cos(\varphi) \right]  \]


\begin{figure}[ht]
    \centering
    \incfig{potentiel}
    \caption{Potentiel}
    \label{fig:potentiel}
\end{figure}


Puisqu'on a un cosinus, les \textit{petites} différence avec l'oscillateur harmonique commence à apparaitre au 4ème ordre

\[ H_{\text{approx}} = 4 E_c \hat n^{2}+ E_J \frac{\hat\varphi^2}{2} + E_j \frac{\hat\varphi^{4}}{8}  \]


De manière analogue à l'oscillateur harmonique on définit

\[ \hat n = - \frac{i}{2} \left( \frac{E_J}{2E_c}  \right)^{\frac{1}{4} } \left( b - b ^{\dagger} \right) \]


\[ \hat \varphi = \left( \frac{2E_c}{E_j}  \right)^{\frac{1}{4} }\left( b + b ^{\dagger} \right)  \]


Le Hamiltonien peut alors se réécrire comme \[ H = \dotsb = \dotsb(\text{RWA} ) \dotsb = \hbar \left( \omega_q - \frac{k}{2}  \right) b b^{\dagger} + \frac{\hbar{}k}{2} \left( b^{\dagger}b \right) ^2 \]

où $\hbar \omega_q \equiv \sqrt{8E_j E_{c}} - E_c $ et $\hbat k = -E_c $

\[ H = \hbar \left( \OMega_q - \frac{k}{2} + \frac{k}{2} b^{\dagger}b \right) b ^{\dagger} b  \]


\begin{figure}[ht]
    \centering
    \incfig{beep-boop-bap}
    \caption{beep boop bap}
    \label{fig:beep-boop-bap}
\end{figure}

Étant donné la séparation ente les niveau énérgétique on appele souvent les qubits transmons des atomes artificiels. 

\[ \frac{K}{\omega_q} = \frac{E_c}{\sqrt{8E_J{}E_c}-E_c} \approx \frac{E_c}{\sqrt{8E_j{}E_c}} = \sqrt{\frac{E_c}{8E_j} }  \]

Dans la limite $\frac{E_j}{E_c} \to \infty$, le  transmon deviens un oscillateur harmonique






\end{document}
