\documentclass{article}    
\usepackage[utf8]{inputenc}    
    
\title{Épisode 4}    
\author{Jean-Baptiste Bertrand}    
\date{\today}    
    
\setlength{\parskip}{1em}    
    
\usepackage{physics}    
\usepackage{graphicx}    
\usepackage{svg}    
\usepackage[utf8]{inputenc}    
\usepackage[T1]{fontenc}    
\usepackage[french]{babel}    
\usepackage{fancyhdr}    
\usepackage[total={19cm, 22cm}]{geometry}    
\usepackage{enumerate}    
\usepackage{enumitem}    
\usepackage{stmaryrd}    
    
%packages pour faire des math    
%\usepackage{cancel} % hum... pas sur que je vais le garder mais rester que des fois c'est quand même sympatique...
\usepackage{amsmath, amsfonts, amsthm, amssymb}    
\usepackage{esint}  


\begin{document}
2022-32-08

\subsection*{Finalisation de la correction d'Erreur}

On a vu comment on notait les codes de correction d'erreur classique. Les codes de correction d'erreur quantiques (linéaires) eux sont caractérisé par \[ \left[ \left[ n,k,d \right]  \right]  \] 

$n$: nombre de qubits physiques 
$k$: nombre de qubits logiques 
$d$: nombre de minimal de qubits sur lequel il faut agir pour faire une opération logique 


\underline{Ex:} 

\begin{itemize}
	\item code de répétition $\left[ \left[ 3,1,1 \right]  \right] $  
	\item code de shor $\left[ \left[ 9,1,3 \right]  \right] $ 
	\item cpde de surface $\left[ \left[ N, 1, \sqrt{N} \right]  \right] $ 
\end{itemize}

En général, on peut corriger \[ \left\lfloor \frac{d-1}{2}  \right\rfloor \] 
Le probabilité d'une erreur logique va comme \[ P_L = \left( \frac{p}{p_{\text{seuil}}}   \right)^{\frac{d-1}{2} +1} \] 

\begin{figure}[ht]
    \centering
    \incfig{graphique-typique-de-correction-d'erreur}
    \caption{Graphique typique de correction d'erreur}
    \label{fig:graphique-typique-de-correction-d'erreur}
\end{figure}

\setcounter{section}{3}

\section{Chapitre 4: Dispositif}

\subsection{Critères de D. Vincerizo}

\begin{center}
	\begin{quantikz}
		\ket0 & \gate[wires=3]{U} &\meter[wires=3]{}\\
		\ket0 &\qw & \meter{} &\\
		\ket0 &\qw & \meter{} &
	\end{quantikz}
\end{center}


\begin{tcolorbox}[title=Pour faire un ordinateur quantique il faut: ]
	\begin{enumerate}
	\item Des qubits bien définis $\left\{ \ket{0}, \ket{1} \right\} $ 
	\item pouvoir initialiser les qubits dans un état précis
	\item avoir un ensemble de portes 
	\item pouvoir mesurer les qubits
	\item des opérateurs de bonne fidélité 
	\end{enumerate} 
\end{tcolorbox}


On prend l'exemple d'un spin $\frac{1}{2}$ 


\begin{enumerate}
	\item cubits bien définit
	\[ \left\{ \ket{\downarrow}, \ket{\uparrow} \right\}  \] 

On applique un champ magnétique pour lever la dégénéressance

\item Initialisation
	On veut $\ket{\psi} \to \ket{0}$ 

Pour cela, on peut mettre le système en contacte avec un réservoir de température \textit{nulle} ( $\hbar \omega \ll k_B T$  ). On laisse alors se produire une relaxation de type $T_1 $  

Un autre stratégie consiste à mesurer le qubits et à aplliquer une porte conditionelle

\begin{center}
	\begin{quantikz}
		{}	& \bend{} & \cwbend{} &\\
		\qw & \meter{} \vcw{-1} & \gate X \vcw{-1} & \qw \ket 0
	\end{quantikz}
\end{center}

\item Portes logiques

\[ H = - \vb{u} \cdot  \vb{B} \] 

\[ U(t) = e^{-iHt} \] 

\[ R_x (\theta) = e^{-i \omega t} \] 

\[ \theta = Bt \] 


Portes à deux qubits

On veut une intéraction du type $H_{\text{int}} = g_{ij} \sigma_i \otimes \sigma_j $ 


ex: couplage $XX$ 

\[ U_{xx} = e^{-i \frac{\pi}{4} \sigma_x \sigma_x} \] 


\[ U_{xx}\left( t- \frac{\pi}{4g_{xx} } \right)  \ket{0} = \frac{\ket{0}-\ket{1}}{\sqrt{2}}  \] 


ex CZ:

\[ H_{\text{int}} = g_{zz} \sigma_z^{(1)} \sigma_z^{(2)}  \] 

\[ CZ = e^{i\frac{\pi}{4} } R_{z_1} \left( \pi/2 \right) R_{z_2} \left( \pi/2 \right)  U_{zz} \left( t = \frac{2}{4g_{zz} }   \right)  \] 

Certaine portes à 2 qubits utilisent un système intermédiaire (un \textit{bus}). L'avantage et que cela permet de coupler des qubits qui sont physiquement très éloignées


\item Mesure de qubit

	Une mesure quantique idéale en mécanique quantique envoie effondre l'état. Par contre en realité, on peut complètement détruire un état en mesurant, en absorbant un photon par exemple.


\item Taux d'erreur faible

erreurs \begin{itemize}
	\item erreur de mesure
	\item control imprécis su système: bruit dans $B$ 
		\[ B \to B + \delta B \qquas t \to t + \delta t \] 
	\item termes parasites du Hamiltonien 
		ex: interaction spin-spin  
	\item relaxation et déphasage P
\end{itemize}
\end{enumerate}

\subsection{Référentiel tournant}

Heu, voir notes de photonique i guess...





\end{document}
