\documentclass{article}    
\usepackage[utf8]{inputenc}    
    
\title{Épisode 4}    
\author{Jean-Baptiste Bertrand}    
\date{\today}    
    
\setlength{\parskip}{1em}    
    
\usepackage{physics}    
\usepackage{graphicx}    
\usepackage{svg}    
\usepackage[utf8]{inputenc}    
\usepackage[T1]{fontenc}    
\usepackage[french]{babel}    
\usepackage{fancyhdr}    
\usepackage[total={19cm, 22cm}]{geometry}    
\usepackage{enumerate}    
\usepackage{enumitem}    
\usepackage{stmaryrd}    
    
%packages pour faire des math    
%\usepackage{cancel} % hum... pas sur que je vais le garder mais rester que des fois c'est quand même sympatique...
\usepackage{amsmath, amsfonts, amsthm, amssymb}    
\usepackage{esint}  

\begin{document}


\setcounter{section}{1}
\setcounter{subsection}{5}

\subsection{Téléportation quantique}


A veut envoyer $\ket{\psi}$ à B  

\begin{figure}[ht]
    \centering
    \incfig{téléportation-quantique-2}
    \caption{Téléportation quantique 2}
    \label{fig:téléportation-quantique-2}
\end{figure}



\begin{align*}
	\ket{\Psi} \otimes \ket{\Phi^+} &= \left( \alpha \ket{0} + \beta \ket{1} \right) \otimes \left( \frac{\ket{00}+\ket{11}}{\sqrt{2}}  \right) \\&= \frac{1}{\sqrt{2}} \left[ \alpha \ket{000} + \alpha \ket{011} + \beta \ket{100} + \beta \ket{111} \right] \\&= \frac{1}{2} \left[ \alpha \left( \ket{\Phi^+} + \ket{\Phi^-} \right) \ket{0}+ \alpha \left( \ket{\Psi^+} + \ket{\Psi^-} \right) \ket{1} + \beta \left( \ket{\Psi^+} - \ket{\Psi^-} \right) \ket{0} + \beta \left( \ket{\Phi^+} - \ket{\Phi^-} \right) \ket{2}   \right] \\&= \frac{1}{2} \left[ \ket{\Phi^-} \left( \alpha \ket{0} + \beta \ket{1} \right) + \dotsb  \right] \\&= \frac{1}{2} \left[ \ket{\Phi^+}\ket{\Psi} + \ket{\Phi^-}Z \ket{\Psi} + \ket{\Psi}X\ket{\Psi} + \ket{\Psi^-} ZX \ket{\Psi} \right]  
\end{align*}

Alice mesure $\{ \ket{\psi^\pm}, \ket{\psi^\pm} \} $ avec 25\% chaque. $$\ket{\Psi^+}: \mathds{1}\quad \ket{\Psi^-}: \text{applique Z} \dotsb $$  



\begin{tcolorbox}[title=Aparté notation tensorielle]

vecteur \verb|-O-|\\
Matrice \verb|-[]-|
	 
état à deux qbits: \verb|0==|
$$\Psi = \sum_{ij} c_{ij} \ket{e_{i}} \otimes\ket{e_j }$$  

 
ket: \verb|O-|\\
bra: \verb|-O|
\\

Produit tensoriel:\\
\verb|O-|\\
$ \otimes$\\
\verb|O-|\\

Contraction: ( $\bra{\psi}\ket{\phi}$  )\\

\verb|(\psi)--(\phi)|\\

Produit matrice-vecteur\\

\verb|(\psi)--[u]-| $= u \ket{\psi}$\\ 

Matrice-Matrice\\

\verb|-[A]-[B]-| = $BA$ = \verb|-[BA]-|\\ 


Trace:\\
\verb|_____|\\
\verb/|   |/\\
\verb|L[M]J|

\end{tcolorbox}

\section{Calcul Quantique}

\subsection{Calcul classique}

ordinateur classique

$$\rho: \{ 0,1 \}^n \to \{ 0,1 \}^,$$ 

Portes universelles
\verb|NAND|

\begin{verbatim}
    ____
a---|   \
    |    )---- a NAND b
b---|___/

COPY:
      ______ a
     /
a---<
     \______ a 


EX: NOT
            ____
      ______|   \
     /      |    \
a---<       |     )------- !a
     \______|    /
            |___/

\end{verbatim}


\begin{tcolorbox}[title=]
	\verb|COPY| est impossible en quantique \centering
	 
\end{tcolorbox}

\underline{Complexité} 


Difficulté de $\rho$: Nombre de portes universelle requisent pour le plus petit circuit réalisant $\rho$  

Famille de Problème ou la taille varie

La circuit ne doit pas être adapté à la taille

\begin{tcolorbox}[title=]
	 $$P: \text{Temps polynomial (facile)} $$ 
	 $$\abs{c_{n}} = n^\alpha $$
u

	 $$NP: \text{Temps non-polynomial} $$ 

	 \underline{NP-difficile}: Au moins aussi difficile que le problème le plus dur de NP (Pas forcément dans NP) 

	 \underline{NP-comlet}: NP difficile \textbf{et} dans NP 
	 
	 clairement $$P \subseteq NP $$ 

	 $$\boxed{\boxed{P =^?NP}}$$ 
\end{tcolorbox}

\begin{figure}[ht]
    \centering
    \incfig{la-complexité}
    \caption{La complexité}
    \label{fig:la-complexité}
\end{figure}
\subsection{Calcul quantique}

Mécanique quantique: Opérateur d'évolution unitaire
$$U^{\dagger}U = \mathds{1}$$ 

La porte \verb|NAND| n'est pas réversible (2bits $\to$ 1bit )

Il existe des porte réversibles classiqus

\begin{figure}[ht]
    \centering
    \incfig{cnot}
    \caption{CNOT}
    \label{fig:cnot}
\end{figure}

\begin{tcolorbox}[title=Note]
	ON peut toujours exprimer une fonction 
	$$\rho: \mathbb{Z}_2^n \mapsto \mathbb{Z}_2^m$$
	sous la forme 
	$$g: \mathbb{Z}^{n+m}_2 \mapsto \mathbb{Z}^{n+m}_2$$  

	$$g(x,0) \mapsto g(x, f(x))$$ 

\end{tcolorbox}

\subsection{Circuits Quantiques}

\begin{figure}[ht]
    \centering
    \incfig{anatomie-d'un-circuit-quantique}
    \caption{Anatomie d'un circuit quantique}
    \label{fig:anatomie-d'un-circuit-quantique}
\end{figure}

a) état initial ( $\ket{0}^{ \otimes n }$  ): Ce choix est arbitraire. Important de commencer dans un état non-intriqué


b) Transformation unitaire U: On décompose $u$ en un ensemble de portes universelles agissant sur 1-3 qubits.
Les U possibles ($U\in SU(2^n)$) forment un groupe continu. On peut générer $U$ à partir d'un circuit fini $C$  
La complexité quantique est définie à partir de $\abs{C} $ 


c) Mesure:
Résultat non-détérministe
On choisit de mesurer dans la base $Z$ $\{ \ket{0}, \ket{1} \}$ 
On peut mesurer différentes bases en chageant $U$. On évite de cacher la complexité dans la mesure. On aurait pu choisir de mesurer durant le circuit.  


\subsection{Complexité quantique}


BQP (bounded-error quantum polynomial time): Esemble des Problèmes faciles pour un ordinateur quantique
(Problèmes tel que $\abs{C} \leq n^\infty $ ) 


\begin{figure}[ht]
    \centering
    \incfig{complexité-quantique}
    \caption{Complexité quantique}
    \label{fig:complexité-quantique}
\end{figure}


L'ordinateur quantique doit donner la bonne réponse la plupart du temps ( $\geq \frac{2}{3} $  ) (pas deterministe). On moyenne sur un grand nombre de calculs

\end{document}
