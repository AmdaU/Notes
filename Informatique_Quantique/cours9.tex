\documentclass{article}    
\usepackage[utf8]{inputenc}    
    
\title{Épisode 4}    
\author{Jean-Baptiste Bertrand}    
\date{\today}    
    
\setlength{\parskip}{1em}    
    
\usepackage{physics}    
\usepackage{graphicx}    
\usepackage{svg}    
\usepackage[utf8]{inputenc}    
\usepackage[T1]{fontenc}    
\usepackage[french]{babel}    
\usepackage{fancyhdr}    
\usepackage[total={19cm, 22cm}]{geometry}    
\usepackage{enumerate}    
\usepackage{enumitem}    
\usepackage{stmaryrd}    
\usepackage{mathtools,slashed}
%\usepackage{mathtools}
\usepackage{cancel}
    
\usepackage{pdfpages}
%packages pour faire des math    
%\usepackage{cancel} % hum... pas sur que je vais le garder mais rester que des fois c'est quand même sympatique...
\usepackage{amsmath, amsfonts, amsthm, amssymb}    
\usepackage{esint}  
\usepackage{dsfont}

\usepackage{import}
\usepackage{pdfpages}
\usepackage{transparent}
\usepackage{xcolor}
\usepackage{tcolorbox}

\usepackage{mathrsfs}
\usepackage{tensor}

\usepackage{tikz}
\usetikzlibrary{quantikz}
\usepackage{ upgreek }

\newcommand{\incfig}[2][1]{%
    \def\svgwidth{#1\columnwidth}
    \import{./figures/}{#2.pdf_tex}
}

\newcommand{\cols}[1]{
\begin{pmatrix}
	#1
\end{pmatrix}
}

\newcommand{\avg}[1]{\left\langle #1 \right\rangle}
\newcommand{\lambdabar}{{\mkern0.75mu\mathchar '26\mkern -9.75mu\lambda}}

\pdfsuppresswarningpagegroup=1


\begin{document}
2022-32-04

\section*{Algorithme de Grover}

On cherche $x^{*}|f(x^{*})=1$  

$f(x) = 0 \forall x \neq x^{*}$ 

On utilise 2 transformations

$$U_{\text{sol}} - \mathds{1}=2\op{x^{*}} $$ 
$$U_i = \mathds{1} - 2 \op{0}$$ 

$$\dotsb$$ 

\setcounter{section}{2}
\setcounter{subsection}{13}

\subsection{Simulation de système quantique}

\underline{Solution classique:} Utiliser des propriétés spécifiques au système

ex: s'il n'y a pas d'intrication, on a besoin de $2n$ pramètres pour spécifier un état. 

S'il y a \textit{un peu} d'intrication, on peut utiliser les réseaux de tenseurs (HPS)

La solution pour un ordinateur quantique pour simuler un système quantique est générale


On simule le Hamiltonien en utilisant un opérateur unitaire (Si le Hamiltonien de dépend pas du temps)

On veut alors, évidement, construire $U(t)$ à partir d'un ensemble de portes élémentairesé

\underline{Exemple: Modèle de Ising} 

$$H = \hbar \sum_{j=1}^{n} \lambda \sigma_{x,j} + J \sigma_{x,j} \sigma_{z,j+1} $$ 

Si $J= 0$ 

$$U = e^{-i H t} = e^{-i \sum_{n=1}^{N} t \simga_{z,j}} = \prod_{j=1}^{n} \dotsb = \pi_{j=1}^{n}R_{x,j} (2\lambda t) $$ 

Si $\lambda =0 $ 

$$\dotsb$$ 

$$U(t) = \prod_{j=1}^{n} e^{- \frac{1}{2} \theta z_1 z_2 }\qquad \theta = 2 Jt$$ 

\underline{Cas générale } 

Dans le cas général on la forme $$e^{A+B} \neq e^{A}e^{B}$$, On ne peut donc pas exprimer nouvel unitaire directement en fonction des $U$ qu'on a déjà trouvé. Cepedant, si on prend des $\Delta t$ très petit, on peut l'approximer comme la mutiplication des deux.  


\subsection{Minimisation (État fondamentale d'un hamiltonien)}


On combine $U(t)$ avec $QPE$ pour trouver $E_0 $ et $\ket{E_0}$

C'est un problème dans $NP$: en générale on peut pas trouver la solution. 

\end{document}
