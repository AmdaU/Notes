\documentclass{article}    
\usepackage[utf8]{inputenc}    
    
\title{Épisode 4}    
\author{Jean-Baptiste Bertrand}    
\date{\today}    
    
\setlength{\parskip}{1em}    
    
\usepackage{physics}    
\usepackage{graphicx}    
\usepackage{svg}    
\usepackage[utf8]{inputenc}    
\usepackage[T1]{fontenc}    
\usepackage[french]{babel}    
\usepackage{fancyhdr}    
\usepackage[total={19cm, 22cm}]{geometry}    
\usepackage{enumerate}    
\usepackage{enumitem}    
\usepackage{stmaryrd}    
    
%packages pour faire des math    
%\usepackage{cancel} % hum... pas sur que je vais le garder mais rester que des fois c'est quand même sympatique...
\usepackage{amsmath, amsfonts, amsthm, amssymb}    
\usepackage{esint}  


\begin{document}
2022-33-05

\setcounter{section}{2}
\setcounter{subsection}{15}


\subsection{solveur variationnel quantique (VQE)}

On cherche à trouver l'énergie et l'état fondamentale d'un hamiltonien $H$ (n'importe quel problème de minimisation peut s'exprimer de cette façon).

C'est un problème NP complèt: On ne trouveras pas un algorithme qui réussit à tout les coups mais un algorithme heuristique qui réussit on l'espère \textit{souvent}.

\underline{VIE}: Algorithme classique utilisant un sous-routine quantique.

\begin{tcolorbox}[title=Rappel: Principe variationnel]
D'abord, on a que,
	$$\mel{\psi}{H}{\psi} \geq E_0 \forall \ket{\psi} \in \mathcal{H} $$ 

	On choisit $\ket{\psi(\vec\theta)}$ avec $\vec\theta$ un vecteur de paramètre et on minimise $$\expval{H}_{\vec{\theta}} \equiv \mel{\psi(\vec\theta)}{H}{\psi(\vec\theta)}$$ 
	alors $$\min_{\vec\theta} \approx E_0 \qq{si $\theta$ nous donne assez de liberté }$$ 
\end{tcolorbox}

	\hrule
	\underline{Ex:} $$H = \frac{\omega}{2} \simga_z$$  
	On pose
	$$\ket{\psi(\vec\theta)} = \cos \frac{\theta}{2} \ket{0} + \sin \frac{\theta}{2} \ket{1} $$ 

	$$\expval{H}_\theta = \frac{\omega}{2} \left( \cos^{2}\frac{\theta}{2} - \sin^{2} \frac{\theta}{2}  \right)  $$ 

	$$\min_{\theta} \expval{H}_\theta = -\frac{\omega}{2} \qquad \theta = \pi$$ 

$$\ket{\psi(\pi)}= \ket{1}$$ 

\underline{sous routine quantique} 

Prend en entré $\vec \theta,\, H$ 
retourne $\expval H _\theta $ 

puisqu'on ne veut que la valeur moyenne, si notre hamiltonien à la forme $$H = \sum_k H_k$$, On peut mesurer individuellement la moyenne sur chacun des $H_k$ et faire la somme après. 


\section{Suprématie quantique}

On veut montrer qu'on peut réaliser une calcul quantique trop long à réaliser classiquement.


On cherche un algorithme qui 

\begin{enumerate}
	\item Peu être réaliser sur les ordinateur quantique actuels. ( $n$ qubits et $q$ portes pas trop grand )
	\item On doit pouvoir vérifier la réponse.
	\item On veut un avantage quantique asymptotique (Un forte suspicion suffit)
\end{enumerate}

ex: Factorisation de Shor: 2 et 3, Google 1 et 3

Stratège actuelle : circuit aléatoire

$$U(\vec \theta ) \text{pris aléatoirement} $$ 

On mesure les qubits en sorties un très grand nombre de fois

Finalement on compare la ditribution avec la distribution idéale 

\subsection{Calcul adiabatique}

\begin{tcolorbox}[title=Théorème adiabatique]
	 Si un système physique est dans un état propre $\ket{\psi_k}$ de $H$ est au temps $t_0$. Il restera dans un état propre de $H(t)$ si $H(t)$ varie lentement.      
\end{tcolorbox}

\begin{figure}[ht]
    \centering
    \incfig{un-graphique-tout-a-fait-quelconque}
    \caption{un graphique tout a fait quelconque}
    \label{fig:un-graphique-tout-a-fait-quelconque}
\end{figure}

$$\Delta = min_t(E_1(t)-E_0(t))$$ 

\begin{tcolorbox}[title=]
	\centering{
	On doit être plus len que $\frac{1}{\Delta} \implies T \gg dfrac{1}{\Delta}  $}
\end{tcolorbox}


On choisi $H(t)$ tel que $\ket{E_0}$ encore la réponse voulue. On choisit $H(0)$  avec un état  fondamentale comme $$H(0) = J \Sum_j X_j = H_0$$ 

l'état fondamentale est $\ket{\psi(0)} = \ket{+}^{ \otimes n }$ 

On choisit $$H(t) = \left(  1- \frac{t}{\tau}   \right) H_0 + \frac{t}{T} H_{\text{final} } $$ 

\underline{Exemple: Modèle de Ising} 

$$H = \sum_i \lambda X_i + J Z_i Z_{i+1} $$ 

On suppose qu'on peut contrôler $\lambda, J$ 

$$J(\lambda) = J \frac{t}{T} $$ 






\end{document}
