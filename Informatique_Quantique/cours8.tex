\documentclass{article}    
\usepackage[utf8]{inputenc}    
    
\title{Épisode 4}    
\author{Jean-Baptiste Bertrand}    
\date{\today}    
    
\setlength{\parskip}{1em}    
    
\usepackage{physics}    
\usepackage{graphicx}    
\usepackage{svg}    
\usepackage[utf8]{inputenc}    
\usepackage[T1]{fontenc}    
\usepackage[french]{babel}    
\usepackage{fancyhdr}    
\usepackage[total={19cm, 22cm}]{geometry}    
\usepackage{enumerate}    
\usepackage{enumitem}    
\usepackage{stmaryrd}    
    
%packages pour faire des math    
%\usepackage{cancel} % hum... pas sur que je vais le garder mais rester que des fois c'est quand même sympatique...
\usepackage{amsmath, amsfonts, amsthm, amssymb}    
\usepackage{esint}  

\begin{document}
2022-09-27
\setcounter{section}{2}
\setcounter{subsection}{10}


\subsection{Estimation de phase (QPE)}

UNe tranformation unitaire $U$, n'est pas nécessairement hérimitique. U n'est donc pas un observable (en générale). On peut toujours écrire $U$ avec un observable $A$ comme

$$U = e^{2\pi i A} = \sum_{\lambda=1}^{N} e^{2\pi i \varphi_{\lambda}} \op{\lambda}$$ 

Si on suppose qu'on peut préparer un état $\ket{\lambda}$ et qu'un oracle puisse évaluer U. On veut estimer $\varphi_\lambda$ 

Pour se faire, on sépare le circuit en 2 registres

\begin{enumerate}
	\item Un registre de t qubits $\ket{0}$

		plus $t$ est grand et plus l'algorithme est précis et plus le succès est probable 
	\item l'état $\ket{\lambda}$ 
\end{enumerate}

\begin{center}
	%\begin{quantikz}
		%\lstick[wires=5]{$\ket{0}$}
		%& \gate{H} & \qw & \qw & \qw & \qw& \ctrl{5}&\qw&\rstick[wires=5]{$\ket{0} + e^{2\pi i 2^n \ket{1}$}\\ 
		%&\vdots\\
		%&\gate{H} & \qw & \qw & \ctrl{3} &\qw \dotsb&\qw&\qw\\
		%&\gate{H} & \qw &\ctrl{2}&\qw&\qw\dotsb&\qw&\qw \\
		%&\gate{H} & \ctrl{1} &\qw&\qw&\qw\dotsb&\qw&\qw &\\
		%\lstick[wires=2]{$\ket{\lambda}$}
		%& \qw\qwbundle[alternate] &\qw & \gate{U^{2^0}}\qwbundle[alternate] &\qw & \gate{U^{2^1}}\qwbundle[alternate] &\qw & \gate{U^{2^2}}\qwbundle[alternate]& \qw &\qw\qwbundle[alternate] &\dotsb  &\gate{U^{2^3}}\qwbundle[alternate] & \qw  & \qw\qwbundle[alternate]& \rstick[wires=2]{$\ket{\lambda}$}
	%\end{quantikz}
\end{center}
\subsection{Algorithme de Shor}

On veut, à partir de $a\in \mathds{N}$, trouver $a_1, a_2 \in \mathds{N}|a = a_1\cdot a_2$  

On combine la QFT et la QPW pour résoudre ce problème avec un ordinateur quantique

\begin{tcolorbox}[title=]
	Classiquement, le meilleur algorithme connu \underline{à ce jour} (number field sieve) à une complexité de $\exp{O(n^{1/3} \log^{2/3} n)}$ 
	Quantiquement, en revanche, la complexité est de $O(n^2\log n \log(\log n))$. Il y a donc un avantage exponentiel.  
\end{tcolorbox}

\begin{tcolorbox}[title=]
	Le record de factorization classique est un nombre de 795 bits! L'avantage quantique devra donc attendre pour des ordinateur quantique beaucoup plus performant que ceux qui existent actuellement.
	
\end{tcolorbox}

L'algorithme de factorisation est basé sur le problème de la recherche d'ordre. On commence donc par cet aspect.



\begin{tcolorbox}[title=Rappel de notions d'arithmétique]
	On travaille ici avec seulement des entiers  positifis ( $\in\mathds{N}^*$  ). Avec 2 nombres $x$ et $n$, il existe une manière unique d'écrire $$x = kn + r$$ où $r$ est le \underline{reste} ( $x \mod n = r$  ) et $0 \leq r\leq n-1. $     

\begin{itemize}
	\item On dit que $a$ divise $b$ (a|b), si $b=ca$ avec $c \in \mathds{N}$
	\item Le plus grand commun diviseur entre 2 nombres ($a,\, b$), noté pgcd$(a,b)$, est $\sup(c | (c|a) (\wedge c|b))$  
	\item Si gcd$(a,b) =1$, on dit que $a$ et $b$ sont co-premiers  
	\item On peut trouver le gcd entre 2 nombres grace à l'algorithme d'Euclide 
	\item Si $ab = cN$, $N$ est composite et $a,b \neq dN$ alors soit gcd($a,N$) ou gcd($b,N$) est un facteur not-trivial de $N$   
\end{itemize}

\end{tcolorbox}

\subsubsection{recherche d'ordre}

Soit $x,n $ des entiers positifs tel que $x < N$ est n'ayant aucun facteur commun. L' \underline{ordre}  de  $x \mod n$ est le plus petit entier positif $r$ tel que $$x^{r}\mod N = 1$$    

On veut déterminer $r$ à partir  de $x$ et $N$ (Un problème difficile classiquement)  

ex: ordre de (4,7) : 3


En général, $x^{r+1}\mod N = x$ ou pour $t= ar + b$ $$x^{t}\mod N = x^{b}\mod N$$   

Ce problème se résout à l'aide de l'algorithme QPE!

$$U_x \ket{y} = \ket{xy \mod N}$$ 

ave $y\in \mathds{Z}^{L}_2$ où $L$ est le nombre de bits nécéssaire pour représenter $N$ ( $L = \lceil\log_2 N\rceil$  )   

\begin{tcolorbox}[colframe=red]
	 \centering Il est important que (x,N) soit co premiers, sinon $U_x$ n'est pas unitaire! 
\end{tcolorbox}


\begin{tcolorbox}[title=]
	 Pour $N \leq y \leq 2^{L}-1 $ on prend la convention que $xy\mod N =y$ 

	 $U_x$ agit non-triviallemnt seulement sur $0 \leq y \leq N-1 $  
\end{tcolorbox}

On peut toujours écrire $U_x = \exp{2\pi i A_x}$ où $$A_x = \mqty(A_x & 0 \\ 0 & \mathds{1}_{2^2-N} )$$ 

Si on applique $U_x$ sur l'état suivant

$$\ket{u_{s}} = \frac{1}{\sqrt{r}} \sum_{k=0}^{r-1} e^{\frac{-2\pi i s k}{r} } \ket{x^{k}\mod N}$$ 

On a $$U_x \ket{u_{s}} = \dotsb =e^{2\pi is/r}\ket{u-x}$$ 


$\implies \ket{u_{s}}$ est un état propre de $U_x$ avec valeur propre $e^{2<pi i s/r}$ 

Pour préparer $\ket{u_s}$, Il faut connaître $r$ (et donc la réponse). Il faut donc utiliser une astuce

\begin{align*}
	\frac{1}{\sqrt{r}} \sum_{s=0}^{r-1} \ket{u_s} &= \frac{1}{r} \sum_{sk} e^{-2 \pi i sk }{r} \ket{x^{k}\mod N}\\ &= \sum_k \left( \frac{1}{r} \sum_s e^{-2\pi isk/r} \right) \ket{x^{k} \mod N }\\
		&=\sum_k \delta_{k,0} \ket{x^{k}\mod N }  \\
		&= \ket{1}
\end{align*}

On peut donc initialiser l'algorithme de phase avec $\ket{1}$, ce qui est simple.

Avec l'algorithme QPE, on obtiens un estimé de $\varphi \approx \frac{s}{r} $ avec un $s$ aléatoire

Il reste à extraire $r$ de $\varphi \approx \frac{s}{r} $

On peut faire cela avec la méthode de l'expension de fraction continues.

\end{document}
