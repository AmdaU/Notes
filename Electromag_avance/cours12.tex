\documentclass{article}    
\usepackage[utf8]{inputenc}    
    
\title{Épisode 4}    
\author{Jean-Baptiste Bertrand}    
\date{\today}    
    
\setlength{\parskip}{1em}    
    
\usepackage{physics}    
\usepackage{graphicx}    
\usepackage{svg}    
\usepackage[utf8]{inputenc}    
\usepackage[T1]{fontenc}    
\usepackage[french]{babel}    
\usepackage{fancyhdr}    
\usepackage[total={19cm, 22cm}]{geometry}    
\usepackage{enumerate}    
\usepackage{enumitem}    
\usepackage{stmaryrd}    
    
%packages pour faire des math    
%\usepackage{cancel} % hum... pas sur que je vais le garder mais rester que des fois c'est quand même sympatique...
\usepackage{amsmath, amsfonts, amsthm, amssymb}    
\usepackage{esint}  

\begin{document}

\section*{Un cas avec deux conducteur (genre un coax mais pas nécessairement )}

$$\begin{cases}
	V(z,t) = V_+ e^{i\left( kz -\omega t \right) } + V_- e^{-i\left( kz + \omega t \right) }\\ 
	I(z,t) = - \frac{V_+}{Z_0} e^{i\left( kz-\omega t \right) } + \frac{V_-}{Z_0} e^{-i\left( kz +\omega t \right) }
\end{cases}$$ 

\begin{figure}[ht]
    \centering
    \incfig{onde-qui-se-propage-dans-un-machin}
    \caption{Onde qui se propage dans un machin}
    \label{fig:onde-qui-se-propage-dans-un-machin}
\end{figure}

$$V_- (z=l, \omega) = \Gamma(\omega) V_+ (z=l, \omega) $$ 

$$V(z=l,\omega) = Z(\omega) I(z=l, \omega)$$ 

\begin{figure}[ht]
    \centering
    \incfig{circuit-éléctrique}
    \caption{Circuit éléctrique}
    \label{fig:circuit-éléctrique}
\end{figure}

$$Z \left( \frac{-V_+ + V_-}{Z_0}  \right) = V_+ V_- $$ 

$$V_- (Z_0 -Z) = V_+ (+Z + Z_0) $$ 

$$\Gamma = \frac{V_-}{V_+} = \frac{-Z + Z_0 }{Z_0 + Z } $$ 

$$\boxed{\Gamma = \frac{Z-Z_0}{Z+Z_0} }$$ 

$$\implies \text{si } Z = Z_0 \text{ alors } \Gamma =0  $$ 

$$Z = \infty \implies \text{circuit ouvert} \implies \Gamma =1 \wedge  I=0 $$ 





\end{document}
