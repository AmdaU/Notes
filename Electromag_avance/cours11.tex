\documentclass{article}    
\usepackage[utf8]{inputenc}    
    
\title{Épisode 4}    
\author{Jean-Baptiste Bertrand}    
\date{\today}    
    
\setlength{\parskip}{1em}    
    
\usepackage{physics}    
\usepackage{graphicx}    
\usepackage{svg}    
\usepackage[utf8]{inputenc}    
\usepackage[T1]{fontenc}    
\usepackage[french]{babel}    
\usepackage{fancyhdr}    
\usepackage[total={19cm, 22cm}]{geometry}    
\usepackage{enumerate}    
\usepackage{enumitem}    
\usepackage{stmaryrd}    
\usepackage{mathtools,slashed}
%\usepackage{mathtools}
\usepackage{cancel}
    
\usepackage{pdfpages}
%packages pour faire des math    
%\usepackage{cancel} % hum... pas sur que je vais le garder mais rester que des fois c'est quand même sympatique...
\usepackage{amsmath, amsfonts, amsthm, amssymb}    
\usepackage{esint}  
\usepackage{dsfont}

\usepackage{import}
\usepackage{pdfpages}
\usepackage{transparent}
\usepackage{xcolor}
\usepackage{tcolorbox}

\usepackage{mathrsfs}
\usepackage{tensor}

\usepackage{tikz}
\usetikzlibrary{quantikz}
\usepackage{ upgreek }

\newcommand{\incfig}[2][1]{%
    \def\svgwidth{#1\columnwidth}
    \import{./figures/}{#2.pdf_tex}
}

\newcommand{\cols}[1]{
\begin{pmatrix}
	#1
\end{pmatrix}
}

\newcommand{\avg}[1]{\left\langle #1 \right\rangle}
\newcommand{\lambdabar}{{\mkern0.75mu\mathchar '26\mkern -9.75mu\lambda}}

\pdfsuppresswarningpagegroup=1


\begin{document}

\section*{Raypnnement}

\subsection*{Potentiels}

\underline{Statique} $$\grad \times E =0 \implies E = -\grad V$$ 
$$\grad \cdot E = - \frac{\rho}{\epsilon_0} \implies \grad^2 B = \frac{\rho}{\epsilon_0} =0$$ 
$$V(P) = \frac{1}{4\pi\epsilon_0} \int \frac{\rho(M)d\tau}{PM} $$ 

$$\grad \cdot \vb{B} = 0 \implies \vb{B} = \grad \times  \vb{A}$$ 

$$\grad x \vb{B} = \mu_{0} j = \grad(\grad \cdot \vb{A}) - \grad^2 A$$ 

$\vb{A}$ n'est pas définis de manière unique. On a un choix de gauge sur la valeur du gradient

$$\grad \cdot \vb{A} = 0 \quad \text{jauge de Coulomb}$$ 

$$\grad^2 \vb{A} + \mu \vb{j} = \vb{0}$$ 

\begin{tcolorbox}[title=Remarque]
	\underline{La condition $\grad \cdot A = 0$ n'est pas trop forte } 
	$$\vb{A}' = \vb{A} + \grad \varphi  $$ 
	$$\grad \times \vb{A}' = \grad \times (\vb{A} + \grad \varphi) = \grad \times \vb{A}$$ 
	Le résultat physique reste le même peut importe le gradient!
\end{tcolorbox}

\underline{dynamique} $$\grad \cdot B =0 \implies \grad \times A $$ 

$$\grad \times B = \grad (\grad \cdot \vb{A}) - \grad^2 \vb{A}$$ 

$$\mu \vb{j} + \epsilon\mu \frac{d\vb{E}}{dt} $$ 

$$\grad \times  \vb{E} = - \frac{d \vb{E}}{dt} = -\grad \times \frac{dB}{dt} $$ 

$$\grad \times \qty(\vb{E} + \pdv{B}{t}) =  0 \implies \vb{E} = \pdv{\vb{B}}{t}= -\grad V$$ 


$$\begin{cases}\grad \times \vb{B} = \mu_0 \vb{j} + \epsilon \mu_{0} \left( -\grad \pdv{V}{t} \right) -\grad \left( \grad \cdot \vb{A} \right) -\grad^2 \vb{A}\\\grad \cdot E = \frac{\rho}{\epsilon_0} = - \grad^2 V-\grad \pdv{\vb{B}}{t}\end{cases}$$ 

On veut annuler le terme en $V$ car c'est celui-là qui mélange les deux composantes 

$$\mu_{0} j = \epsilon_{0} \mu \dv[2]{A}{t} + \grad^2 \vb{A} = \grad \qty[\grad \cdot A + \pdv{V}{t}]$$ 

$$\frac{\rho}{\epsilon_0} + \grad^2 B = \pdv{t}(\grad \cdot A)$$ 

\underline{Gauge de Lorentz} 
$$\grad \cdot A + \pdv{V}{t} = 0$$ 

$$\begin{cases}
	\grad^2 \vb{A} - \epsilon\mu \pdv[2]{\vb{A}}{t} = -\mu_0 \vb{j}\\
	\grad^2 V -\frac{1}{c^2} \pdv[2]{V}{t} = -\frac{\rho}{\epsilon_0} 
\end{cases}$$ 

Si on modifie maintenant le potentiel vecteur, on trouver

$$\vb{A}' = \vb{A} + \grad\varphi \implies \vb{B}' = B$$ 

Cepdant, cela change $\vb{E}$, à moins que $V$ change aussi. On veut que $\vb{E}= \vb{E}'$   
$$\begin{aligned} \vb{E}' &= -\grad V' - \pdv{\vb{A}'}{t}\\
	&= \grad V' - \pdv{\vb{A}}{t} -\grad \pdv{\varphi}{t}	
\end{aligned}$$ 
$$$$ 
On doit donc avoir $$\vb{E}' = - \grad \underbrace{\left( V' +\pdv{\varphi}{t} \right)}_{V} -\pdv{\vb{A}}{t} $$ 

On a donc que $V'$ doit être différent de $V$. $V$ et $\vb{A}$ ne sont plus indépendant!    


\subsection*{Potentiels retardées}

Quand on travail en dynamique le potentiel ne s'applique plus instantanément. L'information voyage à la vitesse de la lumière

$$V(P,t) = \frac{1}{4\pi\epsilon_0} \int \frac{\rho(M,t-\frac{PM}{c} )}{PM} \dd \tau$$ 

$$\vb{A}(P,t) = \frac{\mu_{0}}{4\pi} \int \frac{\vb{j}(M,t - \frac{PM}{c}}{PM} \dd \tau$$ 

On appelle la quantité $t - \frac{PM}{c}$ le temps retardé.

Cette forme d'équation ne s'applique pas sur $\vb{E}$ et $\vb{B}$



\end{document}
