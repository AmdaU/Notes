\documentclass{article}    
\usepackage[utf8]{inputenc}    
    
\title{Épisode 4}    
\author{Jean-Baptiste Bertrand}    
\date{\today}    
    
\setlength{\parskip}{1em}    
    
\usepackage{physics}    
\usepackage{graphicx}    
\usepackage{svg}    
\usepackage[utf8]{inputenc}    
\usepackage[T1]{fontenc}    
\usepackage[french]{babel}    
\usepackage{fancyhdr}    
\usepackage[total={19cm, 22cm}]{geometry}    
\usepackage{enumerate}    
\usepackage{enumitem}    
\usepackage{stmaryrd}    
    
%packages pour faire des math    
%\usepackage{cancel} % hum... pas sur que je vais le garder mais rester que des fois c'est quand même sympatique...
\usepackage{amsmath, amsfonts, amsthm, amssymb}    
\usepackage{esint}  


\begin{document}

On reviens sur le guide d'onde rectangulaire

\begin{figure}[ht]
    \centering
    \incfig{guide-donde-rectangulaire}
    \caption{guide donde rectangulaire}
    \label{fig:guide-donde-rectangulaire}
\end{figure}

Modes:
$$\begin{cases}
	\text{TE} & E_{z} =0 \\
	\text{TM } & B_{z} =0 
\end{cases}$$ 


$$B_{z}(x,y,z) = B_{0} \cos k_{x} x \cos k_{y} y e^{ikz} $$ 
avec $$k^2 = \frac{\omega^2}{c^2} -k_{x}^2 -k_{y}^2 = \frac{\omega^2-\omega_{?}^2}{c^2} \qquad k_{x} = \frac{n\pi}{a} \qquad k_{y} \frac{n\pi}{b} $$ 

ex: $n=0, n=1$


$$B_z(x,y,z) = B_{0} \cos k_{y} y e^{ikz}$$ 
$$B_z(x,y,0)=B_{0} \cos \frac{y\pi}{b} $$ 


Dans le cas d'un métal non-parfait, il y aurat des pertes du au courrants. On s'interesse maintenant à calculer ces courrants.

Comme $\grad \cdot j + \frac{\partial\rho}{\partial t} = 0$ On peut obenir la charge facilement à partir du courrant (à constante près) mais pas l'inverse 

\begin{figure}[ht]
    \centering
    \incfig{graphique-du-champ-magnétique}
    \caption{Graphique du champ magnétique}
    \label{fig:graphique-du-champ-magnétique}
\end{figure}

discontinuités de $B_{z} \to j_z$ 

en $y=b$: $j_{s}$ suivant $\hat x$   

$$B_z(y = b^+)-B_z(y=b^-) = \mu_{0} j$$ 

$\implies j_{s} = \frac{B_{0}}{\mu_{0}} $ 

en $x=0$ $$\vb{j_s} = \frac{B_{0}}{\mu_{0}} \cos\frac{y\pi}{b} \hat y$$  

Comme on est en 2D: on a 

$$\grad \cdot \vb{j_s} + \pdv{\rho}{t} = 0$$ 

En $y = 0, y = b$, on a $J_{s} = cte \implies \sigma=0$  

En $x = 0$ on a $$\grad \cdot \vb{j} = \frac{B_{0}}{mu_0} \frac{\pi}{b} \sin \frac{y\pi}{b} =i\omega\sigma$$  


\underline{Autre approche: Discontinuité de $E$ } 

en $x=a$: $$E_{x} \propto \pdv{B_z}{y} \qquad \text{OK}$$ 

\underline{Remarque: Impédance} 

Quel est l'unité de $\frac{E}{H}$ ? 

$$= \frac{Vm^{-1}}{Am^{-1}} = \Omega$$ 

l'impédance est alors donné par

$$Z = \frac{E_{x}}{H_y} = \mu_{0} \frac{E_{x}}{B_y}= \mu_{0} \frac \omega k$$ 

$$k^2 = \frac{\omega^2-\omega_{?}^2}{c^2} $$ 

Dans le vide $k = \frac{omega}{c} $,  $Z = Z_{0} =\sqrt{\frac{\mu_{0}}{\epsilon_0} }= \mu_{0} c = 377\Omega$ (Exactement? WOW!) 



\end{document}
