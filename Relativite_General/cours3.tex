\documentclass{article}    
\usepackage[utf8]{inputenc}    
    
\title{Épisode 4}    
\author{Jean-Baptiste Bertrand}    
\date{\today}    
    
\setlength{\parskip}{1em}    
    
\usepackage{physics}    
\usepackage{graphicx}    
\usepackage{svg}    
\usepackage[utf8]{inputenc}    
\usepackage[T1]{fontenc}    
\usepackage[french]{babel}    
\usepackage{fancyhdr}    
\usepackage[total={19cm, 22cm}]{geometry}    
\usepackage{enumerate}    
\usepackage{enumitem}    
\usepackage{stmaryrd}    
\usepackage{mathtools,slashed}
%\usepackage{mathtools}
\usepackage{cancel}
    
\usepackage{pdfpages}
%packages pour faire des math    
%\usepackage{cancel} % hum... pas sur que je vais le garder mais rester que des fois c'est quand même sympatique...
\usepackage{amsmath, amsfonts, amsthm, amssymb}    
\usepackage{esint}  
\usepackage{dsfont}

\usepackage{import}
\usepackage{pdfpages}
\usepackage{transparent}
\usepackage{xcolor}
\usepackage{tcolorbox}

\usepackage{mathrsfs}
\usepackage{tensor}

\usepackage{tikz}
\usetikzlibrary{quantikz}
\usepackage{ upgreek }

\newcommand{\incfig}[2][1]{%
    \def\svgwidth{#1\columnwidth}
    \import{./figures/}{#2.pdf_tex}
}

\newcommand{\cols}[1]{
\begin{pmatrix}
	#1
\end{pmatrix}
}

\newcommand{\avg}[1]{\left\langle #1 \right\rangle}
\newcommand{\lambdabar}{{\mkern0.75mu\mathchar '26\mkern -9.75mu\lambda}}

\pdfsuppresswarningpagegroup=1

\begin{document}

2022-09-07

\section*{Transport parallèle}

Le concept de transport parallèle permet de comparer des vecteurs qui sont définis à des points différents (qui viennent de différents espaces tangents). 

\begin{figure}[ht]
    \centering
    \incfig{transport-parallèle}
    \caption{transport parallèle}
    \label{fig:transport-parallèle}
\end{figure}

$$A_i (x) = \vb{A}(x)\cdot \vb{e}_i(x)$$ 

\begin{align*}
	A_i + \partial A_i &= \vb{A}(x)\cdot \vb{e}_i (x+ \dd  x) \\
										 &= \vb{A}(x)\cdot \left( \vb{e}_i (x) + \partial_j \vb{e}_i (x) \dd  x^{j} \right) \\ &= A_i (x) + A_k \underbrace{\vb{e}^k\cdot \partial_j \vb{e}_i (x)}_{\Gamma_{ij}^{k}(x)} \dd  x^i  
\end{align*}

$$\boxed{\delta A_i = \Gamma_{ij}^{k} A_k \dd x^{i}}$$

$$\vb{e}_i = \partial_i \vb{X} \quad \vb{e}^k = \partial^{k}\vb{X} = g^{kj}\partial_j \vb{X}$$ 
$$\partial_j \vb{e}_i = \partial_j \partial_i \vb{X}= \partial_i \vb{e}_j$$ 
$$\Gamma_{ij}^{k}= \partial^{k}\vb{X}\cdot \partial_i \partial_j \vb{X} = \Gamma_{ji}^k$$ 

$$\boxed{\partial A^{i}= - \Gamma_{kj}^{i} A^{k} \dd x^{j}}$$ 

$$\delta (A^{i}B_i ) =0 = \delta^{i}B_i +A^{i}\delta B_i = \left( \delta A^{i}+ \Gamma_{kj}^{i}A^{k} \dd x^j \right)B_i = 0$$ 


\underline{Dérivé covariante} 


\begin{align*}
    DA^{i} &= \text{changement "réel" du vecteur} \\ &= \dd A^{i}- \partial A^{i}\\ &= \partial_j A^{i} \dd x^{j}+ \Gamma_{kj}^{i}A^{k} \dd x^{j} \\ &= \underbrace{\nabla_j A^{i}}_{\partial_j A^{i}+ \Gamma_{kj}^{i}A^{k}}  \dd x^i 
\end{align*}

$$\underbrace{\boxed{\nabla_j A_i = \partial_j A_i - \Gamma_{ij}^{k}A_k}}_{\text{tenseur de rang 2} } $$ 


$$\nabla_i \vb{A} = \text{proj} \partial_i \vb{A} $$ 

\begin{tcolorbox}[title=]
    Les symbols de Christoffel semblent requérir $\vb{X}$ et donc de travailler dans l'espace hôte. Ce n'est pas de cas. On peut tout ré-exprimer en fonction du tenseur métrique.

    $$\Gamma_{ij}^{k}= \frac{1}{2} g^{kl} \left( \partial_j g_{il} +\partial_i g_{jl} - \partial_{l} g_{ij}   \right) $$ 
\end{tcolorbox}


$$\Gamma_{kij} = g_{kl} \Gamma_{ij}^{l} = \partial_k \vb{X}\cdot \partial_j \partial_j \vb{X}$$ 

$$\partial_k g_{ij} = \vb{e}_i \vb{e}_j = \partial_i \vb{X}\cdot \partial_j \vb{X} $$ 
$$\partial_i g_{jk} = \vb{e}_j \vb{e}_k = \partial_j \vb{X}\cdot \partial_k \vb{X} $$ 
$$\partial_j g_{ki} = \vb{e}_k \vb{e}_i = \partial_k \vb{X}\cdot \partial_i \vb{X} $$ 

On addition les deux derniers et on isole $\partial_k \vb{X}\cdot \partial_j \partial_i \vb{X}$ 

pour avoir $$\boxed{\Gamma_{ij}^{k}= \frac{1}{2} g^{kl} \left( \partial_j g_{il} +\partial_i g_{jl} - \partial_{l} g_{ij}   \right)}$$


\begin{tcolorbox}[title=Exemple: $S^{2}$( rayon $a$  ) ]
     Coordonnées sphériques $\theta, \varphi$ 
     $$[g_{ij} ]= \begin{bmatrix} a^{2}& 0 \\ 0 &a^2\sin^2\theta
     \end{bmatrix}$$ 
     $$\Gamma_{jk}^{i}= \frac{1}{2} g^{il} \left( \partial_j g_{lk} +\partial_k g_{jl} - \partial_i g_{jk}  \right) $$ 
         $$[g^{ij}]= \begin{bmatrix} \frac{1}{a^{2}} &0 \\ 0 & \frac{1}{a^2\sin^2\theta}  
         
     \end{bmatrix}$$ 

$$\partial_{\theta} g_{\varphi\varphi} = \partial_1 g22 = 2 a^2\sin\theta\cos\theta$$ 
$$\Gamma_{22}^{1}= - \frac{1}{2} g^{11}\partial_1 g_{22} =- \sin\theta\cos\theta$$ 
$$\Gamma_{12}^{2}= \Gamma_{21}^{2}= \frac{1}{2} g^{22}\partial_1 g_{22} = \frac{\cos\theta}{\sin\theta} = \cot\theta$$ 


\underline{Dérivée covariante} 

$$\nabla_{\theta} A_{\theta} = \partial_{\theta} A^{\theta}+ \Gamma_{\varphi\varphi}^{\theta}A^{\varphi}= \partial_{\theta} A^{\theta}$$ 

$$\nalba_{\varphi} A^{\theta}= \partial_{\varphi} A^{\theta}+ \Gamma_{\varphi\varphi}^{\theta}A^{\varphi}= \partial_{\varphi} A^{\theta}-\sin\theta\cos\theta A^\varphi$$ 
$$\dotsb$$ 
\end{tcolorbox}

\section*{Les géodésique}

La géodésique est une courbe (trajectoire sur un variété) $x^i(\lambda)$ 

\begin{figure}[ht]
    \centering
    \incfig{géodésique}
    \caption{géodésique}
    \label{fig:géodésique}
\end{figure}

vecteur tangenant $\vb{u} = \dv{{}}{\lambda} \vb{X}(x(\lambda)) = \pdv{\vb{X}}{x^i} \dv{x^{i}}{\lambda} = \vb{e}_i \dot x^{i}= \vb{e}_i u^i $ 

$$\abs{\vb{u}} = \sqrt{g_{ij}  u^{i}u^{j}} = \sqrt{g_{ij} \dv{x^{i}}{\lambda} \dv{dx^{j}}{\lambda} } = \sqrt{\dv{s^{2}} {\lambda^{2} }} = \abs{ \dv{s}{\lambda}  }  $$ 

$$\grad_{\lambda} \phi = \dv{\phi}{\lambda} = \partial_i \varphi \pdv{x^{i}}{\lambda} = u^{i}\partial_i \phi$$ 


$$\nabla_{\lambda} A^j = u^{i}\Grad_i A^{j} = u^{i}\partial_i A^{j} + \Gamma_{ki}^{j}A^{k}u^i$$ 



\underline{Géodésique} 

1) Minimise (rend stationaire) la distance entre deux points.

$$S_{AB} = \int_{A}^{B}\dd s \quad \dd s^{2}= g_{ij} \dd x^{i} \dd x^{j}$$ 

$$\delta S_{AB} = 0 \quad x^{i}(\lambda)+ \delta x^{i}(\lambda)$$ 

2) courbe telle que $\vb{u}$ est transporté parallèlement

$$Du^{i}= du^{i}- \delta^{i}= du^{i} + \Gamma_{kj}^{i}u^{k} \dd  x^{k} \dd  x^{j}\propto u^i$$ 

$$\dv{{}}{\lambda} \left( u_i u^i \right) = 2u_i \dot u^{i}$$ 
où $\dot{} \equiv \dv{{}}{\lambda} $ 

$$=2u_i u^{i}f(\lambda) - \Gamma_{jk}^{i}u^{k}u^{j}u_i $$ 


\begin{tcolorbox}[title=]
    $$\Gamma_{ij}^{k\prime} = \pdv{x^{k\prime}}{x^l} \pdv{x^{m}}{x^i\prime} \pdv{x^{n}}{x^{j\prime}} \Gamma_{mn}^{l} - \dotsb$$  
\end{tcolorbox}

On va demander que $f(\lambda) =0 \implies \abs{\vb{u}} = \text{cst}  $ 

$$\bosed{\dot u^{i} + \Gamma_{kj}^{i}u^{k}u^{i}= 0}$$ 
\end{document}

