\documentclass{article}    
\usepackage[utf8]{inputenc}    
    
\title{Épisode 4}    
\author{Jean-Baptiste Bertrand}    
\date{\today}    
    
\setlength{\parskip}{1em}    
    
\usepackage{physics}    
\usepackage{graphicx}    
\usepackage{svg}    
\usepackage[utf8]{inputenc}    
\usepackage[T1]{fontenc}    
\usepackage[french]{babel}    
\usepackage{fancyhdr}    
\usepackage[total={19cm, 22cm}]{geometry}    
\usepackage{enumerate}    
\usepackage{enumitem}    
\usepackage{stmaryrd}    
    
%packages pour faire des math    
%\usepackage{cancel} % hum... pas sur que je vais le garder mais rester que des fois c'est quand même sympatique...
\usepackage{amsmath, amsfonts, amsthm, amssymb}    
\usepackage{esint}  


\begin{document}
2022-35-09


\section*{Géométrie de Kerr}

La géométrie de Kerr est celle d'un \textit{objet} tournant uniformément.


\begin{align*}
	\dd \tau ^2 &= g_{tt} \dd t^{2}+ g_{rr} \dd r^{2}+ g_{\theta\theta} \dd \theta^{2}+ g_{\varphi\varphi}\dd \varphi^{2}+ \textcolor{blue}{2g_{t\varphi}  \dd t \dd \varphi}\\ 
							&= \left( g_{tt}- \frac{g_{t\varphi}^{2}}{g_\varphi\varphi} \right)   \dd t^{2}+ g_{rr} \dd r^{2}+ g_{\theta\theta} \dd \theta^{2}+ g_{\varphi\varphi}\left( \dd \varphi^{2} -\underbrace{\omega}_{- \frac{g_{t\varphi }}{g_{\varphi\varphi} } }   \dd t \right)^2 \\ 
\end{align*}

La métrique est indépendante de $t$ et $\varphi$  

\[ p_t = k = g_{tt} \dot t = g_{t\varphi} \dot \varphi \] 
\[ p_{\varphi} = h = g_{t\varphi} \dot t + g_{\varphi\varphi} \dot \varphi \] 

\begin{tcolorbox}[title=Entraînement des repères]

	\[ h=0 \to \frac{\dot \varphi}{\dot t} = - \frac{g_{t\varphi} }{g_{\varphi\varphi} } = \omega = \dv{\varphi}{t}  \] 
\end{tcolorbox}

\underline{Roy Kerr à dit} 

\[ \dd \tau^{2} = \left( 1 - \frac{\gamma r_{s}}{\rho^2}  \right)  \dd t^{2} + \frac{2a r r_s \sin^{2}\theta }{\rho^2} \dd t \dd \varphi - \frac{\rho^2}{\Delta} \dd r^{2}- \rho^{2}\dd \theta^{2} - \left[ r^{2}- a^{2}\frac{a^{2}r r_s \sin^2\theta }{\rho^2}  \right] \sin^{2}\theta \dd \varphi	\] 



\[ \rho^{2}= r^{2} +a^{2}\cos^{2\theta} \qquad \delta r^{2}+ a^{2}-r r_s  \] 


Cas limite: $ a \to 0 $: On retrouve la métrique de Schwarzschild.

\[ Ma = \text{moment cinétique de l'objet}  \] 


Autre cas limite $r_s \to 0$ 


\[ \dd \tau = \dd t^{2}- \frac{\rho^2}{r^2+a^2} \dd r^{2}- p^{2}\dd\theta^{2} - \rho^{2}\dd \theta^{2}- \left( r^{2}-r^{a} \right) \sin^{2\theta}\dd\varphi   \] 



\[ \Delta = r^{2}+ a^2 \] 

Ces coordonnées décrivent un espace temps plat. Ce sont les coordonnées Boyer–Lindquist

\[ \begin{cases}
	x = \sqrt{r^{2}+a^{2}} \sin\theta \cos\varphi\\
	y= \sqrt{r^{2}+ a^{2}}\sin\theta \sin\varphi\\
	z = r\cos\theta
\end{cases} \] 

\begin{figure}[ht]
    \centering
    \incfig{boyer–lindquist}
    \caption{Boyer–Lindquist}
    \label{fig:boyer–lindquist}
\end{figure}


Singularités intrinsèques 
$\rho =0 \to r=0 \qquad \theta = \frac{1}{2} $ 


Pour être immobile, on doit avoir \[ u^{i} = \left( \dot t ,0,0,0 \right)  \] 

\[ \implies g_{tt} \dot t^{2}  = 1 \implies g_{tt} > 0 \] 

Lorsque $g_{tt} $ deviens négatif, donc, être immobile deviens impossible.


\begin{figure}[ht]
    \centering
    \incfig{surfaces}
    \caption{surfaces. Le processus de d'extraction d'énérgie est représenté en vert (Penrose)}
    \label{fig:surfaces}
\end{figure}

Il est possible d'extraire de l'énérgie d'un trou noir en rotation. 



\end{document}
