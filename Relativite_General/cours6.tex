\documentclass{article}    
\usepackage[utf8]{inputenc}    
    
\title{Épisode 4}    
\author{Jean-Baptiste Bertrand}    
\date{\today}    
    
\setlength{\parskip}{1em}    
    
\usepackage{physics}    
\usepackage{graphicx}    
\usepackage{svg}    
\usepackage[utf8]{inputenc}    
\usepackage[T1]{fontenc}    
\usepackage[french]{babel}    
\usepackage{fancyhdr}    
\usepackage[total={19cm, 22cm}]{geometry}    
\usepackage{enumerate}    
\usepackage{enumitem}    
\usepackage{stmaryrd}    
    
%packages pour faire des math    
%\usepackage{cancel} % hum... pas sur que je vais le garder mais rester que des fois c'est quand même sympatique...
\usepackage{amsmath, amsfonts, amsthm, amssymb}    
\usepackage{esint}  

\begin{document}

2022-09-16

\section*{Exemple: Hyperboloïde de révolution}


\begin{figure}[ht]
    \centering
    \incfig{hyperboloïde-de-révolution}
    \caption{Hyperboloïde de révolution}
    \label{fig:hyperboloïde-de-révolution}
\end{figure}

$$r^2-z^{2}=1 $$ 

$$2r \dd r - 2z -2z \dd z = 0 \;\&\; z= \sqrt{r^2-1}$$ 


$$\dd s^{2}= \dd r^{2}+\dd^{2}+ r^{2}\dd \varphi^{2} =  \left( 1 + \frac{r^2}{r^{2}-1}  \right) \dd r^{2}+ r^{2}\dd \varphi$$ 

$$\implies [g_{ij} ] = \mqty[\frac{2r^{2}-1}{r^2-1} &0\\ 0 & r^2 ]$$ 

$$\Gamma_{11}^{1} = \frac{ r}{\left( r^2-1 \right) (2r^{2}-1)} $$ 
$$\Gamma_{22}^{1}= \frac{ r \left( r^2-1 \right) }{ 2r^2-1} $$ 

$$\Gamma_{12}^{2}= \Gamma_{21}^{2}= -\frac{1}{r} $$ 

$$R_{1212} = \frac{ -r^{2}}{ \left( r^{2}-1 \right) (2r^{2}-1)}  $$ 
$$R = \frac{ -2}{ \left( 2r^{2}-1 \right)^2 } $$ 

$$\dot u_i = \frac{1}{2} \partial_i g_{jk} u^{i}u^{k}$$ 

$$i=2 \implies \quad \dot u_{\varphi} = 0 \implies u_{\varphi=cst} = r^{2}\dot\varphi =h$$ 


\begin{tcolorbox}[title=Coordonnées hyperboliques ]

	$$r =\cosh\theta \quad z =\sinh\theta \quad \theta \in [-\infty,\infty]$$ 

	$$r^{2}-z^{2}=1$$ 
$$\dd s^{2}= \left( \cosh^{2}\theta + \sinh^{2}\theta  \right) \dd \theta^{2}+ \cosh^{2}\theta \dd \varphi = \cosh 2\theta \dd \theta^{2}+ \cosh^{2}\theta \dd \varphi^2$$ 

$$\Gamma_{11}^{1}= -2 \Gamma_{22}^{1}= \tanh 2\theta$$ 

$$\Gamma_{21}^{2}= \Gamma_{12}^{2}=\tanh\theta $$ 

$$R_{1212} = - \frac{\cosh^2\theta}{\cosh^22\theta}  $$ 

$$R = - \frac{ 2 }{\cosh2\theta} $$ 
\end{tcolorbox}

\hrule 
\section*{Sphère}


$$x^{1}= \theta = \text{cst} \quad x^{2}= \varphi \in [0,2\pi]$$ 

$$\nabla_{lambda} A^{i}= \dv{{}}{\lambda} A^{i}+ \Gamma_{jk}^{i}A^{k}i^j$$ 

$$\nabla_{\varphi} A^{i}= \dv{A^{i}}{\varphi} + \Gamma_{k\varphi}^{i}A^{k}= 0$$ 


$$\begin{cases}
	\nabla_{\varphi} A^{\varphi} = \dv{A^{\varphi}\varphi}{+} \Gamma_{12}^{2}A^{\theta}=0 \\
	
\nalba_{\varphi} A^{\theta}= \dv{A^{\theta}}{\varphi} + \Gamma_{22}^{1}A^{\varphi}= 0 
\end{cases}$$

$$\begin{cases}
	
\dv{A^{\varphi}}{\varphi} +A^{\theta} \frac{ \cos\theta }{\sin\theta} =0 \\\dv{A^{\theta}}{\varphi} - A^{\varphi}\sin\theta\cos\theta =0 
\end{cases}$$

$$\begin{cases}
	\dv[2]{A^{\varphi}}{\varphi} + \cos^{2}\tehta A^{\varphi}= 0\\
	\dv[2]{A^{\theta}}{\varphi} + \cos\theta A^{\theta}= 0
\end{cases}$$ 


$$\begin{cases}
A^{\varphi}(\varphi) = \frac{1}{\sin\theta} \cos(\varphi \abs{\cos\theta} )\\
A^{\theta}(\varphi) = \text{sign} (\cos\theta)\sin(\varphi \abs{\cos\theta} )
\end{cases}$$ 

$$A^{\varphi}(\varphi) = \frac{1}{\epsilon} \cos\varphi $$ 

$$A^{\theta}(\varphi) = \sin\theta$$ 


\hrule

\section*{Coordonées polaires planes}


$$\dd s^{2}= \dd r^{2}+ r^{2}\dd \varphi^2$$ 

$$[ g_{ij} ] = \mqty[1 & 0 \\ 0 & r^2]$$ 


$$\dot u_i = \frac{1}{2} \partial_i g_{kj} i^{k}u^{j}$$ 


$$u^{r}= u_r = \dot r $$ 
$$i=1 \implies \dot u_r = r\dot \varphi^{2}= \ddot r$$ 

$$i=2 \implies \dot u_{\varphi} = 0 \implies u_{\varphi} = \text{cst} $$ 
$$U_{\varphi} = g_{\varphi\varphi} u^{\varphi}= r^{2}\dot \varphi = h$$ 

$$\implies \abs{\vb{u}}^{2} =1 = \dot r^{2}+ r^{2}\dot \varphi^{2}= \dot r^{2}+ frac r^2h^{2} r^{4}$$ 




\end{document}
