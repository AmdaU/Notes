\documentclass{article}    
\usepackage[utf8]{inputenc}    
    
\title{Épisode 4}    
\author{Jean-Baptiste Bertrand}    
\date{\today}    
    
\setlength{\parskip}{1em}    
    
\usepackage{physics}    
\usepackage{graphicx}    
\usepackage{svg}    
\usepackage[utf8]{inputenc}    
\usepackage[T1]{fontenc}    
\usepackage[french]{babel}    
\usepackage{fancyhdr}    
\usepackage[total={19cm, 22cm}]{geometry}    
\usepackage{enumerate}    
\usepackage{enumitem}    
\usepackage{stmaryrd}    
    
%packages pour faire des math    
%\usepackage{cancel} % hum... pas sur que je vais le garder mais rester que des fois c'est quand même sympatique...
\usepackage{amsmath, amsfonts, amsthm, amssymb}    
\usepackage{esint}  

\begin{document}

2022-09-21

\setcounter{section}{2}
\section{Principes fondamentaux de la relativité générale}
\subsection{Théorie newtonienne de la gravitation}

Newton ne cherche pas à expliquer le mécanisme de la gravité : il donne simplement une formule.

Le concept de champ gravitationnel nait naturellement de le relativité restraint car la force ne peut pas être instantanée. On a besoin d'un champ pour \textit{contenir} la quantité de mouvement et l'énergie pendant un certain temps.

$$\vb{F}(\vb{r}) = m \underbrace{\vb{g}(\vb{r})}_{\text{champ gravitationnel } } $$ 

$$\vb{g} (\vb{r}) = -G \sum_i \frac{m_i}{\abs{\vb{r}-\vb{r}_i}^{3}} \left( \vb{r}-\vb{r}_i \right) $$ 

$$\grad \cdot \vb{g} = -4\pi G \underbrace{\rho (\vb{r})}_{\text{densité de masse} } $$ 

$$\vb{g} (\vb{r}) = - \grad \Phi$$ 

$$\Grad \Phi = 4\pi G\rho$$ 

$$\Phi(\vb{r}) = -G \int \dd^{3}r' \frac{\rho (\vb{r})}{\abs{\vb{r}-\vb{r}'} } $$ 


\begin{tcolorbox}[title=Unitées ]
	$$E = \rm[G] \frac{M^{2}}{L}  $$
	$$E =\rm  \frac{L^{2}}{T^2} M \xrightarrow{c=1} M $$ 
	$$c =1 \rm \to L= T$$ 
\end{tcolorbox}

On distingue les masses \textit{inertiel} et \textit{gravitationnel} 

$$\vb{F}(\vb{r}) = m_{\text{grav}}  \vb{g}(\vb{r}) = m_{\text{inert}} \vb{a} $$ 

Si $m_{\text{grav}} = m_{\text{inert}} \therefore \vb{a} = \vb{g} $ 


Ce qui nous interesse est réellement le rapport $m_{\text{inert}}/m_{\text{grav}}  $ 

L'expérience de Potvis vise a vérifier si cette masse est identique pour toutes substance.

Il utilise la force centrifuge, qui est une force inertiel pour comparer les rapport de masse. Il a été démontré que les deux sont pareils jusqu'à 10^{-9} 

Récemment, un sonde français a démontré que c'est la même chose jusqu'à 10^{-15}.

Cette égalité est le \textbf{principe d'équivalence faible}

Il suggère qu'un force inertiel est in différentiable d'une force gravitationnelle qui est le \textbf{principe d'équivalence faible}

\section*{coordonnées de Rindler}

$$x = \xi \cosh \theta \qquad t = \xi \sinh\theta $$ 


\begin{figure}[ht]
    \centering
    \incfig{coordonnées-de-rindler}
    \caption{Coordonnées de Rindler}
    \label{fig:coordonnées-de-rindler}
\end{figure}

Observateur à $\xi = \text{cst} $ 

$$u^{i} = \xi \dot\theta \left( \cosh \theta, \sinh\theta \right) $$ 

$$u^{i}u_i =1 = \xi^2\dot\theta^{2} \underbrace{\left( \cosh^{2}\theta- \sinh^{2}\theta \right)}_{1} $$ 

$$\implies 1 = \xi \dot \theta \implies \dot\theta = \text{cst} $$ 
$$\theta = \xi \tau$$ 


$$a^{i}= \dot \theta \left( \sinh\theta, \cosh \theta \right) $$ 

$$a^{i}a_i = \frac{1}{\xi^2} \left( \sinh^{2}-\cosh^2 \right) = - \frac{1}{\xi^2} $$ 

accélération propre $\frac{1}{\xi} $ 

\begin{tcolorbox}[title=]
	Les coordonnées ne sont pas nécessaire en relativité générales et les problèmes peuvent être formulées comme des observateurs s'échangeant des signaux lumineux. 
\end{tcolorbox}

\subsection*{Tétrade}

On peut toujours définir un base locale respectant le produit scalaire de Minkowski. qui différent de celle imposé par \textit{le} système de coordonnées. 


\section*{Coordonnées localement cartésiennes}


On définit

$$x'^p = \left( x^{i}- x_p^{i} \right) + \frac{1}{2} \Gamma_{jk}^{i}(p) (x^{i}- x_p^{j})\left( x^{k}- x_p^{k}  \right)  $$ 

$$\pdv{x'^{i}}{x^j} = \delta_i^{j} + \Gamma_{jk}^{i}(p) \left( x^{k}-k_p^{k} \right) $$ 

$$\pdv[2]{x'^i }{x^{i}}{x^{k}}  = \Gamma_{jk}^{i}(P)$$ 

$$\Gamma'_{jk}^{i}(p)= \pdv{x'^k}{x^l} o\dv{x'^m}{x^i} \pdv{x'^n}{x^j} \Gamma_{mn}^{l}(P) - \pdv{x'^,}{x^i} \pdv{x'^n}{x^j} \pdv{x'^k}{x^m}{x^n}$$ 

$$= \dotsb = 0$$ 

L'équation de la géodésique au point $P$ est donc simplement donnée par $\ddot x^{i}=0$  



\end{document}
