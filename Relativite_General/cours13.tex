\documentclass{article}    
\usepackage[utf8]{inputenc}    
    
\title{Épisode 4}    
\author{Jean-Baptiste Bertrand}    
\date{\today}    
    
\setlength{\parskip}{1em}    
    
\usepackage{physics}    
\usepackage{graphicx}    
\usepackage{svg}    
\usepackage[utf8]{inputenc}    
\usepackage[T1]{fontenc}    
\usepackage[french]{babel}    
\usepackage{fancyhdr}    
\usepackage[total={19cm, 22cm}]{geometry}    
\usepackage{enumerate}    
\usepackage{enumitem}    
\usepackage{stmaryrd}    
\usepackage{mathtools,slashed}
%\usepackage{mathtools}
\usepackage{cancel}
    
\usepackage{pdfpages}
%packages pour faire des math    
%\usepackage{cancel} % hum... pas sur que je vais le garder mais rester que des fois c'est quand même sympatique...
\usepackage{amsmath, amsfonts, amsthm, amssymb}    
\usepackage{esint}  
\usepackage{dsfont}

\usepackage{import}
\usepackage{pdfpages}
\usepackage{transparent}
\usepackage{xcolor}
\usepackage{tcolorbox}

\usepackage{mathrsfs}
\usepackage{tensor}

\usepackage{tikz}
\usetikzlibrary{quantikz}
\usepackage{ upgreek }

\newcommand{\incfig}[2][1]{%
    \def\svgwidth{#1\columnwidth}
    \import{./figures/}{#2.pdf_tex}
}

\newcommand{\cols}[1]{
\begin{pmatrix}
	#1
\end{pmatrix}
}

\newcommand{\avg}[1]{\left\langle #1 \right\rangle}
\newcommand{\lambdabar}{{\mkern0.75mu\mathchar '26\mkern -9.75mu\lambda}}

\pdfsuppresswarningpagegroup=1


\begin{document}
2022-44-12

\subsection*{Orbite circulaire}

$$\frac{1}{2} \dot r^{2} + V_{\text{eff}}(r) = k^2-1 $$ 

Pour une orbite circulaire, $\dot r = 0$ par définition 

$$v_{\text{eff}} = \frac{\bar h^{2}r^{2}}{2 r^2} \left( 1 - \frac{r_s}{r}  \right) - \frac{r_s}{2r}   $$ 


$$0 = V_{\text{eff}}' =i \frac{\bar h^{2}r_s^{2}}{r^3} \left( 1 - \frac{3}{2} \frac{r_s}{r}  \right) + \frac{r_s}{2r^{2}}  $$ 

Pour un $\bar h$ donné: on trouve $r$. Ensuite on substitue dans $ V_{\text{eff}} $ et on trouve $k$.

$$r = 3r_s$$ 


\subsection*{Précession de la périhélie de mercure}



\end{document}
