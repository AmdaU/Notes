\documentclass{article}    
\usepackage[utf8]{inputenc}    
    
\title{Épisode 4}    
\author{Jean-Baptiste Bertrand}    
\date{\today}    
    
\setlength{\parskip}{1em}    
    
\usepackage{physics}    
\usepackage{graphicx}    
\usepackage{svg}    
\usepackage[utf8]{inputenc}    
\usepackage[T1]{fontenc}    
\usepackage[french]{babel}    
\usepackage{fancyhdr}    
\usepackage[total={19cm, 22cm}]{geometry}    
\usepackage{enumerate}    
\usepackage{enumitem}    
\usepackage{stmaryrd}    
\usepackage{mathtools,slashed}
%\usepackage{mathtools}
\usepackage{cancel}
    
\usepackage{pdfpages}
%packages pour faire des math    
%\usepackage{cancel} % hum... pas sur que je vais le garder mais rester que des fois c'est quand même sympatique...
\usepackage{amsmath, amsfonts, amsthm, amssymb}    
\usepackage{esint}  
\usepackage{dsfont}

\usepackage{import}
\usepackage{pdfpages}
\usepackage{transparent}
\usepackage{xcolor}
\usepackage{tcolorbox}

\usepackage{mathrsfs}
\usepackage{tensor}

\usepackage{tikz}
\usetikzlibrary{quantikz}
\usepackage{ upgreek }

\newcommand{\incfig}[2][1]{%
    \def\svgwidth{#1\columnwidth}
    \import{./figures/}{#2.pdf_tex}
}

\newcommand{\cols}[1]{
\begin{pmatrix}
	#1
\end{pmatrix}
}

\newcommand{\avg}[1]{\left\langle #1 \right\rangle}
\newcommand{\lambdabar}{{\mkern0.75mu\mathchar '26\mkern -9.75mu\lambda}}

\pdfsuppresswarningpagegroup=1

\begin{document}

{\Huge Relativité générale}

\section*{Rappels sur la relativité restrainte (vecteurs et tenseurs)}

* On prend comme exemple une espace euclidien de dimension 2 mais la théorie est générale


$$\vb{A} = A^{1}\vb{e}_2 + A^{2}\vb{e}_2 + A^3\vb{e}_3 = A^{i}\vb{e}_i$$ 

les $A^{i}$ sont les composantes \textit{contravariantes } 


Changement de base: $\left\{ \vb{e}_1, \vb{e}_2, \vb{e}_3 \right\} $  

$$\vb{e}_i = \Lambda_{i}^{j} \vb{e}_j' = \Lambda_i^{1}\vb{e}_1' + A_i^{2}\vb{e}_2' +A_i^{3}\vb{e}_3' $$ 

$$\vb{e}_i' = (\Lambda^{-1})_i^{j}\vb{e}_j$$ 

$$\vb{A} = A^j\vb{e}_j = \underbrace{A^{j}\Lambda_j^{i}}_{A^{i\prime}}e_i' $$ 

$$\boxed{A'i = \Lambda_j^{i}A^{j}}$$ 

\subsection*{Base duale}


ayant une base $B$ On peut définir une base duale $\tilde B = \left\{ \vb{e}^1, \vb{e}^2, \vb{e}^3 \right\} | \vb{e}^i \cdot \vb{e}^j = \delta^i_j  $  

$$\vb{A} = \underbrace{A^{i}}_{contravariante} \vb{e}_i = \underbrace{A_j}_{covariantes}  \vb{e}^j$$ 


$$\vb{A} \cdot \vb{e}^i = A^i$$ 
$$\vb{A} \cdot \vb{e}_i = A_i $$ 


On veut démontrer que $\vb{e}^{\prime i} = A_{j}^{i} \vb{e}^j $ 


\underline{Tenseurs:} 


base : $\vb{e}_i \otimes \vb{e}_j$

$$\vb{T}= T^{ij}\vb{e}_j \otimes \vb{e}_j$$

Il y a des représentation covarientes contravarites et mixtes au tenseurs.


$$T^{\prime ij}  = \Lambda_k^{i}\Lambda_l^{j}T^{kl}$$ 
$$T_j^{\prime i} = \Lambda_k^{i} (\Lambda^{-1})_j^{l}T_l^k$$ 


$$T_i^{i} = tr(T) = \cdot =tr (T') $$ 


\subsection*{Tenseur Métrique}


$$\vb{e}_i = g_{ij} \vb{e}^j \iff \vb{e}_i \cdot \vb{e}_j = g_{ik} \underbrace{\vb{e}^k \cdot \vb{e}_j}_{\delta_j^k}= g_ij $$ 

de même: $$\vb{e}^i \cdot \vb{e}^j = g^{ij}\vb{e}^j$$ 

$$\vb{A} \vb{B} = A^{i}\vb{e}_i B^{j}\vb{e}_j = g_{ij} A^{i}B^i$$ 

$$A^{i}=\vb{A}\vb{e}^i = \vb{A} \cdot  (g^{ij}\vb{e}_j) = g^{ij}A_j$$ 


\begin{tcolorbox}[title=]
	 $$A^{i}= g^{ij}A_j$$ 
	 $$A_i = g_{iij} A^j$$ 
$$g_{ik} g^{kj}= \delta_i^{j}$$ 
\end{tcolorbox}

\section*{Espace-Temps (1908)}

Un concept définis par Minkowski après avoir lu les papier de Einstein de 1905. Ce dernier n'aimait pas du tout ce concept. 

\begin{tcolorbox}[title=Quadrivecteur]
$$x^{i}= (ct, x, y, z)$$ 
\end{tcolorbox}


\begin{tcolorbox}[title=Transformation de Lorentz]
$$x^{\prime{}i} = \Lambda_j^{i} x^j$$
\end{tcolorbox}

\begin{tcolorbox}[title=Intervalle]
$$	S^{2}= \cancelto{1}{c^2}t^{2}- x^{2}-y^{2}-z^2$$ 
\end{tcolorbox}


\begin{tcolorbox}[title=unitées Géométriques]
	$$G= 1 \qquad c= 1$$ 
\end{tcolorbox}


\begin{tcolorbox}[title=Transformation de Lorentz]
	 $$\Lambda^{T}g\Lambda =g$$ 
\end{tcolorbox}

On a 16 variables dans une matrice 4x4. On a une contrainte sur 10 d'entres elles. Il reste donc 6 degrés de libertés. Celles ci représente l'alignement des axes et la vitesse.

\underline{Rapidité} 


$$\tanh{\psi} = v$$

$$\Lambda = \cols{-x\sinh{\psi} & t\cosh{\psi} &0 &0\\x\cosh\psi &- t\sinh\psi & 0 &0\\ 0 & 0 &1 &0 \\ 0& 0 &0 &1}$$ 

\underline{Quadrigradient} 

$$\partial_i = \pdv{x^i}$$

$$\boxed{\partial_{i'} = (\Lambda^{-1})_i^{j}}$$ 

\end{document}
