\documentclass{article}    
\usepackage[utf8]{inputenc}    
    
\title{Épisode 4}    
\author{Jean-Baptiste Bertrand}    
\date{\today}    
    
\setlength{\parskip}{1em}    
    
\usepackage{physics}    
\usepackage{graphicx}    
\usepackage{svg}    
\usepackage[utf8]{inputenc}    
\usepackage[T1]{fontenc}    
\usepackage[french]{babel}    
\usepackage{fancyhdr}    
\usepackage[total={19cm, 22cm}]{geometry}    
\usepackage{enumerate}    
\usepackage{enumitem}    
\usepackage{stmaryrd}    
\usepackage{mathtools,slashed}
%\usepackage{mathtools}
\usepackage{cancel}
    
\usepackage{pdfpages}
%packages pour faire des math    
%\usepackage{cancel} % hum... pas sur que je vais le garder mais rester que des fois c'est quand même sympatique...
\usepackage{amsmath, amsfonts, amsthm, amssymb}    
\usepackage{esint}  
\usepackage{dsfont}

\usepackage{import}
\usepackage{pdfpages}
\usepackage{transparent}
\usepackage{xcolor}
\usepackage{tcolorbox}

\usepackage{mathrsfs}
\usepackage{tensor}

\usepackage{tikz}
\usetikzlibrary{quantikz}
\usepackage{ upgreek }

\newcommand{\incfig}[2][1]{%
    \def\svgwidth{#1\columnwidth}
    \import{./figures/}{#2.pdf_tex}
}

\newcommand{\cols}[1]{
\begin{pmatrix}
	#1
\end{pmatrix}
}

\newcommand{\avg}[1]{\left\langle #1 \right\rangle}
\newcommand{\lambdabar}{{\mkern0.75mu\mathchar '26\mkern -9.75mu\lambda}}

\pdfsuppresswarningpagegroup=1


\begin{document}
2022-36-05

\section*{Métrique de Schwarzschild}

Solution des équation de Einstein à symétrie sphérique. Solution découverte par Schwarzschild un peu avant sa mort.

$$\dd \tau = A(r) \dd t^{2} B(r) \dd r^{2}- r^{2}\left( \sint^2\theta \dd \varphi^{2}-\dd \theta ^2 \right)= \left( 1 - \frac{r_s}{r} \dd t^{2} \right) \dd t^{2}- \frac{1}{\left( 1 - \frac{r_s}{r}  \right)} \dd r^2  -r^{2}\left( \sin^2\theta \dd \varphi	^2 + \dd \theta ^2 \right)   $$ 

Cette métrique présente une singularité à $r = r_s $. Cependant c'est une artéfact du système de coordonné et non un singularité \textit{physique}. On peut s'en rendre compte en étudiant des valeurs qui ne dépende pas des coordonnés comme le tenseur de Riemann. En faisant cela, on se rend compte qu'il y a un \textit{vraie} singularité en $r=0$  

$$R^{ijkl}R_{ijkl} = 12 \frac{r_s^{2}}{r^{6}} \qq{singularité géométrique!}$$ 

\begin{tcolorbox}[title=]
	On note la partie spatiale des coordonnées $\vec r = ( x^{1}, x^{2}, x^{3} )$  
\end{tcolorbox}

On peut alors noter la métrique de manière générale (avec des fonction arbitraire de toute les quantité qui sont invariantes par rotation)

$$\dd \tau^{2}= A(r,t) \dd t^{2}- B(r,t ) \dd t (\vb{r} \cdot  \dd \vb{r}) - C(r,t)(\vb{r}\cdot \dd \vb{r})^2 -D(r,t) \dd \vb{r}^2$$ 
On peut faire le changement de variable:
$$\begin{cases}
	x^{1} = r\sin\theta\cos\varphi\\
	x^{2}= r\sin\theta \sin\varphi\\
	x^{3}= r\sin\theta
\end{cases}$$  

Notre métrique est alors
$$\dd \tau^{2}= A\dd t^{2}- V r \dotsb -D\left( \sin^{2\theta}\dd \varphi^{2}+ \dd \theta^2 \right) $$ 

On peut alors toujours faire un changement de coordonné ou $\sqrt{D} =r$. On trouve alors

$$\dd \tau^{2}= A \dd t^{2} + B \dd t \dd r - C \dd r^{2} -r^{2}\left( \sin^2\theta \dd \varphi^{2}\dd \theta^{2} \right) $$ 
Si on impose également un symétrie d'inversion du temps (qui exclus les rotation), $B$ doit être nul car ce terme n'as pas cette symétrie. Ce n'est pas nécessaire de requérir cette symétrie. Plutôt, on peut posser

$$\dd \bar t^{2} = \Pih \left[ A \dd t - \frac{1}{2} B \dd r \right] $$ 

Cela permet de se débarrasser de ce $B$ 

On a donc 

$$\dd \tau^{2}= A(r,t) \dd t^{2}- B(r,t) \dd r^{2} - r^{2}\left( \sin^{2}\tetha \dd \varphi^{2}+ \dd \theta^{2} \right) $$ 

$$g_{ij} = \mqty[\dmat{A(r),-B(r),-r^{2},-r^2\sin^2\theta}]$$ 

$$R_{ij} =0$$ 

$$\begin{cases}
	R_{00} = \frac{A''}{2B} - \frac{A'}{4B} \left( \frac{A'}{A} + \frac{B'}{B}   \right) + \frac{A'}{rB}\\
	R_{11} = - \frac{A''}{2A} + \frac{A'}{4A} \left( \frac{A'}{A} + \frac{B'}{B}  \right) + \frac{B'}{rB} \\
	R_{22} = 1 - \frac{1}{B} - \frac{r}{2B} \left( \frac{A'}{A} - \frac{B'}{B}  \right)\\
	R_{33} = R_{32} \sin \theta 
\end{cases}$$ 


$$R_{00} + \frac{A}{B} R_{11} = 0$$ 

$$\frac{A'}{rB} + \frac{B'A}{rB^2} =0$$ 
$$\frac{1}{r} \left( A'B + B'A \right) = 0$$ 

$$\implies \left( AB \right)' = 0 \implies AB = \text{CST} = \alpha $$ 


$$\dotsb$$ 

$$\boxed{\dd \tau^{2}= \left( 1 - \frac{2M}{r}  \right) \dd t^{2}- \frac{1}{1 - \frac{2M}{r} }\dd r^{2}-r^{2}\dd \Omega^2 }$$ 

\hrule
\section*{Loi de conservation}

$$\boxed{\dot u_i = \frac{1}{2} \partial_i g_{mk} u^{m}u^k}$$ 

$$\dot u_0 =0 \quad u_0 = \text{cst}= k = g_{00} \dot u^{0}= g_{00} \dot t = \underbrace{\boxed{\left( 1 - \frac{r_0}{r} \right)\dv{t}{\tau} =k  }}_{\substack{\text{conservation de l'énergie}\\ \text{cinétique par unité de masse} }  $$ 

$$\dot u_3 = \text{cst} \implies u_3  = g_{33} u^{3}=  -r^2\sin^{2}\theta \dot \varphi = -h $$ 

$$\underbrace{\boxed{ r^{2}\dv{\varphi}{\tau} = h }}_{\text{moment cinétique par masse} } $$ 


\end{document}
 
