\documentclass{article}    
\usepackage[utf8]{inputenc}    
    
\title{Épisode 4}    
\author{Jean-Baptiste Bertrand}    
\date{\today}    
    
\setlength{\parskip}{1em}    
    
\usepackage{physics}    
\usepackage{graphicx}    
\usepackage{svg}    
\usepackage[utf8]{inputenc}    
\usepackage[T1]{fontenc}    
\usepackage[french]{babel}    
\usepackage{fancyhdr}    
\usepackage[total={19cm, 22cm}]{geometry}    
\usepackage{enumerate}    
\usepackage{enumitem}    
\usepackage{stmaryrd}    
\usepackage{mathtools,slashed}
%\usepackage{mathtools}
\usepackage{cancel}
    
\usepackage{pdfpages}
%packages pour faire des math    
%\usepackage{cancel} % hum... pas sur que je vais le garder mais rester que des fois c'est quand même sympatique...
\usepackage{amsmath, amsfonts, amsthm, amssymb}    
\usepackage{esint}  
\usepackage{dsfont}

\usepackage{import}
\usepackage{pdfpages}
\usepackage{transparent}
\usepackage{xcolor}
\usepackage{tcolorbox}

\usepackage{mathrsfs}
\usepackage{tensor}

\usepackage{tikz}
\usetikzlibrary{quantikz}
\usepackage{ upgreek }

\newcommand{\incfig}[2][1]{%
    \def\svgwidth{#1\columnwidth}
    \import{./figures/}{#2.pdf_tex}
}

\newcommand{\cols}[1]{
\begin{pmatrix}
	#1
\end{pmatrix}
}

\newcommand{\avg}[1]{\left\langle #1 \right\rangle}
\newcommand{\lambdabar}{{\mkern0.75mu\mathchar '26\mkern -9.75mu\lambda}}

\pdfsuppresswarningpagegroup=1


\begin{document}
2022-33-28

$$S_m = - \sum_{\alpha} m_\alpha\int \dd \tau_{\alpha} \sqrt{g_{ij} (x_{\alpha} ) \dot x_{\alpha}^i \dot x_{\alpha}^{j}}$$ 

$$S_g = \kappa \int \underbrace{\dd \Omega \sqrt{\abs{g} }}_{\text{invarien de Lorentz} } R $$ 


$$S = S_m + S_g$$ 


$$\frac{\delta S}{\delta g_{ij} } = 0o$$ 


$$\delta S_m = - \frac{1}{2} \sum_{\alpha} m_{\alpha} \int \dd \tau_\alpha \cancelto{1}{\frac{1}{\sqrt{g_{ij}  \dot x_{\alpha}^{i}\dot x_{\alpha}^{j}}}} \dot x_{\alpha}^{k}\dot x_{\alpha}^{l} \delta g_{ki} (x_\alpha)$$ 

On définit le \underline{tenseur énergie-impulsion} 

$$T^{ij} = \frac{1}{\sqrt{\abs{g}}} \sum_{\alpha} m_{\alpha} \int \dd \tau_{\alpha} \dot x_{\alpha}^{i}\dot x_{\alpha}^{j}\delta^{2}(x-x_\alpha(\tau_\alpha)) $$ 

Limite non-relativiste: les particules ne vont pas très vite et toutes les particules ont approximativement le même temps qu'on prend être le temps coordonnée. 

$$T^{ij}_{\text{classique}} = \delta_0^{i}\delta_0^{j} \frac{1}{\abs{g}} \underbrace{\sum_{\alpha} m_{\alpha} \delta^{3}(\vb{r}-\vb{r}_\alpha(t))}_{\text{densité de masse }(\rho(\vb{r})) }   $ $ 

$$\delta \sqrt{\abs{g}} =\hspace{-11pt}\raisebox{6pt}{?}\hspace{5pt} \frac{1}{2} \sqrt{|g|} g^{ij} \delta g_ij $$ 


$$\ln \det M = \tr \ln M$$ 

$$\dleta \tr \ln g = \tr \delta (
\ln g
) \dotsb $$ 



$$R = g^{ij} R_{ij}$$ 

$$\delta R $$ 

\end{document}
