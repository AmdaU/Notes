\documentclass{article}    
\usepackage[utf8]{inputenc}    
    
\title{Épisode 4}    
\author{Jean-Baptiste Bertrand}    
\date{\today}    
    
\setlength{\parskip}{1em}    
    
\usepackage{physics}    
\usepackage{graphicx}    
\usepackage{svg}    
\usepackage[utf8]{inputenc}    
\usepackage[T1]{fontenc}    
\usepackage[french]{babel}    
\usepackage{fancyhdr}    
\usepackage[total={19cm, 22cm}]{geometry}    
\usepackage{enumerate}    
\usepackage{enumitem}    
\usepackage{stmaryrd}    
\usepackage{mathtools,slashed}
%\usepackage{mathtools}
\usepackage{cancel}
    
\usepackage{pdfpages}
%packages pour faire des math    
%\usepackage{cancel} % hum... pas sur que je vais le garder mais rester que des fois c'est quand même sympatique...
\usepackage{amsmath, amsfonts, amsthm, amssymb}    
\usepackage{esint}  
\usepackage{dsfont}

\usepackage{import}
\usepackage{pdfpages}
\usepackage{transparent}
\usepackage{xcolor}
\usepackage{tcolorbox}

\usepackage{mathrsfs}
\usepackage{tensor}

\usepackage{tikz}
\usetikzlibrary{quantikz}
\usepackage{ upgreek }

\newcommand{\incfig}[2][1]{%
    \def\svgwidth{#1\columnwidth}
    \import{./figures/}{#2.pdf_tex}
}

\newcommand{\cols}[1]{
\begin{pmatrix}
	#1
\end{pmatrix}
}

\newcommand{\avg}[1]{\left\langle #1 \right\rangle}
\newcommand{\lambdabar}{{\mkern0.75mu\mathchar '26\mkern -9.75mu\lambda}}

\pdfsuppresswarningpagegroup=1


\begin{document}
2022-00-02

\section*{Horizons et singularités}

Certaines singularité sont \underline{fondamentale}. C'est-à-dire qu'elle ne dépendent pas du choix de coordonnées.

Par exemple. Dans la métrique de Schwartzchild, la quantitée $R_{ijkl} R^{ijkl} = 12 \frac{r_s^{2}}{r^6} $ est un invariant. La singularité en $r=0$ est donc \textit{intrinsèque}.

\underline{Horizon}: hypersurface nulle traversable dans un sens seulement (par de photons)

\underline{Trou noir}: horizon fermé duquel les photons ne peuvent pas sortir 
\underline{Trou blanc}: horizon fermé dans lequel les photons ne peuvent pas rentrer

\begin{figure}[ht]
    \centering
    \incfig{types-d'hyper-surfaces}
    \caption{Types d'hyper-surfaces}
    \label{fig:types-d'hyper-surfaces}
\end{figure}



\end{document}
