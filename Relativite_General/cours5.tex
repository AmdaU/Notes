\documentclass{article}    
\usepackage[utf8]{inputenc}    
    
\title{Épisode 4}    
\author{Jean-Baptiste Bertrand}    
\date{\today}    
    
\setlength{\parskip}{1em}    
    
\usepackage{physics}    
\usepackage{graphicx}    
\usepackage{svg}    
\usepackage[utf8]{inputenc}    
\usepackage[T1]{fontenc}    
\usepackage[french]{babel}    
\usepackage{fancyhdr}    
\usepackage[total={19cm, 22cm}]{geometry}    
\usepackage{enumerate}    
\usepackage{enumitem}    
\usepackage{stmaryrd}    
    
%packages pour faire des math    
%\usepackage{cancel} % hum... pas sur que je vais le garder mais rester que des fois c'est quand même sympatique...
\usepackage{amsmath, amsfonts, amsthm, amssymb}    
\usepackage{esint}  

\begin{document}

2022-09-14

\section*{Courbure}



\underline{Tenseur de Riemann $R_{jkl}^i$ } 

$$R_{kji}^{l}= \partial_i \Gamma_{ki}^{l}-\partial_{i\Gamma_kj}^{l}+ \Gamma_{ki}^{m}\Gamma_{mj}^{l}- \Gamma_{kj}^{m}\Gamma_{mi}^{l}$$ 


\begin{tcolorbox}[title=(1) Non commutativité des dérivées covarientes]

	$$\partial_i \partial_i Y = \partial_j \partial_i Y$$ 
	$$\nabla_i \nabla_j A_k -\nabla_j \nabla_i A_k = R_{kji}^{l}A_l $$ 
	 
\end{tcolorbox}

\begin{tcolorbox}[title= (2) holonomie]
$$\Delta A^{i} = R_{kjl}^{i}A^{k}\dd x^l \dd x'^j$$ 
(voir figure \ref{fig:holonomie})
\end{tcolorbox}

\begin{figure}[ht]
    \centering
    \incfig{holonomie}
    \caption{holonomie}
    \label{fig:holonomie}
\end{figure}


\begin{tcolorbox}[title=(3) déviation géodésique]
\begin{tcolorbox}[title=dérivé intrinsèque]
	$$\vb{A}(\lamda) = A^{i}(lambda) \vb{e}_i(\lambda)$$ 
	$$\dv{{}}{\lambda} \vb{A} = \dv{A^{i}}{\lambda} \vb{e}_i + A^{i} \dv{\vb{e}_i}{\lambda}  = \dot A^{i}\vb{e}_i + A^{i}\partial_j \vb{e}_i \dv{x^{i}}{\lambda}  = \left( \dot A^{k}+ \Gamma)ji^{k} A^{i}\dot x^j \right) \vb{e}_k$$ 
	$$\nabla_k A^{k}= \dot A^{k}+\Gamma_{ji}^{k}A^{i}\dot x^j$$ 
\end{tcolorbox}

$$\boxed{\nabla_{lamda}^{2}\xi^{i}= R_{jkm}^{i}\xi^{m}\dot x^{j}\dot x^{k}}$$ 
	 
\end{tcolorbox}

\begin{figure}[ht]
    \centering
    \incfig{déviation-géodésique}
    \caption{déviation géodésique}
    \label{fig:déviation-géodésique}
\end{figure}


\underline{Tenseur de Rixxi} 
$$R_{ik} = R_{ijk}^{j}$$ 


\underline{Tenseur scalaire} 
$$R=r_i^{i}= g^{ik}R_{ik} $$ 


$$\nabla_j A_k = \partial_j A_k - \Gamma_{jk}^{m}A_m $$ 

$$\nabla_{i}{\nabla_j} A_k = \nabla_i \left[ \partial_j A_k - \Gamma_{jk}^{m}A_m \right] = \partial_i \partial_j A_k - \partial_j \left( \Gamma_{jk}^{m}A_m  \right) -\Gamma_{ij}^{l}\left( \partial_l A_k - \Gamma_{lk}^{m}A_m  \right) - \Gamma_{ik}^{l}\left( \del_i A_l - \Gamma_{jl}^{m}A_m  \right) $$ 

$$= \partial_i \partial_j A_k - \Gamma_{jk}^{m}\partial_i A_m -\partial_i \Gamma_{jk}^{m}A_m -\Gamma_{ij}^{l}\partial_l A_k - \Gamma_{ij}^{l}\partial_j A_l + \Gamma_{ij}^{l}\Gamma_{lk}^{m}A_m + \Gamma_{ik}^{l}\Gamma_{jl}^{m}A_m $$ 

$$\nalba_j \nabla_i A_k = \dotsb(j\leftrightarrow i)$$ 

$$\left( \nabla_i\nabla_j -\nabla_j \nabla_i \right)A_k = \left( - \partial_i \Gamma_{jk}^{m}+ \partial_j \Gamma_{ik}^{m} \right) A_m + \left( \Gamma_{ik}^{l}\Gamma_{jl}^{m}- \Gamma_{jk}^{l}\Gamma_{il}^{m} \right) A_m $$ 


$$= (\partial_i \Gamma_{ki}^{l}-\partial_{i\Gamma_kj}^{l}+ \Gamma_{ki}^{m}\Gamma_{mj}^{l}- \Gamma_{kj}^{m}\Gamma_{mi}^{l})A_l $$ 

\underline{Propriétés } 

A)

$$R_{lkji} = \frac{1}{2} \left( \del_i \partial_j g_{ki} + del_k \partial_j g_{li} - \partial_l \partial_j g_{ki} - \partial_k \partial_i g_{lj}  \right) +g^{mn}\left( \Gamma_{mil} \Gamma_{nkj} - \Gamma_{mjl} \Gamma_{nki} \right)  $$ 


$$R_{lkji} = R_{klji} $$ 
$$R_{lkji} = R_{lkij} $$ 
$$R_{lkji} = R_{jilk} $$ 
$$R_{lkji} + R_{ljik} + R_{likj} =0$$ 

En $d$ dimensions il y a $ \frac{1}{12} d^{2}(d^2-1)$ (20 pour $d=4$ )   

B) Indentité de Bianchi


$$\nabla_m R_{ijkl}+ \nabla_k R_{ijlm} + \nabla_l R_{ijmk} =0 $$ 

\begin{tcolorbox}[title=Exemple 1: sphère de rayon a]

	$$\dd s = a^{2}\dd \theta^{2}+ a^{2}\sin^{2}\theta\dd \varphi^2$$ 
	$$[g_{ij} ] = a^{2}\mqty[1 & 0 \\ 0 & \sin^2\theta]$$ 
	$$[g^{ij}] = \frac{1}{a^2} \mqty[ 1 & 0 \\ 0 & \frac{1}{\sin\theta}  ]$$ 
	$$\Gamma_{22}^{1}= -\sin\theta\cos\theta \qquad \gamma_{12}^{2}=\Gamma_{21}^{2}= \cot\theta$$ 

$$R_{1212}  =a^2\sin\theta$$ 

$$R_{22} = \sin\theta$$ 

$$R_{11} = 1$$ 

$$R = g^{ij}R_{ij} = \frac{2}{a^2} $$ 
	 
\end{tcolorbox}

 
\begin{tcolorbox}[title=Exemple 2: le cylindre de rayon a]
	 $$\dd s = \dd z^{2}+ a^{2}\dd \varphi^2$$ 
	 $$[g_{ij} ] = \mqty[1 &0 \\ 0 & a^2]$$ 
	 Le cylindre est plat!
\end{tcolorbox}

\begin{tcolorbox}[title=Exemple 3: le cône]
	Le cône est plat \textbf{sauf} à l'apex, qui possède un courbure infini

\end{tcolorbox}

\begin{tcolorbox}[title=Exemple 4: tore plongé dans $\mathds{R}^{3} $ ]
	$$\dotsb$$  
\end{tcolorbox}



\end{document}
