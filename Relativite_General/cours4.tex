\documentclass{article}    
\usepackage[utf8]{inputenc}    
    
\title{Épisode 4}    
\author{Jean-Baptiste Bertrand}    
\date{\today}    
    
\setlength{\parskip}{1em}    
    
\usepackage{physics}    
\usepackage{graphicx}    
\usepackage{svg}    
\usepackage[utf8]{inputenc}    
\usepackage[T1]{fontenc}    
\usepackage[french]{babel}    
\usepackage{fancyhdr}    
\usepackage[total={19cm, 22cm}]{geometry}    
\usepackage{enumerate}    
\usepackage{enumitem}    
\usepackage{stmaryrd}    
\usepackage{mathtools,slashed}
%\usepackage{mathtools}
\usepackage{cancel}
    
\usepackage{pdfpages}
%packages pour faire des math    
%\usepackage{cancel} % hum... pas sur que je vais le garder mais rester que des fois c'est quand même sympatique...
\usepackage{amsmath, amsfonts, amsthm, amssymb}    
\usepackage{esint}  
\usepackage{dsfont}

\usepackage{import}
\usepackage{pdfpages}
\usepackage{transparent}
\usepackage{xcolor}
\usepackage{tcolorbox}

\usepackage{mathrsfs}
\usepackage{tensor}

\usepackage{tikz}
\usetikzlibrary{quantikz}
\usepackage{ upgreek }

\newcommand{\incfig}[2][1]{%
    \def\svgwidth{#1\columnwidth}
    \import{./figures/}{#2.pdf_tex}
}

\newcommand{\cols}[1]{
\begin{pmatrix}
	#1
\end{pmatrix}
}

\newcommand{\avg}[1]{\left\langle #1 \right\rangle}
\newcommand{\lambdabar}{{\mkern0.75mu\mathchar '26\mkern -9.75mu\lambda}}

\pdfsuppresswarningpagegroup=1

\begin{document}

2022-09-09

\begin{tcolorbox}[title=À l'épisode précédent:]
	
$$\partial_i A_j, \, \partial_i A^j$$ 

$$\nabla_i A_j = \partial_i A_j - \Gamma_{ji}^{k}A_k$$ 
$$\nabla_i A^{j}= \partial_i A^{j}+ \Gamma_{kj}^{j}A^{k}$$ 
$\Gamma$ est la \underline{connection affine ou symbole de Chritoffel}  
	
$$\Gamma_{ik}^{k}(x) = \frac{1}{2} g^{kl} \left( \partial_i g_{lj} + \partial_j g_{li} - \partial_l g_{ij}  \right) $$ 


\begin{tcolorbox}[title=]
	$\Gamma_ij^{k}$ \textbf{n'est pas un tenseur}\\
	$\nalba_i A^j$ en est un!
\end{tcolorbox}

Il est toujours possible de choisir un référentiel tel que  $\Gamma_{ij}^{k}= 0\; \forall i,j,k$!  
\end{tcolorbox}


\begin{tcolorbox}[title=Théorème du quotient]
si 	$B^{ij}A_j $ est un vecteur $\forall A_j$ qui est un vecteur alors $B^{ij}$ est un tenseur.   
		 
\end{tcolorbox}

\section*{Équation géodésique}
$$\ddot x^{i} + \Gamma_{jk}^{i}\dot x^{j}\dot x^{k} =0$$ 

Cette équation est équivalente à 

$$\dot u^{i}\Gamma_{jk}^{i}u^{j}u^{k}$$ 

$$Du^{i}= \dd u^{i}+ \Gamma_{jk}^{i}u^{i}\dd x^{k}= 0$$ 

$$Du_i=  \dd u_i- \Gamma_{ij}^{k}u_k\dd x^{j}= 0$$ 

On divise par $\dd \lambda$ 
$$\implies \dot u_i - \Gamma_{ij}^{k} u_k u^{i}= 0$$ 
$$= \dot u_i - \Gamma_{kij} u^{k}u^j$$ 
$$=\dot u_i - \frac{1}{2} \left( \partial_i g_{ki} + \partial_j g_{ki} -\partial_k g_{ij}  \right)u^{k}u^{i} $$ 

\begin{tcolorbox}[title=]
	$$A_{kj} = -A_{jk} \qquad S^{kj}=-S^{jk}$$ 
	$$A_{ki} S^{kj}= A_{jk} S^{jk} = -A_{kj} S_{kj} =0$$ 
\end{tcolorbox}

comme les deux derniers termes forment ensemble un tenseur anitsymétrique et qu'ils mutilplient un tenseur symétrique la contribution de ces termes s'annulent 


$$0 = \dot u_i \del_i g_{kj} u^k u^j$$ 

Si $g_kj$ ne dépend pas de $x^{i}$ alors $u_i = \text{cst} $    



\begin{figure}[ht]
    \centering
    \incfig{géodésique-2}
    \caption{Géodésique 2}
    \label{fig:géodésique-2}
\end{figure}


$$S_{AB} = \int_{A}^{B}\dd \lambda L(x, \dot x) = \int_{A}^{B}\dd \lambda \underbrace{\sqrt{g_{ij} (x)\dot x^{i}x^{j}}}_{\abs{\vb{u}} }  = \int_{A}^{B} \sqrt{g_{ij} \dd x^{j} \dd x^{j}}= \int_{A}^{B}\dd s$$ 


$$x^{i}(\lambda) \to x^{i}(\lambda) + \delta x^{i}(\lambda) $$ 

$$\delta S_{AB} = \int_{A}^{B}\dd \lambda \frac{1}{2L} \delta(g_{ij} \dot x^{i}\dot x^{j}) = \int_{A}^{B}\dd \lambda \left\{ \frac{1}{2} \partial_k g_ij \dot x^{i}\dot^{j}\delta x^{k} + g_{ij} \dot x^{i} \delta \dot x^{i}\right\} $$ 

\begin{tcolorbox}[title=]
	$$g_{ij} \dot x^{i}\dv{{}}{\lambda} x^{i}=  \dv{{}}{\lambda} \left( g_ij \dot x^{i}\delta x^j \right) - \dv{{}}{\lambda} \qty(g_{ik} \dot x^{k}) \delta x^l$$ 
\end{tcolorbox}

$$\dotsb$$ 


$$0 = \frac{1}{2} \partial_k g_{ij} u^{i}u^{j}-\left( \frac{1}{2} \partial_j g_{ik} + \frac{1}{2} \partial_i g_{jk}   \right)u^{j}u^{i} - g_{ki} \dot u^j $$ 

$$=  \underbrace{\frac{1}{2}\left( \partial_k g_{ij} - \partial_j g_{ik} - \partial_i g_{jk}  \right)}_{-\Gamma_{kij} } u^{i}u^{j}- g_{kj} \dot u^{i}= 0$$ 


$$\boxed{\Gamma_{ij}^{k}u^{i}u^{j}+ \dot u^{k} =0}$$ 

\hrule

\section*{Vaisseau en accélération constante}

\boxed{$A$} $v(t), x(t)$ avec $t$ le temps terrestre  

4-accélération $$a^{i}= \dv{u^{i}}{\tau} = \dv{{}}{\tau} (\gamma, \gamma \vb{v}) = \left( \frac{\vb{v} \cdot \vb{a}}{(1-v^2)^2}, \frac{\vb{a}}{1-\vb{v}^2} + \frac{\left( \vb{a} \cdot  \vb{v} \right) \vb{v}}{(1-v^{2})^2}    \right) $$ 


$$a_i a^{i}= -\gamma^{4} \left( \vb{a}^2 + \gamma^{2}\left( \vb{v}\cdot \vb{a} \right)^2 \right) $$ 


\begin{tcolorbox}[title=]
	$$\dv{\gamma}{\tau} = -\frac{1}{2} \frac{1}{(1-v^2)^{3/2}} \left( -2 \vb{v} \cdot \dv{\vb{v}}{\tau}  = \frac{1}{(1-v^2)^2} \vb{v}\cdot \vb{a}  \right)  $$ 
\end{tcolorbox}

$$=-\gamma^{6}a^2$$ 


$$\gamma^{3}a=g$$ 

$$\frac{1}{\qty(1-v^2)^{3/2}} \dv{v}{t} =g $$ 
$$g \dd t = \frac{\dd{}v}{(1-v^2)^{3/2}} $$ 

Rapidité:
$$\gamma = \cosh \eta$$ 
$$v\gamma = \sinh\eta$$ 

$$\dd v = \frac{1}{\cosh^2\eta} \dd \eta = \frac{1}{\gamma^{2}} \dd \eta $$ 

$$\int g \dd t = \int \gamma \dd \ets = \int \cosh\eta \dd \eta $$ 

$$gt + \cancel{cst} = \sinh\eta$$ 



$$v(t) = \tanh\eta =\dotsb= \frac{gt}{\sqrt{1+(gt)^2}} $$ 

Comment restaurer les \textit{vrai unités}?

$$\frac{gt}{\sqrt{1 + \left( \frac{gt}{c}  \right)^2}} $$ 


\begin{tcolorbox}[title=]
	 $$gt = \sinh\eta \implies g \dd t = \cosh\eta \dd \eta$$
\end{tcolorbox}


$$x(t)= \int v(t) \dd t = \frac{1}{g} \int\tanh\eta\cosh\eta\dd\eta = \frac{1}{g} \int\sinh\eta = \frac{1}{g} \cosh\eta + \cancelto{0}{cst} = \frac{1}{g}(\cosh\eta -1)$$ 

\clearpage 
\boxed{$B$}


\begin{figure}[ht]
    \centering
    \incfig{milles-mots}
    \caption{Milles mots}
    \label{fig:milles-mots}
\end{figure}


\boxed{ $C$  }



\end{document}
