\documentclass{article}    
\usepackage[utf8]{inputenc}    
    
\title{Épisode 4}    
\author{Jean-Baptiste Bertrand}    
\date{\today}    
    
\setlength{\parskip}{1em}    
    
\usepackage{physics}    
\usepackage{graphicx}    
\usepackage{svg}    
\usepackage[utf8]{inputenc}    
\usepackage[T1]{fontenc}    
\usepackage[french]{babel}    
\usepackage{fancyhdr}    
\usepackage[total={19cm, 22cm}]{geometry}    
\usepackage{enumerate}    
\usepackage{enumitem}    
\usepackage{stmaryrd}    
    
%packages pour faire des math    
%\usepackage{cancel} % hum... pas sur que je vais le garder mais rester que des fois c'est quand même sympatique...
\usepackage{amsmath, amsfonts, amsthm, amssymb}    
\usepackage{esint}  

\begin{document}

2022-09-02

\section*{Espace-temps}

\begin{figure}[H!]
    \centering
    \incfig{espace-temps}
    \caption{Espace-temps}
    \label{fig:espace-temps}
\end{figure}


\begin{figure}[H!]
    \centering
    \incfig{minkowski-2d}
    \caption{Minkowski 2D}
    \label{fig:minkowski-2d}
\end{figure}


$$\dd s^{2}= g_{ij} \dd x^{i}\dd x^{j}= \dd \tau^{2}$$ 

\begin{tcolorbox}[title=Temps propre]
	 Temps qui s'écoule dans le référentielle de l'objet $$\implies x(\tau)$$ 
\end{tcolorbox}

Si on connais $x^{i} (t)$, alors que vaut le temps propre?

\begin{align*}
	d\tau &= \sqrt{g_{ij} \dd x^{i}\dd x^{j}}\\
				&= \sqrt{\dd t^{2}- \dd \vb{r}^2 }\\
        &= \dd t \sqrt{1 - \qty(\pdv{\vb{r}}{t})^2}\\
    &= \dd t \sqrt{1 -\vb{v}^2}\\
    &= \frac{\dd t}{\gamma} 
\end{align*} 

\section*{Action}


$$S = -m \int_{A}^{B}\dd \tau = -m \int_{A}^{B}\dd t \sqrt{1 -\vb{v}^2}$$ 
$$\approx -m \int_{A}^{B}\dd t \left( 1- \frac{1}{2} \vb{v} \right) $$ 
$$= -m \int_{A}^{B}\dd t \frac{1}{2} m \vb{v}^2$$ 

\underline{Lagrangien:}

$$L = -m \sqrt{1-\vb{v}^2} $$ 

$$\vb{p} = \pdv{L}{\vb{v}} = \frac{m\vb{v}}{\sqrt{1-\vb{v}^2}} $$ 

\underline{Hamiltonien} 

$$\vb{p}\cdot \vb{v}=L = H$$ 


\underline{Hamiltonien} 


\begin{align*}
    H &= \vb{p}\cdot \vb{v}-L = \vb{p}\cdot \vb{v}+m\sqrt{1-\vb{v}^2}\\ &= \frac{m}{\sqrt{1-\vb{v}^2}} \left\{ \vb{v}^2 +1 - \vb{v}^2 \right\}\\ &= \frac{m}{\sqrt{1-\vb{v}^2}}\\ &= \sqrt{\vb{p}^2 +m^2} 
\end{align*}

$$H^{2}= \frac{m^{2}}{1-\vb{v}^2} \quad \vb{p}^2 =\frac{m^2v^{2}}{1-\vb{v}^2}  $$ 


\section*{Éléctromagnétisme}

\underline{4-vecteur potentiel:}

$$A^i = (\Phi, \vb{A}), \quad A_i = (\Phi, -\vb{A})$$ 


$$S = \underbrace{S_0 }_{ -m\int \dd \tau  } - e \int_{A}^{B} \underbrace{A_i \dd x^i}_{\text{invarient} } (\vb{E}^2 - \vb{B}^2)  $$ 


\begin{tcolorbox}[title=Tensuer de Faraday  ]
    $$F_{ij} = \partail_i A_j - \partial_j A_i $$ 
    $$F^i_i =0 \quad F_{ij} F^{ij} \text{: invarient} $$ 
    $$\vb{E}= - \vb{v}A_0 - \pdv{\vb{A}}{t} \quad \vb{B} = \grad \times \vb{A}$$ 
\end{tcolorbox}


$\to$ principe de moindre action: 

$$\boxed{ m\ddot x^i  = e F_j^{i} \dot x^{2}}$$ 

$$\boxed{m \dot u^{i}=eF_j^{i}u^{j}}$$ 


\section*{Chapitre 2: géométrie différentielle}

\begin{tcolorbox}[title=Théorème du plongement]
    Nash
    
    Ne vaut que pour des espace euclidien (pas pour l'espace-temps donc) mais le théorème se généralise

    On définit un point $\vb{x} \in \mathbb{R}^{3} $ comme un point de la surface. Où $\mathbb{R}^{3} $ est \textit{l'espace hôte}  

    $$\vb{X} (x^{i})  \quad i \in \{1,\dotsb, d\} $$ 

    Par exemple, la sphère: $$\vb{X}= (\sin\theta\cos\phi, \sin\theta\sin\phi, \cos\theta)$$
    $$x^{1}=\theta \quad x^{2}=\phi$$ 

    {\bf{Il n'existe pas de vecteur position}}
     
    Il est impossible en général de représenter un variété différentiel avec une seule \texit{carte}

\end{tcolorbox}

\begin{figure}[ht]
    \centering
    \incfig{atlas}
    \caption{Atlas}
    \label{fig:atlas}
\end{figure}

\begin{tcolorbox}[title=Espace tangeant]
    $$\vb{e}_i = \pdv{\vb{X}}{x^{i}} = \pdv{\vb{X}}{x^{\prime{}j}} \pdv{x^{\prime{}j}}{x^i} =\vb{e}'_j \underbrace{\pdv{x^{\prime{}j}}{x^i}}_{\Lambda^j}   $$ 
\end{tcolorbox}

\begin{tcolorbox}[title=Truc mémotechnique]
     Quand on divise par un indice inferieur il deviens suppérieur et inversement
\end{tcolorbox}

tenseur métrique

$$g_{ij} = \vb{e}_i \cdot \vb{e}_j = \pdv{\vb{X}}{x^i} \cdot \pdv{\vb{X}}{x^j} = \partial_i \vb{X} \cdot \partial_j \vb{X}$$ 


$$\dd s^{2}= \dd \vb{X}\dd \vb{X} = \left( \partial_i \vb{X} \dd x^i \right) \cdot \left( \partial_j \vb{X} \dd x^{j} \right) = g_{ij} \dd x^{i}\dd x_j = g_{ij} (x) \dd x^{1}\dd x^2 $$ 

fonction: $\phi(x)$ 

$$\partial_{i\phi} = \pdv{\phi}{x^i} $$ 

$$\partial'_i = \pdv{\phi}{x^{\prime{}i}} = \dotsb$$ 

$$\partial_i \phi = \partial'_j \phi \pdv{x^{\prime{}j}}{x^i} $$ 
(vecteur covarient)



\end{document}
