\documentclass{article}
\usepackage[utf8]{inputenc}

\title{Épisode 3}
\author{Jean-Baptiste Bertrand}
\date{\today}

\setlength{\parskip}{1em}

\usepackage{physics}
\usepackage{graphicx}
\usepackage{svg}
\usepackage[utf8]{inputenc}
\usepackage[T1]{fontenc}
\usepackage[french]{babel}
\usepackage{fancyhdr}
\usepackage[total={19cm, 22cm}]{geometry}
\usepackage{enumerate}
\usepackage{enumitem}
\usepackage{stmaryrd}

%packages pour faire des math
%\usepackage{cancel} % hum... pas sur que je vais le garder mais rester que des fois c'est quand même sympatique...
\usepackage{amsmath, amsfonts, amsthm, amssymb}
\usepackage{esint}

\begin{document}

\maketitle

\underline{Projection sur n,l,m}

$$\ket\psi =\mathbb{1}\ket\psi$$

$$ =\sum_{n,l,m,\epsilon} \ket{n,l,m,\epsilon}\underbrace{\braket*{n,l,m,\epsilon}{\psi}}_{c_{n,l,m,\epsilon}}$$

$$\braket{\vec r}{\psi} = \sum_{n,l,m,\epsilon} \underbrace{\bra{\vec r}\ket{n,l,m,\epsilon}}_{R_{n,l}(r)Y_l^m(\theta,\varphi)}\ket{\epsilon}c_{n,l,m,\epsilon}$$

$$= [\psi] = \sum_{n,l,m} \begin{pmatrix}c_{n,l,m,+}R_{n,l}(r)Y_l^m(\theta,\varphi)\\c_{n,l,m,-}R_{n,l}(r)Y_l^m(\theta,\varphi)\end{pmatrix}$$

$$[\psi] = \sum_{l,m} \begin{pmatrix}a_{n,l,+}(r)Y_l^m(\theta,\varphi)\\a_{n,l,-}(r)Y_l^m(\theta,\varphi)\end{pmatrix}$$

$$\dd \mathcal{P}_\epsilon(l,m) = ? $$

$$\boxed{\vb L^2Y_l^m=l(l+1)\hbar^2Y_l^m}$$
$$\boxed{L_zY_l^m=m\hbar Y_l^m}$$


$$\dd \mathcal{P}_\epsilon(l,m) = \abs{\int Y_l^{m*}\sum_{l',m'}a_{l',m',\epsilon}(r)Y^{m*}_{l'}\dd \Omega}^2 r^2 \dd r$$


$$\boxed{\int Y_l^{m*}Y_{l'}^{m'} \dd \Omega = \delta_{ll'}\delta_{mm'}}$$

$$\mathcal{P}_\epsilon(l,m) = \int r^2 \dd r \abs{a_{l,m,\epsilon}(r)}^2$$

$$\mathcal{P}(l,m) = \sum_\epsilon \mathcal{P}_\epsilon(l,m)$$

$$\mathcal{P}(l)\epsilon_{\abs*{m}\leq l \mathcal{P}(l,m)} = \sum_{\abs{m}\leq l}\int r^2\qty(\abs*{a_{l,m,+}(r)}^2 + \abs*{a_{l,m,-}(r)}^2)\dd r$$


\underline{Composition du moment cinétique}

Généralisation et mise en contexte

$\vec P_i$ n'est pas conservé s'il y a de l'interaction. Ce n'est donc pas un bon nombre quantique.

Si le système satisfait: $$\sum_i \vb P_i = \vb P_T$$

Alors $$\dv{\vb P_T}{t}= 0$$. Ce qui signigie que $\vb P_T$ est un bon nombre quantique


$$W_{so} \approx \lambda(L_z S_z + \underbrace{L_xS_x + L_yS_y}_{\frac12L_+S_- +\frac{1}{2}L_-S_+})$$

$L_z(m)$ et $S_z(\epsilon)$ ne sont plus des bons nombre quantiques. Le moment cinétique peut être passsé de l'un à l'autre. Cepandant le moment cinétique total, comme toujours, est conservé. On utilise donc le spin total comme nouveau nombre quantique $$\boxed{\vb J = \vb L + \vb S}$$

\begin{align*}
	\text{ECOC: }\;
	&\vb L^2, L_z, S_s \qquad\to\qquad \vb L^2, \vb J^2, J_z\\
	&\qty{\ket{l,m,\epsilon}} \qquad\to\qquad \qty{\ket{l,J,m}}
\end{align*}

Un exemple simple où cette base pourrait être utilisé est la composition de deux spin.

\end{document}