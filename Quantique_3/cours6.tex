\documentclass{article}    
\usepackage[utf8]{inputenc}    
    
\title{Épisode 4}    
\author{Jean-Baptiste Bertrand}    
\date{\today}    
    
\setlength{\parskip}{1em}    
    
\usepackage{physics}    
\usepackage{graphicx}    
\usepackage{svg}    
\usepackage[utf8]{inputenc}    
\usepackage[T1]{fontenc}    
\usepackage[french]{babel}    
\usepackage{fancyhdr}    
\usepackage[total={19cm, 22cm}]{geometry}    
\usepackage{enumerate}    
\usepackage{enumitem}    
\usepackage{stmaryrd}    
\usepackage{mathtools,slashed}
%\usepackage{mathtools}
\usepackage{cancel}
    
\usepackage{pdfpages}
%packages pour faire des math    
%\usepackage{cancel} % hum... pas sur que je vais le garder mais rester que des fois c'est quand même sympatique...
\usepackage{amsmath, amsfonts, amsthm, amssymb}    
\usepackage{esint}  
\usepackage{dsfont}

\usepackage{import}
\usepackage{pdfpages}
\usepackage{transparent}
\usepackage{xcolor}
\usepackage{tcolorbox}

\usepackage{mathrsfs}
\usepackage{tensor}

\usepackage{tikz}
\usetikzlibrary{quantikz}
\usepackage{ upgreek }

\newcommand{\incfig}[2][1]{%
    \def\svgwidth{#1\columnwidth}
    \import{./figures/}{#2.pdf_tex}
}

\newcommand{\cols}[1]{
\begin{pmatrix}
	#1
\end{pmatrix}
}

\newcommand{\avg}[1]{\left\langle #1 \right\rangle}
\newcommand{\lambdabar}{{\mkern0.75mu\mathchar '26\mkern -9.75mu\lambda}}

\pdfsuppresswarningpagegroup=1


\begin{document}
\section{Théroème de composition du moment cinétique}
Si $\vb j_1$ et $\bv J_2$ deux moments cinétiques alors les valeurs propres à $J^2$ et $J_z$ sibt telles que $$J = J_1 + J_2, J_1+J_2-1, \dotsb, \abs{J_1 - J_2}$$ 
$$-J \leq M \leq J$$ 

vecteur propres: 

$$\ket{J,M}=\sum_{m_1}\sum_{m_2}\ketbra{J_1, J_2; M_1, M_2}\ket{J,M}}$$ 
\section{Exemple, compostition d'un moment orbitale etd'un spin}
$$\vb J_1 = \vb L; \quad \bj J_2 = \vb S$$ 
$$L^2\ket{l,m_2}\otimes \ket{\frac12,m_2} = \hbar^2l(l+1)\ket{l,s,m_1,m_2}$$
$$S^2\ket{l, s, m_1, m_2} = \frac{\hbar^2}{2}\frac.2\dotsb \ket{}$$ 
$$\dotsb$$

$$\vb J = \vb L + \vb S; \quad J_{\rm z}= L_{\rm z}+ S_{\rm z}$$ 
$$J = l + \frac{1}{2} \text{ et } J = l -\frac{1}{2}$$ 
\begin{table}[htpb]
	\centering
	\caption{tableau des vecteur propre}
	\label{tab:val_propre}

	\begin{tabular}{|c|c|c|c}
		\hline
		$m\backslash J$ & $l + \frac12$ & $l - \frac{1}{2} &\dotsb\\ \hline
		$l+\frac{1}{2}$ & $\ket{l+\frac12, l+\frac{1}{2}}$ & &\dotsb \\\hline
		$l-\frac{1}{2}$ & & &\dotsb
	\end{tabular}
\end{table}
\section{Opérateur scalaires et vectoriels (théorème de Wigner-Eckart)}

\textbf{Opérateur scalaire} Si $A$ est scalaire $\implies [A,\vec J] = 0$

Ex $J^2$ 
$$[J^2, \vec J] = [J\cdot J, \vec J] = \vec J [\vec J,\vec J] + [\vec J, \vec J] \vec J =0$$ 

Si $A$ est scalaire $[A, J^2] = \vec J[A,\vec J] + \vec J[\vec J, A]=0$ 


\textbf{Opérateur vectoriel}

$\vec V$ est vectoriel $$[J_i, V_j]  = i\hbar\epsilon_{\rm ijk} J_iV_j$$ 

\end{document}
