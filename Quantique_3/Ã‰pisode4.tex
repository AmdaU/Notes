\documentclass{article}
\usepackage[utf8]{inputenc}

\title{Épisode 4}
\author{Jean-Baptiste Bertrand}
\date{\today}

\setlength{\parskip}{1em}

\usepackage{physics}
\usepackage{graphicx}
\usepackage{svg}
\usepackage[utf8]{inputenc}
\usepackage[T1]{fontenc}
\usepackage[french]{babel}
\usepackage{fancyhdr}
\usepackage[total={19cm, 22cm}]{geometry}
\usepackage{enumerate}
\usepackage{enumitem}
\usepackage{stmaryrd}

%packages pour faire des math
%\usepackage{cancel} % hum... pas sur que je vais le garder mais rester que des fois c'est quand même sympatique...
\usepackage{amsmath, amsfonts, amsthm, amssymb}
\usepackage{esint}

\begin{document}

\maketitle

\section*{Compostition du moment cinétique}

Exemple simple: composition de spins $\frac12$

$$\text{E.C.O.C: } \vb S_1^2\; \vb S_2^2\; S_{1z}\; S_{2z}$$

$$\ket{\frac{1}{2}, \epsilon_1} \otimes \ket{\frac12, \epsilon_2} = \ket{\frac12,\frac12;\epsilon_1,\epsilon_2} \to \ket{\epsilon_1, \epsilon_2} \quad \text{Car les spins sont toujours 1/2 dans notre cas}$$

$$\vb S_1^2 \ket{\epsilon_1,\epsilon_2} = \frac12\qty(\frac12+1)\hbar^2\ket{\epsilon_1,\epsilon_2}$$
$$\vb S_{1z} \ket{\epsilon_1,\epsilon_2} = \frac\epsilon2\hbar\ket{\epsilon_1,\epsilon_2}$$


$$\text{nouvel E.C.O.C: } \vb S_1^2\;, \vb S_2^2\;, \vb S^2\;, S_z$$




On peut vérifier qu'il commutent tous entre eux mais on le feras pas.

On peut également vérifier la complétion. On va le vérifier plus tard.

Cela induit nécessairement une nouvelle base

$$\ket{\frac12,\frac12, S, M} \to \ket{S,M}$$

$$\ket{S,M} =\mathbb{1}\ket{S,M}$$
$$\ket{SM} = \sum_{\epsilon_1, \epsilon_2} \ketbra{\epsilon_1,\epsilon_2}\ket{S,M}$$

Les coefficient de cette expression sont appelées Clebsch-Gordan

$$\vb S^2\ket{S,M} = S(S+1)\hbar^2\ket{S,M}$$
$$S_z \ket{S,M}= M\hbar \ket{SM}$$


$$\boxed{S \geq M \geq -S}$$

$$\text{Contrainte } S_z\ket{\epsilon_1, \epsilon_2} = (S_{1z} + S_{2z})\ket*{\epsilon_1, \epsilon_2} = \underbrace{\qty(\frac{\epsilon_1}2 + \frac{\epsilon_2}{2}) \hbar}_{M\hbar} \ket{\epsilon_1, \epsilon_2}$$

$$\implies M_{\rm max} = \frac 12 + \frac 12 = 1$$

\begin{tabular}[]{|c|c|c|}
	\hline
	$m\backslash S$ & 1 (triplet) &0 (singulet) \\ \hline
	1   & $\ket{1,1} = \ket{+,+}$ &\\ \hline
	0 	& $\ket{1,0} = \frac{1}{\sqrt{2}}\qty[\ket{+-}+\ket{-+}]$ &$\ket{0,0} = \ket{+,-}-\ket{-+}$\\\hline
	-1	& $\ket*{1, -1} = \ket{--}$ & \\\hline
\end{tabular}

Pour savoir comment les nouveau opérateur agissent sur les vecteur, on expirme les nouveaux vecteur et opérateurs en fonctions des anciens

$$\vb S^2\ket{1,1} = \qty(\vb S_1 + \vb S_2)^2\ket{+,+} = \qty(\vb S_1^2 + \vb S_2^2 + 2\vb S_1 \vb S_2)\ket{+,+}
 = \qty(\vb S_1^2 + \vb S_2^2 + 2(\underbrace{S_{1x}S_{2x} + S_{1y}S_{2y}}_{S_{1+}S_{2+} +S_{1-}S_{2+} } + S_{1z}S_{2z}))\ket{+,+}$$

On fait le produit scalaire et on retrouver $S_\pm$


$\ket{0,0} = \alpha\ket{+,-}+\beta\ket{-+}$

On a les contraintes $\alpha^2 + \beta^2 = 1$ et $\frac{\alpha}{\sqrt{2}} + \frac{\beta}{\sqrt{2}} = 0$ par orthogonalité.

\section*{Généralisation à des spins plus grands: spins $J_1$ et $J_2$ fixées}

L'idée reste la même. On part d'un acien ECOC

$$\text{ECOC: } \vb J_1^2,\; J_2^2,\; J_{1z},\; J_{2z}$$

Base $\ket{J_1, m_1} \otimes \ket{J_2, m_2} \to \ket{J_1, J_2; m_1, m_2}$

$$\vb J_1^2 \ket{J_1J_2m_1m_2} = J_{1}(J_{1}+1)\hbar^2\ket{J_1J_2m_1m_2}$$

$$\vb J_{1z}\ket{J_1J_2m_1m_2} = m_1\hbar\ket{J_1J_2m_1m_2}$$


$$\text{nouvel ECOC } \vb J_1^2, \; \vb J_2^2, \; \vb J^2, J_z$$

$$\boxed{-J \leq M \leq J}$$

On fait le même changement de base avec les coefficients de Clebsch-Gordan. Au lieu d'une somme sur epsilon on doit maintenant sommer sur tout les $m_1$ et $m_2$

On trouve, de manière similaire a précédement 
$$\boxed{M = m_1+m_2}$$

Encore une fois, on veut maintenant trouver les nouveau vecteurs prorpes.

\begin{tabular}{|c|c|c|}
	\hline
	$M\backslash J$ & $J_1+J_2$ & $J_1 + J_s -1$\\\hline
	$M_{\rm max} = J_1 + J_2$ & $\ket{J_1+J_2, J_1+J_2}$ &\\\hline
	$J_1 + J_2 -1$& $\ket{J_1+J_2, J_1 + J_2-1}$ & $\ket{J_1+J_2-1, J_1+J_2-1}$\\\hline
	$\dotsb$ & $\dotsb$&$\dotsb$\\\hline
	$-J_1-J_2$ & $\ket{-J_1-J_2, -J_1-J_2}$ &
\end{tabular}

$$\ket{J_1+J_2, J_1+J_2} = \ket{J_1, J_2; J_1, J_2}$$

$$\underbrace{J_-}_{J_{1-}+J_{2-}}\underbrace{\ket{J_1+J_2, J_1+J_2}}_{J_1,J_2; J_1, J_2} = \hbar\underbrace{\sqrt{(J_1+J_2)(J_1+J_2+1)-(J_1+J_2)(J_1+J_2-1)}}_{2(J_1+J_2)}\ket{J_1, J_2; J_1, J_2} $$

$$J_{1-}+J_{2-}\text{S'applique et donne aussi des longues racines, je suis pas trop sur de la conclusion... On verifié que ça marche je crois}$$


\end{document}