\documentclass{article}    
\usepackage[utf8]{inputenc}    
    
\title{Épisode 4}    
\author{Jean-Baptiste Bertrand}    
\date{\today}    
    
\setlength{\parskip}{1em}    
    
\usepackage{physics}    
\usepackage{graphicx}    
\usepackage{svg}    
\usepackage[utf8]{inputenc}    
\usepackage[T1]{fontenc}    
\usepackage[french]{babel}    
\usepackage{fancyhdr}    
\usepackage[total={19cm, 22cm}]{geometry}    
\usepackage{enumerate}    
\usepackage{enumitem}    
\usepackage{stmaryrd}    
\usepackage{mathtools,slashed}
%\usepackage{mathtools}
\usepackage{cancel}
    
\usepackage{pdfpages}
%packages pour faire des math    
%\usepackage{cancel} % hum... pas sur que je vais le garder mais rester que des fois c'est quand même sympatique...
\usepackage{amsmath, amsfonts, amsthm, amssymb}    
\usepackage{esint}  
\usepackage{dsfont}

\usepackage{import}
\usepackage{pdfpages}
\usepackage{transparent}
\usepackage{xcolor}
\usepackage{tcolorbox}

\usepackage{mathrsfs}
\usepackage{tensor}

\usepackage{tikz}
\usetikzlibrary{quantikz}
\usepackage{ upgreek }

\newcommand{\incfig}[2][1]{%
    \def\svgwidth{#1\columnwidth}
    \import{./figures/}{#2.pdf_tex}
}

\newcommand{\cols}[1]{
\begin{pmatrix}
	#1
\end{pmatrix}
}

\newcommand{\avg}[1]{\left\langle #1 \right\rangle}
\newcommand{\lambdabar}{{\mkern0.75mu\mathchar '26\mkern -9.75mu\lambda}}

\pdfsuppresswarningpagegroup=1

\begin{document}

\section*{ \underline{Méthode des variations} }

$$-\dv[2]{\psi}{x} -(\lambda - \abs{x}) \psi = 0  $$ 

$$\psi_{E}(x) =\begin{cases}
	c(a-\abs{x}) & \abs{x}<a\\
	0 & \text{sinon} 
\end{cases}$$ 

$$\int \abs{\psi}^{2} \dd^d n=1$$ 

$$\psi \sim L^{-1/2}$$ 

$$c \sim L^{-1/2} \implies \lambda \propto \alpha^{-3/2}$$ 

$$H = - \dv[2]{{}}{x} + \abs{x} $$ 

$$E(\alpha) = C_1 \alpha^{-2} + C_2 \alpha$$ 

$$\dv{\abs{x}}{x} = \Theta(x) - \Theta(-x) $$ 

$$\dv[2]{\abs{x}}{x} = \delta (x) + \delta(-x) = 2\delta(x) $$ 

Il ne reste qu'a trouver les coefficients $C_1 , C_2$ 

On fait l'intégrale $$\frac{1}{\ket{\psi}\bra{\psi}} t_{-a}-aa^{a} -\psi(s) [2][\psipsx}{x} + \abs{x} \psi^2 \d  x = E(\alpha) $$

On peut ensuite minimiser le résultat par rapport à $\alpha$ 

\section*{ \underline{Particules identiques} }

$$E_n = \hbar \omega \left( n + \frac{1}{2}  \right) $$ 

$$\ket{\varphi_n^{(1)} \epsilon^{(1)}}\ket{\varphi_m^{(2)} \epsilon^{(2)}}$$ 

État fondamental:

$$\ket{\varphi_0^{(1)}, \epsilon^{(1)}, \varphi_0^{(2)}\bar\epsilon^{(2)}}$$ 

$$\ket{\psi} = \frac{1}{\sqrt{2}} \left[ \ket{\varphi_0^{(1)}, \epsilon^{(1)}, \varphi_0^{(2)}, \bar\epsilon^{(2)}} - \ket{\varphi_0^{(1)}\bar\epsilon^{(1)}, \varphi_0^{(2)}\epsilon^{(2)}} \right] $$ 


$$\bra{\psi}H\ket{\psi} = 2\hbar\omega \frac{1}{2} $$ 


\section*{ \underline{Diffusion de potentiel:} }
Approximation de born et potentiel central!

$V (\vb{r}) = v(r)$ 
$$\text{Q:} f_k (\theta, \varphi) = ? $$ 

$$f_k (\theta, \varphi) = \frac{\mu}{2\pi\hbar^2} \int V(\vb{r}) e^{-i \vb{q} \cdot \vb{r}} \dd^3 r$$ 

$$\vb{q} = \vb{k}_f -\vb{k}_i$$ 

$$\sigma(\theta, \varphi) = \abs{f_k (\theta, \varphi)}^{2}$$ 

On a donc,

$$\begin{aligned}
	f_k (\theta, \varphi) &= \frac{\mu}{2\pi\hbar^2} \int \sin\tilde\theta \dd \tilde \theta \dd \tilde\varphi r^2 \dd r V(r) e^{-iqr \cos \tilde\theta}\\ &= \frac{\mu}{\hbar^2} \int_0^\infty\int_{-1}^{1}r2V(r) e^{-iqr} \dd u\dd r = \frac{\mu}{\hbar^2} \int_0^{\infty}\dd r r^2 V(r) \left[ \frac{e^{-iqr} - e^{iqr}}{-2iqur}  \right] 2  \\ &= \frac{2\mu}{\hbar^2q} \int_0^{\infty}\dd r r V(r) \sin(qr) = \\ &= \frac{2\mu}{\hbar^2\sin(\theta/2)} \int_0^{\infty}\dd r r V(r) \sint(kr\sin(\theta/2) 
\end{aligned}$$ 

\begin{tcolorbox}[title=Intégrale commune]
	 	$$I_1 = \int_0^{\infty}\dd r r e^{-\alpha r^2 + \beta r} $$
On peut obtenir le résultat de cette intégrale à partir d'une intégrale connue:
		$$I_0 =\int_0^{\infty}\dd r e^{-\alpha r^2 + \beta r} = \frac{1}{2} \sqrt{\frac{\pi}{\alpha} }e^{\frac{\beta^2}{4\alpha} }$$ 

$$I_1 = \dv{I_0}{\beta} = \frac{\beta}{4\alpha} = \sqrt{\frac{\pi}{\alpha} } e^{\beta^2/4\alpha} $$ 

\end{tcolorbox}

$$f_k (\theta) = \frac{\mu}{\hbar^2 k \sin(\theta/2) 2i}\int_0^{\infty} \dd r r V_0 e^{-r^2/a^2} \left[ e^{iqr} - e^{-iqr} \right] $$ 

$$\dotsb$$ 








\end{document}
