\documentclass{article}    
\usepackage[utf8]{inputenc}    
    
\title{Épisode 4}    
\author{Jean-Baptiste Bertrand}    
\date{\today}    
    
\setlength{\parskip}{1em}    
    
\usepackage{physics}    
\usepackage{graphicx}    
\usepackage{svg}    
\usepackage[utf8]{inputenc}    
\usepackage[T1]{fontenc}    
\usepackage[french]{babel}    
\usepackage{fancyhdr}    
\usepackage[total={19cm, 22cm}]{geometry}    
\usepackage{enumerate}    
\usepackage{enumitem}    
\usepackage{stmaryrd}    
\usepackage{mathtools,slashed}
%\usepackage{mathtools}
\usepackage{cancel}
    
\usepackage{pdfpages}
%packages pour faire des math    
%\usepackage{cancel} % hum... pas sur que je vais le garder mais rester que des fois c'est quand même sympatique...
\usepackage{amsmath, amsfonts, amsthm, amssymb}    
\usepackage{esint}  
\usepackage{dsfont}

\usepackage{import}
\usepackage{pdfpages}
\usepackage{transparent}
\usepackage{xcolor}
\usepackage{tcolorbox}

\usepackage{mathrsfs}
\usepackage{tensor}

\usepackage{tikz}
\usetikzlibrary{quantikz}
\usepackage{ upgreek }

\newcommand{\incfig}[2][1]{%
    \def\svgwidth{#1\columnwidth}
    \import{./figures/}{#2.pdf_tex}
}

\newcommand{\cols}[1]{
\begin{pmatrix}
	#1
\end{pmatrix}
}

\newcommand{\avg}[1]{\left\langle #1 \right\rangle}
\newcommand{\lambdabar}{{\mkern0.75mu\mathchar '26\mkern -9.75mu\lambda}}

\pdfsuppresswarningpagegroup=1


\begin{document}

\section*{`Opérateur vectoriles}

$\vec v$ est vectoriel si $[v_{\rm, i}, V_j] = i\hbar\epsilon_{\rm ijk}$ 

sous-espace: $\mathcal E (k,j) = \{\ket{k,j,m}, m = -j,\dotsb,j\}$ 


$$P_\mathcal{E} = \sum_{-j}^j \ketbra{k,j,m}$$ 

$$\boxed{P_\mathcal E \vec v P_\mathcal E = \alpha P_\mathcal E \vec J P_\mathcal E}$$ 
	
On considère $P_{\rm \mathcal{E} }^2 \vec J \cdot \vec v$

$$ = P_{\rm \mathcal{E} } \vec J P_{\rm \mathcal{E} } \vec v P_{\rm \mathcal{E} } = \alpha P_{\rm \mathcal{E} } \vec J\cdot \vec J P_{\rm \mathcal{E} } \equiv \alpha$$

$$\implies <\vec J \cdot \vec v>_{\mathcal{E}(k,j)} = \alpha j(j+1)\hbar^2$$

\underline{Application} Multiplet des spins et facteur de ???

Atomes à plusieurs éléctrons

$$\vec L = \sum_{i=1}^z \vec L_i\quad \vec S = \sum_{i=0}^{z}\vec S_i$$ 

$$\vec J = \vec L + \vec S$$ 

$$\mathcal{E}(k,j) \to \mathcal{E} (E_0, L, S, J) \to \{\ket{E_0, L, S, J, M}\quad J \geq M \geq -J\}$$ 

champ mangétique

$$H = H_0 - \gamma \sum_{i=1}^z\qty(\vec L_{\rm i}+ g\vec S_i) \cdot \vec B$$ 

dans $\mathcal{E} (E_0, L, S, J) :P_{\rm \mathcal{E} }\qty[-\gamma\qty(\vec L +g\vec S)]P_{\rm \mathcal{E}} = -\gamma\alpha_L\vec J-\gamma g \alpha_s\vec J$ 

On remplace $\vec L$ et $\vec S$ par $\alpha \vec J$ dans le Hamiltonien  

On réécrit les $\alpha$s en fonction de produit scalaires.

Les produits scalairs impliquent de calculer:

$$<\vec L ^2>_\epsilon = L(L+1)\hbar^2 \quad <\vec L \cdot \vec S>_{\epsilon_0} = ?$$

Si $\vec L + \vec S = \vec J \implies \vec J ^2 = \vec L^2 + \vec S^2 + 2\vec S \cdot \vec L \implies \vec L \cdot \vec S = \frac{1}{2} \qty(J^2 - L^2 - s^2)$ 


On a finalement que $$H = H_0 - \gamma g_{\rm L}\vec J \cdot \vec B \quad \text{ dans }\mathcal{E} (E_0, L, S,J)$$  

Si $\vb B$ est orienté en $z$ on trouve

$$H = H_0 -\gamma g_{\rm L}J_{\rm z}B \implies H\ket{E_0, L, S, J, M} = \qty(H_0 - \gamma g_{\rm L}M\hbar B)\ket{E_0, L, S, J, M}$$  


\section*{Théorie des parturbation}

En général,

$$H \ket{\psi}= E\ket{\psi}$$ 

n'est pas soluble exactement.

On prend $$H = \underbrace{H_{0}}_{\text{soluble}}+\underbrace{W}_{\ll H}$$ 

$$H_{0} \ket{\varphi_n} = E_n^0\ket{\varphi_n} \quad \bra{\varphi_n}\ket{\varphi_n'} = \delta_{nn'}$$ 

on pose $w = \lambda \bar w\quad \lambda \ll 1$

On postule $$E = E_{n}^0 +\lambda E^{(1)} + \lambda^2 E^{(2)} + \dotsb$$ 
$$\ket{\psi} = \ket{\varphi_n} + \lambda \ket{\varphi^{(1)}} +\lambda^2 \ket{\varphi^{(2)}}+ \dotsb$$ 

Choix: $$\bra{\varphi_n}\ket{\psi} =1 = \underbrace{\bra{\varphi_n}\ket{\varphi_n}}_1 + 0 +0 + \dotsb$$ 


$$\qty(H_{0}+ \lambda\bar W) \qty[\ket{\varphi_n} + \lambda \ket{\varphi^{(1)}} +\lambda^2 \ket{\varphi^{(2)}}+ \dotsb] = \qty(E_{\lambda^0}+ \lambda E ^{(1)} + \lambda^2 E^{(2)} + \dotsb)(\ket{\varphi_n} + \lambda \ket{\varphi^{(1)}} +\lambda^2 \ket{\varphi^{(2)}}+ \dotsb)$$ 


$$O(\lambda^0): \quad H_{0}\ket{\varphi_n} = E_{n}^0\ket{\varphi_n}$$ 
$$O(\lambda^1): \quad H_{0}\ket{\varphi^{(1)}} + \bar W \ket{\varphi_n} = E^0_n \ket{\varphi^{(1)}} + E^{(1)}\ket{\varphi} \implies \dotsb \implies E^{(1)} = \bra{\varphi_n}\bar W \ket{\varphi_n}$$ 


$$O(\lambda^2): \quad H_0\ket{\varphi^{(2)}} + \bar W \ket{\varphi^{(1)}} = E_n^{?}? + ? + ? \implies \dotsb$$ 


Bon, je note pas tout ça, je l'ai déjà fait une fois, pas une deuxième

$$\implies \ket{\varphi^{(1)}} = \sum_{g_n}\sum_{m\neq n} \frac{\ket{\varphi_m}\bra{\varphi_m}\bar W \ket{\varphi_n}}{E^0_n-E^0_m}$$ 



$$\implies E^{(2)} = \sum_{g_n} \sum_{m\neq n} \norm{\frac{\bra{\varphi_n}\bar W \ket{\varphi_n}}{E^0_n =E^0_m}}^2$$ 




\end{document}
