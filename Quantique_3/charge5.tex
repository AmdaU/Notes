\documentclass{article}    
\usepackage[utf8]{inputenc}    
    
\title{Épisode 4}    
\author{Jean-Baptiste Bertrand}    
\date{\today}    
    
\setlength{\parskip}{1em}    
    
\usepackage{physics}    
\usepackage{graphicx}    
\usepackage{svg}    
\usepackage[utf8]{inputenc}    
\usepackage[T1]{fontenc}    
\usepackage[french]{babel}    
\usepackage{fancyhdr}    
\usepackage[total={19cm, 22cm}]{geometry}    
\usepackage{enumerate}    
\usepackage{enumitem}    
\usepackage{stmaryrd}    
    
%packages pour faire des math    
%\usepackage{cancel} % hum... pas sur que je vais le garder mais rester que des fois c'est quand même sympatique...
\usepackage{amsmath, amsfonts, amsthm, amssymb}    
\usepackage{esint}  

\begin{document}
\section*{2 :Atome d'Hydrogène}


Atome d'H dans l'état $1S$ 

$$W(t) = -\alpha q E \hat z \sum_{n=0}^{N} \delta (t-n\tau) e^{t/z_{0?} }$$ 

$$\lambda = \bra{n,l,m} hat z \ket{1,0,0} = \int \dd^3 r R_{m,l} (r) Y_l^{m} (\theta, \varphi) r \cos\theta \varphi_{1S} (r) $$ 

$$\lambda = \int r^3 \dd r R_{n,l} (r) R_{1,0} (r) \underbrace{\int\dd \Omega Y_1^{0} (\theta, \varphi) Y_l^{m}(\theta \varphi)}_{\delta_{m0} \delta_{l1} }  \sqrt{\frac{3}{4\pi} } frac 1 \sqrt{4\pi}$$ 

\underline{Conclusions}:
Transision possibles: $1S \to n p_z$ 

1er Transition $$1S \to 2p_z$$ 


On va se placer dans la limite des temps longs

$$P_{1s\to2p_{z}(t\to \infty} = \frac{1}{\hbar^2} (q\alpha E \lamba)^2 \abs{\int_0^{\infty}\sum_{n=0}^{\infty} \delta (t'-n\tau) e^{t'\left( i\omega_{1S}, 2p_z^{-1/\tau_{0}}  \right) }\dd t'}^{2}$$ 

l'interieur de la valeur absule est une série géométrique qui vaut

$$ \frac{1}{1-e^{\tau \left( i\omega_{1s} 2pz - \text{il a effacé}  \right) }}$$ 


$$$$ 

\end{document}
