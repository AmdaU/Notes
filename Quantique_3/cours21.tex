\documentclass{article}    
\usepackage[utf8]{inputenc}    
    
\title{Épisode 4}    
\author{Jean-Baptiste Bertrand}    
\date{\today}    
    
\setlength{\parskip}{1em}    
    
\usepackage{physics}    
\usepackage{graphicx}    
\usepackage{svg}    
\usepackage[utf8]{inputenc}    
\usepackage[T1]{fontenc}    
\usepackage[french]{babel}    
\usepackage{fancyhdr}    
\usepackage[total={19cm, 22cm}]{geometry}    
\usepackage{enumerate}    
\usepackage{enumitem}    
\usepackage{stmaryrd}    
\usepackage{mathtools,slashed}
%\usepackage{mathtools}
\usepackage{cancel}
    
\usepackage{pdfpages}
%packages pour faire des math    
%\usepackage{cancel} % hum... pas sur que je vais le garder mais rester que des fois c'est quand même sympatique...
\usepackage{amsmath, amsfonts, amsthm, amssymb}    
\usepackage{esint}  
\usepackage{dsfont}

\usepackage{import}
\usepackage{pdfpages}
\usepackage{transparent}
\usepackage{xcolor}
\usepackage{tcolorbox}

\usepackage{mathrsfs}
\usepackage{tensor}

\usepackage{tikz}
\usetikzlibrary{quantikz}
\usepackage{ upgreek }

\newcommand{\incfig}[2][1]{%
    \def\svgwidth{#1\columnwidth}
    \import{./figures/}{#2.pdf_tex}
}

\newcommand{\cols}[1]{
\begin{pmatrix}
	#1
\end{pmatrix}
}

\newcommand{\avg}[1]{\left\langle #1 \right\rangle}
\newcommand{\lambdabar}{{\mkern0.75mu\mathchar '26\mkern -9.75mu\lambda}}

\pdfsuppresswarningpagegroup=1

\begin{document}

\section*{Diffusion élastique}


\begin{figure}[h]
	\centering
	\includegraphics[width=0.8\textwidth]{figures/Diffusion}
	\caption{}
	\label{fig:}
\end{figure}

$$\dd n  =F_i \simga(\theta. \varphi)\dd \Omega$$ 

$$u_{\vb{k}}(r) \sim \\left( e^{ikz} + f(\theta \varphi) \frac{e^{ikr}}{r}  \right) $$ 

$$\simga(\theta, \varphi) = \abs{f_k (\theta \varphi)}^{2}$$ 

\underline{Première approche: Approximation de Born} 

$$\simga( \theta, \varphi) = \frac{\mu^2}{(2\pi)^2\hbar^4} \abs{\dd r^3 e^{-i \vb{q} \cdot \vb{r}} v(\vb{r})}^{2}$$ 


\underline{Deuxième approche: Méthode des déphasages ou d'ondes partielles} 

Pour des $V(\vb{r})$ central $(V(\vb{r}) ⁼ V(r))$ on a que la base $\left\{ H, \vb{L}^2, L_z \right\} $ forme un E.C.O.C   

tandis que $ \left\{ \varphi_{k,l,m}  \right\} $ forme une fase de fonctions propres (ondes partielles)

$$\varphi_{k,l,m} = R_{kl} (r) Y_l^m (\theta, \varphi)$$ 

En pricipe en devrait décomposer sur trois incdie mais on n'elève $k$ parce que c'est élastique et $m$ je sais plus pourquoi.

$$v_k (\vb{r}) = \sum_{\ell} C_{k,\ell} \varphi_{klm} (\vb{r}) $$ 


Loin du potentiel $\varphi_{klm} \to \varphi_{klm}^{(0)}$: Ondes sphérique libres.



Ces derniéres fonctions d'ondes ( $\varphi_{klm}^{(0)} $ sont des fonction propres de $H_0 = \frac{\vb{p}^2}{2m} $  


$$H_0 \ket{\varphi_{klm}^{(0)}} = E_k \ket{\varphi_{klm}^{(0)}}$$ 

$$E_k = \frac{\hbar^2k^2}{2\mu} $$ 

$$\bra{\vb{r}}H_0 \ket{\varphi_{klm}^{(0)}} = E_k \bra{\vb{r}} \ket{\varphi_{klm}^{(0)}}$$ 

$$\boxed{- \frac{\hbar^2}{2\mu} \grad^2 \varphi_{klm}^{(0)} = E_k \varphi_{klm}^{(0)}}$$ 

$$- \frac{\hbar^2}{2\mu} \left[ \frac{1}{r} \pdv[2]{r} r \frac{1}{r^2} \left( \pdv[2]{\theta} + \frac{1}{?} \pdv{\theta} + \frac{1}{\sin^2\theta} \pdv[2]{\varphi}\right)  \right] \varphi_{klm}^{(0)} = \frac{\hbar^2k^2}{2\mu} \varphi_{klm}^{(0)}$$ 

$$\left[ -\frac{\hbar^2}{2\mu} \frac{1}{r} \pdv[2]{r} r + \frac{\hbar^2}{2\mu r^2} l(l+1) \right]R_{kl}^{(0)} (r) = \frac{\hbar^2k^2}{2\mu} R_{kl}^{(0)} $$ 


\begin{figure}[ht]
    \centering
    \incfig{potentiel-effectif}
    \caption{Potentiel effectif}
    \label{fig:potentiel-effectif}
\end{figure}

$$-\frac{\hbar^2}{2\mu} \pdv[2]{u_{kl}^{(0)}}{r} + \frac{\hbar^2}{2\mur^2} l(l+1) u_{kl}^{(0)}  $$ 

Lorsqu'on fait tendre le rayon vers l'infini, on trouve

$$- \frac{\hbar^2}{2\mu} \pdv[2]{u_{kl}^{(0)}}{r} \approx \frac{\hbar^2k^2}{2\mu} u_{kl}^{(0)}    $$ 

Les solution de cette équation différentielle sont bel est bien des ondes sphériques!

Dans le cas $r \to 0$

$$u_{kl}^{(0)} (r) \sim r^s$$ 

$$s(s-1) = l(l+1)$$ 

$$s_1 = l+1 \implies r^{l+1} \qq{régulière a l'origine)$$ 

$$s_1 = -l \implies r^{-l}$$ 

$$r^2 \dv[2]{R_{kl}^{(0)}}{r} + 2r \dv{R_{kl}^{(0)}}{r} + k^2 r^2 R_{kl}^{(0)} - l(l+1) R_{kl}^{(0)} =0  $$ 

C'est une équation à point régulier sigulier! Il est pertinant de faire un changement de variable.

$$\rho^2 \dv{R_{kl} (\rho)}{\rho} + 2\rho \dv{R_{kl}^{(0)}}{\rho} + \rho^2 R_{kl}^{(0)} - l(l+1) R ^{(0)} = 0 $$ 


\end{document}
