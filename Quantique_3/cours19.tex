\documentclass{article}    
\usepackage[utf8]{inputenc}    
    
\title{Épisode 4}    
\author{Jean-Baptiste Bertrand}    
\date{\today}    
    
\setlength{\parskip}{1em}    
    
\usepackage{physics}    
\usepackage{graphicx}    
\usepackage{svg}    
\usepackage[utf8]{inputenc}    
\usepackage[T1]{fontenc}    
\usepackage[french]{babel}    
\usepackage{fancyhdr}    
\usepackage[total={19cm, 22cm}]{geometry}    
\usepackage{enumerate}    
\usepackage{enumitem}    
\usepackage{stmaryrd}    
\usepackage{mathtools,slashed}
%\usepackage{mathtools}
\usepackage{cancel}
    
\usepackage{pdfpages}
%packages pour faire des math    
%\usepackage{cancel} % hum... pas sur que je vais le garder mais rester que des fois c'est quand même sympatique...
\usepackage{amsmath, amsfonts, amsthm, amssymb}    
\usepackage{esint}  
\usepackage{dsfont}

\usepackage{import}
\usepackage{pdfpages}
\usepackage{transparent}
\usepackage{xcolor}
\usepackage{tcolorbox}

\usepackage{mathrsfs}
\usepackage{tensor}

\usepackage{tikz}
\usetikzlibrary{quantikz}
\usepackage{ upgreek }

\newcommand{\incfig}[2][1]{%
    \def\svgwidth{#1\columnwidth}
    \import{./figures/}{#2.pdf_tex}
}

\newcommand{\cols}[1]{
\begin{pmatrix}
	#1
\end{pmatrix}
}

\newcommand{\avg}[1]{\left\langle #1 \right\rangle}
\newcommand{\lambdabar}{{\mkern0.75mu\mathchar '26\mkern -9.75mu\lambda}}

\pdfsuppresswarningpagegroup=1


\begin{document}

\section*{Effet photoéléctrique}

Application de la règle d'or de Fermi!

$$\partial \mathcal{P}_{1S\to \vb{p}_f} (\vb{p}_f, t) = \frac{\pi{}t}{2\hbar} \int_{Df} \rho(E_f ) \norm{\bra{\vb{P}_f}W\ket{\varphi_{1S}}}^2 \partial (E_f - E_i - \hbar \omega) \dd E_f \dd \Omega $$ 

$$W = -\frac{q}{(c)m} e^{i \vb{k} \cdot  \vb{R}} \vb{A}_0 \cdot \vb{p}$$ 

$$kr \sim ka_0 \ll 1 \to \text{Approximation diploaire} $$ 

$$\bra{\vb{p}_f}{W}\ket{\varphi_{1S} } = \int \dd r^3 \frac{e^{i \vb{p}_f \cdot \vb{r}}}{\left( 2\pi\hbar \right)^{3/2} } \bra{\vb{r}} \vb{D} \cdot  \vb{E} \ket{\varphi_{1S} }$$ 

$$\vb{r} \cdot  \vb{E} = xE_X+ yE_{y} + ZE_z = rE_0 \left( \cos\theta\cos\theta_0 + \sin\theta\sin\theta_0\cos(\varphi-\varphi_0) \right) $$ 

\begin{figure}[ht]
    \centering
    \incfig{identification-des-vecteurs-et-angles}
    \caption{Identification des vecteurs et angles}
    \label{fig:identification-des-vecteurs-et-angles}
\end{figure}


$$\bra{\vb{p_{{f}}}W\ket{\varphi_1S} } = -\frac{12qE_0}{\hbar^{2/3}\pi(2a_{0)}^{3/2}} \times \frac{15k_{f}a_{0}^5}{[1+k_f^2a_0^2]^3}\cos\theta_0 $$ 

$$\dotsb$$ 

\begin{equation*}
	\boxed{\frac{\partial\mathcal{P}}{\partial t \partial \Omega_f}  = \frac{256a_0^3}{4\pi\hbar} \underbrace{\frac{E_0^2q^2}{e^2} }_{E_0^2} \left( \frac{\omega}{\omega_0} -1 \right)^{3/2} \left( \frac{\omega}{\omega_0}  \right)^6 \cos^2\theta_0   }
\end{equation*}

\begin{figure}[ht]
    \centering
    \incfig{transition-de-l'effet-photoéléctrique}
    \caption{Transition de l'effet photoéléctrique}
    \label{fig:transition-de-l'effet-photoéléctrique}
\end{figure}

On remarque que l'expression est imaginaire pour $\omega < \omega_0$ 


\section*{Théorie de la diffusion (élastique)}


\begin{figure}[ht]
    \centering
    \incfig{diffusion}
    \caption{Diffusion}
    \label{fig:diffusion}
\end{figure}


$$\dd n \propto F_i \dd \Omega$$ 

$$\dd n = \simga(\theta, \varphi) F_i \dd\Omega$$ 

$$\sigma(\theta, \varphi): \text{Section efficace de diffusion} $$ 

$\left[ \simga \right] = $ surface (barr = $10^{-24}\rm{cm}^2$  

Considérations physiques

\begin{figure}[ht]
    \centering
    \incfig{onde-plane-icidente}
    \caption{Onde plane icidente}
    \label{fig:onde-plane-icidente}
\end{figure}


$$\dotsb$$ 

\end{document}
