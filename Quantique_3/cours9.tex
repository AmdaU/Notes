\documentclass{article}    
\usepackage[utf8]{inputenc}    
    
\title{Épisode 4}    
\author{Jean-Baptiste Bertrand}    
\date{\today}    
    
\setlength{\parskip}{1em}    
    
\usepackage{physics}    
\usepackage{graphicx}    
\usepackage{svg}    
\usepackage[utf8]{inputenc}    
\usepackage[T1]{fontenc}    
\usepackage[french]{babel}    
\usepackage{fancyhdr}    
\usepackage[total={19cm, 22cm}]{geometry}    
\usepackage{enumerate}    
\usepackage{enumitem}    
\usepackage{stmaryrd}    
\usepackage{mathtools,slashed}
%\usepackage{mathtools}
\usepackage{cancel}
    
\usepackage{pdfpages}
%packages pour faire des math    
%\usepackage{cancel} % hum... pas sur que je vais le garder mais rester que des fois c'est quand même sympatique...
\usepackage{amsmath, amsfonts, amsthm, amssymb}    
\usepackage{esint}  
\usepackage{dsfont}

\usepackage{import}
\usepackage{pdfpages}
\usepackage{transparent}
\usepackage{xcolor}
\usepackage{tcolorbox}

\usepackage{mathrsfs}
\usepackage{tensor}

\usepackage{tikz}
\usetikzlibrary{quantikz}
\usepackage{ upgreek }

\newcommand{\incfig}[2][1]{%
    \def\svgwidth{#1\columnwidth}
    \import{./figures/}{#2.pdf_tex}
}

\newcommand{\cols}[1]{
\begin{pmatrix}
	#1
\end{pmatrix}
}

\newcommand{\avg}[1]{\left\langle #1 \right\rangle}
\newcommand{\lambdabar}{{\mkern0.75mu\mathchar '26\mkern -9.75mu\lambda}}

\pdfsuppresswarningpagegroup=1


\begin{document}

\section*{Théorie des perturbation (?)}

Il fait un rappel de la théorie des perturbation, qu'on a fait au dernier cours

\begin{figure}[ht]
    \centering
    \incfig{spectre-énérgétique}
    \caption{spectre énérgétique}
    \label{fig:spectre-énérgétique}
\end{figure}

\section*{Cas dégénéré}

On pose:
$$\ket{\varphi_{n,\alpha}} = \sum_{i=1}^{g_n} c^\alpha_{n,i} \ket{\varphi_n^i}$$ 

On fait un chanement de base pour utilisel les ket $\alpha$ au lieu d'utiliser les ket $i$  

$$H_0\ket{\varphi_n,\alpha} = E_n^0\ket{\varphi, \alpha}$$ 
	
$$H_0\ket{\varphi^{(1)}} + W\ket{\varphi_{n,\alpha}}= E_n^0\ket{\varphi_n^i} + E^{(1)} \ket{\varphi_{n,\alpha}}$$ 
$$\bra{\varphi^i_n}H_0\ket{\varphi^{(1)}} + \bra{\varphi^i_n}W\ket{\varphi_{n,\alpha}}= \bra{\varphi^i_n}E_n^0\ket{\varphi_n^i} + \bra{\varphi^i_n}E^{(1)} \ket{\varphi_{n,\alpha}}$$ 


$$\sum_{i=1}^{g_n} \bra{\varphi^i_n}\bar W \ket{\varphi_n^{i\prime}}\bra{\varphi_n^{i\prime}}\ket{\varphi_{n,\alpha}}= E^{(1)}\braket{\varphi_n^i}{\varphi_{n,\alpha}}$$ 

C'est essentiellement un produit matriciel 


$$\det(P_{\mathcal{E}}\qty(\bar W = E^(1)\mathbb{1})P_{\mathcal{E} }) = 0 \to E^{(1)} \text{ valeur propres}$$ 

On va se limiter en ordre 1 en énérgie, et donc en ordre 0 en état dans le cadre du cours.

L'odre 0 n'est pas trivial même à l'ordre 0 dans le cas dégénéré.

\section*{Algorithme}

si $$H = H_{0}+W$$

si $\ket{\varphi_{n}}$ est non-dégénéré : formule
sinon $$E_{0}= E_{n}^0 + \lambda E_\alpha^{(1)}$$ 

\section*{Application: structure fine de l'atome $H$}

rappel: eq dirac: $$(c\vec\alpha\cdot\vec p +\beta mc^2 + V(r))\psi = E\psi\quad V = - \frac{e^2}{r} $$ 

$$H_{sf}= \frac{\vec{p}^2}{2m} + V +\underbrace{W_{mv}+W_{D}+ W_{SD}}_{\text{perturbation}}$$ 

\begin{figure}[h!]
    \centering
    \incfig{spectre-de-l'atome-d'hydrogene}
    \caption{spectre de l'atome d'hydrogene}
    \label{fig:spectre-de-l'atome-d'hydrogene}
\end{figure}

$$\boxed{\ket{n=1, l=0, n=0, \pm}=\ket{\varphi_{1s}}}$$ 

$$\ket{n=2, l=0, m=0, \pm} = \ket{2s}$$ 

$$\ket{n=2, l=1, m\in \{1,0,-11\}, \pm} = \ket{2p}$$ 


on définit $$E_{n}^{0}=-\frac{E_{I}}{n^{2}}\qquad E_{I}=\frac{m e^{4}}{2 \hbar^{2}}=\frac{1}{2} m c^{2} \alpha^{2}$$ 

et $$\alpha=\frac{e^{2}}{\hbar c} \simeq \frac{1}{137}$$ 

\underline{Niveau 1s} 

$$E_{1s}= E_{1s}^0\bra{1,0,0,\pm} W_{mv}+W_{0}\ket{1,0,0,\pm}$$ 


$$\bra{1,0,0}\otimes\bra{1,0,0}{\pm}W_0\ket{1,0,0}\otimes\ket{\pm} = \bra{1,0,0}W_0\ket{1,0,0}$$ 

$$=\int \dd^3 r \bra{1,0,0} W_D\ket{\vec r}\bra{\vec r}\ket{1,0,0} = \int \dd^3 r \varphi_{1s}(r) \frac{\hbar^2e^2\pi}{2m^2c^2} \delta(\vec r) \varphi_{1s}(r) = \frac{\hbar^2e^2\pi}{2m^2c^2}\underbrace{\norm{\varphi_{1s}(0)}^2}_{\frac{1}{\pi a^2_0} } = \frac{ 1}{2} mc^2 \alpha^4$$ 

$$\bra{1,0,0,\pm}\underbrace{W_{mv}}_{\frac{-\bar p^4}{8m^3c^2} }\ket{1,0,0,\pm}$$ 

si $\frac{p^2}{2m} H_{0-V}\implies P^4 = (2m)^2(H_{0}-V)^2= 4m^2(H_{0}^2-H_{0}V- VH_{0}+V^2)$ 

\begin{align*}\bra{1,0,0}W_{mv}\ket{1,0,0} = - \frac{1}{2mc^2} \bra{1,0,0}H_{0^2}- H_{0} V - VH_{0}+ V^2\ket{1,0,0}=\\ -\frac{1}{2mc^2} \qty(E_{1s}^2+ E_{1s}\bra{1,0,0}V\ket{1,0,0} + \bra{1,0,0}V^2\ket{1,0,0}) = -\frac{5}{8} mc^2 \alpha^4\end{align*}

(On obtien le résultat après avoir intergrés sur $V$)


Donc: $$E_{1s}= E_{1S}^0 + \qty(\frac{1}{2} - \frac{5}{8}) mc^2\alpha^2$$ 

\underline{Niveau n=2} 


$$2s: \ket{2,0,0,\pm},\quad g=2$$ 
$$2p: \ket{1,2,(\pm1, 0),\pm},\quad g=6$$ 

$$[\vb{L}^2, \vb{P}^4] = [\vb{L}^2, P^2P^2] = p^2 [L^2, P^2] + [L^2, P^2] P^2$$ 

$$\vb P^2 = P_{r^2}+L^2 \implies \text{tout commute}$$ 

$$\implies P^4 \text{ conserve }l$$ 

$$[L^2, \vb L\cdot \vb S] = \[L^2, \vb L] \cdot \v S + \vb L [\vb L, \vb S] =0$$ 

$$\implies W_{so} \text{ conserve } l$$ 

\begin{figure}[ht]
    \centering
    \incfig{matrice-de-wsf}
    \caption{matrice de Wsf}
    \label{fig:matrice-de-wsf}
\end{figure}
	
$$\bra{\pm, 2,0,0}W_D\ket{2,0,0,\pm} = \bra{2,0,0}W_d\ket{2,0,0}$$ 

$$\varphi_{2s}(r) = \frac{1}{\sqrt{8\pi a_0^3}} \qty(1-\frac{r}{2a_{0}}) r^{-\frac{r}{2a_{0}} }$$ 
\end{document}
