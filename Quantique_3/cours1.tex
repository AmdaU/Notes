\documentclass{article}
\usepackage[utf8]{inputenc}

\title{Épisode 0}
\author{Jean-Baptiste Bertrand}
\date{\today}

\usepackage{physics}
\usepackage{graphicx}
\usepackage{svg}
\usepackage[utf8]{inputenc}
\usepackage[T1]{fontenc}
\usepackage[french]{babel}
\usepackage{fancyhdr}
\usepackage[total={19cm, 22cm}]{geometry}
\usepackage{enumerate}
\usepackage{enumitem}

%packages pour faire des math
%\usepackage{cancel} % hum... pas sur que je vais le garder mais rester que des fois c'est quand même sympatique...
\usepackage{amsmath, amsfonts, amsthm, amssymb}
\usepackage{esint}

\begin{document}

\maketitle

\section{Spin de l'electron : 2 confirmations}

\subsection*{Problème de S-f de qqch}

La théorie de Bohr n'est pas relativiste. C'est un problème si on considère que les éléctrons vont à $\sim 10^6$ m/s. Si on iclus la relativité, les niveaux d'énériges sont décallés correctement, cependant, la dégénéressance n'est pas levée comme observé expérimentalement.

POur arriver à le faire, on doit considérer l'effet Zeeman.

L'effet Zeeman est la levé des dégénéressance par l'application d'un champ magnétique.


$$-l \leq m \leq l $$

$$2l + 1 \text{ Projections possibles}$$

Il y a toujours un nombre impair de projections.

\includesvg{E0_1.svg}



On suppose que la sep des niv de H est similaire à celle de l'effet Zeeman.

On a donc pensé à l'ajout du nombre quantique du \textit{spin} pour explique cette levé de dégénéressance.


$$\ket{n,l,m} \to \ket{n,l,m,m_s}$$

\subsection*{Équation de dirac}

$$i\hbar \del \psi = H\psi \; \psi = \psi(\vec r, t)$$





\end{document}