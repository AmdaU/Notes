\documentclass{article}
\usepackage[utf8]{inputenc}

\title{Épisode 2}
\author{Jean-Baptiste Bertrand}
\date{\today}

\usepackage{physics}
\usepackage{graphicx}
\usepackage{svg}
\usepackage[utf8]{inputenc}
\usepackage[T1]{fontenc}
\usepackage[french]{babel}
\usepackage{fancyhdr}
\usepackage[total={19cm, 22cm}]{geometry}
\usepackage{enumerate}
\usepackage{enumitem}

%packages pour faire des math
%\usepackage{cancel} % hum... pas sur que je vais le garder mais rester que des fois c'est quand même sympatique...
\usepackage{amsmath, amsfonts, amsthm, amssymb}
\usepackage{esint}

\begin{document}

\maketitle

\underline{Spineurs, bases et rerésentations}

\begin{align}
	&\text{ECOC: } X, Y, Z, S_z, (S^2): \mathcal{E}_{\vec r}\otimes \mathcal{E}_s=\mathcal{E}\; \ket{\vec r, s}\\
	&\text{ECOC: } P_x, P_y, P_z, S_z; \ket{\vec p, s}\\
	&\text{ECOC: } H_0, \vb L^2, L_z, S_z ; \ket{n,l,m,s}
\end{align}

Relation de fermeture dans $\mathcal{E}:$ $$1 = 1_{\vec r} \otimes 1_S = \int \dd^3 r \ket{\vec r}\bra{\vec r} \otimes \sum_\epsilon \ket{\epsilon}\bra{\epsilon}$$

$$\implies 1 = \sum_\epsilon\int\dd^3r \ketbra*{\vec r\epsilon}{\vec r, \epsilon}$$

Preuve très similaire pour les autres bases.

$$\ket\psi = 1 \ket \psi = \sum_\epsilon \int \dd^3r \ket{\vec r, \epsilon} \underbrace{\braket*{\vec r, \epsilon}{\psi}}_{\Psi_\epsilon(\vec r)}$$

Représentation matricielle:

$$
	\ket\psi = \int \dd^3 r \begin{pmatrix}
		\psi_+(\vec r)\\
		\psi_-(\vec r)
	\end{pmatrix}\ket{\vec r}
$$

$$\braket*{\vec r}{\psi} = \begin{pmatrix}
	\psi_+(\vec r)\\
	\psi_-(\vec r) 
\end{pmatrix} = [\psi] \; \text(Spineur!)$$ 

$$
	\bra\psi = \int \dd^3 r \begin{pmatrix}
		\psi^*_+(\vec r) &
		\psi^*_-(\vec r)
	\end{pmatrix}\bra{\vec r}
$$


\begin{equation}
	\ket\psi = 1\ket\psi = \sum_\epsilon\sum_{n,l,m} \ket{n,l,m,\epsilon}\overbrace{\braket*{n,l,m,\epsilon}{\psi}}^{C_{n,l,m,\epsilon}}
\end{equation}

si

$$\ketbra*{\vec r}{n,l,m} = R_n,l(r)Y_l^m(\theta,\phi)$$

$$\braket*{\vec r}{\psi} = \sum_{n,l,m}\sum_\epsilon \underbrace{\braket*{\vec r}{n,l,m}}_{R_n,l(r)Y_l^m(\theta,\phi)}\ket\epsilon C_{n,l,m,\epsilon} = \sum_{n,l,m}\begin{pmatrix}
	c_{n,l,m,+}R_n,l(r)Y_l^m(\theta,\phi)\\
	c_{n,l,m,-}R_n,l(r)Y_l^m(\theta,\phi)
\end{pmatrix}$$

\end{document}