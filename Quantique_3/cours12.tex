\documentclass{article}    
\usepackage[utf8]{inputenc}    
    
\title{Épisode 4}    
\author{Jean-Baptiste Bertrand}    
\date{\today}    
    
\setlength{\parskip}{1em}    
    
\usepackage{physics}    
\usepackage{graphicx}    
\usepackage{svg}    
\usepackage[utf8]{inputenc}    
\usepackage[T1]{fontenc}    
\usepackage[french]{babel}    
\usepackage{fancyhdr}    
\usepackage[total={19cm, 22cm}]{geometry}    
\usepackage{enumerate}    
\usepackage{enumitem}    
\usepackage{stmaryrd}    
    
%packages pour faire des math    
%\usepackage{cancel} % hum... pas sur que je vais le garder mais rester que des fois c'est quand même sympatique...
\usepackage{amsmath, amsfonts, amsthm, amssymb}    
\usepackage{esint}  


\begin{document}

\section*{Particules identiques}

\underline{Définition}: Des particules identiques ont la même mase, charge et spin... etc. Rien ne distingue une de l'autre.


\begin{figure}[ht]
    \centering
    \incfig{mécanique-clasique-versus-quantique}
    \caption{Mécanique clasique versus quantique}
    \label{fig:mécanique-clasique-versus-quantique}
\end{figure}

{\bf Cas simples avec $n=2$ }

Les particules "1" et "2" sont des les états $\ket{\varphi^1_r}$ et $\ket{\varphi_{r'}^2}$ respectivement 

Espace de Hilbert $$ \mathcal{E}^{\otimes 2} = \mathcal{E}^1 \otimes \mathcal{E}^2$$ 

Dégénéresance d'échange $$\ket{\varphi^1_r} \otimes \ket{\varphi_{r'}^2} = \ket{\varphi^1_r, \varphi_{r'}^2}$$ 

On prend le vecteur $\ket{\Psi} = a_{1} \ket{\varphi_{r}^1,\varphi_{r'}^2} + a_{2} \ket{\varphi_{r'}^1, \varphi_{r}^2}$ 

Avec $|a_{1} | = |a_{2} |$ Comme dicté par le \underline{postulat de symétrisation}  

La dégénérecese d'échange induit l'utilité des opérateur de permutation $P_{\pi}$ ou $\pi$ représente les difféentes permutation.

Par exemple pour $n=2$ on a $$\{P_{\pi} \} = \{\mathds{1}, P_{21}\} = \{P_{\pi1} P_{\pi2} \}$$ 

$$P_{21} \ket{\varphi^1_r, \varphi_{r'}^2} = \ket{\varphi^2_r, \varphi_{r'}^1}$$ 

$$P_{21}^2 = \mathds{1}$$ 

$$\implies P_{21} = P_{21}^{-1}$$ 

\underline{Hérmiticité de $P_2$ } 


$$\bra{\varphi_{i}^1, \varphi_{j}^2} P_{21}^\dagger \ket{\varphi_{i}^1, \varphi_{j}^2} = \qty(\bra{\varphi_{i}^1, \varphi_{j}^2} P_{21} \ket{\varphi_{i}^1, \varphi_{j}^2})^* = \dotsb = \delta_{ij'} \delta_{ji'} $$ 


Les état propres de opérateur de permutations sont les états complètement symétriques et les états complètement antisymétriques

$$\begin{cases}
	P_{21} \ket{\Psi}_+ = \ket{\Psi}_+ & \text{État symétrique}\\ 	
	P_{21} \ket{\Psi}_- = -\ket{\Psi}_- & \text{État antisymétrique}
\end{cases}$$ 


On définit mainetanant deux projecteurs: $S_{\pm} $ 

$$S_{\pm} \equiv \frac{1}{2} (\mathds 1 \pm P_{21} )$$ 

On démontre facilement que $S_{\pm}^2 = S_{\pm} $ 

On trouve que l'effet de ces projecteur est de (anti)symétriser les états!

$$P_{21} S_\pm\ket{\Psi} = P_{21} \frac{1}{2} (\mathds 1 \pm P_{21} ) \ket{\Psi} = S_{\pm} \ket{\Psi}$$ 


On trouve la propriété importante que 

$$S_{+S_-} = \frac{1}{4} \left( \mathds 1 + P_{12}  \right) \left( \mathds 1 - P_{21}  \right) = \frac{1}{4} \left( \mathds 1 -P_{21}^2 +P_{21} -P_21\right) =0 $$ 

Ce qui est tout à fait logique car on projete sur des sous espace disjoint! On ne peux pas avoir des particules qui sont des bosons et des fermions en même temps.

$$\mathcal{E}_+^{\otimes2} \cap \mathcal{E}_-^{\otimes2} = 0 \qquad \mathcal{E}_+^{\otimes2} \cup  \mathcal{E}_-^{\otimes2} = \mathcal{E}^{\otimes 2}$$ 


\begin{tcolorbox}[title = Septième postulat de la mécanique quantique; postulat de symétrisation]
Les vecteurs d'était pour $n=2 $ particules identiques sont soit symétriques (bosons) soit antisymétique (fermions) 
\end{tcolorbox}

\underline{Généralisation à plusieurs particules ($n > 2$) } 

$$\ket{\varphi_{r_1}^2,\varphi_{r_2}^2, \dotsb, \varphi_{r_n}^n} \implies \text{ dégénéressance d'échange}$$ 

$$\ket{\Psi} = \sum_{i}^{n!} a_{i} P_{\pi i} \ket{\Psi_{\pi i} }$$ 


$$|a_i| = |a_{j} | \forall i,j$$ 


\begin{tcolorbox}[title=Illustation avec $n\equiv3$ ]	
	$$P_{321} \ket{\varphi_{r_1}^1,\varphi_{r_2}^2,\varphi_{r_3}^3}= \ket{\varphi_{r_1}^3,\varphi_{r_2}^1,\varphi_{r_3}^2} = \ket{\varphi_{r_2}^2,\varphi_{r_3}^2, \varphi_{r_1}^3}$$ 
	Les $P_{\pi}$ ne sont pas commutatif:
	$$P_{132} P_{312} \ket{1,2,3} = P_{132} \ket{3,1,2} = \ket{3,2,1}$$ 
	$$P_{321} P_{132} \ket{1,2,3} = P_{321} \ket{1,3,2} = \ket{2,3,1}$$ 
	$$\ket{3,2,1} \neq \ket{2,3,1} \implies P_{\pi_i} P_{\pi_j} \neq P_{\pi_j} P_{\pi_i} $$ 
\end{tcolorbox}


Les permutaitons peuvent toujours être décomposé en traspoition (échange de deux éléments seulement) Ex: $P_{321} = P_{132} P_{213} $. La parité d'une permutation correspond alors à la parité du nombre de transposition dont elle est composé.



En général $P_{pi} \neq P_{pi}^\dagger$ (n'est pas hérmitique) même si c'est le cas pour les transposition. 

\underline{Unitarité}: $$P_{\pi}^\dagger P_{\pi} = \mathds{1}$$ $$P_{321}^\dagger P_{321} = (P_{123} P_{213})^\dagger P_{321} = (P_{132} P_213)^\dagger (P_{132} P_{213} ) = P_{213}^\dagger P_{132}^\dagger P_{132} P_{213} = \mathds{1}$$  
La preve général suit exactement le même raisonement.


ON cherche les état symétique et antisymériques
$$P_{\pi} \ket{\Psi}_\pm = \pm^\pi \ket{\Psi}_\pm$$ 


On introduit encore une fois des projecteurs

$$S_{+} = \frac{1}{n!} = \sum_{\pi} P_{\pi} \qquad S_{-} = \frac{1}{n!} \sum_{\pi} (-1)^\pi P_{\pi} $$ 


$$S_{\pm}^2 = \frac{1}{n!} \left( \frac{1}{n!} \sum_{\pi} \pm^\pi P_{\pi}  \right) \sum_{p'} \pm^{\pi'} P_{\pi'} = \frac{1}{n!} \left( \frac{1}{n!} \sum_{\pi} \pm^{\pi+\pi'} \underbrace{P_{pi} P_{\pi'}}_{P_{\tilde \pi}} + \dotsb \right) = \frac{1}{n!} (S_{\pm} + \dotsb) = \frac{n!}{n!} S_{\pm} = S_{\pm}  $$ 


	
\end{document}
