\documentclass{article}
\usepackage[utf8]{inputenc}

\title{Charge 1}
\author{Jean-Baptiste Bertrand}
\date{\today}

\setlength{\parskip}{1em}

\usepackage{physics}
\usepackage{graphicx}
\usepackage{svg}
\usepackage[utf8]{inputenc}
\usepackage[T1]{fontenc}
\usepackage[french]{babel}
\usepackage{fancyhdr}
\usepackage[total={19cm, 22cm}]{geometry}
\usepackage{enumerate}
\usepackage{enumitem}
\usepackage{stmaryrd}

%packages pour faire des math
%\usepackage{cancel} % hum... pas sur que je vais le garder mais rester que des fois c'est quand même sympatique...
\usepackage{amsmath, amsfonts, amsthm, amssymb}
\usepackage{esint}

\newcommand{\col}[1]{\begin{pmatrix}#1\end{pmatrix}}

\begin{document}

\maketitle

$$H_p \frac{1}{2m} = \qty{\vec p \cdot \underbrace{\qty[\vec p - \frac qc \vec A]}_{\vec \Pi}}^2 +qV(\vec R)$$


$$(\vec \sigma \cdot \vec \pi)^2 = \vec \pi \cdot \vec \pi + i\vec\sigma\cdot(\vec\pi\cross\pi)$$

\underline{Preuve:}

$$\boxed{\begin{matrix}\sigma_i^2 = 1\\ \sigma_i\sigma_j =-\sigma_j\sigma_i\\\sigma_1\sigma_2 = i\sigma_3 \end{matrix}}$$


$$(\vec\sigma \cdot \vec \pi)^2 = \sum_{ij}\sigma_i\pi_i\sigma_j\pi_j = \sum_{ij} + \sum_{i\neq j}$$


... Pas le temps de retranscrire

$$\vec \pi \cross \vec \pi \begin{pmatrix}
	f_x\\f_y\\f_z
\end{pmatrix} =  \qty(\frac\hbar i \vec \nabla -\frac q c \vec A) \cross \qty(\frac\hbar i \vec \nabla -\frac q c \vec A)\begin{pmatrix}
	f_x\\f_y\\f_z
\end{pmatrix} $$


$$=\underbrace{\qty(\frac \hbar i)^2 \grad \cross \grad(f)}_{0} - \frac q c \vec A \cross \frac \hbar i \grad (f) - \frac \hbar i \grad \cross \grad \frac qc \vec A(F) + \underbrace{\qty(\frac qc )^2 \vec A \cross A (f)}_0$$

Expenssion du produit vectorielle: On se rend compte que sur $A$ ou que sur $f$

% $$\grad \cross \vb A (f) = \begin{pmatrix}
% 	(\del_yA_z -\del_zA_y)f_x\\
% 	(\del_zA_x -\del_xA_z)f_y\\
% 	\dots
% \end{pmatrix}$$

$$= -\frac \hbar i \frac qc \underbrace{\grad \cross \vb A}_{\vb B}(f)$$


$$H_p = \frac{1}{2m} \qty(\vec p -\frac qc \vb A)^2 + \frac{i\vec\sigma}{2m}\cdot\qty(-\frac \hbar i \frac qc \vb B) + q V(\vb R)$$

Le 2eme terme est genre $S\cdot B$ ou dequoi

$$\boxed{Y_0^0 = \frac{1}{\sqrt{4\pi}}}$$


\section{Spineurs et mesures}

$$[\psi](\vec r) = Ne^{-\alpha r^2/2}\begin{pmatrix}\sin\theta\cos\varphi + \sin\theta\sin\varphi\\ 1+\cos\theta\end{pmatrix} = \begin{pmatrix}
	\psi_+(\vec r)\\\psi_-(\vec r)
\end{pmatrix}$$

$$\psi_0(\vec r) = f_0(\vb r) \sum_{l,m} Y_l^m(\theta, \varphi)a_{lm\sigma}$$

$$\mathcal{N}([\phi]) = \int \dd^3 r \qty(\abs{\psi_+(\vb r)}^2 + \abs{\psi_-(\vb r)}^2) = \int \dd r r^2\qty(\sum_{lm\sigma}f_0(r)^2\abs*{a_{lm\sigma}}^2)$$


$$P(l,m,\sigma) = \frac{1}{\mathcal{N}[\psi]}\cross \int\dd r r^2 f_0(r)^2\abs*{a_{lm\sigma}}^2$$


\end{document}
