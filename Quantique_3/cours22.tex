\documentclass{article}    
\usepackage[utf8]{inputenc}    
    
\title{Épisode 4}    
\author{Jean-Baptiste Bertrand}    
\date{\today}    
    
\setlength{\parskip}{1em}    
    
\usepackage{physics}    
\usepackage{graphicx}    
\usepackage{svg}    
\usepackage[utf8]{inputenc}    
\usepackage[T1]{fontenc}    
\usepackage[french]{babel}    
\usepackage{fancyhdr}    
\usepackage[total={19cm, 22cm}]{geometry}    
\usepackage{enumerate}    
\usepackage{enumitem}    
\usepackage{stmaryrd}    
    
%packages pour faire des math    
%\usepackage{cancel} % hum... pas sur que je vais le garder mais rester que des fois c'est quand même sympatique...
\usepackage{amsmath, amsfonts, amsthm, amssymb}    
\usepackage{esint}  

\begin{document}

\section*{Diffusion}


$$\dd n = F_i \simga(\theta, \varphi)\dd \Omega$$ 


$$\varphi_{klm}^{(0)} \quad \text{fonctions propres en absence de $V$ } \quad \left\{ H_0, \vb{L}^2, L_z\right\} $$ 

$$-\frac{\hbar^2}{2\mu} \grad^2 \varphi^{(0)} = E_k \varphi^{(0)}$$ 

On le récrit en coordonnées sphériques:

$$\dotsb$$ 

On fait le changement de variable $\rho = kr$ 

On obtiens équation de Bessel sphérique, $\rho = 0$ est un point régulier singulier.


On trouve un équation indicielle.

$$R_{kl}^{(0)} (kr) = \sqrt{\frac{2k^2}{\pi}} J_l(Kr)$$ 

$$\varphi_{klm}^{(0)} ( \vb{r}) = \sqrt{\frac{2k^2}{\pi} }J_{l(kr} Y_l^m $$ 

les $\varphi_{klm}^{(0)}$ sont orthonormées 
$$J_l (kr) \to_{\rho\to\infty} = \frac{1}{\rho} \sin \left( \rho - l \frac{\pi}{2}  \right) $$ 

$$\varphi_{klm}^{(0)} (r\to \infty, \theta, \varphi) \to \sqrt{\frac{2k^2}{\pi} } \frac{1}{kr} \sin(kr -\frac{l\pi}{2} )$$ 

Le $\sin$ est une somme d'exponentielle

$$\dotsb$$ 



\section*{Exemple: Sphère dure}

$$V(\vb{r})\begin{cases}
	\infty & \abs{r} \leq r_0\\
	0 & \abs{r} > r_0 
\end{cases}$$ 

Dans la limite des ??????????????? $$kr_0 \ll 1$$ 

\begin{tcolorbox}[title=Parenthèse sur une formule]
	$$\vb{L} = \vb{r} \times \vb{p} = b\hbar k$$  
	$$\abs{\vb{L}} = \hbar\sqrt{l \left( l+1 \right) }$$ 
	$$ka \ge \sqrt{k(l+1)}$$ 
	La seule valeur qui va être pertinente à la diffusion est $l=0$, les autre sont négligeables 
\end{tcolorbox}



\end{document}
