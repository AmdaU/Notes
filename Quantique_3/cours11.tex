\documentclass{article}    
\usepackage[utf8]{inputenc}    
    
\title{Épisode 4}    
\author{Jean-Baptiste Bertrand}    
\date{\today}    
    
\setlength{\parskip}{1em}    
    
\usepackage{physics}    
\usepackage{graphicx}    
\usepackage{svg}    
\usepackage[utf8]{inputenc}    
\usepackage[T1]{fontenc}    
\usepackage[french]{babel}    
\usepackage{fancyhdr}    
\usepackage[total={19cm, 22cm}]{geometry}    
\usepackage{enumerate}    
\usepackage{enumitem}    
\usepackage{stmaryrd}    
    
%packages pour faire des math    
%\usepackage{cancel} % hum... pas sur que je vais le garder mais rester que des fois c'est quand même sympatique...
\usepackage{amsmath, amsfonts, amsthm, amssymb}    
\usepackage{esint}  


\begin{document}

\section*{Méthode variationnelles}


Approche approximative à $H \ket{\varphi_n} = E_{n} \ket{\varphi_n}$ 


En générale, cette équation des difficile à solutionner.


$H$ n'est pas toujours décomposable en $H_{0} + W$ 

$$W \sim H_{0}, W > H_{0} \implies \text{ La théorie des perturbation n'est pas utile}$$ 


$$\text{Intuition phyisque + considération} \implies \text{ ket d'essai} \ket{\psi_{\alpha} } \text{ pour le fondamentale}$$ 

${\alpha}$ est un esemble de paramètre variationnels.

Comme le ket d'essai n'est pas nécéssairement normalisé de base (puisqu'on le postule) on doit d'abord le normaliser.


Si le ket d'essai est une fonction de $\alpha$, alors l'énergie de cette état aussi.


$$\frac{\bra{\psi_\alpha}H\ket{\psi_\alpha}}{\braket{\psi_{\alpha} }} = \avg{H}(\alpha)$$ 


On peut alors trouver l'état fondamentale en minimisant l'énérgie de notre ket.


$$\eval{\dv{\avg{H}}{\alpha}}_{\alpha_{0} } =0 $$ 

On alors que $\ket{\alpha_{0} }$ est un état fondamental.


On a, par définition de l'état fondamentale que 
$$\avg{H} \geq E_{0} \text{ où } = \iff \ket{\psi_\alpha} = \ket{\Psi_0}$$ 


\underline{Théorème de Ritz} 

Pour le ket d'essai $\ket{\psi}$ où $\avg{H}$ est un extremum 


$$H \underbrace{\ket{\psi}}_{\text{État propre de }H} = \underbrace{\avg{H}}_{\text{Valeur propre}} \ket{\psi}$$ 
Cela suggère que l'équation de Shrödinger peut être trouvé par principe variationnel.


\underline{Exemple: oscillateur harmonique 1D} 

$$H = \frac{p^2}{2m} + \frac{1}{2} kx^2$$ 

les énérgies sont donnnes par $E_{n} = \hbar\omega\left( n + \frac{1}{2}  \right) $ 

On sait que la fonction d'onde de l'état fondamentale doit être maximale en $0$ et qu'elle doit s'annuler à l'infini. On postule donc que la solution est une gaussienne, 

$$\braket{x}{\varphi_E} = e^{-\alpha x^2}$$ 

On part de 

$$\frac{\bra{\varphi_E}H\ket{\varphi_E}}{\braket{\varphi_E}} = \avg{H}(\alpha)$$ 

$$\int \varphi_E(\alpha) \qty(- \frac{\hbar^2}{2m} \dv[2]{x} + \frac{1}{2} k x^2)\varphi_E(\alpha) \dd x$$ 

$$\dotsb$$ 


$$\avg{H}_E(\alpha) = \frac{\hbat^2}{2^{3/2}} \sqrt{\pi} \sqrt{\alpha} + \frac{K}{42^{3/2}} frac \sqrt{\pi} \alpha^{3/2}}$$ 


$$\eval{\frac{\del\avg{H}}{\del\alpha}}_{\alpha_{0} } = 0 \implies \alpha_{0} = \frac{1}{2\hbar} \sqrt{km}$$ 


$$\avg{H}_E (\alpha_{0}) = \frac{\hbar}{2m} \alpha_{0} + \frac{k}{8} \frac{1}{\alpha_{0}} = \frac{1}{2} \hbar\omega$$ 


On trouve donc la bonne énérgie est la bonne fonction d'onde.


Si on aurait pris une aute fonction d'onde d'essai, on aurrait trouvé une fonction d'onde propre à l'Hamiltonien associée à une énérgie plus élevée.	





	
\end{document}
