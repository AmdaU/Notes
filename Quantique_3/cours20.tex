\documentclass{article}    
\usepackage[utf8]{inputenc}    
    
\title{Épisode 4}    
\author{Jean-Baptiste Bertrand}    
\date{\today}    
    
\setlength{\parskip}{1em}    
    
\usepackage{physics}    
\usepackage{graphicx}    
\usepackage{svg}    
\usepackage[utf8]{inputenc}    
\usepackage[T1]{fontenc}    
\usepackage[french]{babel}    
\usepackage{fancyhdr}    
\usepackage[total={19cm, 22cm}]{geometry}    
\usepackage{enumerate}    
\usepackage{enumitem}    
\usepackage{stmaryrd}    
    
%packages pour faire des math    
%\usepackage{cancel} % hum... pas sur que je vais le garder mais rester que des fois c'est quand même sympatique...
\usepackage{amsmath, amsfonts, amsthm, amssymb}    
\usepackage{esint}  

\begin{document}

\underline{Diffusion élastique}

\begin{figure}[ht]
    \centering
    \incfig{diffusion}
    \caption{diffusion}
    \label{fig:diffusion}
\end{figure}


$$\dd n = F_i \dd \underbrace{\sigma(\theta, \varphi)}_{\text{section efficasse} } \Omega$$ 

Les unitées de la section efficasse sont $$\left[ \sigma \right]  = \text{surface (ban = $10^{-24} \rm{cm}^2$)} $$ 

L'énérgie est conservé $$\frac{\hbar^2k^2}{2m} = E_k$$ 


\underline{Argument physiques pour l'amplitude de diffusion} 

$$u_r \sim \left( e^{ikz} + f_l (\theta,\varphi) \frac{e^{e^{ikr}}}{r}  \right) $$ 

$$\sigma(\theta, \varphi) \iff^? f_k (\theta, \varphi) $$ 


Courrant de probabilité de incident ($i$) et diffusé ($d$) $$\vb{J}_{i,d} = \frac{1}{\mu} \Re \left[ \varphi_{i,d}^*(\vb{r}) \frac{\hbar}{i} \grad \varphi_{i,d}  \right] $$   

$$\abs{\vb{J}_i} = \frac{\abs{A}^2 i\hbar k}{\mu}  $$ 

Si $\varphi=A e^{ikz}$ 

$$\vb{J}_d = \frac{i}{\mu} \frac{\abs{A}^2}{2\mu} \abs{f_k (\theta , \varphi)}^2 \left[ \frac{e^{-ikr}}{r} \frac{\hbar{}ki}{i} \frac{e^{ikr}}{r} - \cancel{\frac{e^{-ikr}}{r} \frac{hbar}{i} \frac{e^{ikr}}{r^2}}  + \text{c.c.} \right] \hat r $$ 

$$\left( \vb{J}_d \right)_r = \frac{\abs{A}^2\hbar k}{\mu} \frac{1}{r^2} \abs{f_{k(\theta,\varphi)}}^2$$ 

$$\implies \dd n = C \frac{\abs{A}^2\hbar{}k}{\mu} \frac{1}{\cancel{r^2}} \abs{f_k }^2 \cancel{r^2} \dd \Omega$$ 

$$\sigma(\theta \varphi) = \abs{f_{k(\theta, \varphi)}^2 $$ 

$$f_k (\theta, \varphi)\to\begin{cases}
	\text{Théorie des perturbations }\to \text{approximation (règle d'or de Fermi) de Born}\\ \text{déphassages (ondes partielles)}   
\end{cases}$$ 

\subsection*{Théorie des perturbation (Approximation de Born)}

$$\norm{\vb{P}_i} = \norm{\vb{P}_f} \qq{Élastique}$$ 

$$\partial \mathcal{P}_{i\to f} = \int_{DF}  \dd^3 P_F \abs{\bra{\vb{P}_i}\ket{\psi(t)}}^2 $$ 
$$\partail \mathcal{P}_{i \to f} (t)  =\underbrace{\int_{Df} P_f^2 \dd P_f \dd \Omega}_{\int \rho(E_{F)} \dd E_f \dd \Omega}  \abs{\bra{\vb{P}_f}\ket{Psi(t)}}^2$$ 

$$\boxed{\rho(E_f ) \dd E_f \dd\Omega}$$ 



$$\partial \mathcal{P}_{i\to f} \approx \int_{Df} \rho(E_{f)} \dd E_f \dd \Omega \times  \frac{1}{\hbar^2} \abs{\int_0^t e^{i\omega_{fi} t'} W_{fi} (t') \dd t'}^2$$ 

$$W(t) = \frac{W}{2} e^{i\omega t} + \frac{W^\dagger}{2} e^{-i\omega t}$$ 



$$\abs{\bra{P_{f}} \ket{\psi(t)}}^2 = \frac{\abs{\bra{\vb{P}_f}W\ket{\vb{P}_i}}^2}{4\hbar^2} \abs{e^{i\left( \omega_{fi} -\omega \right) \frac{t}{2} } \frac{\partial m \left( \omega_{fi} -\omega \right) \frac{t}{2}  }{\left( \omega_{fi} -\omega \right)/2} +e^{i \left( \omega_{fi} + \omega \right)\frac{t}{2} } \frac{\partial m \left( \omega_{fi} + \omega \right) \frac{t}{2} }{\left( \omega_{fi} + \omega \right)/2}  }^2$$ 


$$= \frac{\bra{\vb{P}_f}W \ket{\vb{P_i}}}{\hbar^2} \frac{\partial m^2 \omega_{fi} \frac{t}{2} }{\left( \omega_fi \right)^2/2^2}  $$ 


$$\dotsb??????$$ 


$$\bra{\vb{P_{f}}} V(\vb{R}) \ket{\vb{P_{i}}} = \int \bra{\vb{P}_f}\ket{\vb{r}}\bra{\vb{r}}V(\vb{r}) \ket{\vb{P}_i} \dd^3 r = \int \frac{e^{i \vb{P}_f \cdot \vb{r} /\hbar}}{\left( 2\pi\hbar \right)^{3/2}} V(\vb{r}) \frac{e^{i \vb{P_i \cdot \vb{R}}/\hbar}}{\left( 2\pi\hbar \right)^{2/3}} \dd^3 r    $$ 


$$\int \frac{e^{-i \frac{P_f - P_i}{\hbar} \cdot \vb{r}}}{\left( 2\pi \hbar \right)^3} V(\vb{r}) \dd^3 r = \frac{1}{(2\pi\hbar)^3} \int \dd^3 r e^{-i \vb{q} \cdot \vb{r}} V( \vb{r}) \qq{Tansformé de fourier du potentiel} $$ 

$$\frac{\vb{P}_f-\vb{P}_i}{\hbar} = \vb{K_{f}}- \vb{K}_i \equiv \vb{q}$$ 

$$\sigma(\theta, \varphi) = \frac{\# ?? \text{diffusées/temps}  \partial\Omega }{\# \text{??? incidents/temps ???} } $$ 


$$\vb{J}_i = \frac{1}{\mu(2\pi\hbar)^3} \hbar k \hat z = \frac{1}{(2\pi\hbar)^3} \underbrace{\frac{\sqrt{2\muE}}{\mu}}_{\frac{\sqrt{2}\sqrt{E}}{\sqrt{\mu}} }  \hat z$$ 

$$ \boxed{\frac{\partial\mathcal{P}_{i\to{}f}(t)}{\partial t \partial \Omega}/\abs{\vb{J}_i} = \sigma(\theta \varphi) = \frac{\mu^2}{(2\pi)^2\hbar^4} \abs{\int \dd^3 r e^{-i\vb{q} \cdot \vb{r}} V(\vb{r})}^2} $$ 



$$\vb{q}^2 = \dotsb 4K^2 \sin^2 \frac{\theta}{2} $$ 

\underline{Diffusion nucléons-nucléons (pos de Yukawa)} 

si $\sigma(\theta \varphi) = \norm{f_k (\theta, \varphi)}^2    $  





\end{document}


