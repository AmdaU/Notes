\documentclass{article}    
\usepackage[utf8]{inputenc}    
    
\title{Épisode 4}    
\author{Jean-Baptiste Bertrand}    
\date{\today}    
    
\setlength{\parskip}{1em}    
    
\usepackage{physics}    
\usepackage{graphicx}    
\usepackage{svg}    
\usepackage[utf8]{inputenc}    
\usepackage[T1]{fontenc}    
\usepackage[french]{babel}    
\usepackage{fancyhdr}    
\usepackage[total={19cm, 22cm}]{geometry}    
\usepackage{enumerate}    
\usepackage{enumitem}    
\usepackage{stmaryrd}    
    
%packages pour faire des math    
%\usepackage{cancel} % hum... pas sur que je vais le garder mais rester que des fois c'est quand même sympatique...
\usepackage{amsmath, amsfonts, amsthm, amssymb}    
\usepackage{esint}  


\begin{document}
2024-04-04

\begin{tcolorbox}[title=Rappels]
	\(B^{*}\) forme bilinéaire sur \(h^{*}\) non dégénéré 
\(B^{*}\) est défini-positif sur \(\mathds{R} \expval{\alpha}_{\alpha\in R}  \)
	\(\implies \mathds{R}\expval{\alpha}\) est une espace euclidien sous ensemble de \(R\) ensemble de racines


	\[ R = R^{+} \cup R^{-} \] pour \(h \in \mathds{E}\) t.q. \((h,\alpha) \neq 0\)

	\[ R^{+} = \{ \alpha | (h,\alpha) > 0 \} = -R^{-} \]


Racines simples: \(S \in R\) racines qui ne se décompose pas en une somme de racines positives

\[ \{ \frac{\alpha}{\norm{\alpha} }_{\alpha\in S}   \}  \]
est configuration admissible; ensemble de vecteurs tel quel \(\angle (u,v \in \{ \frac{\pi}{2}, \frac{2\pi}{3} ,  \frac{3\pi}{4} \frac{5\pi}{6}  \} \)

Diagramme de conxeter

Nombre de lien entre les points est (0,1,2,3) et correpond à l'idince de la liste d'angles

Est 
\begin{enumerate}
	\item Acylcique
	\item degré de chaque sommet \(\leq 3\)
\end{enumerate}

\end{tcolorbox}

On va essayer de reistreindre les Diagramme de Coxeter encore plus

\underline{Lemme}: 

Si \(v_{1}, \dotsb , v_n \) est une configuration admissible et \(V_i - V_j \) dans le Diagramme de Coxeter alors alors 

\(V_1 , \dotsb \hat V _i , \dotsb \hat v_j , \dotsb v_j , v_i + v_j \)

est une configuration admissible dont le Diagramme est identique sont que les sommets \(v_i \) et \(v_j \)

\begin{figure}[h!]
    \centering
    \incfig{exemple-lemme}
    \caption{exemple lemme}
    \label{fig:exemple-lemme}
\end{figure}

\underline{Démonstration}: Si \(v_k\)  n'est pas relié à \(v_i, v_j\)
\((v_k , v_j + v_i ) = 0   \)

si \((v_k - v_i - v_j )\), \( (v_k, v_i + v_j ) = (v_k , v_i ) + (v_k + v_j ) = (v_k , v_i ) \)

idem pour \( \left( v_k = v_i -v_j  \right)\)

de plus \(\left(  v_i + v_j , v_i + v_j \right) = \dotsb = 1\)

On déduit que que le Diagramme de Coxeter d'une configuration admissible irréductible fait partie de la liste 

\[ A_n :1 - 2 -  \dotsb - n-1 - n\]
\[ BCF_n  :1 - 2 -  \dotsb - i = i+1 - \dotsb- n-1 - n\]
\[ DE_n  :. - . -  \dotsb - . \perp . - \dotsb-. - .\]
\[ G_2: . \equiv . \]

\underline{Lemme}: Le Diagramme d<une configuration admissible ne peut par contenir comme sous graphe 

\begin{enumerate}
	\item \( \circ  - \circ = \circ - \circ -\circ\)
	\item \( \circ  - \circ \perp^{2} \circ - \circ \)
	\item \( \underbrace{\circ  - \circ}_{2}  \perp^{1} \underbrace{\circ - \circ}_{5}  \)
	\item \( \underbrace{\circ  - \circ}_{3}  \perp^{1} \underbrace{\circ - \circ}_{3}  \)
\end{enumerate}

\underline{Démonstration}: 

2) On 7 vecteurs 

la matrice \((v_i ,v_j ) \) est (voir figure) dégénéré


\begin{figure}[ht]
    \centering
    \incfig{matrice}
    \caption{matrice}
    \label{fig:matrice}
\end{figure}

Finalement, on a les cas
\[ A_n :1 - 2 -  \dotsb - n-1 - n\]
\[ BC_n  : 1 = 2 - \dotsb- n-1 - n\]
\[ F_4 = . - . = . - . \]
\[ D_n  : 1 \perp^{1}- . - \dotsb- . - .\]
\[ E_6 : . - . -\perp^{2} -. - . \]
\[ E_7 : . - . -\perp^{2} -. - . -. \]
\[ E_8 : . - . -\perp^{2} -. - . -. -. \]
\[ G_2: . \equiv . \]



\begin{tcolorbox}[title=Rappel]
	Dans un sytème de racine, si \(\angle \alpha, \beta = \frac{2\pi}{3} \) alors \(\norm{\alpha}  = \norm{\beta} \) 
	si \(\angle \alpha, \beta = \frac{3}{\pi} 4 \) alors \(\norm{\alpha} = \sqrt{2} \norm{\beta}  \)
	\[ \dotsb \]
\end{tcolorbox}

Conséquence de ce rappel:

Dans les cas \(A_n, D_n , E_n \) toutes les racines sont de la même longeure 

On garde une flèche sur \(=\) et\(\equiv\) qui pointe vers la racine la plus courte. On obtiens les \underline{Diagrammes de Dynkin} 


\[ B_n  : 1 =<= 2 - \dotsb- n-1 - n\]
\[ C_n  : 1 =>= 2 - \dotsb- n-1 - n\]
\[ F_4 : . - . =>= . - . \]
\[ G_2: . \equiv>\equiv . \]


\underline{Exemples} 
Les Diagrammes \(A_n \) est le Diagramme de \(\mathfrak{sl}(n+1 ,\mathds{C})\)

\[ h = \{ \begin{pmatrix} \alpha_1 \\ & \ddots \\&&\alpha_{n+1}  \end{pmatrix} | \sum_{\alpha_i} =0 \}  \]

\[ h^{*} = \expval{L_1 , \dotsb , L_{n+1} } \]
\[ R = \{ L_i - L_j | i \neq j \}  \]
\[ S = \{ L_1 - L_2 , L_2 - L_{3}, \dotsb , L_n - L_{n+1}  \}  \]

Le diagramme de \(B_n \) est le diagramme de \(\mathfrak{so}(2n+1 , \mathds{C}) \). Les \(C_n \) c'est pour \(\mathfrak{sp}(2n \mathds{C})\)

\[ B_2 = C_2 \implies \mathfrak{so(?)} = \mathfrak{sp}(4, \mathds{C})\]

\section*{Construction de \(	\mathfrak{g}_2\)}
\end{document}

