\documentclass{article}    
\usepackage[utf8]{inputenc}    
    
\title{Épisode 4}    
\author{Jean-Baptiste Bertrand}    
\date{\today}    
    
\setlength{\parskip}{1em}    
    
\usepackage{physics}    
\usepackage{graphicx}    
\usepackage{svg}    
\usepackage[utf8]{inputenc}    
\usepackage[T1]{fontenc}    
\usepackage[french]{babel}    
\usepackage{fancyhdr}    
\usepackage[total={19cm, 22cm}]{geometry}    
\usepackage{enumerate}    
\usepackage{enumitem}    
\usepackage{stmaryrd}    
\usepackage{mathtools,slashed}
%\usepackage{mathtools}
\usepackage{cancel}
    
\usepackage{pdfpages}
%packages pour faire des math    
%\usepackage{cancel} % hum... pas sur que je vais le garder mais rester que des fois c'est quand même sympatique...
\usepackage{amsmath, amsfonts, amsthm, amssymb}    
\usepackage{esint}  
\usepackage{dsfont}

\usepackage{import}
\usepackage{pdfpages}
\usepackage{transparent}
\usepackage{xcolor}
\usepackage{tcolorbox}

\usepackage{mathrsfs}
\usepackage{tensor}

\usepackage{tikz}
\usetikzlibrary{quantikz}
\usepackage{ upgreek }

\newcommand{\incfig}[2][1]{%
    \def\svgwidth{#1\columnwidth}
    \import{./figures/}{#2.pdf_tex}
}

\newcommand{\cols}[1]{
\begin{pmatrix}
	#1
\end{pmatrix}
}

\newcommand{\avg}[1]{\left\langle #1 \right\rangle}
\newcommand{\lambdabar}{{\mkern0.75mu\mathchar '26\mkern -9.75mu\lambda}}

\pdfsuppresswarningpagegroup=1


\begin{document}
2024-01-22

\section*{Rappel}

Un \underline{morphisme}  de représentation est une application linéaire $\varphi:V \to U$ (qui est compatible avec les deux représentation) t.q.

\[ \rho_2 (g) \circ \varphi = \varphi \circ \rho_2 (g) \]
^
$\varphi$ est appelée une application \underline{équivariante} 

\underline{Lemme de Shur} 

\begin{enumerate}
\item	Si $\rho_1,\, \rho_2$ sont irréductible $\varphi$ morphisme $\implies \varphi = 0$ ou isomorphe

\item Si $V=U$ alors $\varphi = \lambda \mathds{1}$
\end{enumerate}

\underline{Prop:} Tout représentation irréductible d'un groupe abélien est de dimension (rang) 1.

Les repr ??? de $S_3$ (à iso près) sont $\rho_?, \rho_? \qq{et} \rho_?$

\section*{Caractère d'une représentation:}

\[ \chi_{\rho} : G \to \mathds{C}  \]
\[ g \mapsto \tr(\rho (g)) \]

$\chi_{\rho} $ est un exemple de fonction \underline{centrale} (class function) c-à-d $\forall h \in Ga,\, \chi_{\rho} (h g h^{-1}) = \chi_{\rho} (g)$

Dans $S_n$ permutation de $n$ éléments la conjugasion correspond à un "changement d'étiquette" 

La \underline{tables des caractères} d'un groupe fini $G$ est un  tableau où les \underline{lignes} sont les représentations irréductibles et les \underline{colonnes} sont les calsses de conjugaison dans $G$. Les entrées sont $\chi_\rho(g)$   

\underline{Exemple:} $S_3$ 

\begin{table}[htpb]
	\centering
	\label{tab:car_s3}

	\begin{tabular}{c|c|c|c}
	&	1 & 3 & 2 \\
	&	e & (12) & (123)\\\hline
		$\chi_?$ & 1 &1 &1 \\\hline 
		$\chi_?$ & 1 & -1 & 1 \\ \hline
		$\chi_{\rho_{\text{std}}}$ & 2 &0 & -1
	\end{tabular}
	\caption{tables des caractères de $S_3$}
\end{table}

\begin{tcolorbox}[title=Remarques]
	\begin{itemize}
		\item Dans la première colonne on lit les dimensions des représentation irréductible  
		\item les colonnes sont orthogonales par le produit scalaire standard 
		\item Autant de lignes que de colonnes
		\item chaque lignes est un vecteur de norme $\abs{G} $
	\end{itemize}
\end{tcolorbox}

\underline{Exemple:} $\mathds{Z}_{4}$ 

\begin{table}[htpb]
	\centering
	\label{tab:label}

	\begin{tabular}{c|c|c|c|c}
	& 1 & 1 & 1 &1 \\ 
	& 0 & 1 & 2 & 3\\ \hline
		$\chi_?$ &1 &1&1&1 \\\hline
		$\chi_?$ & 1 & i & -1 & -i \\\hline
		$\chi_?$ & 1 & -1 & i & -1 \\\hline
		$\chi_?$ & 1 & -i & -1 & i
	\end{tabular}
	\caption{Table des caractères de $\mathds{Z}_{4}$}
\end{table}

\section*{Rappels et suppléments d'algèbre linéaire}

$V$ un $(k)$espace vectoriel est un groupe abélien muni d'une multiplication par un scalaire
\[ k \times V <to V \]
\[ (\lambda, \vb{v}) \mapsto \lambda \cdot \vb{v} \]

satisfaisant 
\begin{enumerate}
	\item $(\lambda \vb{u}) \cdot \vb{v} = \lambda \cdot (u \cdot \vb{v})$
	\item $1 \cdot \vb{v} = \vb{v}$
	\item $\lambda(u +v) = \lambda u + \lambda v$
	\item $(\lambda + \mu ) = \lambda v + \mu v$
\end{enumerate}


Soit $U,V$ deux k-espaces vectoriels 

\[ \rm{Hom}(U,\, V) := \{ L:U \to V | L \text{application linéaire}  \}  \] est un k-espace vectoriel lorsque muni des opérations

\[ (L_1 + L_{2)} (u) = L_1 (u) + L_2 (u) \]
\[ (\lambda\cdot L ) (u) = \lambda \cdot (L(u)) \]
\[ \rm{dim}(\rm{Hom} (u,v)) = \rm{dim} (u) \rm{dim} (v) \]
	
Le produit Tensoriel de $U$ et $V$ est un $k$-espace vectoriel $U \otimes V$ muni d'une application bilinéaire 

$U \times V \to U \otimes V$

$(u,v) \mapsto u \otimes v$

et satisfaisant la \underline{propriété universelle} : Pour tout application bilinéaire $b: U \times V \to W$

Je vois pas ... 


\underline{En pratique}: Si $e_1 , \dotsb , e_n $ est une base de $U$, $f_1, \dotsb, f_m $ est une base de $V$ alors $\{ e_i \otimes f_g \} $ est une base de $U \times V$ 

\underline{Exemple:} 

J'ai pas envie de l'écrire

\[ \begin{pmatrix} a \\b  \end{pmatrix} \otimes \begin{pmatrix} c \\ c \end{pmatrix} = \dotsb ac e_1 \otimes f_1  + \dotsb \]

\underline{Exemple:}  produit scalaire standard dans $\mathds{C}^{2}$ est bilinéaire ($(\begin{pmatrix} a\\b \end{pmatrix},\, \begin{pmatrix} c\\d \end{pmatrix} ) \to ac+ bc$)

Quelle est $\bar b \mathds{C}^{2} \otimes \mathds{C}^{2} \to \mathbb{C}$

$(\begin{pmatrix} a\\b \end{pmatrix} \otimes \begin{pmatrix} c\\d \end{pmatrix} ) \to ac+ bc$

\begin{tcolorbox}[title=Attention]
	Il est des éléments de $\mathds{C}^{2} \otimes \mathds{C}^{2}$ qui n'écrivent pas comme des états factorisables 
\end{tcolorbox}




\end{document}
