\documentclass{article}    
\usepackage[utf8]{inputenc}    
    
\title{Épisode 4}    
\author{Jean-Baptiste Bertrand}    
\date{\today}    
    
\setlength{\parskip}{1em}    
    
\usepackage{physics}    
\usepackage{graphicx}    
\usepackage{svg}    
\usepackage[utf8]{inputenc}    
\usepackage[T1]{fontenc}    
\usepackage[french]{babel}    
\usepackage{fancyhdr}    
\usepackage[total={19cm, 22cm}]{geometry}    
\usepackage{enumerate}    
\usepackage{enumitem}    
\usepackage{stmaryrd}    
\usepackage{mathtools,slashed}
%\usepackage{mathtools}
\usepackage{cancel}
    
\usepackage{pdfpages}
%packages pour faire des math    
%\usepackage{cancel} % hum... pas sur que je vais le garder mais rester que des fois c'est quand même sympatique...
\usepackage{amsmath, amsfonts, amsthm, amssymb}    
\usepackage{esint}  
\usepackage{dsfont}

\usepackage{import}
\usepackage{pdfpages}
\usepackage{transparent}
\usepackage{xcolor}
\usepackage{tcolorbox}

\usepackage{mathrsfs}
\usepackage{tensor}

\usepackage{tikz}
\usetikzlibrary{quantikz}
\usepackage{ upgreek }

\newcommand{\incfig}[2][1]{%
    \def\svgwidth{#1\columnwidth}
    \import{./figures/}{#2.pdf_tex}
}

\newcommand{\cols}[1]{
\begin{pmatrix}
	#1
\end{pmatrix}
}

\newcommand{\avg}[1]{\left\langle #1 \right\rangle}
\newcommand{\lambdabar}{{\mkern0.75mu\mathchar '26\mkern -9.75mu\lambda}}

\pdfsuppresswarningpagegroup=1


\begin{document}
2024-01-15

\section*{retour sur le dernier cours}
\[ (G, \bullet) \quad \text{c'est un groupe} \]

\[ S_E = \{ \sigma: E \to E | \sigma \text{ inversible }  \}  \quad \text{ est une groupe pour la composition } \]


Un \underline{cycle} est un élément de $S_n$ de la forme \[ \sigma(a_1 ) = a_{i\neq1},\, \sigma(a_{k}) = a_{1,\,} i = 1, \dotsb, k \] 
On le note $(a_1 \, a_2 \, a_3 \, \dotsb\, a_k)$

\begin{tcolorbox}[title=Fait important]
	Toute permutation se décompose de manière unique en cycles disjoint 
	\underline{Exemple:}
	\[ \sigma = \begin{pmatrix} 1 & 2 &3 &4 &5 \\ 2 & 1 &5 &4 &3 \end{pmatrix} = (1\, 2) \circ (3\, 5) = (3\, 5) \circ (1\, 2) \]
	\[ \begin{pmatrix} 1 & 2 & 3 &4 &5 &6 &7\\ 7 &3 &4&1 &6 &2 &5 \end{pmatrix}  = (1\, 7\, 5\,6\,2\,3\,4) \]
\end{tcolorbox}


Le signe (ou la signature) d'un cycle de longeur $\ell$ est \[ 	(-1)^{\ell -1} \begin{cases} +1: \text{la permutation est paire}\\ -1: \text{la permutation est imparire}  	
\end{cases}\]


On a la relation $\rm{sgn}(\sigma_1 \circ \sigma_{2}) = \rm{sgn}(\sigma_1 ) \rm{sgn}(\sigma_2 )$

On peut utiliser une manière graphique pour calculer la signature d'une permutation (graph: compter le nombre d'intersections)

Action de $G$ sur $X$: deux définitions 

\begin{enumerate}
	\item $\bullet: G \times X\to X $
	\item homomorphisme $f: G \to S_x$
\end{enumerate}

\underline{Représentation de $G$:} action linéaire de $G$ sur un espace vectoriel $V$ 

\underline{Exemple:} La \underline{Représentation vectoriel} sur $V$  

\[ g \circ \vb{v} = \vb{v} \quad \forall g \in G,\, v\in V \]
\[ \rho: G \to GL(V) \]
\[ g\mapsto \mathds{1} \]


Pour $G$ fixé, on a la représentation \underline{régulière} ($R$) (pour chaque élément du groupe on a un vecteur) 

\[ \expval{e_{g_{1},}  \dotsb, e_{g_n} }  \qq{où} G = \{ g_1, \dotsb, g_n  \} \]

On définit $g \bullet e_g = e_{g \bullet g}$

\underline{Exemple:} \[ \mathds{Z}_3 = \{ 0,1,2 \}  \] 
\[ V = \expval{e_0 \, e_1 \, e_2 } \]
\[ R(0) = \begin{pmatrix} 1 & 0  &0 \\ 0 & 1 & 0 \\ 0 & 0 &1 \end{pmatrix}  \]

\[ R(1) = \begin{pmatrix} 0 & 0  &1 \\ 1 & 0 & 0 \\ 0 & 1 &0 \end{pmatrix}  \]


\[ R(2) = \begin{pmatrix} 0 & 1  &0 \\ 0 & 0 & 1 \\ 1 & 0 &0 \end{pmatrix}  \]

Les éléments du groupe $\mathds{Z}_3$ sont ici representé par les matrices 3x et l'addition (modulaire) est remplacé par la multiplication matriciel des éléments de la représentation.

\underline{Autre exemple:} 

\[ S_3 = \{ e, (12), (13), (23), (123), (132) \}  \]

\[ R(12)  = \begin{pmatrix} 0 & 1 & 0 & 0 &0 &0 \\ 1 & 0&0 &0&0 &0\\ 0 &0 &0 &0 &0 & 1\\ 0&0&0&0&1&0 \\ 0&0&0&1&0&0 \\ 0&0&1&0&0&0 \end{pmatrix} \]

Plus généralement , si $G$ agit sur $E$ (ensemble fixé), on définit une \underline{représentation de permutation} sur $\expval{e_1 , e_2 , \dotsb , e_{n}} \quad E = \{ e_1 , \dotsb , e_n \} $ par $\rho(g)(e_i ) = g \bullet e_1 \qq{(action de $G$ sur $E$)}$

\underline{exemple:} $V = \mathds{C}$ Ou on prend $\mathds{C}$ comme un espace vectoriel 


\[ G = \mathds{Z}_3 \]
\[ \rho : \mathds{Z}_3 \to \mathds{C}^{*} = \rm{GL(1, \mathds{C})} \]
\[ n \mapsto \omega^{n}\qq{où} \omega = e^{2\pi i /3} \]

\underline{Définition:} Un \underline{sous-représentaation} de \[ \rho: G \to \rm{GL}(V) \]  est la restriction de $\rho$ à un sous-espace $U\subset V$ invariant par $\rho$. c-à-d, si $u \in U$, alors $\rho(g) u \in U\forall g \in G$

\underline{Exemple:} Pour $R: S_3 \to \rm{GL}(6, \mathds{C})$ 
Le sous-espace $\{ \begin{pmatrix} z\\z\\z\\z\\z\\z \end{pmatrix} \in \mathds{C}^6 | z \in \mathds{C} \} $ est une sous représentation \underline{triviale} 

Le sous-espace $U_0 = \left\{ \begin{pmatrix} z_1 \\ z_2 \\ \vdots \\ z_6 \end{pmatrix} \in \mathds{C}^6 | z_1 + z_2 + \dotsb + z_6 = 0 \right\} $	est aussi une sous-représentation de $R$ de dimension 5


\underline{Définition:} Une représentation est \underline{irréductible} si elle n'admet aucune sous représentation \underline{propre} ($\neq 0, \neq V$)   


\underline{Exemple:} $S_3$:

$\rho: S_3 \to \rm{GL}(3, \to \mathds{C})$ la représentation de permutation induite par l'action \underline{???} de $S_3$ sur $\{ 1,2,3 \} $ 
$\rho(12) = \dotsb 3x3,\, \rho(123) = \dotsb 3x3$

$\rho$ est elle irréductible ? non, 

\[ \expval{ \begin{pmatrix} 1 \\ 1 \\1 \end{pmatrix}  } = \{ 	\begin{pmatrix} z\\z\\z \end{pmatrix} \in \mathds{C}^3 | z \in \mathds{C} \}  \]
est invariant est irréductible

Également, $U_0 = \expval{\begin{pmatrix} 1 \\ -1 \\ 0 \end{pmatrix} , \begin{pmatrix} 0 \\ 1 \\ -1 \end{pmatrix}   } =  \{ \begin{pmatrix} z_1 \\ z_2 \\ z_3 \end{pmatrix} | z_1 + z_2 +z_3 = 0 \} $ est invariant


Es-ce que $U_0$ est irréducibleÉ

Cherchons un sous-espace invariant de dim 1

\[ \rho (12) \begin{pmatrix} z_1 \\ z_2 \\ z_3 \end{pmatrix}  = \begin{pmatrix} z_2 \\ z_1 \\ z_1 - z_2 \end{pmatrix} = \lambda  \begin{pmatrix} z_1 \\ z_2 \\ -z_1 - z_2 \end{pmatrix}   \]

\[ \dotsb \]

Conculsion : $U_0$ est une représentation irréductible. On l'appelle repreésentation \underline{standard} de $S_3$

\underline{Ex:} $S_3$


\[ \rm{sgn}: S_3 \to \mathds{C}^{*} = \rm{GL}(1, \mathds{C}) \]
\[ \sigma \mapsto \rm{sgn}(\sigma) \]

Si $\rho_{1:} G \to \rm{GL}(u), \quad \rho_{2:} G \to \rm{GL}(v)$ sont 2 représentation de $G$, leurs \underline{somme directe} est la représentation $\rho_1 \oplus \rho_{2:} G \rm{GL}(u \oplus v)$ 

\[ (\rho_1 \oplus \rho _2 )(g) (u \oplus v) = \rho_1 (g) u \oplus \rho_2 (g) v \]


\underline{Exemple:} si $U = \mathds{R}^{n}$ $V = \mathds{R}^{m}$ 

\[ U \oplus V  = \mathds{R}^{n+m}\]

$U \oplus v$ contient $u \oplus 0$ et $0 \oplus v$ comme sous représentation 


\underline{Proposition}: Soit $U \subset V$ une sous-repr/sentation de $\rho: G \to \rm{Gl}(V)$. Alors, il existe une sous-représentation $W\subset V$ telle que $V = U \oplus W$  


\begin{tcolorbox}[title=Attention!]
	 
	Faux en général pour les groupes infinis
\end{tcolorbox}


\underline{Exemple:} $\rho: \mathds{Z} \to \rm{GL}(2, \mathds{C})$
\[ n \mapsto \begin{pmatrix} 1 & n \\ 0 & 2 \end{pmatrix}  \]
est une représentation de $\mathds{Z}$, $\expval{e_1}$ est une sous-représentation triviale, mais il n'en existe par d'autre 

\[ \begin{pmatrix} 1 &0 \\ 0 & 1 \end{pmatrix} \begin{pmatrix} x \\ y \end{pmatrix}  = \begin{pmatrix} x+y \\ y \end{pmatrix}   \]

\underline{Démonstration:} 

Soit $V_0 \subset V$ n'importe quel complément de $U$ ($V = U \oplus W_0 $)

Ce n'est \textbf{pas} un sous-espace en général

\[ \rho(g) w \notin W_o  \qq{pour} w \in W_0\]

Soit $\pi: V \to U$ la projection complémentaire à $W_0$

Définissons $\pi' = \frac{1}{\abs{G}} \sum_{g\in G}^{} \rho(g) \circ \pi \circ \rho (g^{-1})$
si $u \in U$
\[ \pi' (u) = \frac{1}{\abs{G}} \sum_{g\in G}^{\infty} \rho(g) \pi \left[ \rho (g') u \right]   \]
\[ \frac{1}{\abs{G}} \sum_{g\in G}^{} \cancel{\rho(g) \rho(g^{-1})} u  \]

 \[ 	\frac{1}{\abs{G}} \abs{G} u = u   \]

 \[ \implies \pi' : V\to U \qq{est surjectif et indentité sur $$} \]

 \[ W = Ker(\pi') \qq{est notre candidat de sous-représentation}\]

Vérifions que $W$ est $\rho(G)$ invariant

\[ h \in G \quad V \in \rm{Ker} \pi' \]

\[ \pi'(\rho(h) V) = \frac{1}{\abs{G}} \sum_{g \in G}^{\infty} \rho(g) \pi \rho(g') \rho(h) v  = \dotsb = 0\]          

comme $\pi'/ i = \mathds{1}_u$

\[ U \cup ,,,,,, \]



\end{document}
