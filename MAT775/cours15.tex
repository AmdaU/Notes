\documentclass{article}    
\usepackage[utf8]{inputenc}    
    
\title{Épisode 4}    
\author{Jean-Baptiste Bertrand}    
\date{\today}    
    
\setlength{\parskip}{1em}    
    
\usepackage{physics}    
\usepackage{graphicx}    
\usepackage{svg}    
\usepackage[utf8]{inputenc}    
\usepackage[T1]{fontenc}    
\usepackage[french]{babel}    
\usepackage{fancyhdr}    
\usepackage[total={19cm, 22cm}]{geometry}    
\usepackage{enumerate}    
\usepackage{enumitem}    
\usepackage{stmaryrd}    
\usepackage{mathtools,slashed}
%\usepackage{mathtools}
\usepackage{cancel}
    
\usepackage{pdfpages}
%packages pour faire des math    
%\usepackage{cancel} % hum... pas sur que je vais le garder mais rester que des fois c'est quand même sympatique...
\usepackage{amsmath, amsfonts, amsthm, amssymb}    
\usepackage{esint}  
\usepackage{dsfont}

\usepackage{import}
\usepackage{pdfpages}
\usepackage{transparent}
\usepackage{xcolor}
\usepackage{tcolorbox}

\usepackage{mathrsfs}
\usepackage{tensor}

\usepackage{tikz}
\usetikzlibrary{quantikz}
\usepackage{ upgreek }

\newcommand{\incfig}[2][1]{%
    \def\svgwidth{#1\columnwidth}
    \import{./figures/}{#2.pdf_tex}
}

\newcommand{\cols}[1]{
\begin{pmatrix}
	#1
\end{pmatrix}
}

\newcommand{\avg}[1]{\left\langle #1 \right\rangle}
\newcommand{\lambdabar}{{\mkern0.75mu\mathchar '26\mkern -9.75mu\lambda}}

\pdfsuppresswarningpagegroup=1


\begin{document}
2024-03-11


\begin{tcolorbox}[title=Rappels]
	\(\mathfrak{sl}(3 \mathds{C}) = h \oplus \bigoplus_{\alpha\in \Phi} \mathfrak{g}_\alpha  \) 
	osti, je suis deja done 
\[ \dotsb \]

On a montré que les poids diffèrent par une combinaison de racines:

Si \(v \in V_{\alpha}, C \in g_{\beta}\) \(\beta\)-racine, \(\alpha\)-poids

alors \( X \cdot v \in V_{\alpha+\beta} \)

Le \textit{poids le plus haut} est une poids maximal pour l'ordre induit l'évaluation sur \(\begin{pmatrix} a_0 \\ & b_0 \\ & & c_0  \end{pmatrix} \in h\) t.q. \(a_0 > b_0 > C_0 \)

Il existe un \underline{vecteur de plus haut poids} \(v\) qui satisfait 

\begin{itemize}
	\item \(v \in V_{\alpha} \) pour \(\alpha \in h^{*}\)
	\item \(E_{23} v = E_{13} v = E_? v  = 0\)
\end{itemize}

\end{tcolorbox}

\underline{Proposition}:

\(V\) est engendré par \(v\) (vecteurs de plus haut poids) et toutes ses images par \underline{tout les mots possible} en \(E_{2,1}, E_{3,2}, E_{3,1} \) 

\underline{Démonstration}

\(W\) le sous-espace engendré par \(v\) et tout les motes possibles en \(E_{2,1}, E_{32}, E_{31}  \) appliqué à \(V\)


\[ W = \expval{v, E_{21} v, e_{32} v, E_{31} v, E_{21} E_{32} v, \dotsb} \]

On veur montrer que \(W\) est \(\mathfrak{sl}(3, \mathds{C})\)-invarient 

Partie facile, \(W\) est invariant par \(h\) et par \(E_{21} , E_{31}, E_{32} \)

Reste à montrer que \(W\) est invarient par \(E_{1,2} , E_{2,3} \)


\(E_{1,3} = [E_{1,2}, E_{2,3}] \), il suffit donc de vérifier \(E_{1,2} W \subseteq W\) et \(E_{23} W \subseteq W\)

Posons \(W_n\) le sous-espace engendré par \(v\)a et tout les mots en \(E_{21}, E_{32}  \) de la longeure \(\leq n \) appliqué à \(v\)


Par récurence, on montre \(E_{12} \cdot W_n \subseteq W_{n-1} \), \(E_{2,3} \cdot W_{n\subseteq} W_{n-1} \)


Soit \(w\in W_n \)

\[ \implies w= E_{21}\cdot w' \qq{pour} w' \in W_n-1 \]
ou
\[ w = E_{32} \cdot w' \]

\begin{enumerate}
	\item \[ E_{1,2} \cdot w = E_{1,2}  \cdot E_{2,1} \cdot w' = \left( [E_{12}, E_{21}] + E_{21} \vdot E_{12}  \right) w'  \]
		\begin{tcolorbox}[title=]
			\[ E_{1,2} \in g_{L_1-L_2}  \]
			\[ E_{21} \in G_{L_2-L_1}  \]
			\[ \implies [E_{1,2}, E_{21} ] \in h = g_e   \] 
		\end{tcolorbox}
		\[ = \in W_{n-1} + \in W_{n-1}   \]


		\[ E_{2,3} \cdot w = E_{2,3} \cdot E_{1,2} \cdot w '  \]
		\[=\left( \underbrace{[E_{23}, E_{21} ]}_{0}  + E_{2,1} + E_{23}  \right) \cdot w' \]
		\[ =\underbrace{E_{21} \cdot \underbrace{\left( E_{21} \cdot w' \right)}_{W_{n-2} }}_{W_{n-1} }    \]
	\item même chose

Puisque \(W = \bigcup_n W_n \), \(W\) est stable par \(\mathfrak{sl}(3 \mathds{C})\) \(\implies W = V \) \(\blacksquare\)
\end{enumerate}


De la preuve, on déduit: 

Pour \(V\) une représentation (pas nécéssairement irréductible), si \(v\) est un vecteur de plus haut poidsm alors le sous espace engendré par \(v\) est ses images par \(E_{21} \) et \(E_{3,2} \) est une sous représentation irréductible


Il existe un \(n\) pour lequel \(\left( E_{2,1}  \right) ^n \cdot v = 0\) mais \(\left( E_{2,1}  \right) ^{n-1} \cdot v \neq 0\)

Observation: \(V_{\alpha+m(L_2-L_1)} \) est de dim 1 ou 0 (car il existe un seul \textit{chemin} entre \(\alpha \) et \(\alpha + m(L_2 - L_1 )\)

\begin{align*}
	\begin{matrix}
		E_{21} & E_{12} & E_{11} - E_{22} \\
		\begin{pmatrix} 0 & 0 & 0 \\ 1 & 0 &0 \\ 0&0&0 \end{pmatrix} & \begin{pmatrix} 0 & 1&0\\0&0&0\\0&0&0 \end{pmatrix} & \begin{pmatrix} 1 &0 &0\\ 0 &-1&0\\0&0&0 \end{pmatrix} \\
	Y & X & H
	\end{matrix}
\end{align*}

engendrent une sous-algèbre de Lie de \(\mathfrak{sl}(3 \mathds{C}) \) isomorphe à \(\mathfrak{sl}(2 \mathds{C})\)

En restreignant à cette sous-algèbre, on obtient une représentation de \(\mathfrak{sl}(2,\mathds{C}) \) sur \(V\) (par nécéssairement irréductible) 

\underline{Rappel} 
Les valeurs propres pour \(H\) dans un représentation de \(\mathfrak{sl}(2 \mathds{C})\) sont entière et symétriques par rapport à 0


Les valeurs propres de \("H" = E_{11} - E_{22} \) sont \(\alpha (H) , \left( \alpha + L_2 - L_1  \right) (H), \dotsb, \left( \alpha +n(L_2 - L_1) \right) (H)\)

on réécrit \(\alpha(H), \alpha(H) -2, \alpha(H) -4, \dotsb, \alpha(H) -2n \)

\[ \implies \alpha(H) -2 n = -\alpha (H) \]
\[ \implies n = \alpha(H)  \]

L'arrête entre \(\alpha\) et \(\alpha + n(L_2 - L_1 ) \) est symétrique par rapport à la droite \(\beta(H_{12} ) = 0 \)

Posons \(\alpha + \alpha \left( J_{1,2}  \right) \left( L_2 -L_1  \right) = \alpha_2  \) et \(v_2 = E_{2,1}^{???}\cdot v \in V_{\alpha_2} \)

On a \( E_{21} \cdot v_2 = 0  \), \(E_{2,3} \cdot  v_2 = 0\) , \(E_{1,2} \cdot v_2 =0 \)

\(v_2\) est une \textit{vecteur de plus haut poids} pour l'ordre définis par \(\begin{pmatrix} a\\&b \\ &&c \end{pmatrix} \), \(b>a>c\)

Les espaces de poids sont contenus dans l'hexagone  des sommets \(\alpha \) et ses réflexions dans les 3 droites 

Les espace de poids sur les arêtes sont de dimension 1

On déduit que \(\alpha(H)_{i,j} \in \mathds{Z} \forall H \in h\) 
\[ \implies \alpha = a L_1 + b L_2 + c L_3 \quad a,b,c\in \mathds{Z}\]

\end{document}
