\documentclass{article}    
\usepackage[utf8]{inputenc}    
    
\title{Épisode 4}    
\author{Jean-Baptiste Bertrand}    
\date{\today}    
    
\setlength{\parskip}{1em}    
    
\usepackage{physics}    
\usepackage{graphicx}    
\usepackage{svg}    
\usepackage[utf8]{inputenc}    
\usepackage[T1]{fontenc}    
\usepackage[french]{babel}    
\usepackage{fancyhdr}    
\usepackage[total={19cm, 22cm}]{geometry}    
\usepackage{enumerate}    
\usepackage{enumitem}    
\usepackage{stmaryrd}    
\usepackage{mathtools,slashed}
%\usepackage{mathtools}
\usepackage{cancel}
    
\usepackage{pdfpages}
%packages pour faire des math    
%\usepackage{cancel} % hum... pas sur que je vais le garder mais rester que des fois c'est quand même sympatique...
\usepackage{amsmath, amsfonts, amsthm, amssymb}    
\usepackage{esint}  
\usepackage{dsfont}

\usepackage{import}
\usepackage{pdfpages}
\usepackage{transparent}
\usepackage{xcolor}
\usepackage{tcolorbox}

\usepackage{mathrsfs}
\usepackage{tensor}

\usepackage{tikz}
\usetikzlibrary{quantikz}
\usepackage{ upgreek }

\newcommand{\incfig}[2][1]{%
    \def\svgwidth{#1\columnwidth}
    \import{./figures/}{#2.pdf_tex}
}

\newcommand{\cols}[1]{
\begin{pmatrix}
	#1
\end{pmatrix}
}

\newcommand{\avg}[1]{\left\langle #1 \right\rangle}
\newcommand{\lambdabar}{{\mkern0.75mu\mathchar '26\mkern -9.75mu\lambda}}

\pdfsuppresswarningpagegroup=1


\begin{document}
2024-01-29

\begin{tcolorbox}[title=Rappels]
	
	$P$ projection, apli linéaire $P: V \to V$ t.q. $P^{2}= P$
	
	
	\[ \tr(P) = \rm{dim}(\rm{Im}P) \]
	
	\[ \rho: G \to \rm{GL}(V) \]
	
	\[ P = \frac{1}{\abs{G}} \sum_{g\in G} \rho(g)   \]
	est une projection avec $\rm{Im}P = V^{G}= ?$
	
	\[ \rm{dim} V^{G}= \frac{1}{\abs{G}} \sum_{g \in G} \chi_{\rho}(?)  \]
	Nombre de représentation triviales dans les décomposition de $\rho$
	
	En particulier si $\rho$  est irréductible et non-trivial \[ \sum_{g \in G} \chi_{\rho} (g) = 0 \]
	 
\end{tcolorbox}

$\rho_{1,\,} \rho_2$ deux représentations et on s'intéresse à la représentation 

\[ \rm{Hom}(\rho_1,\, \rho_2 ): G \to \rm{GL}(\rm{Hom}(U,V)) \]

\begin{tcolorbox}[title=Rappel ]
	 Si $U = \mathds{C}^n,\, V = \mathds{C}^m$
	 \[ \rho_{1(g)} \in \rm{GL}_n (\mathds{C})\qquad \rho_{2(g)} \in \rm{GL}_m (\mathds{C})  \]
	 \[ \rm{Hom}(U,V) = \rm{Mat}_{n \times m}(\mathds{C}) \]
	 \[ \rm{Hom}(\rho_1 ,\rho_2 )(g) (M) = \rho_2 (g) \cdot M \cdot \rho_1 (g)^{-1}  \]
\end{tcolorbox}


\underline{Proposition}: 
\[ \rm{Hom}(U,V)^G  = \{ \varphi: u \to v | \text{$\varphi$ est une morphisme de représentation}  \} \]


\underline{Démonstration}:

\[ M \in \rm{Hom}(U,V)^G \iff \rho_2 M \rho_1 (g) \in v = M  \forall g \in G \iff \rho_2(g) M = M \rho_1 (g) \iff M \text{ est une morphisme de représentations } \]


Si $\rho_1,\, \rho_2$ sont irréductibles, le lemme de Shor dit 

\[ \rm{dim}(\rm{Hom}(U,V)^G ) = \begin{cases} 0 & \text{si} \rho_1 \ncong \rho_2 \\ 1 & \text{si} \rho_1 \cong \rho_2 
	
\end{cases} = \tr P = \tr( \frac{1}{\abs{G}} \sum_{g \in G} \rm{Hom}(\rho_1 ,\, \rho_2 )(g)  ) =  \frac{1}{\abs{G}} \sum_{g \in G} \tr \rm{Hom} (\rho_{1},  \rho_2 )(g) (\text{à démontrer} )\]

\[ =  \frac{1}{\abs{G}} \sum_{g \in G} \bar{\chi_{\rho} (g)} \]



\[ \therefore \frac{1}{\abs{G}} \sum_{g \in G} \bar{\chi_{\rho} (g)} \chi_{\rho} (g)  = \begin{cases} \dotsb
	
\end{cases}\]

Les caractères de représentations irréductibles sont \underline{orthonormés} par le produit scalaire 

\[ \expval{f_1 , f_{2}} = \frac{1}{\abs{G}} \sum_{g \in G} \bar{f_1 }(g) f_2 (g)   \]

sur l'espace $f: G \to \mathds{C}$


\begin{tcolorbox}[title=Exemple: $S_3$]
	\[ \rho_{^{\text{triv}}} = \frac{1}{6} \left( 1^2 + 3\cdot 1^2 + 2 \cdot 1^3 \right)  = 1 \qquad \dotsb\] 
\end{tcolorbox}
 
\[ \mathds{C}_C (G) = \{ f: G \to \mathds{C} | f(hgh^{-1}) = f(g) \forall g \in G \}  \]

\[ \rm{dim}(\mathds{C}_C (G)) = \text{\# de classes de conj} \]


\underline{Corrollaire} 

\[ \text{\# de repr irr homo-isomorphe de $G$} \leq \text{\# de classe de conj}   \]

(même $=$ mais ça reste à démontrer!)

\underline{Démonstration}: (je vois pas lol) 

\underline{Corrollaire 2}: Toute représentation est derterminé (à iso près) par son caractère $\chi_{\rho} $ 

\underline{Démonstration}: On sait que $\rho = \rho_1^{m_{1}} \oplus \dotsb \oplus \rho_k^{m_k}$ 

De plus $ \chi_{\rho} = m_1 \chi_{\rho_{1}} + m_2 \chi_{\rho_{2}} + \dotsb + m_k \chi_{\rho_k}  $

On peut retrouver $m_i$ avec le produit scalaire 

\[ \expval{\chi_\rho, \chi_{\rho_i} } = m_i  \]

\begin{tcolorbox}[title=Exemple]
	Décomposons $R:S_3 \to \rm{GL}( \mathds{C}^{6} )$ (la repr régulière) en irréductible 

	\begin{itemize}
		\item $\chi_R (e) =6,\, \chi_R (12) = 0, \chi_R (123) = 0$ (les générateurs n'ont pas de points fixes)
		\item $\expval{\chi_R, \chi_{\text{triv}} } = \frac{1}{6} (6 + 0 + 0)$
			\[ \expval{} = \frac{1}{6} (6 + 0 + 0) \]
			\[ \expval{} = \frac{1}{6} (6*2 + 0 + 0) \]

			\[ \implies \chi_R = \chi_{\text{triv}}  + \chi_? + 2 \chi_? \]
	\end{itemize}

\end{tcolorbox}

\begin{tcolorbox}[title=Exemple]
	 Décomposons $\rho: S_3 \to \rm{GL}(\mathds{C}^3)$ la représentation de permutation canonique 
	 \begin{itemize}
	 	\item \[ \chi_{\rho}(e) = 3 \quad \chi_{\rho} (12) =1 \quad \chi_{\rho}(123) = 0  \]

			\[ \chi_{\rho} = \chi_{\text{triv}} + \chi_{\text{std}}   \]
			\[ \rho = \rho_{std} \oplus \rho_{\text{triv}}  \]
	 \end{itemize}
\end{tcolorbox}

\begin{tcolorbox}[title=]
	Calculons $\rho_{\text{std}} \otimes \rho_{\text{std}}  $
	
	\[ \text{(J'ai pas envie d'écrire des matrices à la main)}  \] 
\end{tcolorbox}


\underline{Corollaire 3}: $\rho$ est irréductible ssi $\expval{\chi_{\rho,} \, \chi_{\rho} } = 1$

\underline{Démonstration}: 

\[ \expval{\chi_{\rho} ,\, \chi_{\rho} } = m_1^{2}+ \dotsb + m_k^{2} = 1 \]

puisque $m_i \in \mathds{N}$, un des $m_i =1 $, tout les autres =0

\[ \iff \chi_{\rho} = \chi_{\rho,i}: \text{irréductible}  \]


\underline{Corollaire 4}:

Tout représentation irréductible apparait dans les décompostion de $R$ avec multiplicité $\rm{dim} \rho_i $ et $\abs{G} (= \rm{dim}(R)) = \sum_{\rho_i \text{irre} } \rm{\dim}(\rho_i ) ^2   $




\end{document}
