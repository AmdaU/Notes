\documentclass{article}    
\usepackage[utf8]{inputenc}    
    
\title{Épisode 4}    
\author{Jean-Baptiste Bertrand}    
\date{\today}    
    
\setlength{\parskip}{1em}    
    
\usepackage{physics}    
\usepackage{graphicx}    
\usepackage{svg}    
\usepackage[utf8]{inputenc}    
\usepackage[T1]{fontenc}    
\usepackage[french]{babel}    
\usepackage{fancyhdr}    
\usepackage[total={19cm, 22cm}]{geometry}    
\usepackage{enumerate}    
\usepackage{enumitem}    
\usepackage{stmaryrd}    
    
%packages pour faire des math    
%\usepackage{cancel} % hum... pas sur que je vais le garder mais rester que des fois c'est quand même sympatique...
\usepackage{amsmath, amsfonts, amsthm, amssymb}    
\usepackage{esint}  


\begin{document}
2024-02-29

\[ {\rm{}sl}(3, \mathds{C}) = h \oplus_{\alpha} \mathfrak{g}_\alpha \]

\[ h = \{ \begin{pmatrix} a \\ & b \\ &&c \end{pmatrix} | a+b+c =0 \}  \]
où \(\forall X \in \mathfrak{g}_\alpha, \forall H \in h \)
\[ {\rm{}ad} (H) X = [H, X] = \alpha (H) X  \]

\underline{exemple}:

\[ X =E_{1,2}  \]

\[ \left[ \begin{pmatrix} a \\ &b \\&&v \end{pmatrix}, E_{1,2}  \right] (a-b) E_{1,2} \]

\[ X \in g_{\alpha} \qq{où} \alpha\begin{pmatrix} a \\ & b\\ &&c \end{pmatrix} = a -b \]

On définit \(L_i \begin{pmatrix} a \\ &b \\&&c \end{pmatrix} = i\)

\[ L_1 , L_2 , L_3 \in h^{*} \]

\[ \alpha = L_1 - L_2 \]

les \(\alpha\) dans la décomposition (\(*\)) s'apellent des \underline{racines} de \[ {\rm{}sl(3 \mathds{C})} \] 

La liste des racines et 
\[ L_1-L_2 , L_1 - L_3 , \dotsb\]

\begin{tcolorbox}
	dans \({\rm{}sl}(2, \mathds{C})\), une racine est un nombre complex car \({\rm{}dim}(h)=1\). Les racines de \[ {\rm{}sl(2,\mathds{C})} \]  sont -2 et 2
\end{tcolorbox}

Les vecteur propres associé à une racine s'apellent des vecteurs de racine

\[ E_{i,j},\, i\neq j  \] est un vecteur de racine pour \(L_i -L_j \)

Supposons que \(X \in g_{\alpha} \) et \(Y \in g_{\beta} \) et \(H \in h\)

\[ [H, [X, Y] ] = [X, [H, Y] ] + [Y, [X, H] ] = [X, \beta(H) Y] - [Y, \alpha(H)X] =  \beta(H) [X, Y] - \alpha(H) [Y, X] = \left( \alpha + \beta \right) (H) [X, Y]   \]


Si \(X\) vecteur de racine \(\alpha\), \(Y\) vecteur de racine \(\beta\) alors \([X, Y] \) vecteur de racine \(\alpha + \beta\)

\({\rm{}ad}(X) \) agit \textit{par translation} de la racine de \(Y\)

\[ [{}, {}]: g_{\alpha} \times  g_{\beta} \to g_{\alpha+\beta}  \]

Revenons 'a une représentation irréductible \(V\) de \({\rm{}sl}(3 \mathds{C})\)

\[ \rho: {\rm{}sl(3, \mathds{C})} \to {\rm{}gl}(V) \]


On décompose \(V = \oplus_{\alpha} V_{\alpha}  \) où \(\alpha \in h ^{*}\) et \(v \in V_{\alpha} \), \( H \in h\)
\[ \implies H_v = \alpha(H) v \]

Les valeurs propres \(\alpha\) s'apellent les racines de la représentation. Les vecteur prorpres sont des \underline{vecteur de poids} 


Une racine est donc un poids pour la représentation ad

soit \(X \in g_{\alpha} \) et \(v \in V_{\beta} \)

\[ H \cdot (Xv) = [H, X] \cdot v + X \cdot (H \cdot v) = \alpha(H) Xv + C(\beta(H) v) = \left( \alpha + \beta \right) (H) (Xv ) \]

\(X \in g_{\alpha} \) agit par translation de \(\alpha\) sur le poids \(\beta\) de \(V\)

\underline{Conséquence:} Pour \(V\) irréductible, tout les poids diffèrent d'une combination entière de racine de \(L_i - L_j \) 

Le réseau \(\Lambda_R\) engendré par les racines est appelé réseaux des racines. 

\underline{Exemple}: 
\(V \in \mathds{C}^3 \) et \(\rho: {\rm{}sl}(3, \mathds{C}) \to {\rm{}gl}(\mathds{C}^3) \) l'inclusion  
\(e_1 , e_2 , e_3\) does des vecteurs propres de poids pour les poids \(l_1, L_2 , L_3 \)

\[ \begin{pmatrix} a \\ & b \\ &&c  \end{pmatrix} \begin{pmatrix} 1 \\0 \\0  \end{pmatrix} = a \begin{pmatrix} 1 \\0 \\0 \end{pmatrix} = L_1 (\begin{pmatrix} a\\&b \\&&c \end{pmatrix} ) \begin{pmatrix} 1 \\0 \\0  \end{pmatrix}  \]

En effet, \(L_2 , = L=1 + \left( L_2 - L_1 \right) \)

\[ L_3 = L_1 + \left( L_3 - L_1  \right)  \]

\underline{Exemple 2}:

\(\Lambda^{2} (\mathds{C}^3) = \expval{e_1 \wedge e_{2,} e_1, \wedge e_2, e_2 \wedge e_3 }\)


\[ \begin{pmatrix} a\\ &b \\&&c  \end{pmatrix} \left( e_1 \wedge e_2  \right) = a e_1 \wedge e_2 +  b e_1 \wedge e_2 = \dotsb = -L_3 \begin{pmatrix} a \\& b\\&&c \end{pmatrix}  \]

\[ \dotsb \]


Pour imiter ce qu'on a fait dans \({\rm{}sl(2, \mathds{C})}\) on cherche un poids \textit{maximal}. On définit la maximalité. On fixe \[ H_0 \begin{pmatrix} a\\&b\\&&c \end{pmatrix} \in h  \] et on considère l'ordre partiel sur \(h^{*}\) \[ \alpha < \beta \iff \Re(\beta(H_0 ) - \alpha(H_0 )) > 0  \] 

En choisissant \(a > b > c\), les racines \(L_1 - L_2 , L_1 - L_3 , L_2 - L_3  \) sont positives alors que les trois autres sont négatives 

\underline{Lemme}: Pour \(V\) une représentation irréductible de \({\rm{}sl}(3 \mathds{C}) \), il existe un vecteur de poids\( v \in V_{\alpha} \), \(v\neq 0\) t.q. \(E_{1,2} (v) = 0, E_{1,3} (v) =0, E_{2,3} (v) =0 \)


\underline{Démonstration}: Soit \(\alpha\) maximal parmis les poids t.q. \(V_{\alpha} \neq \{ 0 \}  \) par l'ordre \(<\) \(\alpha\) existe car \(V\) est de dimension finie. Soit \(v \in V_{\alpha} \). Alors, \(E_{1,2} \cdot V \in V_{\alpha+L_1-L_2} \)

Si \(E_{1,2} v \neq 0 \) alors \(\alpha + L_1 +L_2 > \alpha\) et \(v_{\alpha} \neq 0\) contredit la maximalité

De même \( E_{1,3} v= 0, \quad E_{2,3} v = 0\)


ON appelle v un \underline{vecteur de plus haut poids}  ou \underline{vecteur maximal}  

\underline{Proposition}: \(V\) est engendré par \(v\) et toutes les images de \(v\) par toutes les combination possibles de \(E_{2,1} ,\, E_{3,2},\, E_{3,1} \) 


\underline{Démonstration}: Soit \(W\) le sous-espace engendré par \(V\) et toutes ses images par des combinaisons de \(E_{21},\, E_{3,2}, \, E_{3,1}\)

Il suffit de montrer que \(W\) est stable par \({\rm{}sl}(3 \mathds{C})\)

\begin{enumerate}
	\item \(W\) est stable par \(h\) (\(W\) est engendré par des espaces de poids)
	\item \(W\) est stable par \(E_{2,1}, \, E_{3,2},\, E_{3,1}  \) par définition 
	\item Il reste à montrer que \(W\) est stable par \(E_{1,2},\, E_{2,3},\, E_{3,2}  \). Il suffit de le montrer pour \(E_{1,2} \) et \( E_{2,3} \) car \(E_{1,3} = [E_{1,2}, E_{2,3}] \) 
\end{enumerate}

\begin{center}
	\huge\textbf{À suivre... }	
\end{center}

\end{document}
