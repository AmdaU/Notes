\documentclass{article}    
\usepackage[utf8]{inputenc}    
    
\title{Épisode 4}    
\author{Jean-Baptiste Bertrand}    
\date{\today}    
    
\setlength{\parskip}{1em}    
    
\usepackage{physics}    
\usepackage{graphicx}    
\usepackage{svg}    
\usepackage[utf8]{inputenc}    
\usepackage[T1]{fontenc}    
\usepackage[french]{babel}    
\usepackage{fancyhdr}    
\usepackage[total={19cm, 22cm}]{geometry}    
\usepackage{enumerate}    
\usepackage{enumitem}    
\usepackage{stmaryrd}    
\usepackage{mathtools,slashed}
%\usepackage{mathtools}
\usepackage{cancel}
    
\usepackage{pdfpages}
%packages pour faire des math    
%\usepackage{cancel} % hum... pas sur que je vais le garder mais rester que des fois c'est quand même sympatique...
\usepackage{amsmath, amsfonts, amsthm, amssymb}    
\usepackage{esint}  
\usepackage{dsfont}

\usepackage{import}
\usepackage{pdfpages}
\usepackage{transparent}
\usepackage{xcolor}
\usepackage{tcolorbox}

\usepackage{mathrsfs}
\usepackage{tensor}

\usepackage{tikz}
\usetikzlibrary{quantikz}
\usepackage{ upgreek }

\newcommand{\incfig}[2][1]{%
    \def\svgwidth{#1\columnwidth}
    \import{./figures/}{#2.pdf_tex}
}

\newcommand{\cols}[1]{
\begin{pmatrix}
	#1
\end{pmatrix}
}

\newcommand{\avg}[1]{\left\langle #1 \right\rangle}
\newcommand{\lambdabar}{{\mkern0.75mu\mathchar '26\mkern -9.75mu\lambda}}

\pdfsuppresswarningpagegroup=1


\begin{document}
2024-02-01

\begin{tcolorbox}[title=typo devoir 1]
	2.1 \[ \Lambda^{n} = \{  \alpha \in V^{\otimes n} | \sigma \bullet \alpha  = ?(\sigma) \alpha\}  \] 
Exemples: \[ 	\mathds{R}^{2} = \expval{e_1 \, e_2 } \]
\[ \rm{Sym}(\mathds{R}^{2}) \ni  e_i \otimes e_2 + e_2 \otimes e_1\]
\[ \sigma(e_1 \otimes e_2 + e_2 \otimes e_1) = \sigma(e_1 \otimes e_{2} + \sigma (e_2 \otimes e_1 ) = e_1 \otimes e_2 - e_2 \otimes e_1 \]
\[ \Lambda^{2}(\mathds{R}^{2}) \ni e_1 \otimes e_2 - e_2 \otimes e_1  \]
\end{tcolorbox}


\begin{tcolorbox}[title=Rappels]
	 $\rho_1 , \rho_2 $ reps indestructibles de $G$

	 alors \[  \expval{\chi_{\rho} } \]
	 \[ \dotsb \]
\end{tcolorbox}


\underline{Corollaire 5}: 
si $g \neq e$

\[ \sum_{\rho_i \text{irred} } \rm{dim} (\rho_i )\chi_{\rho_i}(g) = 0  \]

\underline{Démonstration}: 

\[ 0 = \chi_R (g) = \sum_{\rho_i \text{irred} } \rm{dim} (\rho_i )\chi_{\rho_i}(g)\quad (g \neq e) \]

Permet de trouver une caractère manquant dnas le table si on connaît tout les autres

\begin{tcolorbox}[title=Plus d'algèbre linéaire ]
	$e_1 ,\, \dotsb e_n$ base de $V$  $f_1 ,\, \dotsb f_m$ base de $W$ 
	$e_i \otimes f_j $ base de $V \otimes W$

	\[ M \in \rm{GL}(V) \qquad N \in \rm{GL}(W)  \]
	\[ M \otimes N \in \rm{GL}(V \otimes W) \]

\end{tcolorbox}

\underline{Proposition:}

\[ \tr(M \otimes N) = (\tr M)(\tr N) \]
\[ \chi_{\rho_1 \otimes \rho_2 } = \chi_{\rho_1} \cdot \chi_{\rho_2}  \]

\underline{Démonstration} 
 
\[ \tr(M \otimes N) = \sum_{ij} \left[ (M \otimes N) (e_i \otimes f_j ) \right]_{i,j } = \sum_{i,j} M_{i,i} M_{j,j} = \left( \sum_i M_{ii}  \right) \sum_j (M_{jj} )  = \tr M \tr N \]

\underline{Définition} 

L'espace dual de $V$ est $\rm{Hom}(V, \mathds{C})$ noté $V^{*}$

Si $M \in \rm{GL}(V) $ 

$M^{*} \in \rm{GL}(V^{*}) $  

$M^{*} \cdot \alpha = \alpha \circ M ^{-1}$

De même, si $\rho_{i} G \to \rm{GL}(V)$ est une repr. La repr \underline{dual} est $\rho^{*}: G \to \rm{GL}(V^{*})$ 
\[ g \mapsto \rho(g)^{*} \]

\underline{Proposition}: \[  \chi \rho ^{*} = \bar \chi_{\rho}  \] 

\underline{Démonstration}: 	
$g \in G$, $\rho(g) \in \rm{GL}(V)$ est une matrice d'ordre \underline{finie} 
\[ \left( \exists n | \rho(g)^n = I\right)  \]

\[ \implies \rho(g) \text{est diagonalisable est ses valeurs propres sont des racines de 1}   \]

\[ \chi_{\rho} (g) = \tr(\rho(g)) = \lambda_1 + \dotsb \lambda_d\]

\[ \rho^{*}(g) = (\rho(g)^{-1})^t  \]

\[ \tr(\rho^{*}(g)) = \lambda_1^{-1} + \dotsb + \lambda_d^{-1} = \bar \lambda_1 + \dotsb + \bar \lambda_d = \bar \chi_{\rho} (g) \]

\underline{Corrolaire } $\rho$ est irréductible $\iff \rho^{*}$ est irréductible

\[ 1 = \expval{\chi_{\rho}, \chi_{\rho}}  = \frac{1}{\abs{G}	} \sum_{g \in G} \bar \chi_{\rho} (g) \chi_{\rho} (g)   \]

\[ \iff \expval{\bar \chi_{\rho} , \bar \chi_{\rho} }  = \sum_{g \in G} \chi_{\rho} (g) \bar \chi_{\rho} (g) = 1\]

\[ \tr(A \otimes B) = \tr(A) + \tr(B) \]

\underline{Proposition}: 
\[ \chi_{\rho_1 \oplus \rho_{2}} = \chi_{\rho_{1}} + \chi_{\rho_2}   \]


\underline{Proposition}:

\[ \rm{Hom}(V, W) \cong V^{*} W \]

\underline{Démonstration}:

\[ f: V^{*} \otimes W \to \rm{Hom}(V, W)\]
\[ \alpha \otimes  w \mapsto (v \mapsto \alpha(v) w) \]

est linéaire 

\[ e_1 ^{*}, \dotsb , e_n ^{*}  \text{ base de V}  \]
\[ w_1 , \dotsb , w_m  \text{ base de W}  \]

\[ f(e_i ^{*} \otimes w_j ) = (v \mapsto e_i ^{*} (v) w_j  ) = (v ) \]

confus

\underline{Exemples:} $S_4 $ et $A_4 $

Les classes de conjugaisons dans $S_4 $ sont 

\[ \overbrace{(e)}^{1} ,\, \overbrace{(12)}^{6} ,\, \overbrace{(123)}^{8} ,\, \overbrace{(1234)}^{6} ,\, \overbrace{(12)(34)}^{3}     \]

(Toutes les traspotitions sont coujugés )


\begin{table}[htpb]
	\centering
	\label{tab:label}
	\begin{tabular}{c|c|c|c|c|c}
		 & 1  & 6 & 8 & 6 & 3\\
		 & $e$ & (12) & (123) & (1234) & (12)(34)\\\hline
		$\chi_0$ & 1 & 1 & 1 & 1 & 1 \\\hline 
		$\chi_{\text{sym}}$ &1 & -1 & 1 & -1 & 1 \\\hline
		$\chi_{\text{std}}$ &3 & 1 & 0 & -1 & -1 \\\hline
		$\chi_{\text{sym} \otimes \text{std} }$ &3 & -1 & 0 & 1 & -1 \\\hline
		$\chi_{4}$ &2 & 0 & -1 & 0 & 2 \\\hline
	\end{tabular}
	\caption{char de $S_4$}
\end{table}



Regardons la representation  $\rho_?$ de dim 4

\[ \rho_? : S_4 \to \rm{GL}(\mathds{C}^{4}) \]

on sait que $\rho_?$ se décompose en $\rho_{\text{triv}} \oplus \rho_{\text{std}} $


\[ \chi_{\rho_{?}} = \chi_{\rho_{?}} - \chi_0   \]

\[ = (4\, 2\, 1\, 0\, 0) - (1\,1 \, 1\, 1\,1 \,1) \]
\[  = (3\, 1\, 0 \, -1\, -1) \]

\[ \expval{\chi_{\text{std}} \chi_{\text{std}}  } = \frac{1}{24} \left( 3^2 + 6^2 + \dotsb \right)  =1  \]


Pour trouver $\rm{di}(\rho_4 )$

on utilise $\abs{G} = \sum_{\rho \text{irred} }\dim(\rho_i)^2$

\[ 23 = 1^2 + 1 ^2 + 3^2 + 3^2 +d^{2} \]

$d = 2$

On trouve les autres coeffs avec \[ 0 = \sum_{g \text{irred} } \rm{dim} (\rho_i) \chi_{\rho_i} (g) \]

Calculons $\rho_4$

On a $\rho((12)(34)) = I$
\[  \tr(\rho((12)(34))) = 2\]

$M $ est conjugé à \[ \begin{pmatrix} x & 0 \\ 0 & 2-x \end{pmatrix}  \]

mais \[ \begin{pmatrix} x^{2}& 0 \\ 0 &(2-x)^2 \end{pmatrix} = \mathds{1} \]

\[ \implies M = \mathds{1} \]

Quand une representation


a une noyau $\rm{Ker} \rho \subset G$

elle se factorise

$\rho_4$ ne se factorise pas 

\end{document}
