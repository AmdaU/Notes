\documentclass{article}    
\usepackage[utf8]{inputenc}    
    
\title{Épisode 4}    
\author{Jean-Baptiste Bertrand}    
\date{\today}    
    
\setlength{\parskip}{1em}    
    
\usepackage{physics}    
\usepackage{graphicx}    
\usepackage{svg}    
\usepackage[utf8]{inputenc}    
\usepackage[T1]{fontenc}    
\usepackage[french]{babel}    
\usepackage{fancyhdr}    
\usepackage[total={19cm, 22cm}]{geometry}    
\usepackage{enumerate}    
\usepackage{enumitem}    
\usepackage{stmaryrd}    
    
%packages pour faire des math    
%\usepackage{cancel} % hum... pas sur que je vais le garder mais rester que des fois c'est quand même sympatique...
\usepackage{amsmath, amsfonts, amsthm, amssymb}    
\usepackage{esint}  


\begin{document}
2024-01-11

\section{Rappel de théorie des groupes et de leurs actions}

Un groupe est une paire $(G, *)$, ou $G$ est un ensemble et $*$ est une opération $(*: G \times G \to G)$

3 axiomes:
\begin{enumerate}
	\item $a * (b *c) = (a * b) *c \quad \forall a,b,c \in G$	
	\item $\exists e \in G | e * a = a * e = a \forall a \in G$
	\item $\forall a \in G, \exists b \in G | a*b = e$
\end{enumerate}

Ex: $(\mathds{Z}, +), (\mathds{Q}, +), (\mathds{R}, +), (\mathds{C}, +), (\mathds{R}, +), \dotsb$

Les groupes matriciels sont très importants 

Tout les groupes mentionné jusqu'à maintenant sont infini, un exemple de groupe fini est $(\mathds{Z}_n, +)$

\[ S_E = \{ f: E\to E | f \text{ est inversible }  \}  \]
avec l'opération de composition $\circ$

On l'appel le groupe symétrique de $E$

$S_n = S_{\{ 1,2, \dotsb, n \} }$

Est le groupe des permuations de $n$ éléments

Notation pour désigner les éléments $\sigma \in S_n$:

\[ \sigma = \begin{pmatrix} 1 &2 &3 & 4 & \dotsb & n\\ \sigma(1)& \sigma(2)& \dotsb & \dotsb & \dotsb & \sigma(n) \end{pmatrix}  \]

\underline{Définition:} Un \underline{morphisme/homomorphisme} de groupes ($G,\, H$) est une fonction $f: G \to H$ t.q. $f(a *_G b) = f(a) *_H f(b)$. Si $f$ est inversible alors $f^{-1}$ est aussi un morphisme et on dit alors que $f$ est un \underline{isomorphisme} 

\underline{Exemples:} 

\begin{itemize}
	\item $\rm{det}: \rm{GL}_n (\mathds{R}) \to \mathbb{R}^{*}$
	\item $\abs{\cdot} : \mathbb{C} \to \mathbb{R} ^{*}$
	\item $\mathbb{Z} \to \mathbb{Z}_n$
\end{itemize}


\underline{Définition:} Une \underline{action} d'une groupe $G$ sur un ensemble $X$ est une application \[ \bullet  : G \times \to X  \]  satisfaisant \[ e \bullet x = x \quad \forall x \in X \] et \[ a \bullet (b \bullet x)  = (a * b) \bullet x \]

\underline{Exemple:}

\[ G = \rm{GL}_n (\mathbb{R})\quad X =\mathbb{R}^n \]

\underline{Définition}: Une \underline{action} de $G$ sur $x$ est un homomorphisme $f: G \to S_x$  

Les deux définition sont équivalentes

On définit $f(g) = (x \mapsto g \bullet x)$

\begin{align*}
	f(g_1 * g_{2)(x)} =& (g_1 * g_{2)} \bullet x\\ =& g_1 \bullet (g_2 \bullet x) \\&= g_1 \bullet f(g_2)(x) \\ &= f(g_{1)} (f(g_{2)(x))} \\&= \left[ f(g_1 ) \circ f(g_2 ) \right] (x )\quad \forall x \in X  
\end{align*}

\[ \implies f(g_1 * g_2) = f(g_1) \circ f(g_2 )\]

Si $X$ a plus de structure et qu'on a une action de de $G$ sur X qui preserve la structure lors on dit que $G$ agit par (homéomorphise, isométrie, application linéaire, ... (linéairement)) sur $X$

\underline{exemple:} $G = S_3 $ agit par isométrie sur un triangle équilatéral (voir \ref{fig:triangles-équilatérals})

\begin{figure}[ht]
    \centering
    \incfig{triangles-équilatérals}
    \caption{Triangles équilatérals}
    \label{fig:triangles-équilatérals}
\end{figure}

\textbf{ATTENTION}: $S_4$ n'agit pas (fidelement, injectivement) sur le carré par isométrie (certaines permuations \textit{brisent le triangle}) S. Par contre $S_4$ agit par isométries sur le cube!


%\begin{figure}[ht]
    %\centering
    %\incfig{cube}
    %\caption{cube}
    %\label{fig:cube}
%\end{figure}


$A_n \subset S_n$ et est groupe des permuations paire

$A_5$ agit par isométrie sur le dodécaèdre  

\underline{Théorème:} [Cayley] Tout groupe est isomorphe à un \underline{sous-groupe} d'un groupe de permutation $S_E$    


\underline{Démonstration}: On considère l'action de $G$ sur lui-même $(x = G)$

\[ g_1 \bullet g_2 = g_1 * g_2 \]

on obtiens $f: G\to S_G$: homomorphisme injectif car si $f(g_{1)} = f(g_1)$ alors $f(g_{1)(e)} = f(g_2)(e)$, $g_1 \bullet e = g_2 \bullet e$, $g_1 = g_2$

\[ \implies f(G) \subset S_G \text{ est isomorphe a $G$ } \]

\underline{Définition:} Une \underline{représentation} d'un groupe $G$ est une actions linéaire de $G$ sur un espace vectoriel $V$. Autremenet dit, un homomorphisme $ \rho: G \to \rm{GL}(V)$. Le rang d<une représentation est $\rm{dim}V$

\underline{exemples}: \[ \rho \mathds{C}^{*} \to \rm{GL}(2, \mathds{R}) \] 
\[ a+ib \to \begin{pmatrix} a & -b \\ b & a \end{pmatrix}  \]

Si $G$ est un groupe fini, il admet la \underline{représentation régulière}: 

\[ G = \{ g_1 , g_2, \dotsb , g_n  \}  \]

\[ V = \expval{e_{g_1}, \dotsb} \]


\end{document}
