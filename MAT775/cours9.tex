\documentclass{article}    
\usepackage[utf8]{inputenc}    
    
\title{Épisode 4}    
\author{Jean-Baptiste Bertrand}    
\date{\today}    
    
\setlength{\parskip}{1em}    
    
\usepackage{physics}    
\usepackage{graphicx}    
\usepackage{svg}    
\usepackage[utf8]{inputenc}    
\usepackage[T1]{fontenc}    
\usepackage[french]{babel}    
\usepackage{fancyhdr}    
\usepackage[total={19cm, 22cm}]{geometry}    
\usepackage{enumerate}    
\usepackage{enumitem}    
\usepackage{stmaryrd}    
    
%packages pour faire des math    
%\usepackage{cancel} % hum... pas sur que je vais le garder mais rester que des fois c'est quand même sympatique...
\usepackage{amsmath, amsfonts, amsthm, amssymb}    
\usepackage{esint}  


\begin{document}
2024-02-12

\begin{tcolorbox}[title=Rappels]

	\begin{itemize}
		\item Groupe de Lie matriciel \(G \ni I\)
			\(\to\) sous-groupe fermé de \(\rm{GL}(n \mathds{C})\)
		\item G est une sous-variété 
		\item \underline{Exemples}
			\(\rm{GL}(n, \mathds{R})\)
			\(\rm{Sl}(n,\mathds{R}), \rm{SL}(n,\mathds{C})\)
			\(\rm{O}(n), \rm{O}(n,\mathds{C})\)
			\(\rm{SO}(n), \rm{SO}(n, \mathds{C})\)
			\(\rm{U}(n), \rm{SU}(j)\)
			\(\rm{Sp}(2n, \mathds{R})\)
			\(\rm{Sp}(2n, \mathds{C})\)
			Groupe des matrice triangulaire superieur
			\((\rm{S})\rm{O}(p,q) = \{ M \in \rm{GL}(p+q, \mathds{R}) | M^{t}I_{pq} M^{t}= I_{pq}  \} \)
			\((\rm{S})\rm{U}(p,q) = \{ M \in \rm{GL}(p+q, \mathds{C}) | M^{*} I_{pq} M = I_{pq}  \} \)

		\item \rm{G} Connexe si \(\exists \gamma: [0,1] \to G\)
			avec \(\gamma(0) = I, \quad \gamma(1) = A \quad \forall A \in G\)
		\item \(G^{0} \subseteq G\) (composantes connexe de \(I\)) est un sous-groupe normal 
	\underline{exemple}: 
	\[\rm{O}(1,1) = \{ M= \begin{pmatrix} a &b \\ c & d \end{pmatrix}  | M^{t} \begin{pmatrix} 1 & 0 \\ 0 & -1 \end{pmatrix} M = \begin{pmatrix} 1 & 0 \\ 0 & -1 \end{pmatrix} \} \]
	On résous le système d'équations:

	\[ M = \begin{pmatrix} a & \pm c\\ c & \pm a \end{pmatrix} \qq{avec} a^{2}- c^{2} =1  \]
	\end{itemize}
	 
	\underline{Exercice}:

	\(\rm{O}(2)\)

\end{tcolorbox}

Étant donné \(f:G \to H\) un morphisme de groupe de Lie. On lui associe une application linéaire \[ \eval{\dd f}_{I} : T_I G \to T_{I} H \]. En fait cette application détermine uniquement \(f\). 

\boxed{\centering\text{Un voisinage arbitrairement petit autour de \(I\) engendre \(G\)} }

\begin{tcolorbox}[title=Attention, colframe=red]
	Pas tout les applications linéaires \(L: T_I G \to T_I H \) sont la dérivé d'un morphisme 
\end{tcolorbox}

On cherche une condition pour que \[ L = \eval{\dd f}_{I}   \]


Étant donnée \(g \in G\), on définit la \underline{multiplication à gauche} \(L_g : \to  G\) c'est une application lisse 
mais 
\[ \dd \eval{L_g }_{I} : T_I G \to T_g G  \]

On va plutôt regarder la conjugaison par \(g \in G\)

\[ \rm{Ad}(g) : G \to G \]
\[ h \to g h g ^{-1} \]

\[ \dd \eval{\rm{Ad}(g)}_{I} : T_I G \to T_I G   \]
\[ X \to gX g ^{-1} \]

\[ \gamma(t) \in G | \gamma(0) = I \quad \gamma' (0) = X \]
\[ \rm{Ad}(g) (\gamma(t)) = g \gamma(t) g ^{-1} \]

\[ \dd \eval{\rm{Ad}(G)}_{I} = \eval{\dv{{}}{t}}_{t=0} g \gamma (t) g ^{-1} = g X g ^{-1}  \]

Pour obenir une condition sur \(T_I G\) uniquement, on dérive \(\rm{Ad}(f)\) par rapport à \(g\) en fixant \(X\)

\[ G \to T_I G \]
\[ g\mapsto g X g ^{-1} \]

pour dériver cette appilcation on prend 

\(\gamma (-\epsilon, \epsilon) \to G \)

\[ \gamma(0) = I \]
\[ \gamma'(0) = U  \in T_I G \]

\[ \eval{\dv{{}}{t} }_{t=0} \gamma(t) X \gamma(t) ^{-1}  = \left[ \gamma'(t) X \gamma(t) ^{-1} + \gamma(t) X (\gamma(t)^{-1}  )'\right]_{t=0}  \]



\[ = YXI^{-1} + -IXI^{-1}Y I^{-1} \]
\[ = YX - XY \in T_I G \]


L'opération sur \(T_I g\)
\[ [X, Y] = XY - YX  \] s'appelle le \underline{crochet} 



Comme le crocher est définit en termes de la multiplication dans \(G\) et ses dérivées, pour tout morphisme de groupe de Lie \( f: G \to H\) la dérivé \( \dd f \eval{}_{I}: T_I G \to T_I H \) satisfaisant \( \dd f \eval{}_{I} [X, Y] = [ \dd f \eval{}_{I} X, \dd f \eval{}_{I} Y   ] \)


En fait \(L : T_I G \to T_I H\) est la dérivé d'un morphisme de groupe de Lie \(\iff L( [X, Y]  ) = [ L(X), L(Y) ] \forall X, Y \in T_I G \)

Le crochet a toutes les propriétés suivantes 

\begin{enumerate}
	\item Bilinéaire 
	\item antisymétrique
	\item Identité de Jacobi 
\end{enumerate}



\underline{Définition}: Une algèbre de Lie complexe est un espace vectoriel \(\mathfrak{g}\) complexe muni d'une application sur \(\mathds{C} \) 

\[ [{}, {}] : \mathfrak{g} \times  \mathfrak{g} \to \mathfrak{g}  \]

\underline{Exemple}: Si \(G\) est une groupe de lie matriciel, \(g = T_I G\) muni de \( [X, Y] = XY - YX \) est une algèbre de lie


Si \(f: G \to H\) est un morphisme d'algèbre de Lie (linéaire et \( \dd  f \eval{}_{I} [X, Y] = [	\eval{\dd{}f}_{I}X ,\eval{\dd{}f}_{I} Y]    \))


\underline{Exemple}:  

\[ G = \rm{GL}(n, \mathds{C}) \quad \mathfrak{g} = M_n(\mathds{C})\]

\(G = \rm{SL}(n, \mathds{C})\)
\[ \gamma(t) \in \rm{SL}(n ,\mathds{C}) \]

\[ \gamma(0) =1\]

\[ \det(\gamma(t)) = 1 \]

\[ \eval{\dv{{}}{t} }_{t=0} \det (\gamma(t)) = 0 = \eval{\dd \det(0)}_{\gamma (0)} = \tr \circ \gamma '(0) = \tr(\gamma '(0))\]

\[ \tr(\gamma'(0)) = 0 \quad \forall \gamma'(0) \in T_I \rm{SL}(n, \mathds{C}) \]


\[ T_I \rm{SL}(n \mathds{C}) \subseteq \{ X \in M_n (\mathds{C}) | \tr X = 0 \}  \]


En fait on a l'égalité


\end{document}
