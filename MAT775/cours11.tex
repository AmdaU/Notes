\documentclass{article}    
\usepackage[utf8]{inputenc}    
    
\title{Épisode 4}    
\author{Jean-Baptiste Bertrand}    
\date{\today}    
    
\setlength{\parskip}{1em}    
    
\usepackage{physics}    
\usepackage{graphicx}    
\usepackage{svg}    
\usepackage[utf8]{inputenc}    
\usepackage[T1]{fontenc}    
\usepackage[french]{babel}    
\usepackage{fancyhdr}    
\usepackage[total={19cm, 22cm}]{geometry}    
\usepackage{enumerate}    
\usepackage{enumitem}    
\usepackage{stmaryrd}    
    
%packages pour faire des math    
%\usepackage{cancel} % hum... pas sur que je vais le garder mais rester que des fois c'est quand même sympatique...
\usepackage{amsmath, amsfonts, amsthm, amssymb}    
\usepackage{esint}  


\begin{document}
2024-02-22

\begin{tcolorbox}[title=Rappels ]
	\[ \dotsb \]
\end{tcolorbox}

\underline{Proposition}: Soit \(0 \neq V \in V_\beta\), alors \(\{ V, \rho(\gamma) v, \rho(y)^2 v, \dotsb \} \) engendre \(V\) 

\underline{Démonstration:} On montre que \(U = \expval{v, \rho(y)v, \rho(y)^2 v, \dotsb}\) est stable pour \(\rho(x), \rho(y), \rho(H)\)

\begin{enumerate}
	\item \(\rho(H) (\rho(y)^m v) = \left( \beta -2m \right)\rho(Y)^m c \in U \)
	\item \(\rho(y) \rho(y)^m v = \rho(y)^{m+1}v \in U\)
	\item \(\rho(x) \rho(y)^m v =?\)

		On va montrer par récurrence que \(\rho(x) \rho(y)^m v = m(\beta-m+1) \rho(y)^{m-1}\)

	pour \(m=0 \) \(\rho(x) v = 0\)
	pour \(m=1\) \( \rho(x)\rho(y) = \left( \rho(H) + \rho(Y)\rho(x) \right)v \)

	\[ \rho(x) \rho(y)^{m+1} v = \left( \rho(H) + \rho(y) \rho(x)) \rho^{m} \right)  \]
	\[ \dotsb \]

	\[ \left[ \left( m+1 \right) \left( \beta -m \right) \rho(y)^m V \right]  \]

\(\implies U \subseteq \) est stable pour \(\rho\) comme \(\rho\) est irréductible, \(U=V\)


\underline{Conséquences} 

\begin{itemize}
	\item \(V_{\alpha} =1\)
	\item \(\rho\) est uniquement déterminé par \(\beta = {\max} up(\rho(H))\)
\end{itemize}

De plus, comme \(V\) est de dimension finie, il existe \(m\) t.q.\(\rho(y)^m v = 0\) et \(\rho(y)^{m-1} v = 0\)

\[ 0 = m(\beta -m+1)\rho(y)^{m+1} v \]

\[ \implies m(\beta-m+1) =0 \]
\[ \implies \beta = m -1 \qquad \beta \in \mathds{N}\]

\begin{figure}[ht]
    \centering
    \incfig{ladder}
    \caption{ladder}
    \label{fig:ladder}
\end{figure}

\end{enumerate}

Il y a au plus une représentation irréductible de dimenention \(n\) et les espaces propres de \(\rho(H)\) sont
\[ V_{1-n}, V_{2-n}, \dotsb V_{n-2}, V_{n-1}  \]

On va montrer qu'ils existent

\section*{Produit tensoriels de représentation d'algèbre de Lie}

\underline{Rappel} 

\[ \rho_i : G \to {\rm{}GL}(V_i ) i \in \{ 1,2 \}    \]

\[ \rho_1 \otimes \rho_2 : G \to {\rm{}GL}(V_1 \otimes V_{2})  \]

est définie par \(\rho_1 \otimes \rho_{2}(g) \left( V_1 \otimes V_2  \right)  = \rho_{1}(g)v \otimes \rho_2 (g) v_2 \) 

Si \(G\) est un groupe de Lie \(\mathfrak{g}\) son algèbre de Lie 

Calculons \(\eval{\dd(\rho_1 \otimes \rho_2)}_{I}  \mathfrak{g} \to {\rm{}gl} V_1 \otimes V_2\)

Soit \(\gamma(t) \in G\), \(\gamma(0)=I\), \( \gamma'(0) = X \in G \) 

\[ \dv{{}}{t} \left( \rho_1 \otimes \rho_2  \right) \gamma(t) (V_1 \otimes V_2)  = \dotsb = \left( \dd \eval{\rho_1 }_{I} (x) V_1  \right) \otimes V_2 + V_1 \otimes ( \dotsb )\]


\underline{Définition}: 


Si \(\rho_i: \mathfrak{g} \to {\rm{}gl} (V_i)\) sont 2 représentation d'algèbre de Lie, alors \(\rho_1 \otimes \rho_2 \) est définie par \(\left( \rho_1 \otimes \rho_2  \right) X  \left( V_1 \otimes V_2  \right) \)

On a également \(sym^{n}(\rho) \subseteq \rho^{ \otimes n}\), \(\Lambda^{n}(\rho) \subseteq \rho^{ \otimes n}\) sous-représentation comme pour \(G\) un groupe
On introduite la notation 

\[ v_1 \cdot v_2 \dotsb \cdot v_n := Sym^{n} (v_1 \otimes v_2 \dotsb v_n) \in Sym^n(V)\]
et \[ v_1 \wedge v_1 \dotsb = Alt(v_1 \dotsb) \] 


Revenons à \({\rm{}sl}(2,\mathds{C})\)

la représentation \underline{?????} est \(i :  \dotsb\) 

\[ i(H) = \begin{pmatrix} 1 & 0 \\ 0 & -1 \end{pmatrix} \]

a les valeurs propres \(1, -1\)


\[ \mathds{C}^2 = V_1 \oplus v_2  \]

est la représentation irréductive de dimension 2

\[ sym(\mathds{C}^2) = \expval{e_1 \cdot e_1, e_1 \cdot e_{2,} e_2 \cdot e_2} \]

\[ \left( Sym(i)(H) \right)(e_1^{2}) = H^{ \otimes 2}(e_1 \otimes e_{1)} = 2 e_1^{2}  \]

sur \(e_1 \otimes e_2\) c'est \(0\)
sur \(e_2 \otimes e_2\) c'est \(-2e_2^{2}\)

\[ \implies sym(\mathds{C}^2) = \expval{e_1^{n-i}, e_2^{i}} \]

Chacun est une vecteur propre de \(sym(H)\) et 

\[ sym(H)(e_1^{n-1} \cdot {e_2^i}) = \left( H \underbrace{e_1 e_1 e_1 e_2^{i}}_{n_1}  \right) + \left( e_1 H e_1 \dotsb e_2^{i} \right)  + \dotsb \]


\[ = \dotsb = (n-2i) e_1^{n-i} e_2^{i} \]

Je vois pas

\underline{Exemple}:
Quelle est la d/composition de \(sym^{2}(\mathds{C}^2) \otimes sym^{2}(\mathds{C}^2)\) en irréductibles?

On calcule les valeurs propres de \(\rho(H)\)

pour \(sym^{2}(\mathds{C}^2: -2, 0 2\)
pour \(sum^{2}(\mathds{C}^2): -3, -1, 1\)

Si \(\rho_1 (H) v = \lambda_1 v, \rho_{2}(H) u = \lambda_2 u\)
\[  \left( \rho_1 \otimes \rho_2  \right) H \left( v \otimes u \right) = \rho_1(H) v \otimes u + v \otimes \rho_2 (H) u = \lambda_1 v \otimes u + v \otimes \lambda_2 u = \left( \lambda_1 + \lambda_2 \right) \left( v \otimes u  \right)  \]

\begin{figure}[ht]
    \centering
    \incfig{valeurs-propres}
    \caption{valeurs propres}
    \label{fig:valeurs-propres}
\end{figure}

\end{document}
