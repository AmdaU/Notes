\documentclass{article}    
\usepackage[utf8]{inputenc}    
    
\title{Épisode 4}    
\author{Jean-Baptiste Bertrand}    
\date{\today}    
    
\setlength{\parskip}{1em}    
    
\usepackage{physics}    
\usepackage{graphicx}    
\usepackage{svg}    
\usepackage[utf8]{inputenc}    
\usepackage[T1]{fontenc}    
\usepackage[french]{babel}    
\usepackage{fancyhdr}    
\usepackage[total={19cm, 22cm}]{geometry}    
\usepackage{enumerate}    
\usepackage{enumitem}    
\usepackage{stmaryrd}    
    
%packages pour faire des math    
%\usepackage{cancel} % hum... pas sur que je vais le garder mais rester que des fois c'est quand même sympatique...
\usepackage{amsmath, amsfonts, amsthm, amssymb}    
\usepackage{esint}  


\begin{document}
2024-03-25

\begin{tcolorbox}[title=Rappel]
	\underline{Forme de Killing}
	\[ B: \mathfrak{g} \times  \mathfrak{g} \to \mathds{C} \]
	\[ (X,Y) \mapsto \tr({\rm{}ad}(X) \circ {\rm{}ad}(Y)) \]

	\underline{Porpriétés}: 
		\(\alpha, \beta\) avec \(\beta \neq \alpha\) alors si \(X \in \mathfrak{g}_\alpha\), \(Y \in \mathfrak{g}_\beta\), \(B(X,Y) = 0\)
	(autrment dit, si \(\beta \neq \alpha\) \(\mathfrak{g}_\alpha \perp g_{-\alpha} \)

\underline{cas spéciaux} 
\begin{enumerate}
	\item si \(\alpha = 0, \ \beta \neq 0\) \(\mathfrak{g}_0 \perp \mathfrak{g}_{\beta} \)
	\item si \(\alpha = \beta \neq 0\), \(\mathfrak{g}_\alpha\) est isotrope (\(\mathfrak{g}_\alpha \perp \mathfrak{g}_\alpha \))
\end{enumerate}

Si on restreint à \(\mathfrak{h}\) (\(X,Y \in \mathfrak{h}\))

\(B(X,Y) = \sum_{\alpha\in R} \alpha(X) \alpha(Y)\)
	
\(\implies \text{sur } \mathds{R}\expval{H_\alpha}, \ B\) est défini positive (non-dégénéré) 
\end{tcolorbox}

\begin{tcolorbox}[title=Rappel d'algèbre linéaire]
	\(V\) espace vectoriel, \(b\) forme bilinéaire symétrique	 
	\[ \varphi_b : V \to V^{*} \]
	\[ v \mapsto b(v,-) \]
\(b\) est non dégénéré \(\iff\) \(\varphi_b\) est un isomorphisme

On définit la forme bilinéaire duale de b, \(b^{*}\) donné parameters

\[ b^{*} (\alpha, \beta) = b(\varphi_b^{-1}(\alpha) , \varphi_b ^{-1} (\beta ) \]

Autrement dit, si \(\alpha = \beta(u,-) \), \(\beta = b(v, -)\) alors
\(b^{*}(\alpha, \beta) = b(u,v) =\alpha (v) = \beta (u) \)
\end{tcolorbox}

\underline{Proposition}: 
Si \(\alpha(H) = 0\) (\(\alpha \in R\)) alors \(B(H, H_\alpha) =0\)

Autrement dit \(H_{\alpha}^{\perp} = {\rm{}Ker}(\alpha)\)

\underline{Démonstration}:

\[ H_{\alpha} = [X_{\alpha},, Y_\alpha]  \]
Supposons \(\alpha (H) = 0\)

\(B(H, H_\alpha) = B(H, [X_{\alpha}, Y_{\alpha} ] = B( [H, X_{\alpha} ], Y_{\alpha} )= \alpha(H) B(X_{\alpha}, Y_{\alpha}) = 0 \)


\underline{Corollaire}:

Une racine \(\alpha \in R \) est orhtogonale à l'hyperplan 

\[ \Omega_{\alpha} = \{ \beta \in h ^{*} | \beta(H_\alpha) = 0 \}  \]


\begin{proof}
Soit \(\beta \in \Omega_\alpha\)

\[ \implies \beta(H_{\alpha}) = 0 \]

\[ \exists X, Y \in \mathfrak{h} \text{ t.q. } \alpha = B(X, -), \ \beta =B(Y, -)   \]

\(0 = \beta (H_{\alpha} ) = B(Y, H_{\alpha} ) \)

\[ Y \in H_{\alpha}^{\perp}  \]
\[ \alpha(Y) = 0 = B(X,Y) = B(\alpha, \beta)  \]

	
\end{proof}
\underline{Proposition}:

\[ \varphi_B^{-1} (\alpha) = \frac{2}{B(H_{\alpha}, H_\alpha) } H_{\alpha}   \]

\[ \varphi_B (H_\alpha) = \frac{2}{B(\alpha,\alpha) } \alpha \]

où \( \varphi_{\beta} (H) = B(H, -) \)


\underline{Démonstration}: Par définition, si \(\varphi_B ^{-1} (\alpha) = T_\alpha\) 

\[ B(T_{\alpha} , - ) = \alpha(-) \]


n a \(\forall H \in h\), \[ B(H_{\alpha}, H) = B(  [X_{\alpha} , Y_\alpha] , H) = B(X_{\alpha} , [ Y_{\alpha} , H] = B(X_{\alpha}, - [H, Y_{\alpha} ] =B(X_{\alpha} , - (-\alpha (H )) Y_{\alpha} ) = \alpha (H) B(X_{\alpha} ,Y_{\alpha)}   \]



De plus, \(B(H_{\alpha} , H_{\alpha} ) = \alpha (H_{\alpha} ) B (X_{\alpha} ,Y_{\alpha} ) = 2  B( X_{\alpha} ,Y_{\alpha}) \)


\[ \implies B(H_{\alpha} ,H) = \alpha (H) \frac{B(H_{\alpha} , H_{\alpha} ) }{2}  \]

\[ \implies B\left( \frac{2}{B(H , H_{\alpha} ) } H_{\alpha} , H  \right) = \alpha (H)  \]

\[ \implies T_{\alpha} = \frac{2}{B(H_{\alpha} , H_{\alpha} ) }  \]

2) exercice!


ON peut donc réécrire les générateurs du groupe de Weyl 

\[ W_{\alpha} (\beta ) = \beta (H_{\alpha} ) \alpha = \beta - \beta(H_{\alpha} ) \alpha = \beta - 2 \frac{B(\beta , \alpha) }{B (\alpha , \alpha ) } \alpha   \]

Réflexion dans l'hyperplan \(\alpha^{\perp}\)


\underline{Exemple}:

Calculons \(B\) sur \(\mathfrak{sl}(2 \mathds{C})\)

\[ {\rm{}ad}: \mathfrak{sl}(2 \mathds{C}) \to \mathfrak{gl}(\mathds{C}^3)  \]
\[ H \mapsto \begin{pmatrix} 0 & & \\ & 2 & \\ &&-2 \end{pmatrix}  \]
\[ X \mapsto \begin{pmatrix} 0 & 0 & 1 \\ -2& 0 & 0\\ 0&0&0 \end{pmatrix}  \]
\[ Y \mapsto \begin{pmatrix} 0 & -1 & 0 \\ 0& 0 & 0\\ 2&0&0 \end{pmatrix}  \]
(dans la base \(H, X, Y\))

\[ B(H,H) = \tr \left( \begin{pmatrix} 0 \\ & 2 \\ && -2 \end{pmatrix}^2 \right) = 8  \]

\[ B(H,X) = \tr(\begin{pmatrix} 0 \\ & 2 \\ &&-2 \end{pmatrix} \begin{pmatrix} 0 &0 &1 \\ -2 &0 &0 \\0 &0 &0 \end{pmatrix} ) = \tr \begin{pmatrix} 0 &0 & 1 \\ -4 & 0 &0 \\ 0 &0 &0) \end{pmatrix} = 0 \]
\[ B(X,X) = B(Y,Y) = 0 \]

\[ B(X,Y) = \tr\begin{pmatrix}2 & \dotsb & \dotsb \\ \dotsb & 2 & \dotsb \\ \dotsb & \dotsb & 0\end{pmatrix} =4 \]

\[ B = \begin{pmatrix} 8 &0 &0 \\ 0&0&2 \\ 0 &2&0 \end{pmatrix}  \]
\section*{\(B\) sur \(\mathfrak{h} \subseteq \mathfrak{sl}(3, \mathds{C})\)}

\(H_1 = \begin{pmatrix} 1 \\ & -1 \\&&0 \end{pmatrix},\, H_2 = \begin{pmatrix} 0 \\ & 1 \\ && -1 \end{pmatrix}  \)

\[ B(H_1, H_1 ) = \sum_{\alpha \in R} \alpha (H_1)^{2} = 2^2 + 1^2 + (-1)^{2}= 12 \]

\[ B(H_1, H_2 ) = \sum_{\alpha\in R} \alpha (H_1) \alpha (H_2 ) = 2 \left[ 2 \cdot -1 + 1 \cdot 1 + -1 \cdot 2\right] =2 \cdot -3 = -6  \]
\[ B(H_2, H_2 ) =12 \]

\[ B = \begin{pmatrix} 12 & -6 \\ -6 & 12  \end{pmatrix}  \]

C'est une matrice définit positive


On peut alors vérifier que les racine sont orthogonale à leur plans de réflexion, \(L_1 - L_2\) est la racine qui pointe vers le haut (comme on le dessine habituellement) . En se fiant au dessin habituelle, cette racine devrait être orthogonale à \(L_1\).

\begin{tcolorbox}[title=Rappel d'algèbre linéaire ]
	si \(b\) est donné par une matrice, \(b(u,v) = u^{t}b v \)
	\[ \varphi_b = V \mapsto V ^{*} \]
	\[ V \mapsto k b\]
	\[ b(u,v) = u^{t}b v = b^{*} (\alpha ^{t} b , v^t b) = u^{t}b (b^{*}) b^{t}v \implies b^{t}= (b^t) ^{-1}\]
\end{tcolorbox}

\(B^{-1} = \frac{1}{108} \begin{pmatrix} 12 & 6 \\ 6 & 12 \end{pmatrix} \)

La base duale de \(H_1, H_2\) est \(L_1 , - L_3\)
la matrice dans cette base est \(\frac{1}{108} \begin{pmatrix} 12 &6 \\ 6 & 12  \end{pmatrix} \)

On calcule \(B(L_1 , L_2 - L_3 )  = B(L_1, -L_1 +2 (-L_3) ) = \begin{pmatrix} 1 & 0 \end{pmatrix}  \frac{1}{108} \begin{pmatrix} 12 & 6 \\ 6 & 12 \end{pmatrix} \begin{pmatrix} -1 \\ 2 \end{pmatrix} = 0 \)

\textit{jazz hands}

%Ça devrait donner 0 mais visiblement on a fait une erreur de signe

On a également 

\[ B(L_2 - L_3 , L_2 - L_3 )  = \frac{1}{108} \begin{pmatrix} -1 & 2 \end{pmatrix} \begin{pmatrix} 12 & 6 \\ 6 & 12 \end{pmatrix} = \frac{1}{108} \begin{pmatrix} 0 & 18 \end{pmatrix} \begin{pmatrix} -1 \\ 2 \end{pmatrix} = \frac{36}{108} = \frac{1}{3}  \]

\[ \implies \norm{L_2 - L_3 } = \frac{1}{\sqrt{3 }}   \]

\underline{Corollaire de} \[ \beta (H_{\alpha} ) = \frac{2 B(\beta , \alpha ) }{B(\alpha ,\alpha ) }  \] 

Si \(\alpha ,\beta \) deux racines alors 
\[ \frac{2 B(\beta , \alpha ) }{B(\alpha ,\alpha )} \in Z  \]


\section*{Classification des algèbres de Lie simples complexes}

soit \(\mathfrak{g}\) une algèbre de Lie semi-simple, \(\mathfrak{h} \in \mathfrak{g}\) sous algèbre de Cartan.

Notons \(\mathbb{E}\) le sous-espace euclidien de \(h^{*}\) engendré par \(R\) munie de \(B^{*}\) qui (dual de Killing) qu' on va noter \((\ ,\ )\)

\[ B(\alpha, \beta)  = (\alpha ,\beta)  \]

On 
\begin{itemize}
	\item \(R\) est finie et engendre \(\mathds{E}\)
	\item \(\dotsb\)
\end{itemize}

\end{document}
