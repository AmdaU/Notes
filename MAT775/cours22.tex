\documentclass{article}    
\usepackage[utf8]{inputenc}    
    
\title{Épisode 4}    
\author{Jean-Baptiste Bertrand}    
\date{\today}    
    
\setlength{\parskip}{1em}    
    
\usepackage{physics}    
\usepackage{graphicx}    
\usepackage{svg}    
\usepackage[utf8]{inputenc}    
\usepackage[T1]{fontenc}    
\usepackage[french]{babel}    
\usepackage{fancyhdr}    
\usepackage[total={19cm, 22cm}]{geometry}    
\usepackage{enumerate}    
\usepackage{enumitem}    
\usepackage{stmaryrd}    
    
%packages pour faire des math    
%\usepackage{cancel} % hum... pas sur que je vais le garder mais rester que des fois c'est quand même sympatique...
\usepackage{amsmath, amsfonts, amsthm, amssymb}    
\usepackage{esint}  


\begin{document}
2024-04-08

\section*{Construction de \(G_2\) (suite)}

\[ \alpha_3 = \alpha_2 + \alpha_1 \]
\[ \alpha_4 = \alpha_1 + \alpha_3  \]
\[ \alpha_5 = \alpha_1 + \alpha_5  \]
\[ \alpha_6 = \alpha_2 +\alpha_5 \]


\[ x_1 \in g_{\alpha_1} \quad X_2 \in g_{\alpha_{2}}  \]
On peut choisir 

\[ Y_1 \in g_{-\alpha} \quad Y_2 \in g_{-\alpha_2 }  \]

tel que \(H_{i} = [X_i, Y_i] \quad [H, X/Y] = \pm 2 X/Y \)

On définit \(X_3 = [X_1, X_2] \in g_{\alpha_3} , X_4 = [X_1, X_3] , X_5 = [X_1, X_4] , X_6 = [X_2, X_5] \)

idem pour les \(Y\)

on sait que \(\mathfrak{g}_2 = \expval{H_1, H_2 , X_{1\dotsb6} , Y_{1\dotsb6} }\)

On calcule tout les crochets

Tout ce qui tombe sur pas une racine c'est 0

Pour trouver les crochet avec $H_1$ et $H_2$, on considère des petit \(\mathfrak{sl}2\)

\end{document}
