\documentclass{article}    
\usepackage[utf8]{inputenc}    
    
\title{Épisode 4}    
\author{Jean-Baptiste Bertrand}    
\date{\today}    
    
\setlength{\parskip}{1em}    
    
\usepackage{physics}    
\usepackage{graphicx}    
\usepackage{svg}    
\usepackage[utf8]{inputenc}    
\usepackage[T1]{fontenc}    
\usepackage[french]{babel}    
\usepackage{fancyhdr}    
\usepackage[total={19cm, 22cm}]{geometry}    
\usepackage{enumerate}    
\usepackage{enumitem}    
\usepackage{stmaryrd}    
    
%packages pour faire des math    
%\usepackage{cancel} % hum... pas sur que je vais le garder mais rester que des fois c'est quand même sympatique...
\usepackage{amsmath, amsfonts, amsthm, amssymb}    
\usepackage{esint}  


\begin{document}
2024-02-15

\begin{tcolorbox}[title=Rappels]
	\(G\) groupe de liea\\
	\(\mathfrak{g} = T_I G\) algèbre de Lie pour \( [X, Y] = XY - YX\)

	En général, une algèbre de LIe est un espace vectoriel muni d'un crochet \( [., .]: V \times V \to V\) satisfaisant

\begin{enumerate}
	\item bilinéaire
	\item antisymétrique
	\item Jacobi
\end{enumerate}
\end{tcolorbox}

\begin{tcolorbox}[title=Exercice, colframe=green]
	\begin{enumerate}
		\item Montrer que \(\mathds{R}^{3}\) muni du produit vectoriel \(\times \) est une algèbre de lie
		\item Construire un isomorphisme entre \( \left( \mathds{R}^{3}, x \right) \text{ et } \left( \square(3), [., .]  \right)    \)	
	\end{enumerate} 
\end{tcolorbox}


\begin{tcolorbox}[title=tentative]
	\begin{enumerate}
		\item On doit montrer que \(\times \) respecte les trois conditions 
			\begin{enumerate}
				\item \(\vb{x}, \vb{a}, \vb{b} \in \mathds{R}^{3}\)

					\[ \vb{x} \times \lambda\left( \vb{a} + \vb{b} \right)  = \begin{pmatrix} x_1 \\ x_2 \\ x_3  \end{pmatrix} \times\lambda  \left( \begin{pmatrix} a_1 \\ a_2 \\ a_3   \end{pmatrix} + \begin{pmatrix} b_1 \\ b_2 \\ b_3 \end{pmatrix}  \right)  = \begin{pmatrix} x_1 \\ x_2 \\ x_3  \end{pmatrix} \times \begin{pmatrix} \lambda a_1 + \lambda b_1 \\ \lambda a_2 + \lambda b_2 \\ \lambda a_3 + \lambda b_3 \end{pmatrix}= \begin{pmatrix} \\ \dotsb \\ {} \end{pmatrix} = \lambda (\vb{x} \times  \vb{a} + \vb{x} \times \vb{b}  \]


			\end{enumerate}
	\end{enumerate} 
\end{tcolorbox}


\section*{L'application exponentielle}

\(G\) groupe de Lie, \(\mathfrak{g}= T_I G\) sont algèbre de Lie


\underline{Définition}:

\[ \exp : \mathfrak{G} \to G \]
est l'unique application lisse satisfaisant

\begin{enumerate}
	\item \(\exp(0) = I\)
	\item \(\eval{\dd \exp}_{0} : \mathfrak{g} \to \mathfrak{g} \) est l'application identité
	\item \(\forall X \in g\) l'application \(t \to \exp(tx)\) est un homomorphisme de groupes 
		\[ \exp(t+s) X = \exp tX + \exp sX  \]

		(l'existence et l'unicité sont à démontrer)
\end{enumerate}

\underline{Proposition}:

Pour \(G = \rm{GL}(n, \mathds{C} \), \(\exp(X) = \sum_{k=0}^{\infty} \frac{1}{k!} x^{k}= e^{X}\)

\begin{tcolorbox}[title=Rappels sur l'exponentiation de matrices]
	\begin{enumerate}
		\item 
	\end{enumerate}
	 
\end{tcolorbox}

\underline{Proposition}:

\[ f: G \to H \] est un morphisme de groupe de Lie alors 

\[ \begin{pmatrix} \mathfrak{g} & \to \dd f \eval_I \to & \mathfrak{h} \\ \downarrow \exp_g & & \downarrow\exp_H \\ G & \to f \to & H \end{pmatrix}  \]

commute, c-à-d, \(f \circ \exp_G = \exp_H \circ \dd f \eval_I \)

\underline{Conséquence:}

Si \(G \subseteq \rm{GL}(n, \mathds{C})\)


\(\implies i \circ \dotsb \)


tout à été effacé  dasfefefwefeffsfefrgqp

\underline{Démonstration}:

\[ \dotsb \]

\section*{Représentation de groupe/algèbre de Lie} 

\underline{Définition} 

Une représentation de \(G\) est un morphisme \(G \to \rm{GL}(n, \mathds{C}\)

Une représentation de \(\mathfrak{g}\) est une morphisme d'algèbre de Lie \( \mathfrak{g} \to \mathfrak{g}\rm{l}(n, \mathds{C})\)


\underline{Exemeple:} Représentation adjointe 


\(\rm{Ad}: G \to \rm{GL}(\mathfrak{g}\)
\[ g \mapsto \rm{Ad}(g) \]

où \(\rm{Ad}(g) (X) = g X ^{-1} g\)

on peut vérifier la linéairité et \(\rm{Ad} = \left( \rm{Ad} g\right) \left( \rm{Ad} h \right)  \)



\end{document}
