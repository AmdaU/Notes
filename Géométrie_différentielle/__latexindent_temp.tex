\documentclass{article}
\usepackage[utf8]{inputenc}

\title{Cours 3}
\author{Jean-Baptiste Bertrand}
\date{December 2021}

%\setlength{\parindent}{4em}
\setlength{\parskip}{1em}


\usepackage[utf8]{inputenc}
\usepackage[T1]{fontenc}
\usepackage[french]{babel}
\usepackage{fancyhdr}
\usepackage[total={19cm, 22cm}]{geometry}
\usepackage{enumerate}
\usepackage{enumitem}
\usepackage{svg}
\usepackage{physics}
\usepackage{mathrsfs}  

%packages pour faire des math
%\usepackage{cancel} % hum... pas sur que je vais le garder mais rester que des fois c'est quand même sympatique...
\usepackage{amsmath, amsfonts, amsthm, amssymb}
\usepackage{esint}

\begin{document}

\maketitle

courbe régulière: $\alpha' \neq 0 \forall t$\\


\underline{longeure d'arc}
$$\mathscr{L} = \int_a^b ||a'(t)|| \dd t$$

Approximtion avec une partition

$P = (t_0, t_1, ... t_n)$

$$\mathscr{L}(\alpha,P) = \sum_{i=0}||\alpha(t_i)-\alpha(t_{i-1})||$$


\underline{Prop}

Si alpha est $C^1$ alors $\alpha$ est rectifiable et 

$\mathscr{L}(\alpha) = sup_p \mathscr{L}(\alpha,P)$


On a montré que pour toute partition $P$ : $\mathscr{L}(\alpha, P) \leq \int_a^b||\alpha(t)\dd t||$

\underline{Lemme}: $||\int_a^b\alpha(t)\dd t|| \leq \int_a^b ||\alpha(t)\dd t$

Reste à montrer que $\forall \epsilon > 0 \exists P t.q.$
$$\mathscr{L}\alpha,P) \geq \int_a^b |\alpha(t)|\dd t -\epsilon$$

Continuité uniforme de alpha prime 

$\exists \delta > 0$ t.q. si 

\underline{Proposition:} Une courbe paramétré $\alpha$ admetant une reparamétrisation par longeur d'arc ssi elle est régulière


\underline{Dem} ($\implies$)

Si $\alpha$ admet une reparamétrisation par longeure d'arc $\tilde \alpha$

et $\tilde \alpha = \alpha \circ \varphi\; \varphi:[a,b] \to [c,d]$

\begin{align*}
	\tilde \alpha(t) =\alpha (\varphi(t))\\
	\tilde \alpha'(t) =\alpha'(\varphi(t))\varphi'(t)\\
	\underbrace{||\tilde \alpha'(t)||}_1 =||\alpha'(\varphi(t))|| |\varphi'(t)||\\
	\implies ||\alpha'(\varphi(t))|| \neq 0
\end{align*}

($\Longleftarrow$)

Trop long, trop loin

\underline{Exemple}: Calculer la paramétrisation par longeure d'arc d'une hélice 

$$\alpha(t)=(a\cos(t),a\sin(t),bt) \; (a, b >0) (t \in \mathbb{R})$$


\begin{align*}
	\Psi(t) &=\int_0^t || \alpha'(x)|| \dd x\\
			&= \int_0^t(-a\sin x, a\cos x, b) \dd x\\
			&=\int_0^t \sqrt{a^2 +b^2} \dd x\\
			&= t\sqrt{a^2+b^2}\\
			&\implies \Psi^{-1}(s) = \frac{s}{\sqrt{a^2+b^2}}\\
			&\implies \tilde{\alpha}(s) = (a\cos\frac{s}{\sqrt{a^2+b^2}}, a\sin\frac{s}{\sqrt{a^2+b^2}}, b\frac{s}{\sqrt{a^2+b^2}})
\end{align*}


\underline{Courbe du jour:} Caténoide

\underline{Repère de Frenst}

Un repère adapté à la courbe.

Le premier vecteur est le vecteur tangeant.

Le second vecteur est le vecteur \textit{accélération}. En effet, il est toujours pependiculaire au déplacement dans le cas d'une courbe paramétré par longeure d'arc (vitesse constante)

Le troisième est celui qui reste ($\cross$)

\underline{Lemme:} Soient $f,g:(a,b) \to \mathbb{R}$ differentiable. Si $f(t) \circ g(t)$ est constante alors $f'(t)\circ g(t) = -f(t)\circ g('t)$

\underline{Dem} 
$(f(t) \circ g(t))'=0 \implies f'(t)\circ g(t) +f(t)\circ g('t) = 0 \blacksquare$ 

Soit $\alpha$ paramétré par longueure d'arc

$$T(s) := \alpha(s)$$

$$k(s) := ||T'(s)|| = ||\alpha''(s)||$$ est la courbure de $\alpha$ au point $\alpha(s)$

$$N(S) :=\frac{T'(s)}{k(s)}$$

On dur que $\alpha$ est birégulière si $k(s) \neq 0 \forall s$ 

$$B(s) := T(s)\cross N(s)$$

$T,N,B$ est le \underline{repère de Frenet} de $\alpha$


\begin{equation*}
	||T(s)|| = 1\\
	T(s)\cdot T(s) = 1\\
	T(s)\cdot T'(s) = 0\\
	\implies k(s) T(s)\cdot N(s)  = 0
\end{equation*}


$$T,N,B \text{ sont } \perp$$

$$||B|| ||T\cross N|| = ||T||N||sin(\phi) =1$$
Orthonormé!

On a, par définition que

$$T'(s) = k(s)N(s)$$

$$N'(s)\cdot T(s) =-N(s)\cdot T'(s)\\N'(s)\cdot N(s) = 0 \\ N'(s)\cdot B(s) =:\tau(s)$$

$\tau$:torsion

\dots

On obtiens les \underline{Équations de Frenet-Serra}

\begin{align*}
	T'(s) &= k(s)N(s)\\
	N'(s) &= k(s)T(s) + \tau(s)B(s)\\
	B'(s) & -\tau(s) N(s)
\end{align*}

$$\begin{pmatrix}
	T'\\N'\\B'
\end{pmatrix} = \begin{pmatrix}
	0 & k(s) & 0 \\ k(s) &0 & \tau(s) \\0 & -\tau(s) &0
\end{pmatrix}$$

\end{document}

