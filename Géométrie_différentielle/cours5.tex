\documentclass{article}
\usepackage[utf8]{inputenc}

\title{Cours 5}
\author{Jean-Baptiste Bertrand}
\date{\today}

%\setlength{\parindent}{4em}
\setlength{\parskip}{1em}


\usepackage[utf8]{inputenc}
\usepackage[T1]{fontenc}
\usepackage[french]{babel}
\usepackage{fancyhdr}
\usepackage[total={19cm, 22cm}]{geometry}
\usepackage{enumerate}
\usepackage{enumitem}
\usepackage{svg}
\usepackage{physics}
\usepackage{mathrsfs}  
\usepackage{tcolorbox}

%packages pour faire des math
%\usepackage{cancel} % hum... pas sur que je vais le garder mais rester que des fois c'est quand même sympatique...
\usepackage{amsmath, amsfonts, amsthm, amssymb}
\usepackage{esint}

\begin{document}

\maketitle

La dernière fois, on s'interessait à ce qui se passe quand un courbe n'est pas paramétrisé par longueur d'arc. On suppose qu'il existe un paramétisation par longueur d'arc: $\alpha(s(t))$ où $s$ est la longeur d'arc

$$s(t) = \int_0^t ||\alpha(s(x))|| \dd x$$

$$s'(t) = ||\alpha(s(t))'|| = v(t)$$

$$\alpha(s(t))'=\alpha'(s(t))s'(t)=T(s(t))v(t)$$
$$\alpha(s(t))'' = T'(s(t))v(t)^2 + T(s(T))v'(t) = \kappa(s(t))N(s(t))v(t)^2 + v'(t)T(s(t))$$

Pour calculer $N$ et $\kappa$ sans passer par la longeure d'arc, on utilise

$$\kappa(s(t))N(s(t))= \frac{\alpha(s(t))''-v'(t)T(s(t))}{v(t)^2}$$

\underline{Exercice:} Finir l'exemple

$$\alpha(t) = (3t-t^2, 3t^2, 3t+t^2)$$

On devrais trouver

$$\kappa(s(t))=\tau(s(t))=\frac{1}{3(1+t^2)^2}$$


\underline{Proposition:} La courbure d'une coure $\alpha$ (non-paramétrée par longueur d'arc) est donnée par $$\kappa(t)=\frac{\norm*{\alpha' \cross \alpha''}}{\norm{\alpha'}}^2$$

\underline{Démonstration} 

On a, par ce qu'on a fait ci-haut

$$\alpha(s(t))=vT$$
$$\alpha(s(t))''=v'T + \kappa v^2 N$$

$$\alpha' \cross \alpha'' = v^3 \kappa(T\cross N)= v^3 \kappa B$$

$$\implies \norm*{\alpha' \cross \alpha''} = v^3\kappa$$
$$\implies \frac{\norm{\alpha'\cross\alpha''}}{\norm*{\alpha'}^3}=\kappa \quad \text{car } v = \norm*{\alpha'}$$


\underline{Conséquence des formules de Frenet-Serret}

\underline{Prop}: Une courbe est un droite $\iff \kappa =0$

\underline{Démonstration} ($\implies$) Si $\alpha$ est une droite

$$\alpha(s) = \rho_0 + sV$$
$$\alpha'(s) = v-T(s) \implies T'(s)=0 \implies \kappa =0$$

($\Longleftarrow$)
si $\kappa(s) =0 \forall s$

$$T'(s)=0 \implies T(s) = T_0$$

$$\alpha(s) = \int_0^sT*(x)\dd x = sT_0 + \rho_0$$

\underline{Exemple}: Que peut-on dire d'une courve $\alpha$ dont toutes les tangeantes passent par un même point?

Sans pertes de généralité, les tangeantes passent par $\vec O \in \mathbb{R}$

$$\implies \alpha(s) + \lambda(s)T(s) = 0$$
$$\implies T(s) + \lambda'(s)T(s) + \lambda(s)T'(s) =0$$
$$\implies (1 + \lambda'(s))T(s) + \lambda(s) + \lambda(s)(\kappa(s)N(s)) =0$$

$$\implies 1+\lambda'(s)=0 \quad \text{ou } \quad \lambda(s)\kappa(s) = 0$$
$$\lambda(s) = -s +c$$
$$\lambda =0 \text{ si } s=c \implies \kappa = 0 \text{ sauf si } \dotsb$$

\underline{Prop} 1)Une courbe birrégulière $\alpha$ est \underline{planaire} $\iff \tau \equiv 0$. 2) Les courbes planaires de courbure constante sont des cercles.

\underline{Démonstration} 1) $\implies$ Si $\alpha$ est planaire, $T$ et $N$ engendrent le plan qui contiens $\alpha$. Cela signifique que $T\cross N=B$ est constant. C'est le vecteur normal au plan qui contient la courbe $\alpha$.

$$\implies B'(s)=0=-\tau N \implies \tau = 0$$

Donc la torsion est nulle $\blacksquare$

($\Longleftarrow$) Inveserment, si $\tau \equiv0$

$$B'(s) = 0 \implies B(s) =B_0 \text (est constant)$$
$$\implies (\alpha(s)\cdot B(s))' = T(s)\cdot B(s) + \alpha(s)\cdot B'(s) =0$$

$$\alpha(s)\cdot B(s) =\alpha(s)\cdot B_0 = C$$

C'est l'équation d'un plan dans $\mathbb{R}\quad \blacksquare$

2) $\Longleftarrow$

Un cercle est paramétré par longeur d'arc avec l'équation suivante:

$$\alpha(s) = \qty(r\cos(\frac sr), r\sin(\frac sr),0)$$
$$\alpha'(s) = T(s)=\qty(-\sin(\frac sr), \cos(\frac sr), 0)$$

$$T'(s)=\qty(-\frac 1r\cos(\frac sr), -\frac 1r\sin(\frac sr), 0)$$

$$\implies \kappa = \norm*{T'(s)} = \frac 1r \text{ est constante}$$

Cela donne une interprétation à la courbure qui est que en chaque point, il existe un cercle de rayon $r$ qui est une meilleur approximation de la courbe.

$$\implies$$

Soit $\alpha(s)$ ime courbe planaire avec $$\kappa_s=\kappa_0$$

Commme on sait déja que cela doit ête un cercle, on s'aide en cherchant le centre du cercle.

On pose $\beta(s)=\alpha(s) + \frac 1{\kappa_0}N(s)$

$$\beta'(s) = T(s) = \frac{1}{\kappa_0} (-\kappa T + \tau B)$$

$$\norm*{\alpha(s)-\beta(s)} = \norm{\frac{1}{k_0}N(s)} = \frac 1{k_0}$$

$\implies \alpha(s)$ est sur le cercle de rayon $\frac 1{k_0}$ centré en $B_0$

\underline{Courbe du jour: Tractrice}
UN chien enterre un os à $(0,1)$, son maître à $(0,0)$ la tire par une laisse en de déplaçant vers $x>0$. Comme le chien tire très for, la laisse est toujours tangenante à la trajectoire du chien.

Soit $\theta$ l'angle formé par la laisse et l'axe des $x$

$$\alpha(t) = (t+\cos\theta(t), \sin\theta(t))$$
$$\alpha'(t) = 1-(\sin\theta)\theta', \cos\theta \theta'$$

La laisse est dans la direction $(\cos\theta, \sin\theta)$. Comme la trajectoire $\alpha$ est tangeante à la laisse. 

$$\frac{\cos\theta\theta'}{1-(\sin\theta)\theta'} = \frac{\sin\theta}{\cos\theta} = \tan\theta$$

$$\cos\theta\theta' = \sin\theta -(\sin^2\theta)\theta'$$
$$\theta' = \sin\theta$$
$$\dv{\theta}{t} = \sin\theta$$
$$-\ln(\csc\theta + \tan\theta) = t+c \quad t=0, \theta=\frac\pi2 \to c=0$$ 


$$\alpha = \qty(-\ln(\csc\theta + \tan\theta) +\cos(\theta), \sin\theta) \quad \frac\pi2\leq\theta\leq\pi$$

En reparamétrisant 

$$\alpha(t) = \qty(t - t\sinh(t), \sinh(t))$$

\underline{Forme locale canonique d'une courbe}

\underline{Proposition:} Soit $\alpha$ une courbe birrégulière paramétrée par longueur d'arc t.q. $\alpha(0) =0$ alors 

$$\alpha(s) = (s -\frac{k_0^2}6s^2 +o(s^2))T(o) + \qty(\frac{k_0}{2}s^2 + ) \dotsb$$

C'est vraiment laid, c'est loin pis il y a du soleil, sorry.

\underline{Démonstration} Le théorème de Taylor nous dit $$\alpha(s) = sa'(0) + \frac{s^2}2\alpha''(0) + \frac{s^2}{6}\alpha'''(0) +O(s^4)$$

$\alpha'(0) = T(0)$
$\alpha''(s)= T'(s) = \kappa(s)N(s)$
$\alpha'''(s) = \kappa'(s)N(s) + \kappa(s)N'(s) = \kappa'(s)N(s)+ \kappa'(0)N(0) + \kappa_0 \tau_0 B(0)$

$$\implies \alpha(s)=\qty(s-\frac{s^2}6\kappa_0^2+O(s^3))T(0) + \qty(k_0\frac{s^3}{2} + k_0' \frac{s^2}6 + o(s^3))N(0) + \qty(\kappa_0\tau_o\frac{s^3}6 + O(s^3))B(0)$$

\underline{Le théorème fondamentale des courbes dans $\mathbb{R}^3$}

Si j'ai deux courbes donc je connais la même courbure est la même torsion en tout point alors c'est la même courbe à une isométrie près.

Montrons d'abord que les isométies de $\mathbb{R}^3$ préservent la courbure est la torsion. 

\underline{Rappel} Une isométrie de $\mathbb{R}^3$ e st de la forme $\vb x \mapsto A \vb x + \vb b$ où $A \in O(3) \iff AA=\text{id}, \; b\in \mathbb{R}^3$

Une isométrie est \underline{directe} où une \underline{transformation directe} si $A \in SO(3) \iff \det A =1$

Soit $\alpha$ une courbe paramétré par longeure d'arc 

On définit $$\alpha^*(s) = A\alpha(s) + b$$

$$\alpha^{*'}(s) = A\alpha(s)$$
$$T'(s) = AT(S)$$
$$T^{*'}(s) = A T'(s)$$
$$\norm{\kappa^*N^*(s)} = \norm*{\kappa(s)AAN(s)}$$

$$\kappa^* = \kappa$$


$$B^* = T^*\cross N (AT)\cross (AN) =A(T\cross N) = A B$$

$$(B^*)' = -\tau^*N^*$$
$$(AB)' = AB' = -\tauAN \implies \tau = \tau^*$$

\end{document}