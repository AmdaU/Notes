\documentclass{article}
\usepackage[utf8]{inputenc}

\title{Cours 4}
\author{Jean-Baptiste Bertrand}
\date{\today}

%\setlength{\parindent}{4em}
\setlength{\parskip}{1em}


\usepackage[utf8]{inputenc}
\usepackage[T1]{fontenc}
\usepackage[french]{babel}
\usepackage{fancyhdr}
\usepackage[total={19cm, 22cm}]{geometry}
\usepackage{enumerate}
\usepackage{enumitem}
\usepackage{svg}
\usepackage{physics}
\usepackage{mathrsfs}  
\usepackage{tcolorbox}

%packages pour faire des math
%\usepackage{cancel} % hum... pas sur que je vais le garder mais rester que des fois c'est quand même sympatique...
\usepackage{amsmath, amsfonts, amsthm, amssymb}
\usepackage{esint}

\begin{document}

\maketitle

\section*{Rappel}

\begin{itemize}
	\item Une courbe est régulière ($\alpha'(t)\neq 0 \iff $) elle peut être oaramétrisé pr longeur d'arc ($||\tilde\alpha(s)|| \equiv 1$)
	\item Repère de Frenet de $\alpha$ paramétré par longeur d'arc $$T = \alpha'(s),\; N = \frac{T'(s)}{||T'(s)||} (\norm*{T'(s) = k(s)}),\; B=T\cross N$$
	\item Courbe \underline{birrégulière} $\to k(s)\neq 0$
	\item Équations de Frenet-Serret $$\begin{matrix}T' = && kN &\\ N' = &-kT& &+\tau B\\ B' = & &-\tau N &\end{matrix}$$
	\item $N'(s)\cdot B(s) =\tau(s)$
\end{itemize}


La torsion ($\tau$) mesure à quel point on sort d'un plan. La courbure ($k$) mesure à quel point on dévie d'une droite.

\underline{Exemple: Hélice}

$$\alpha(s) = \qty(a\cos(\frac sc), a\sin(\frac sc), b\frac sc)$$ où $c = \sqrt{a^2 + b^2}$

est paramétrisé par longueure d'arc

$$T(s)= \alpha'(s) = \qty(-\frac qc \sin(\frac s c), \frac ac\cos(\frac sc), \frac bc)$$

$$T'(s) = \qty(-\frac{a}{c^2}\cos(\frac sc), -\frac{a}{c^2}\sin{\frac sc}, 0)$$

$$\kappa(s) = \norm*{T'(s)} \frac{a}{c^2}$$

$$N = \qty(-\cos(\frac sc), -\sin(\frac sc),0)$$

$$B = T \cross N = \qty(\frac bc \sin(\frac sc), -\frac bc\cos(\frac sc), \frac ac)$$

$$N'(s)= \qty(\frac 1c\sin(\frac sc),-\frac 1c\cos(\frac sc), 0)$$

$$\tau(s)= N'\cdot B = \frac{b}{c^2}\sin^2(\frac sc)+ \frac{b}{c^2}\cos^2(\frac sc)+0 = \frac{b}{c^2}$$

\begin{tcolorbox}[title=Remarque]
	La courbure d<une coubre de $\mathbb{R}^3$ est \underline{toujours positive} (C'est une nrome) mais la torsion \underline{a un signe}. La torsion renseigne sur la chiralité.
\end{tcolorbox}


$$T'= \kappa N \checkmark$$ 
$$\cdots$$

\underline{Courbes non-paramétrées pas longueur d'arc}

Soit $\alpha$ une courbe birrégulière. On note $s(t)$ la reparamétrisation par longueur d'arc.

\begin{equation*}
	\dv{\alpha}{t} = \dv{\alpha(s(t))}{t} = \dv{\alpha(s(t))}{s}\dv{s}{t} \tag{*}
\end{equation*}

$$\norm*{\dv{\alpha}{t}} = 1 \abs{\dv{s}{t}}$$

la fonction $\dv{s}{t} = v(t)$ est la vitesse de $\alpha$
$$\dv{\alpha}{t} = T(s(t))v(t)$$

Pour calculer $N$

$$\dv{T(s(t)}{t} = \dv{T(s(t))}{s}\dv{s}{t} = \kappa(s(t))N(s(t))v(t)$$

$$\implies N(s(t)) =\frac{1}{v(t)}\dv{T(s(t)}{t} $$

On peut ensuite calculer $B$ et $\tau$

\pagebreak
\underline{Exemple}

$$\alpha(t) = \qty(3t-t^3, 3t^3, 3t+t^3)$$

$$\alpha'(t) = \qty(3-3t^2, 6t, 3+3 t^2) = 3\qty(1-t^2, 2t, 1+t^2)$$

$$v(t)= \norm*{\alpha'(t)}=\dots = 3\sqrt{2}(1+t^2)$$

$$T=\frac{\alpha'}{v}= \frac{1}{\sqrt 2(1+t^2)}\qty(1-t^2, 2t, 1+t^2)$$

$$\kappa N(t) = \frac{1}{v(t)}T'(t)=\dots=\frac 16 \qty(\frac{-4t}{1+t^2},\frac{2-2t^2}{(1+t^2)^2})$$


$$\kappa(t)=\norm*{k(t)N(t)}=\frac{1}{3(1+t^2)^2}$$

On calcul B, pas le temps de retranscrire

\end{document}