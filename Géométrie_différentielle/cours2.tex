\documentclass{article}
\usepackage[utf8]{inputenc}

\title{Cours 2}
\author{Jean-Baptiste Bertrand}
\date{December 2021}

\usepackage[utf8]{inputenc}
\usepackage[T1]{fontenc}
\usepackage[french]{babel}
\usepackage{fancyhdr}
\usepackage[total={19cm, 22cm}]{geometry}
\usepackage{enumerate}
\usepackage{enumitem}
\usepackage{svg}
\usepackage{physics}

%packages pour faire des math
%\usepackage{cancel} % hum... pas sur que je vais le garder mais rester que des fois c'est quand même sympatique...
\usepackage{amsmath, amsfonts, amsthm, amssymb}
\usepackage{esint}

\begin{document}

\maketitle


\underline{Courbe du jour}: Cycloïde: Trajectoire d'un point sur une roue qui tourne sans gilisser


Paramétrisation: $\alpha(t) = \underbrace{(tr, r)}_{\text{Centre du cerlce}} + (-r\sin t, -r \cos t)$

%\includesvg{cycloide.svg}

\headrule

\underline{Def:} La \underline{longeur d'arc} d'une courbe $\alpha: [a,b] \to \mathbb{R^3}$ est $l(\alpha) = \int_a^b ||a'(t)|| \dd t$

$\tilde{a}:[\tilde{a}, \tilde{b}] \to [a,b]$ est une \underline{reparamétrisation} de $\alpha$ sèil existe $\phi [\tilde a, \tilde b]\to [a,b]\; C^3$, bijective $C^3$ ...

\end{document}