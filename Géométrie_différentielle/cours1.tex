\documentclass{article}
\usepackage[utf8]{inputenc}

\title{meme}
\author{Jean-Baptiste Bertrand}
\date{December 2021}

\usepackage[utf8]{inputenc}
\usepackage[T1]{fontenc}
\usepackage[french]{babel}
\usepackage{fancyhdr}
\usepackage[total={19cm, 22cm}]{geometry}
\usepackage{enumerate}
\usepackage{enumitem}

%packages pour faire des math
%\usepackage{cancel} % hum... pas sur que je vais le garder mais rester que des fois c'est quand même sympatique...
\usepackage{amsmath, amsfonts, amsthm, amssymb}
\usepackage{esint}

\begin{document}


\section{Chapitre 1}

On s'interesse à qualifier des courbes sans étudier les propriétés des fonctions. Par exemple, on veut considérer $y=x^2$ et $y^2=x$ comme identique à roatition près malgré le fait qu'elle soit définis comme deux équations assez différentes.

On va distingues les propriétés intrinsèques et extrinsèques d'une surface.

Une propriété intrinsèques pourrait être détécté par quelqu'un vivant dans la surface.

La distance de longueure d'arc est une quantité intrinsèque à la sphère tandis que la \textit{logueure cordale} est une quantitée intrinsèque.

La \underline{corubure gaussiènne} est la plus importante quantitié intrinsèque associé à une surface.

La courbure gaussienne ne change pas si on la déforme de manière rigide.

\subsection{Courbure d'un polyèdre}

Défault d'angle:

$c(s)=2 \pi-\sum_{T \text { face }} \theta_{T}(s)$


\underline{La caractéristique d'Euleur} d'un polytope $P$ est la quantité

$$\Chi(P) = V - E + F$$


\end{document}