\documentclass{article}    
\usepackage[utf8]{inputenc}    
    
\title{Épisode 4}    
\author{Jean-Baptiste Bertrand}    
\date{\today}    
    
\setlength{\parskip}{1em}    
    
\usepackage{physics}    
\usepackage{graphicx}    
\usepackage{svg}    
\usepackage[utf8]{inputenc}    
\usepackage[T1]{fontenc}    
\usepackage[french]{babel}    
\usepackage{fancyhdr}    
\usepackage[total={19cm, 22cm}]{geometry}    
\usepackage{enumerate}    
\usepackage{enumitem}    
\usepackage{stmaryrd}    
\usepackage{mathtools,slashed}
%\usepackage{mathtools}
\usepackage{cancel}
    
\usepackage{pdfpages}
%packages pour faire des math    
%\usepackage{cancel} % hum... pas sur que je vais le garder mais rester que des fois c'est quand même sympatique...
\usepackage{amsmath, amsfonts, amsthm, amssymb}    
\usepackage{esint}  
\usepackage{dsfont}

\usepackage{import}
\usepackage{pdfpages}
\usepackage{transparent}
\usepackage{xcolor}
\usepackage{tcolorbox}

\usepackage{mathrsfs}
\usepackage{tensor}

\usepackage{tikz}
\usetikzlibrary{quantikz}
\usepackage{ upgreek }

\newcommand{\incfig}[2][1]{%
    \def\svgwidth{#1\columnwidth}
    \import{./figures/}{#2.pdf_tex}
}

\newcommand{\cols}[1]{
\begin{pmatrix}
	#1
\end{pmatrix}
}

\newcommand{\avg}[1]{\left\langle #1 \right\rangle}
\newcommand{\lambdabar}{{\mkern0.75mu\mathchar '26\mkern -9.75mu\lambda}}

\pdfsuppresswarningpagegroup=1


\begin{document}
2022-28-28

\section*{Règle d'or de Fermi}

Les fluctuations du vides sont essentielles pour expliquer la relaxation spontané

Pour prendre ces fluctuations en compte, on utilise la théorie des perturbations.


$$P_{e\to{}g} = \frac{1}{\har^2} \abs{\int_{0}^{t}\dd t' \mel{g}{H_{\text{int}}}{e} e^{-i\omega t} }^{2}$$ 



$$S_{ge} = \frac{W_{ge}}{\hbar} \left[ e^{i\varphi} \int_{0}^{t}\dd t' e^{i\left( \omega - \omega_0 \right)t' } + e^{-i\varphi} \int_0^{t}\dd t' e^{-i\left( \omega+ \omega_0 \right)t' } \right] $$ 

$$= \frac{W_{ge}}{i\hbar} \left[ e^{-i\varphi} \frac{e^{i\left( \omega - \omega_0 \right)t' }-1}{i \left( \omega-\omega_0	 \right) } - e^{-i\varphi} \cancel{\frac{e^{-i\left( \omega + \omega_0 \right)t' }-1}{i \left( \omega+\omega_0	 \right) }}  \right] $$ 

$$P_{e\to{}g} = \frac{\abs{W_{ge} }^{2}}{\hbar^2} t^{2}\sinc^{2}(\delta t/2) \qquad t_0 =0 $$ 

Pour tenir compte du champ électromagnétique :

$$\mathcal{H}_a \to \mathcal{H} _a \otimes \mathcal{H}_\gamma \qquad \ket{e} \to \ket{e,0} \qquad \ket{g} \to \ket{g,1}$$

$$P_{\text{emission}} = \frac{\abs{\mel{g, 1_{\lambda,\vb{k} }}{H_{\text{int}}{} }{e,0}}^{2}}{\hbar^2} \frac{\sinc^{2}(\delta t/2)}{\Delta^2}  $$ 

Il existe un continuum de modes $\lambda, \vb{k}$ 

\underline{Concept de ka densité de modes électromagnétiques} 

$$\rho (E) = \dv{N(E)}{E} $$ 

Sur un intervalle $\dd E$ 

$$\dd P_{\text{emission}} = \dd N(E) \frac{\abs{\mel{g,1_E}{H_{\text{int}} }{e,0}}^{2}}{\Delta^2} \frac{\sin^2(\Delta /2)}{\Delta ^2} $$ 

Donc

$$P_{\text{emission}} \frac{1}{\hbar^2} \int \dd E \rho (E) \abs{\mel{g,1_E}{H_{\text{int}} }{e,0}}^{2}$$ 

\underline{Fonction de dirac} 

$$\frac{\sin^{2}\Delta t/2}{\Delta^{2}} \xrightarrow{T\to \infty} \frac{\pi}{2} \hbar t \delta(\Delta) $$ 

\underline{Taux d'émission} 

$$\Gamma = \frac{\dd P_{\text{emission}} }{\dd t} $$ 

\begin{figure}[ht]
    \centering
    \incfig{densité}
    \caption{densité}
    \label{fig:densité}
\end{figure}


\underline{Atome à proximité d'un miroir} 

\setcounter{section}{2}

\section{Sources de photons uniques}

\begin{tcolorbox}[title=Objectifs]
	 \begin{itemize}
	 	\item Importance de ce type de source
		\item Différentes réalisations
		\item Characterization en sources de photons 
	 \end{itemize}
\end{tcolorbox}

\subsection{Pourquoi?}

La cryptographie quantique (BB84) et une application qui apporte énormément d'intérêt.

Un autre utilisation importante des sources de photons uniques est la production d'état quantique.


Un couplage optomécanique permet $$\ket{1 \text{photon}, 0	\text{phonon}   } \to \ket{0 \text{photon}, 1 \text{phonon}  }$$ 

\underline{Préparation d'état plus complexes à partir d'un photon} 

\begin{figure}[ht]
    \centering
    \incfig{lame-séparatrice}
    \caption{lame séparatrice}
    \label{fig:lame-séparatrice}
\end{figure}


\underline{atome à trois niveau} 

2 transitions permettent d'avoir un \textit{photon annonciateur}

Cependant, on a pas de controle sur le mode. Aussi même avec le photon annonciateur, on est pas certain du temps ou le second va arriver.

Pour corriger le manque de contrôle de la fréquence, on utilise un cavité. Pour qualifier la \textit{qualité} de cette fréquence on utilise le facteur de Purcell: $$F_p = \frac{\Gamma_{\text{cavité}} }{\Gamma_{\text{autre} + \Gamma_{\text{cavité}} } } $$ 


\underline{SPDC: Spontaneous Parametric Down Conversion} 



Cristal non-linéaire 


\begin{figure}[ht]
    \centering
    \incfig{guy-bernier}
    \caption{Guy bernier}
    \label{fig:guy-bernier}
\end{figure}

\setcounter{subsection}{2}


\subsection{Optique non-linéaire}

Exemple de système non-linéaire
\begin{itemize}
	\item Doubleur de fréquence
	\item Laser pulsé (Q switch)
	\item Amplification fibré 
\end{itemize}


La polarisation d'un milieu linéaire $$P(t) = \epsilon_0 \chi^{(1)} E(t) \qquad \chi^{(1)}: \text{susceptibilité} $$ 

Non linéaire

$$P(t)= \epsilon_0 \chi^{(1)} E(t) + \underbrace{\epsilon_0 \chi^{(2)}(t) + \epsilon_0 \chi^{(3)}(t) + \dotsb}_{\text{Non linéarie} } $$ 



$$\vb{E} = \vb{E}_0 e^{-i\omega t } + c.c. $$ 

$$P^{(2)}(t) = \underbrace{2\epsilon_0 \chi^{(2)} \vb{E}_0 \vb{E}_0^* }_{\text{rectification optique} } + \epsilon_0\chi^{(2)} \underbrace{\left( E_0^{2}e^{-2 i \omega t}+ E_0^{*2}e^{2i\omega t} \right)}_{\text{génération de seconde harmonique} }  $$ 





\end{document}
