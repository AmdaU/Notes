\documentclass{article}    
\usepackage[utf8]{inputenc}    
    
\title{Épisode 4}    
\author{Jean-Baptiste Bertrand}    
\date{\today}    
    
\setlength{\parskip}{1em}    
    
\usepackage{physics}    
\usepackage{graphicx}    
\usepackage{svg}    
\usepackage[utf8]{inputenc}    
\usepackage[T1]{fontenc}    
\usepackage[french]{babel}    
\usepackage{fancyhdr}    
\usepackage[total={19cm, 22cm}]{geometry}    
\usepackage{enumerate}    
\usepackage{enumitem}    
\usepackage{stmaryrd}    
\usepackage{mathtools,slashed}
%\usepackage{mathtools}
\usepackage{cancel}
    
\usepackage{pdfpages}
%packages pour faire des math    
%\usepackage{cancel} % hum... pas sur que je vais le garder mais rester que des fois c'est quand même sympatique...
\usepackage{amsmath, amsfonts, amsthm, amssymb}    
\usepackage{esint}  
\usepackage{dsfont}

\usepackage{import}
\usepackage{pdfpages}
\usepackage{transparent}
\usepackage{xcolor}
\usepackage{tcolorbox}

\usepackage{mathrsfs}
\usepackage{tensor}

\usepackage{tikz}
\usetikzlibrary{quantikz}
\usepackage{ upgreek }

\newcommand{\incfig}[2][1]{%
    \def\svgwidth{#1\columnwidth}
    \import{./figures/}{#2.pdf_tex}
}

\newcommand{\cols}[1]{
\begin{pmatrix}
	#1
\end{pmatrix}
}

\newcommand{\avg}[1]{\left\langle #1 \right\rangle}
\newcommand{\lambdabar}{{\mkern0.75mu\mathchar '26\mkern -9.75mu\lambda}}

\pdfsuppresswarningpagegroup=1

\begin{document}

2022-09-16

\section*{Résolution du Hamiltonien avec la théorie des perturbations}


$$H = \underbrace{-\hbar \left( \omega -\omega_0 \right) \op e - d_{eg} \vb{E}_0 \op{e}{g} -d_{eg}^{*}\vb{E}_0^*\op{g}{e}}_{H_{0}}  \underbrace{ - d_{eg}^{*}\op{e}{g} - d_{eg}\vb{E}_0^* e^{-2i\omega t} \op{g}{e}}_{V}  $$ 

états propres de $H_0$  

$$\ket{\pm} = (\cos,\sin) \frac{\theta}{2} e^{-i \frac{\varphi}{2} }\ket{e} \pm (\sin,\cos) \frac{\theta}{2}  e^{i \frac{\varphi}{2} }$$ 



L'approximation séculaire (qui consiste à négliger $V$), reviens à dire que la probabilité de transition $\ket{\pm} \to \ket{\mp}$ qu'il cause est très faible.



On veut connaître $\mathcal{P}_{\ket{+}\to \ket{-}}(t)$ 

$$\mathcal{P}_{\ket{+}\to \ket{-}} = \frac{1}{\hbar} \abs{\int_{0}^{t}\dd t' e^{-i \frac{E_+ - E_-}{ \hbar} t}\mel{+}{V}{-}}^2$$ 

à claculer à $\Delta =0$ et $\Omega \in \mathds{R} (\varphi=0)$  

\end{document}
