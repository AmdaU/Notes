\documentclass{article}    
\usepackage[utf8]{inputenc}    
    
\title{Épisode 4}    
\author{Jean-Baptiste Bertrand}    
\date{\today}    
    
\setlength{\parskip}{1em}    
    
\usepackage{physics}    
\usepackage{graphicx}    
\usepackage{svg}    
\usepackage[utf8]{inputenc}    
\usepackage[T1]{fontenc}    
\usepackage[french]{babel}    
\usepackage{fancyhdr}    
\usepackage[total={19cm, 22cm}]{geometry}    
\usepackage{enumerate}    
\usepackage{enumitem}    
\usepackage{stmaryrd}    
    
%packages pour faire des math    
%\usepackage{cancel} % hum... pas sur que je vais le garder mais rester que des fois c'est quand même sympatique...
\usepackage{amsmath, amsfonts, amsthm, amssymb}    
\usepackage{esint}  


\begin{document}
2022-11-18

\section*{États Habillés de Lumière}


\[ \ket{g} \otimes \ket{1} = \ket{g,1} \qquad \ket{e} \otimes \ket{0} = \ket{e,0} \]

\[ H_{\text{JC}} = \hbar \omega c \mqty(1 & 0 \\ 0  &1) + \frac{\hbar}{2} \mqty(\delta & \Omega \\ \Omega & -\delta)  \]

\[ \ket{+} = \sin\theta \ket{g,1} + \cos\theta \ket{e,0} \]
\[ \ket{+} = -\cos\theta \ket{g,1} + \sin\theta \ket{e,0} \]


\section*{Régime dispersif}

Comment utiliser le Hamiltonien des James Cummings pour mesurer l'état de l'atome.


\[ H_{\text{JC}} = \hbar \omega_c \left( a ^{\dagger} a + \frac{1}{2}  \right) + \frac{\hbar \omega}{2} \sigma_z + \frac{\hbar\Omega}{2} \left( \sigma^{+} ^{\dagger} _ \sigma a ^{\dagger} \right) 	   \]

à $\delta = 0$

\[ \ket{+} = \ket{g,1} + \ket{e,0} \qquad \ket{-} = - \ket{g,1} + \ket{e, 0} \]

\begin{tcolorbox}[title=]
	Une mesure projective du nombre de photon projette L'état de l'atome
	 
\end{tcolorbox}






\end{document}
