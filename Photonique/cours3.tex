\documentclass{article}    
\usepackage[utf8]{inputenc}    
    
\title{Épisode 4}    
\author{Jean-Baptiste Bertrand}    
\date{\today}    
    
\setlength{\parskip}{1em}    
    
\usepackage{physics}    
\usepackage{graphicx}    
\usepackage{svg}    
\usepackage[utf8]{inputenc}    
\usepackage[T1]{fontenc}    
\usepackage[french]{babel}    
\usepackage{fancyhdr}    
\usepackage[total={19cm, 22cm}]{geometry}    
\usepackage{enumerate}    
\usepackage{enumitem}    
\usepackage{stmaryrd}    
\usepackage{mathtools,slashed}
%\usepackage{mathtools}
\usepackage{cancel}
    
\usepackage{pdfpages}
%packages pour faire des math    
%\usepackage{cancel} % hum... pas sur que je vais le garder mais rester que des fois c'est quand même sympatique...
\usepackage{amsmath, amsfonts, amsthm, amssymb}    
\usepackage{esint}  
\usepackage{dsfont}

\usepackage{import}
\usepackage{pdfpages}
\usepackage{transparent}
\usepackage{xcolor}
\usepackage{tcolorbox}

\usepackage{mathrsfs}
\usepackage{tensor}

\usepackage{tikz}
\usetikzlibrary{quantikz}
\usepackage{ upgreek }

\newcommand{\incfig}[2][1]{%
    \def\svgwidth{#1\columnwidth}
    \import{./figures/}{#2.pdf_tex}
}

\newcommand{\cols}[1]{
\begin{pmatrix}
	#1
\end{pmatrix}
}

\newcommand{\avg}[1]{\left\langle #1 \right\rangle}
\newcommand{\lambdabar}{{\mkern0.75mu\mathchar '26\mkern -9.75mu\lambda}}

\pdfsuppresswarningpagegroup=1

\begin{document}

2022-09-09

\subsection*{Modèle de Lorentz, retour}


On se place à $\omega \gg \omega_ij$ 

Le déplacement est très faible, on peut négliger $\omega_{0}, \gamma$ 


\setcounter{section}{2}
\setcounter{subsection}{3}

\subsection{Modèles quantiques}

\begin{tcolorbox}[title=Objectifs]
\begin{itemize}
	\item Réintroduire la théorie des perturbation
	\item $\mathcal{P}_{\ket{g}\to \ket{e}}$ 
	\item Oscillations de Rabo
\end{itemize}	 
\end{tcolorbox}


		$$H_0 = \hbar\omega_e \ket{e}\bra{e} \left( + \hbar \cdot 0 \ket{g}\bra{g} \right) $$ 

	$$\ket{\psi(t)} = \gamma_g \ket{g} + \gamma_e e^{-i\omega_e t} \ket{e}$$ 

	Comment le système se couple à un champ E.M.?


$$H = H_0 + H_{\text{int}} $$ 

$$H_{\text{int}} = -\hat D \hat E(\vb{r},t)$$ 

$$\hat D: \text{ Opérateur de moment dipolaire } = q \hat r $$ 

Problème à deux niveaux

$$H = \hbar\omega_e \ket{e}\bra{e} -\hat D \hat E(r,t)$$ 


Approche perturbative : $H_{\text{int}}: \text{ faible }  $ 
$$H_{\text{int}} \to \lambda H_{\text{int}} \quad \lambda \ll 1$$ 


$$\psi(t) = \sum_n \gamma_n (t)\ket{n}$$ 

$$i\hbar \dv{{}}{t} \psi \ket{\psi(t)} = \left( H_0 + \lambda H_{\text{int}}  \right) \ket{\psi(t)}$$ 

On projette sur un $\ket{k}$ quelconque 

$$i\hbar \dv{{}}{t} \bra{k}\ket{\psi(t)} = \bra{k}H_0 \ket{\psi(t)} + \lambda \bra{k} H_{\text{int}}\ket{\psi(t)} $$ 

$$= E_k \bra{k} \ket{\psi} + \lambda \sum_n \bra{k} H_{\text{int}} \ket{n}\bra{n}\ket{\psi(t)}$$ 

$$i\hbat \left[ - \frac{E_k}{\hbar} + \dv{{}}{t} \gamma_k (t) \right]e^{-iE_k t/\hbar} = e_k \gamma_{k(t)} e^{-i E_k t/\hbar} + \lambda \sum_m \bra{k} H_{\text{int}} \ket{n} \psi_n (t) e^{(E_n -E_k )t/\hbar} $$ 


donc,

$$\forall \ket{k},\quad \dv{{}}{t} \gamma_k (t) = \lambda \sum_n \bra{k} H_{\text{int}} \ket{n} \gamma_n (t) e^{-i \frac{E_n -E_k }{\hbar}t } $$ 

Cela est la solution exacte et n'est, évidemment, pas facile à résoudre en général.

On fait donc une série en $\lambda$ 

$$\gamma_{k(t)} = \gamma_k^{(0)}(t) +\lambda \gamma_k^{(1)}(t) + \lambda^{2}\gamma_k^{(2)}(t) + \dotsb$$ 


$$\gamma_e^{(1)} = \frac{1}{i\har} \int_{t_0}^{t}\dd t' \bra{e}H_{\text{int}}\ket{e} \gamma_e^{(0)} e^{-i \delta E_{eg} t/\hbar} + \dotsb   $$ 

$$\gamma_e^{(1)} (t) = \frac{1}{i\hbar} \int_{t_0}^{t}\dd t \bra{\psi}{e} H_{\text{int}} \ket{g} e^{-i \Delta E_{ge} t/\hbar} $$ 

On va considérer un champ éléctrique de la forme

$$\vb{E} (\vb{r},t) = \vb{E}_0 \cos(\omega t + \varphi)$$ 

$$H_{\text{int}} = \hat W \cos(\omega t \pm \varphi) $$ 
$$\hat W = \hat D \vb{E}_0 = q \hat r \vb{E}_0$$ 


$$\gamma_e (t) = \frac{W_{eg}}{i\hbar} \\int_{t_0}^{t}\dd t' \cos(\omega t' + \varphi) e^{-i \frac{E_g -E_{e}{\hbar}} t' }$$ 

$$\gamma_{e(t)\approx} \frac{W_{eg}}{2i\hbar} \int_{t_{0}}^{t} \dd t' \left[ e^{i\psi} e^{i\omega t'}+ e^{-i\varphi-i\omega t'} \right] e^{i\omega_{eg} t'} $$ 


$$\dotsb$$ 




\end{document}
