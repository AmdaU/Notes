\documentclass{article}    
\usepackage[utf8]{inputenc}    
    
\title{Épisode 4}    
\author{Jean-Baptiste Bertrand}    
\date{\today}    
    
\setlength{\parskip}{1em}    
    
\usepackage{physics}    
\usepackage{graphicx}    
\usepackage{svg}    
\usepackage[utf8]{inputenc}    
\usepackage[T1]{fontenc}    
\usepackage[french]{babel}    
\usepackage{fancyhdr}    
\usepackage[total={19cm, 22cm}]{geometry}    
\usepackage{enumerate}    
\usepackage{enumitem}    
\usepackage{stmaryrd}    
\usepackage{mathtools,slashed}
%\usepackage{mathtools}
\usepackage{cancel}
    
\usepackage{pdfpages}
%packages pour faire des math    
%\usepackage{cancel} % hum... pas sur que je vais le garder mais rester que des fois c'est quand même sympatique...
\usepackage{amsmath, amsfonts, amsthm, amssymb}    
\usepackage{esint}  
\usepackage{dsfont}

\usepackage{import}
\usepackage{pdfpages}
\usepackage{transparent}
\usepackage{xcolor}
\usepackage{tcolorbox}

\usepackage{mathrsfs}
\usepackage{tensor}

\usepackage{tikz}
\usetikzlibrary{quantikz}
\usepackage{ upgreek }

\newcommand{\incfig}[2][1]{%
    \def\svgwidth{#1\columnwidth}
    \import{./figures/}{#2.pdf_tex}
}

\newcommand{\cols}[1]{
\begin{pmatrix}
	#1
\end{pmatrix}
}

\newcommand{\avg}[1]{\left\langle #1 \right\rangle}
\newcommand{\lambdabar}{{\mkern0.75mu\mathchar '26\mkern -9.75mu\lambda}}

\pdfsuppresswarningpagegroup=1

\begin{document}

\section*{Solution non-pertubative}


$$H_0 = \hbar \omega_0 \ket{e}\bra{e} = \hbar\omega_0 \mqty(0& 0\\ 0 & 1)$$ 

\begin{tcolorbox}[title=]
	$$\ket{e}= \mqty(0\\1) \qquad \ket{g} = \mqty(1\\0)$$  
\end{tcolorbox}

$$E = - \mel{e}{\hat D \cdot \vb{E}(\vb{r},t)}{g}$$ 

$$H_{\text{int}} = - \hat D \cdot \vb{E} (\vb{r}, t) = - \hat D \left( \vb{E}_0(\vb{r})e^{-i\omega t} +cc \right) $$ 

$$= \left( \op{g} + \op{e} \right) \left[ -\hat D \vb{E} (\vb{r},t) \right] \left( \op{g} + \op{e} \right)  $$ 

$$= - \op{g} \hat d \cdot \vb{E} \op{g} - \op{g} \hat \cdot \vb{E} \op{e} - \dotsb$$ 


$$H = \hbar \mqty(0 & 0 \\ 0 &\omega_{0}) - \mqty(\mel{g}{\hat D \vb{E}}{g} & \mel{g}{\hat D \vb{E}{e}}{e}\\ \mel{e}{\hat D \vb{E}}{g} &\mel{e}{\hat D \vb{E}}{e})$$ 


\underline{importance des symétries} 

On va regarder l'effet de l'opérateur parité sur notre système.

$$\hat H_e = \frac{P^{2}}{2m} + V_{\text{coul}}(\vb{r}) $$ 

on compare $H_e \Pi$ et $\Pi H_e$: si $H_e$ commute avec l'opérateur parité, le système à un symétrie d'inversion spatiale. 

$$H_e \Pi f(x) = H_e f(-x) = - \frac{\hbar^{2}}{2m} f''(-x) + V_{\text{coul}} f(x) $$ 

dans l'aute sens

$$\Pi H_e f(x) = \Pi \left( -\frac{\hbar^{2}}{2m} f''(x) + V_{\text{coul}}(x) f(x)  \right) = -\frac{\hbar^{2}}{2m} f''(-x) + V_{\text{coul}} (x) f(-x) = H_e \Pi f(x)   $$ 

Donc $[\Pi, H_e] =0$ si $V(x)=V(-x)$, ce qui est vrai pour les atomes. 

Pour un vecteur propre $\ket{n}$ 
$$H_e \Pi \ket{n} = \Pi H_e \ket{n} = \Pi E_n \ket{n} = E_n \left( \Pi \ket{n} \right) $$ 

Donc si $\ket{n}$ est un vecteur propre, alors $\Pi\ket{n}$ l'est aussi.  

$$\Pi^{2}\ket{n} =\ket{n}$$ 

$$\Pi \ket{n} = \pm \ket{n}$$ 

Si $\ket{n}$ est un vecteur propre de $H_e$, c'est aussi un vecteur propre de $\Pi$ avec une valeur propre de $\pm 1$   

\begin{tcolorbox}[title=]
	Pour un atome $\psi_e$ et $\psi_g$ sont soit pair, soit impair.   \centering
\end{tcolorbox}

$$\mel{e}{\hat D}{e} = q \int \psi_e^{*}\hat r \psi_e =0 = \bra{g} \hat D \ket{g}$$ 

$$H_{\text{int}} = -\mqty(0 & \mel{g}{D \cdot E}{e}\\ \mel{e}{D \cdot E}{g}& 0) $$ 

Les orbitales $\psi_e$ et $\psi_g$ ne présente pas de moments dipolaire permanents.  

$E_n$ posant $d_{eg} = \mel eDg$  

$$\hat H = \hbar \omega_0 \op e - d_{eg} \vb{E}(\vb{r},t) \op{e}{g} - d_{eg}^{*}\vb{E}(\vb{r},t) \op{g}{e}$$ 

\underline{Changement de base pour $H$: Base tournante avec la pompe $\left( e^{-i\omega t} \right) $  } 

Si on prend un unitaire $U$  

Dans la nouvelle base (changement de base définis par $U$ )

$$H' = UHU^{\dagger}+i\hbar \dv{U}{t} U$$ 


Shro:

$$\dv{\psi}{t} = \frac{-i}{\hbar} H\psi$$ 


$$U \dv{\psi}{t} = \frac{-i}{\hbar}  UH\underbrace{\mathds{1}}_{U^{\dagger}U} \Psi $$ 

$$\dv{{}}{t} (U\psi)  = \dv{U}{t} \psi + U \dv{\psi}{t} = \dv{U}{t} = \dv{U}{t} U^{\dagger}U \psi + U \dv{u}{t} $$ 

$$\dv{{}}{t} (U\psi) = \frac{-i}{\hbar} UHU^{\dagger}U\psi + \dv{U}{t} U^{\dagger}U \psi = \frac{-i}{\hbar} \left( U H U^{\dagger}+ i\hbar \dv{U}{t} U^\dagger \right) U \psi $$ 


$$U = e^{i\omega t \op e}$$ 

$$U \ket{e} = e^{i\omega t \op e} \ket{e}  = \sum_k \frac{1}{k!} \left( i\omega t \ope \right) ^k \ket{e} = \sum_k \frac{1}{k!} \left( i\omega \right) ^k \ket{e} = e^{i\omega t}\ket{e}$$ 


$$\bra{e} U^{\dagger}= e^{-i\omega t} \bra{e}$$ 


$$i\hbar \left( \dv{{}}{t} U \right) U^{\dagger}= i\hbar \left( i\omega \op e  \right) e^{i\omega t \op e} e^{-i\omega t \op e} = -\hbar \omega \op e$$ 

$$UHU^{\dagger} = \dotsb$$ 


$$\dotsb$$ 


$$H' = -\hbar \underbrace{\Delta}_{\omega-\omega_0 }  \op e - d_{eg} \vb{E}_o \op {e}{g} - d_{eg} \vb{E}_0^* e^{2i\omega t} \op{e}{g} + d_{eg}^{*}\vb{E}_0 e^{-2i\omega t} \op{g}{e} + d_{eg}^{*}\vb{E}_0 e^{-2i\omega t} \op{g}{e}$$ 

\begin{tcolorbox}[title=]
	\centering Dans la base tournante on \textit{renormalise} l'énérgie $\hbar \omega_{0\to} \hbar(\omega-\omega_0)$ 
\end{tcolorbox}

\begin{tcolorbox}[title=Approximation Séculaire (Rotating wave approximation | RWA)]
	 \centering Négliger les termes en $2\omega t$ 
\end{tcolorbox}


$$\boxed{H_{\text{séculaire}} = -\hbar \Delta \op e - d_{eg} \vb{E}_0 \op{e}{g} d_{eg}^{*} E_0^{*} \op{g}{e} }$$ 

Pour obtenir $\mathcal{P}_{g\to e}(t)$, il faut diagonaliser $H_{\text{sec}} $  

On pose $\underbrace{\Omega}_{\text{Freq. de Rabi} }  = - \frac{2d_{eg}\vb{E}_0}{\hbar} $  

$$H_{\text{sec}} = \hbar \mqty(0 & \frac{\Omega}{2} \\ \frac{\Omega}{2} & -\Delta)$$ 

$$\implies E_{\pm} = - \frac{\hbar\Delta}{2} \pm \frac{\hbar}{2} \sqrt{\abs{\Omega}^{2}+\Delta^2}$$ 


Pour $\Delta =0$  

$$\ket{+} = \frac{\sqrt{2}}{2 } \left( e^{-i \frac{\varphi}{2 } } \right) $$ 

$$a f g/af$$ 

\end{document}
