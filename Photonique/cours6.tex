\documentclass{article}    
\usepackage[utf8]{inputenc}    
    
\title{Épisode 4}    
\author{Jean-Baptiste Bertrand}    
\date{\today}    
    
\setlength{\parskip}{1em}    
    
\usepackage{physics}    
\usepackage{graphicx}    
\usepackage{svg}    
\usepackage[utf8]{inputenc}    
\usepackage[T1]{fontenc}    
\usepackage[french]{babel}    
\usepackage{fancyhdr}    
\usepackage[total={19cm, 22cm}]{geometry}    
\usepackage{enumerate}    
\usepackage{enumitem}    
\usepackage{stmaryrd}    
\usepackage{mathtools,slashed}
%\usepackage{mathtools}
\usepackage{cancel}
    
\usepackage{pdfpages}
%packages pour faire des math    
%\usepackage{cancel} % hum... pas sur que je vais le garder mais rester que des fois c'est quand même sympatique...
\usepackage{amsmath, amsfonts, amsthm, amssymb}    
\usepackage{esint}  
\usepackage{dsfont}

\usepackage{import}
\usepackage{pdfpages}
\usepackage{transparent}
\usepackage{xcolor}
\usepackage{tcolorbox}

\usepackage{mathrsfs}
\usepackage{tensor}

\usepackage{tikz}
\usetikzlibrary{quantikz}
\usepackage{ upgreek }

\newcommand{\incfig}[2][1]{%
    \def\svgwidth{#1\columnwidth}
    \import{./figures/}{#2.pdf_tex}
}

\newcommand{\cols}[1]{
\begin{pmatrix}
	#1
\end{pmatrix}
}

\newcommand{\avg}[1]{\left\langle #1 \right\rangle}
\newcommand{\lambdabar}{{\mkern0.75mu\mathchar '26\mkern -9.75mu\lambda}}

\pdfsuppresswarningpagegroup=1

\begin{document}

2022-09-21
\section*{Retour}

$$H= -\hbar (\omega-\omega_{0}) \op e  -\vb{d}_{eg} \vb{E}_0 \op{e}{g} - \vb{d}_{eg} \vb{E}_0 \op{g}{e} + V$$ 


$$\Omega =\abs{\frac{\vb{d}_{eg}\vb{E}_0}{\hbar} } \qq{,} \omega \gg \sqrt{\abs{\Omega}^2 + \Delta^2 }$$ 

Comment calcule-t-on la probabilité d'exciter l'atome? ( $\mathcal{P}_{\ket{g}\to\ket{e}}(t) $  )



$$H \ket{+} = e^{-i \frac{E_+}{\hbar} t} \ket{+}$$ 
$$H \ket{-} = e^{-i \frac{E_-}{\hbar} t} \ket{-}$$ 

$$\text{État initial:} \ket{\psi(0)}=\ket{g} = \left( \op + + \op - \right) \ket{g}$$ 

Pour un $t$ quelconque: $$\ket{\psi(t)}=  e^{-i Ht/\hbar} \left( \op + + \op- \right) \ket{g} = \ip{+}{g}e^{-i E_+ t/\hbar} \ket{+} + \ip{-}{g}e^{-iE_+ t/\hbar} \ket{-}$$  

$$\mathcal{P} _{g \to e} (t) = \abs{\ip{e}{\psi(t)}}^{2}$$ 


$$= \ip{+}{g}e^{-i E_+ t/\hbar} \bra{e}\ket{+} + \ip{-}{g}e^{-iE_+ t/\hbar} \bra{e}\ket{-}$$ 

$$\ip{e}{\psi(t)} = \dotsb = 2\sin \frac{\theta}{2} \cos \frac{\theta}{2} e^{i \varphi} \left\{ \frac{e^{-iE_+ t /\hbar} - e^{-i E_- t/\hbar}}{2}  \right\} $$ 

$$\boxed{ \mathcal{P} _{g\to e}(t) = \sin^{2}\theta \sin^{2}\left( \frac{t}{2} \sqrt{\Delta^{2} + \abs{\Omega}^2 } \right)  }$$ 

valable si $\tilde \Omega = \sqrt{\Delta^{2}+ \abs{\Omega}^{2}}\ll \omega \text{ : Fréquence de la pompe } $ 


Es-ce possible que cette approximation ne soit plus valide? ( $\tilde \Omega \sim \omega$  ) 

Pour pomper l'atome, on prendre $\Delta = 0$ 

$$\abs{\Omega} \sim \omega ?$$ 

$$\abs{\Omega} = \abs{\frac{\vb{d}_{eg} \cdot \vb{E}_0}{\hbar} }  $$ 

Possible en utilisant un laser toujours plus puissant!
$$\mathcal{P} _{g\to e} (t) = \frac{\abs{\Omega}^{2}}{\abs{\Omega}^{2}+\Delta^2} \sin^{2}\left( \frac{t}{2} \sqrt{\abs{\Omega}^2 + \Delta^2 } \right)  $$ 

Limites grand décalages : $$\mathcal{P} _{g \to e} (t) = \frac{\abs{\Omega}^{2}}{\Delta^{2}+\abs{\Omega}^{2}} $$ 


Limites petit décalages : $$\mathcal{P} _{g \to e} (t) = 1 - \frac{\Delta^{2}}{\abs{\Omega}^{2}} $$ 

\begin{figure}[ht]
    \centering
    \incfig{boule-de-bloch}
    \caption{boule de Bloch}
    \label{fig:boule-de-bloch}
\end{figure}

\section*{Règle d'or de Fermi}

Atome en interaction avec un champ $\vb{E}$ :
$$H = E_+ \ket{+} + E_- \ket{-}$$ 

Avec le champ, on peut faire $\ket{g}\to \ket{e}$ 

Une fois dans l'état $\ket{e}$, l'atome y reste si  le champ est nul puisque les états $e$ et $g$ ne sont plus couplées . 

Comment alors, peut-il y avoir un émission spontané?: Les fluctuations du vide couple les états $e$ et $g$  



\end{document}
