\documentclass{article}    
\usepackage[utf8]{inputenc}    
    
\title{Épisode 4}    
\author{Jean-Baptiste Bertrand}    
\date{\today}    
    
\setlength{\parskip}{1em}    
    
\usepackage{physics}    
\usepackage{graphicx}    
\usepackage{svg}    
\usepackage[utf8]{inputenc}    
\usepackage[T1]{fontenc}    
\usepackage[french]{babel}    
\usepackage{fancyhdr}    
\usepackage[total={19cm, 22cm}]{geometry}    
\usepackage{enumerate}    
\usepackage{enumitem}    
\usepackage{stmaryrd}    
\usepackage{mathtools,slashed}
%\usepackage{mathtools}
\usepackage{cancel}
    
\usepackage{pdfpages}
%packages pour faire des math    
%\usepackage{cancel} % hum... pas sur que je vais le garder mais rester que des fois c'est quand même sympatique...
\usepackage{amsmath, amsfonts, amsthm, amssymb}    
\usepackage{esint}  
\usepackage{dsfont}

\usepackage{import}
\usepackage{pdfpages}
\usepackage{transparent}
\usepackage{xcolor}
\usepackage{tcolorbox}

\usepackage{mathrsfs}
\usepackage{tensor}

\usepackage{tikz}
\usetikzlibrary{quantikz}
\usepackage{ upgreek }

\newcommand{\incfig}[2][1]{%
    \def\svgwidth{#1\columnwidth}
    \import{./figures/}{#2.pdf_tex}
}

\newcommand{\cols}[1]{
\begin{pmatrix}
	#1
\end{pmatrix}
}

\newcommand{\avg}[1]{\left\langle #1 \right\rangle}
\newcommand{\lambdabar}{{\mkern0.75mu\mathchar '26\mkern -9.75mu\lambda}}

\pdfsuppresswarningpagegroup=1

\begin{document}

2022-09-02

\section*{Emission spontanée + Absorbtion + Émission spontannées}

\begin{figure}[ht]
    \centering
    \incfig{rebop}
    \caption{rebop}
    \label{fig:rebop}
\end{figure}

$$\dv{n_{2}{t}} = -A_{21} n_2 -I_{\nu} B_{12} \Delta n$$ 

\setcounter{subsection}{4}
\setcounter{section}{1}

\subsection{Inversion de population}
    
Comment obtenir $\Delta n >0$ 

Equilibre thermodynamique local : $n_2 = e^{-\Delta E /kT}n_1 \leq n_1 \implies \Delta n \leq 0$ 

État stationnaire: $n_2 = \frac{Bh\nu{}n_{p}{A_{21}+Bh\nu} n_p}  n_1 < n_1 \implies \Delta n \leq 0$  


\begin{figure}[ht]
    \centering
    \incfig{diagramme-énérgétique-typique-d'un-laser}
    \caption{Diagramme énérgétique typique d'un laser}
    \label{fig:diagramme-énérgétique-typique-d'un-laser}
\end{figure}

On veut une cavité qui correspond au mode du photo $\gamma_{21}$ pour que les photons resent  et maximisent le processus d'émission spontané.



\begin{align*}
    \dv{n_{3}{t}} &= \gamma_{\omega} \left( n_0 -n_3 \right) -\gamma_{32} n3\\ \dv{n_{2}{t}} &= \gamma_{32} n_3 - \gamma_{31} n_2 + Bh \nu n_p (n_1 - n_2 ) \\ \dv{n_1 }{t} &= \gamma_{21} n_2 -\gamma_{10} n_1 - B h\nu n_p (n_1 -n_2 )\\ \dv{n_0 }{t} &= \gamma_{10} n_1 - \gamma_{\omega} (n_0 -n_3)
\end{align*}


\section{Émetteurs à deux niveaux}

\begin{tcolorbox}[title=Objectifs]
    \begin{itemize}
     \item Montrer comment certains modèles classiques peuvent donner des predictions exacte dans l'interaction atom/lumière (dans certaines limites)
         \begin{itemize}
             \item Indice de réfraction (nuage d'atome)
                \item radiation d'un atome
                \item effets mécaniques de la lumière
                \item refroidissement d'atome (ralentire le centre de masse)
                \item (Emission collective)
         \end{itemize}
    \end{itemize}
\end{tcolorbox}




\subsection{Oscillateur harmonique}

$$m \ddot x + m\omega_0^{2}x=0 $$ 

$$ m = \frac{m_{e} m_n }{n_e + m_{n}} \approx n_e  $$ 

$$\vb{E}(r,t) = \vb{E}^+ (\vb{r}) e^{i\omega t} + \vb{E}(\vb{r}) e^{-i\omega t} \qquad \vb{E}^+ = \left( \vb{E}^- \right)^*  $$ 


$$m \ddot x + m\omega_0^{2}x=q \vb{E}(\vb{r},t) $$ 

$$\dotsb$$ 

$$\boxed{x_0^{+}= \frac{eE^+/m}{\omega^{2}-\omega_0^{2}} }$$ 

Moment dipolaire élécrique

$$d \sim 1 \text{[e \AA]} $$ 

$$\vb{d}^+ = -ex^+$$ 
$$ = -\frac{e^{2}}{m} \frac{E^+}{\omega^2-\omega_0^2} $$ 


La desnité de polarisation est donc $$\vb{P} = N d^{+}= \hat\epsilon \frac{Ne^{2}}{m} \frac{\vb{E}_0^+}{\omega_0^2} e^{-i \omega t} $$ 

$$\chi = \frac{\vb{P}}{\vb{E}} = \frac{Ne^{2}}{m} \frac{1}{\omega_0^{2}- \omega^2} = \frac{N}{\epsilon_0} \alpha(\omega)$$ 

\subsection{Modèle de Loretz}

On ajoute de la dissipation

$$x_0^{+}= \frac{e}{m} \frac{E_0^{+}}{\omega^{2}-\omega_0^{2}+i\gamma\omega} $$ 

$$\alpha(\omega) = \frac{e^{2}}{m} \frac{1}{\omega_0^{2}-\omega^{2}-i\gamma\omega} $$ 


indice de réfraction : $$\tilde n (\oemga) = \sqrt{1 +\chi(\omega)} \approx 1 + \frac{\chi(\omega)}{2}$$ 

$$\tilde n (\omega = 1 + \frac{Ne^{2}}{2m\epsilon_0} \frac{(\omega_0 -\omega)}{\left( \omega_0^{2}-\omega^2 \right)^2 +\gamma^2\omega^2} +i \frac{Ne^{2}}{2m\epsilon_0} \frac{\gamma \omega}{\left( \omega_0^{2}-\omega^2 \right) + \gamma^2\omega^2} $$ 


coefficient d'absorption $a(\omega)$ : $$\dv{I}{z} = -a(\omega)I$$  
$$a(\omega) \equiv 2k_0 \Im \left[ \tilde n (\omega) \right] $$ 
$$n (\omega) \equiv \Re \left[ \tilde n (\omega) \right] $$ 

\underline{retard de phase} 

$$\delta = tan^{-1} \left( \frac{\gamma\omega}{\omega_0^{2}-\omega^2}  \right) $$ 

\begin{tcolorbox}[title=]
    Le modèle de Lorentz et valide à \texit{à basse puissance} 
\end{tcolorbox}

\begin{figure}[ht]
    \centering
    \incfig{indice-de-réfraction-complexe}
    \caption{indice de réfraction complexe}
    \label{fig:indice-de-réfraction-complexe}
\end{figure}




\begin{figure}[ht]
    \centering
    \incfig{delta}
    \caption{delta}
    \label{fig:delta}
\end{figure}
\clearpage



\pagebreak

\subsection{Limite de l'approche classique}

Reproduit beaucoup d'effets à faible intensité.


Le coefficient d'absorption correct?


$$a(\omega) = \sigma(\omega) N$$ 


$$\sigma_c = \eval{\frac{e^2}{m\epsilon_0 c\gamma}}_{\omega=\omega_0}  $$ 


Avec un traitement quantique, on obtiens

$$\sigma_q = \frac{2\pi{}c^{2}}{\omega_{12}^{2}} $$ 

Définissons un terme de correction $$f_{12} = \frac{\sigma_{q}}{\sigma_{c}} =  \dotsb \frac{g_2}{g_1}  $$ 


Pour des amplitude très faibles, on trouve le comportement de l'oscillateur harmonique 

$$\chi (\omega) \to - \frac{N e^2}{m \epsilon_0 \omega^{0}} \implies \sum_{i} f_{1i} = 1 $$ 

\textit{fudge factor}


\subsection{Modèle quantique}


Modèle simple de l'atome + approche pertubatice pour calculer la probabilité de transition $1 \to 2$ 

Entrevoir les oscillations de Rabi





\end{document}
