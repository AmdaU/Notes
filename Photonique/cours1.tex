\documentclass{article}    
\usepackage[utf8]{inputenc}    
    
\title{Épisode 4}    
\author{Jean-Baptiste Bertrand}    
\date{\today}    
    
\setlength{\parskip}{1em}    
    
\usepackage{physics}    
\usepackage{graphicx}    
\usepackage{svg}    
\usepackage[utf8]{inputenc}    
\usepackage[T1]{fontenc}    
\usepackage[french]{babel}    
\usepackage{fancyhdr}    
\usepackage[total={19cm, 22cm}]{geometry}    
\usepackage{enumerate}    
\usepackage{enumitem}    
\usepackage{stmaryrd}    
\usepackage{mathtools,slashed}
%\usepackage{mathtools}
\usepackage{cancel}
    
\usepackage{pdfpages}
%packages pour faire des math    
%\usepackage{cancel} % hum... pas sur que je vais le garder mais rester que des fois c'est quand même sympatique...
\usepackage{amsmath, amsfonts, amsthm, amssymb}    
\usepackage{esint}  
\usepackage{dsfont}

\usepackage{import}
\usepackage{pdfpages}
\usepackage{transparent}
\usepackage{xcolor}
\usepackage{tcolorbox}

\usepackage{mathrsfs}
\usepackage{tensor}

\usepackage{tikz}
\usetikzlibrary{quantikz}
\usepackage{ upgreek }

\newcommand{\incfig}[2][1]{%
    \def\svgwidth{#1\columnwidth}
    \import{./figures/}{#2.pdf_tex}
}

\newcommand{\cols}[1]{
\begin{pmatrix}
	#1
\end{pmatrix}
}

\newcommand{\avg}[1]{\left\langle #1 \right\rangle}
\newcommand{\lambdabar}{{\mkern0.75mu\mathchar '26\mkern -9.75mu\lambda}}

\pdfsuppresswarningpagegroup=1

\begin{document}

\underline{{ \Huge Photonique et optique quantique}} 


2022-08-31

\begin{tcolorbox}[title=Références]
\begin{itemize}
	\item D. Steck : quantum optics note
	\item G. Milbur : Quantum optics
	\item Aspect Fabres: Introduction to Quantum Optics
\end{itemize}
	 
\end{tcolorbox}

\section*{Contenu du cours}

\begin{itemize}
	\item Interaction lumière-matière
	\item Les degrées de liberté internes
		\begin{itemize}
			\item LASER
			\item \underline{LDOS} local density of optical states
			\item Source de photon unique (cryptographie quantique)
		\end{itemize}
	\item Propriété des émetteurs à deux niveaux
		\begin{itemize}
			\item matière classique
		\end{itemize}
	\item effet d'optique non-linéaire.
		\begin{itemize}
			\item SPDC (source de pair de photons)
		\end{itemize}
	\item Effet mecaniques
		\begin{itemize}
			\item Refroidissement doppler	
			\item Pince optique
			\item optomécanique
		\end{itemize}
\end{itemize}

\section*{Table des Matière}

\bf{Chapitre 1}: Physique des LASER

\bf{Chapitre 2}: émetteurs à 2 niveaux

Chapitre 3: Source de photon unique

Chapitre 4: Cryptographique quantique et clef quantique

Chapitre 5: Modèle de Jaynes-Cummings et mesure dispersive

Chapitre 6: Mesure quantique et non démolition (QND)

Chapitre 7: Optomécanique


\section{Physique des LASER}


\subsection{Histoire}

L'emission des atomes est introduit en 1926 en s'inspirant de la radioactivité. 

$$\frac{d {N_k}}{d {t}} = - A_k N_k$$

$$N_k(t) = N_k(u) e^{-A_k t}$$ 


On s'imagine le sytéme a deux niveau (atome) comme pouvant soit se désexiter ou pas avec 50\% de chance après une temps $\Delta t$. Ce modèle mène directement à la décroissance exponentielle.

\begin{figure}[ht]
    \centering
    \incfig{probabilites}
    \caption{probabilites}
    \label{fig:probabilites}
\end{figure}

L'état 1 est l'état desecité et comprende $n_1$ atomes, similaire pour $E_2$  

Processus d'absorption

$$\pdv{n_{2}{t}} = +I_j B_{12} n_1$$ 

$B_{12} $ Coefficient de Einstein 
$$I_v = \frac{1}{4\pi} \iint i_{V(k')} \dd k' \underbrace{\psi(\nu)}_{\text{chevauchement frequence phot et at} } \dd\nu$$ 


Taux d'absorption doit dépendre des photons incidents (densité, mode, fréquence)

$$\dv{n_{2}}{t} = -A_{21} n_2 + I_{\nu} B_{12} n_1$$ 
$$\dv{n_2 }{t} = A_{21} n_2 - U\nu B_{12} n_1$$ 

$A,B$ sont des constantes 

Que ce passe-t-il à l'équilibre thermodynamique local.

État stationnaire $$\frac{d {n_2}}{d {t}} = - \frac{d {n_2}}{d {t}} =0$$ 

Éqilibre thermodynamique: $$\frac{n_2}{n_1} = \frac{g_2}{g_1} e^{-\frac{E_2-E_1}{kT} }$$ 


\underline{Rayonnement du corps noir} 

$$I_{\nu} = \frac{2h\nu^{3}}{c^{2}} \frac{1}{e^{h\mu/kT}-1}  $$ 

$$A_{21} n_2 + I_{\nu} B_{12} n_1 =0 \iff \frac{n_2}{n_1} \frac{A_{21}}{ B_{12}} = I_{\nu} $$ 

$$\implies \frac{g_1}{g_2} e^{\Delta E/ kT } \frac{A_{21}}{B_{12}} = \frac{2\hbar\nu^{3}}{c^2} \frac{1}{e^{\hbar\nu/kT} -1} $$ 

Ce résultat n'a aucun sens. Le problème est qu'il manque l'émission stimulée. 

\begin{figure}[ht]
    \centering
    \incfig{emission-stimulée}
    \caption{emission stimulée}
    \label{fig:emission-stimulée}
\end{figure}

Nouvelle équation
$$ - A_{21} n_2 + I_{\nu} B_{12} n_1 -I_{\nu} B_{21} n_2$$ 

Équilibre thermodynamique local

$$A_{21} n_2 = I_{\nu} B_{12} n_1 - I+\nu B_{21} N_2 \iff I_{\nu} = \frac{A_{21} n_{2}}{{B_12} n_1 -B_{21} n_{2}}= \frac{A_{21}}{B_{12}} \frac{1}{\frac{n_1}{n_2} - \frac{B_{12}}{B_{21}} }  $$ 

$$\frac{n_2}{n_1} = \frac{g_2}{g_1} e^{-\Delta E /kT}$$ 

$$\frac{A_{21}}{B_{12}} \frac{1}{\frac{g_1}{g_2} e^{-\Delta/ kt} - \frac{B_{21} }{b_12} }  = \frac{2(\Delta{}E)^3}{h^2c^2} \frac{1}{e^{\Delta E/kt}-1} $$ 


Puisque c'est vrai pour toute température, on doit avoir que $$g_2 B_{21} = g_1 B_{12} $$ 

On peut écrire 

$$\pdv{n_{2}{t}} = -A_{21} n_2 + I_{\nu} B_{21} \Delta n$$

Si $\Delta n > 0$ on a pas que des perte et on peut avoir un laser. On appelle ça une inversion des population. 
\setcounter{subsection}{4}

\subsection{Équation de taux et inversion de population}


\begin{figure}[ht]
    \centering
    \incfig{bop}
    \caption{bop}
    \label{fig:bop}
\end{figure}


On prende $$g_1 = g_2 \implies B_{21} = B_{12} =B$$ 

$$pdv{n_{2}}{{t}} = A_{21} n_2 - I_{\mu} B \Delta n$$ 

\underline{Inversion de population:} $n_2 > n_1 \quad (\Delta n > 0)$  

On s'intéresse au nombre de photons stimulés


\end{document}
