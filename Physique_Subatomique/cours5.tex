\documentclass{article}    
\usepackage[utf8]{inputenc}    
    
\title{Épisode 4}    
\author{Jean-Baptiste Bertrand}    
\date{\today}    
    
\setlength{\parskip}{1em}    
    
\usepackage{physics}    
\usepackage{graphicx}    
\usepackage{svg}    
\usepackage[utf8]{inputenc}    
\usepackage[T1]{fontenc}    
\usepackage[french]{babel}    
\usepackage{fancyhdr}    
\usepackage[total={19cm, 22cm}]{geometry}    
\usepackage{enumerate}    
\usepackage{enumitem}    
\usepackage{stmaryrd}    
    
%packages pour faire des math    
%\usepackage{cancel} % hum... pas sur que je vais le garder mais rester que des fois c'est quand même sympatique...
\usepackage{amsmath, amsfonts, amsthm, amssymb}    
\usepackage{esint}  

\begin{document}

2022-09-15
\section*{Accél érateur et detecteur de particules}

\subsection*{Sources naturelles}

Avant les collisionuers,on utilisait des sources naturelles de paticules

\begin{itemize}
	\item rayons $\alpha,\beta,\gamma$ 
		\begin{itemize}
			\item noyaux instables
			\item \sim 100 - 200 keV. (Max de quelques MeV)
		\end{itemize}
	\item rayons cosmiques
		\begin{itemize}
			\item énergie jusqu'à $10^{19}$ eV mais incontrolables
			\item surtout des protons
			\item pas de concensus sur les origine
		\end{itemize}
\end{itemize}

\subsection*{Générateur de Cockroft-Walton}
	\begin{itemize}
		\item AC $\to$ DC
		\item Initiallement utilisé comme accélérateur 
		\item  Encore utilsié comme premier stage
		\item max 1 MeV
		\item Courrant dans les appareil à rayons x
	\end{itemize}

\subsection*{Générateur van de Graaf}

10-20 MeV avec pression de Gaz inerte

\subsection*{Accélérateur tandem}
 (Accélérer en attirant puis en repoussant)
30 à 40 MeV


\subsection*{Accélérateurs linéaires (LINAC)}

\begin{itemize}
	\item Utilisé dans tout les complexes comme injecteurs
	\item Peu de perte radiatives
	\item SLAC (Stanford) $\sim$ 50 GeV ( $e^-$  )  
\end{itemize}

\subsection*{Cavité accélératrice}

Cavité en Mode TM

\begin{figure}[ht]
    \centering
    \incfig{relations-de-dispertion}
    \caption{relations de dispertion}
    \label{fig:relations-de-dispertion}
\end{figure}

\section*{Pertes radiatives}
100 MeV par tours


\section*{Cyclotron}

Principe de l'indépendce de la fréquence vs Énérgie

Ne marche plus dans le domaine relativiste
$$\gamma \sim 1.5$$ 


\section*{Syncotrons}

Éléments discrets
\begin{itemize}
	\item aiment dipolaire
	\item aiment quadripolaire
	\item cavité EM.
\end{itemize}

Champ $B$ et fréquence sont ajustés. 


\section*{Focalisaton Magnétique}

On peut focuser un flux de particules avec des quatrupoles magnétiques alternants.

\section*{Collisionneurs}
Deux faisceaux de sens opposées.

Maximise l'énérgie disponible lors de la création de particules



\section*{Pertes par ionisation des particules chargées}

Formule de Bethe


\section*{Absorption des rayons gamma}
\begin{itemize}
	\item Compton
	\item Photoéléctrique
	\item Pair
\end{itemize}


\section*{Chambre à fils}


\section*{Scintillateur}

\section*{Détécteurs à été solide }

\section*{Laboratoires}

\hrule

\section*{Diffusion de Neutron de basse énérgie (1KeV)}

Le neutron est incident sur un noyeau de rayon de $5$fm 


\begin{figure}[ht]
    \centering
    \incfig{potenteil}
    \caption{potentiel}
    \label{fig:potenteil}
\end{figure}


$$\dv{\simga}{\Omega} \popto \abs{\tilde V(\vb{q})}^{2}$$ 

$$\boxed{\vb{q}^2 = 4 \vb{p}^2 \sin^{2}\frac{\theta}{2} }$$ 


$$\tilde V(\vb{q}) = \int \dd^{3}r v(\vb{r}) e^{-i \vb{q}\cdot \vb{r}} \propto a^{3}f(x)$$ 


$$x = \abs{\vb{q}}a \quad \text{sans unités} $$ 

le maximum de $q$ est de $2pa$ ? vraiment pas sur

$$2a \sqrt{2 m_n T_{n}} = 2 \cdot 5 \text{fm} \sqrt{2 \cdot 939 \text{MeV} * 0.001 \text{MeV} }/(197 \text{MeV}\cdot \text{Fm}  ) \approx 0.07 $$  

Puisque $a$ est petit, $V(\vb{r})$ est \textit{piqué} et donc $\tilde V$ est presque constant .   


\end{document}
