\documentclass{article}    
\usepackage[utf8]{inputenc}    
    
\title{Épisode 4}    
\author{Jean-Baptiste Bertrand}    
\date{\today}    
    
\setlength{\parskip}{1em}    
    
\usepackage{physics}    
\usepackage{graphicx}    
\usepackage{svg}    
\usepackage[utf8]{inputenc}    
\usepackage[T1]{fontenc}    
\usepackage[french]{babel}    
\usepackage{fancyhdr}    
\usepackage[total={19cm, 22cm}]{geometry}    
\usepackage{enumerate}    
\usepackage{enumitem}    
\usepackage{stmaryrd}    
\usepackage{mathtools,slashed}
%\usepackage{mathtools}
\usepackage{cancel}
    
\usepackage{pdfpages}
%packages pour faire des math    
%\usepackage{cancel} % hum... pas sur que je vais le garder mais rester que des fois c'est quand même sympatique...
\usepackage{amsmath, amsfonts, amsthm, amssymb}    
\usepackage{esint}  
\usepackage{dsfont}

\usepackage{import}
\usepackage{pdfpages}
\usepackage{transparent}
\usepackage{xcolor}
\usepackage{tcolorbox}

\usepackage{mathrsfs}
\usepackage{tensor}

\usepackage{tikz}
\usetikzlibrary{quantikz}
\usepackage{ upgreek }

\newcommand{\incfig}[2][1]{%
    \def\svgwidth{#1\columnwidth}
    \import{./figures/}{#2.pdf_tex}
}

\newcommand{\cols}[1]{
\begin{pmatrix}
	#1
\end{pmatrix}
}

\newcommand{\avg}[1]{\left\langle #1 \right\rangle}
\newcommand{\lambdabar}{{\mkern0.75mu\mathchar '26\mkern -9.75mu\lambda}}

\pdfsuppresswarningpagegroup=1

\begin{document}

2022-09-12
\section*{Section différentielle de diffusion}


\begin{figure}[ht]
    \centering
    \incfig{diffusion-par-un-potentiel-fixe}
    \caption{diffusion par un potentiel fixe}
    \label{fig:diffusion-par-un-potentiel-fixe}
\end{figure}



$$\dv{\sigma}{\Omega} = \frac{1}{\Phi} \dv{\Gamma}{\Omega} $$ 

$$\Gamma = 2\pi \frac{1}{\nu} \int \frac{\dd ^3 P_? }{(2\pi)^3} \abs{M_{fi} }^{2} \delta(E_2-E_1 ) $$ 

L'intégrale deviens, en coord sphérique: $$\frac{1}{\nu} \int_{}^{} \frac{p_2^{2}\dd p_2 \dd \Omega}{(2\pi)^3   } $$ 

Non relativiste: $E_2 = \frac{p_2^{2}}{2m} $ 
$$\dd E_2 = \frac{p_2}{m} \dd p_2$$ 


$$\dv{\Gamma}{\Omega} = \frac{1}{(2\pi)^3} \nu \int \abs{M_fi}^{2} p_2 m \dd E_3 \delta(E_{2-E_1)} = \frac{1}{(2\pi)^3} \nu \abs{M_fi}^{2}\abs{\vb{p}} m $$ 


$$\dv{\sigma}{\Omega} = \left( \frac{m}{2\pi}  \right)^2 \nu^{2}\abs{M_{fi} }^{2}$$ 

$$\text{Flux: }\rho \underbrace{v}_{\text{vitesse } = \frac{\abs{\vb{p}}}{m}   }   $$ 

$$M_{fi} = \bra{f} V \ket{i} = \bra{\vb{p}_2} V \ket{\vb{p}_1} = d^{3}r \bra{\vb{p}_2}\ket{\vb{r}} V(\vb{r})\bra{\vb{r}}\ket{\vb{p}_1}$$ 

$$= \frac{1}{\nu} \int \dd^{3}r e^{-i(\vb{p}_2 -\vb{p}_1 ) \cdot \vb{r}}$$ 
$$= \frac{1}{\nu} \tilde V (\underbrace{\vb{p}_2 - \vb{p}_1 }_{\vb{q} =\text{tansfert de $p$ } } )$$ 


$$\boxed{\dv{\sigma}{\Omega} = \left( \frac{m}{2\pi}  \right) ^2 \abs{\tilde V(\vb{q})}^{2}}$$ 

\begin{tcolorbox}[title=Exemple: Loi de Coulomb]
$$V(\vb{r}) = \frac{e_1 e_{2}}{4\pi r} $$ 

$$\nabla^{2}\phi = - \delta(\vb{r}) \qquad \phi(\vb{r})= \frac{1}{4\pi{}r} $$ 
$$- \vb{q}^2 \tilde \phi (\vb{q}) = -1\to \tilde\phi(\vb{q})= \frac{1}{\abs{q}^2}  $$ 

\hrule 
$$\vb{q}^2 = \dotsb 4 = \vb{p}^2\sin^{2}\frac{\theta}{2} $$ 

$$\dv{\simga}{\Omega} = \left( \frac{m e_1 e_2 }{ 8\pi p^{2} }  \right)^2 \text{cosec} ^4 \frac{\theta}{2} $$ 

$$\sigma = \int \dd \Omega \dv{\simga}{\Omega} \to \infty$$ 

\underline{distribution de charge} 

$$V(\vb{r}) = \frac{e_1 e_2 }{4\pi r} \to \frac{e_1 e_2 }{4 \pi r} \int \dd^{3} r' \frac{\rho (\vb{r}')}{\abs{\vb{r}-\vb{r}'} } $$ 
c'est une convolution!

$$\tilde V (\vb{q}) = \frac{1}{\abs{\vb{q}}^{2}} \tilde \rho(\vb{q})$$ 

On obtiens donc un simple facteur de correction
\end{tcolorbox}


\subsection*{Diffusion à plusieurs particules}

\begin{figure}[ht]
    \centering
    \incfig{diffusions-à-plusieurs-particules}
    \caption{diffusions à plusieurs particules}
    \label{fig:diffusions-à-plusieurs-particules}
\end{figure}


$$\dd \Gamma = \abs{\mathcal{M}_{fi}}^{2} \frac{\dd^{3}P}{(2\pi)^3} \left( 2\pi \right) ^4 \delta^4\left( p_{1+p_2} -p_3 -p_4 - \dotsb - p_N  \right) \qquad \text{[N.C.]}  $$ 

\begin{tcolorbox}[title=]
	$$\text{NC} \to \text{NR} \qquad \ket{\vb{p}}_{\text{NC}} = \frac{1}{\sqrt{2E}} \ket{\vb{p}}_{\text{NR}}    $$ 

	 
\end{tcolorbox}


$$\dd \sigma= \abs{\mathcal{M}_{fi}}^{2} \frac{E_1}{\abs{\vb{p}}} \frac{\dd^{3}p_{3}}{(2\pi)^3} \dotsb  $$ 

On veut trouver une quanité qui est egale à $\vb{p}_1$ dans le référentiel du laboratoire mais est aussi un invariant 


$$(\underbrace{\vb{p}_1}_{(E_1 , \vb{p}_1 )}  \underbrace{\vb{p}_2}_{(m_2 , \vb{0})} )^2 - (m_1 m_2 )^2 $$ 

$$E_1^{2}m_2^{2}- m_1^{2}m_2^{2}= (E_1^{2}-m_1^{2}) m_2^{2}= \vb{p}_1^2 m_1^{2}(m_2 , \vb{0})$$ 

$$\dd \simga = \abs{ \mathcal{M}_{fi} }^{2} \frac{1}{4\sqrt{(p_1 p_2 )^2 - (m_1 m_2 )^2 }} \frac{\dd ^3 p_2 }{2 E_3 (2\pi) ^3} \dotsb (2\pi)^4 \delta(p_1 + p_2 - p_3 - \dotsb - p_N )  $$ 

\subsection*{Résonances \& masse invariante}

Masse invariente de $N$ particules 



$$M^{2}= \underbrace{\left( p_1 + p_2 + \dotsb + p_N \right)}_{p_{\text{tot}} } ^2 = \left( E_1 + \dotsb + E_n  \right) - \left( \vb{p}_1 + \vb{p}_2 + \dotsb + \vb{p}_N \right)^2$$ 

$$\rho(E) = \frac{1}{2\pi} \frac{\Gamma}{(E-M)^2 + \left( \frac{\Gamma}{2}  \right) ^2} $$ 



\begin{figure}[ht]
    \centering
    \incfig{désintégration-2}
    \caption{Désintégration 2}
    \label{fig:désintégration-2}
\end{figure}


\clearpage

\section*{Chaîne de masse \mu}



\begin{figure}[ht]
    \centering
    \incfig{chaîne-de-masse}
    \caption{Chaîne de masse}
    \label{fig:chaîne-de-masse}
\end{figure}


$$\mathcal{L}  = \frac{1}{2} \mu \sum_{r=1}^{N} \left\{ \dot u_r^{2} -\Omega^{2} u_r^{2} -\Gamma ^2 \left( u_r -u_{r+1}  \right) ^2 \right\} $$ 

$$\dv{{}}{t} \pdv{\mathcal{L}}{\dot u_{r}} - \pdv{ \mathcal{L}  }{u_{r}} = 0  $$ 

On tourne la manivelle: 

$$\omega_q = \sqrt{\Omega^{2}+ 2\Gamma^{2}(1-\cos q)}$$ 

\begin{figure}[ht]
    \centering
    \incfig{relation-de-dispersion}
    \caption{relation de dispersion}
    \label{fig:relation-de-dispersion}
\end{figure}

\end{document}
