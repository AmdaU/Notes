\documentclass{article}    
\usepackage[utf8]{inputenc}    
    
\title{Épisode 4}    
\author{Jean-Baptiste Bertrand}    
\date{\today}    
    
\setlength{\parskip}{1em}    
    
\usepackage{physics}    
\usepackage{graphicx}    
\usepackage{svg}    
\usepackage[utf8]{inputenc}    
\usepackage[T1]{fontenc}    
\usepackage[french]{babel}    
\usepackage{fancyhdr}    
\usepackage[total={19cm, 22cm}]{geometry}    
\usepackage{enumerate}    
\usepackage{enumitem}    
\usepackage{stmaryrd}    
    
%packages pour faire des math    
%\usepackage{cancel} % hum... pas sur que je vais le garder mais rester que des fois c'est quand même sympatique...
\usepackage{amsmath, amsfonts, amsthm, amssymb}    
\usepackage{esint}  

\begin{document}

2022-09-26

\section*{QFT: champ scalaires avec interaction}



$$L = \frac{1}{2} <mu \sum_{r=1}^{N} \left\{ \dot u_r^{2} - \Omega^{2}u_r^{2}-\Gamma^{2}\left( u_r - u_{r+1}  \right) ^2 - \lambda u_t^3 \right\} $$ 

C'est un développement à l'ordre 3 et non un résultat exacte


$$H = H_0 + \underbrace{H_1}_{perturbation} $$ 

$$H_1  = \frac{g}{6} \int \dd x \phi^{3}(x) \qquad \frac{g}{6} = \lambda \sqrt{\frac{a}{\mu} }$$ 


$$\boxed{\text{3D} \to H_1 = \frac{g}{6} \int \dd^3 r \phi^3(\vb{r})}$$ 


$$\phi(\vb{r}) = \frac{1}{\sqrt{\nu}} \sum_{\vb{p}} \frac{1}{\sqrt{2\omega p}} \left( a_p e^{i \vb{p} \cdot \vb{r}} + a_p ^{\dagger} e^{-i \vb{p}\cdot \vb{p}} \right) $$ 


$$H_1 = \frac{g}{6} \frac{1}{\nu^{3}{2}} \int \dd^{3}r \sum_{p,p',q} } \frac{1}{\sqrt{8\omega_p \omega_{p'} \omega_{q}}  } \left( a_p e^{i\vb{p}\vb{r}} + a_p^{\dagger} e^{-i\vb{p}\vb{r}} \right) \left( a_{p'} e^{i\vb{p}'\vb{r}} + a_{p'}^{\dagger}e^{-i\vb{p}'\vb{r}
}  \right) \left( a_q e^{i\vb{q}\vb{r}} + a_q^{\dagger} e^{-i\vb{q}\vb{r}} \right) $$ 
On intègre sur $\vb{r}$ 

$$H_1  = \frac{g}{6} = \frac{1}{2\sqrt{2\nu} } \sum_{p,q} \left\{ \frac{1}{\sqrt{\omega_p \omega_{p-q}  \omega_q}   } \left( a_p a_{p-q} a_q  + a_q ^{\dagger} a_{p-q} ^{\dagger} a_p^{\dagger}\right) + \dotsb\right\} $$ 

Tout les termes qui sont une suite d'opérateur de création, créent au total un quantité de mouvement nulle.

Dans les différents termes, on fait des changmenents de varaibles du type $q \to q+p$ 


$$H_1  = \frac{g}{6} \frac{1}{2\sqrt{2\nu}} \sum_{p,q} \frac{1}{\sqrt{\omega_{p\omega_{q}\omega_{p+q} } } } \left[ a_p a_{-p-q} a_q +a^{\dagger}_p+ a_q a_p a_q + a_p a^{\dagger}_{p+q} a_q + \dotsb \right] $$ 


Cette perturbation représente l'interaction entre différente excitation du champ. Des genres de \textit{collisions}. On considère, puisqu'on fait de la théorie des perturbation, que ces collision sont assez peu fréquente et contribuent peu à l'énérgie totale.

$$\Gamma_{i\to f} = 2\pi \abs{M_{fi} }^{2} \delta(E_f -E_i ) $$ 


$$M_{fi} = \mel{f}{H_{1}}{i} + \dotsb$$ 

$$\ket{i} = a_{p_{1}}^{\dagger} a_{p_2} ^{\dagger} \ket{0}= a_1^{\dagger} a_2 ^{\dagger} \ket{0}$$ 
$$\ket{f}=  a^{\dagger}_{p_{3}} a_{p_4}^{\dagger} \ket{0} = a_3 ^{\dagger} a_4^{\dagger} \ket{0}$$ 

$\mel{f}{H_{1}}{i} =0$ car les états qui n'ont pas le même nombre de particules sont orthogonaux. On doit donc aller au second ordre de perturbation. \textit{L'état intermédiaire} $\ket{n}$ permet de créer des particules de manière seulement \textit{temporaire}.

$$M_{fi} = \sum_n \frac{\mel{f}{H_1}{n}\mel{n}{H_1}{i}}{E_1 -E_n} $$ 

\begin{figure}[ht]
    \centering
    \incfig{diagramme-pas-de-feynmann}
    \caption{diagramme pas de Feynmann}
    \label{fig:diagramme-pas-de-feynmann}
\end{figure}


$$(A)\begin{cases}
	\begin{align*}
		\mel{n}{H_{1}}{i} =& \bra{0} a_3 a_2 a_q (a^{\dagger}_q a_1 a_3 ^{\dagger}) a_1^{\dagger} a_2^{\dagger} \ket{0}\\
											& \bra{0} a_p a_p^{\dagger} \ket{0} =1\\ 
											& \bra{0} [a_p, a_p^{\dagger}] + a_p^{\dagger}a_p \ket{0}\\
											&\cdot g \frac{1}{\sqrt{8 \nu \omega_1 \omega_2 \omega_3} }
	\end{align*}\\
	\begin{align*}
		\mel{f}{H_{1}}{n} = &\bra{0}a_3 a_4 (a_4^{\dagger}a_2 a_{q)} a_3^{\dagger} a_2^{\dagger} a_q^{\dagger} \ket{0} \\
												&\cdot g \frac{1}{\sqrt{8\nu \omega_2 \omega_q \omega_{1-3} } }
	\end{align*}\\
	M^{(A)} = \frac{g^2}{8\nu \sqrt{\omega_{1\omega_2\omega_3\omega_4}}} \frac{1}{\omega_{1-3} } \frac{1}{\omega_1 -\omega_3 -\omega_{1-3} }  
\end{cases}$$ 


Les autres diagramme nous mène presqu'exactement à la même équation ex:


	M^{(A)} = \frac{g^2}{8\nu \sqrt{\omega_{1\omega_2\omega_3\omega_4}}} \frac{1}{\omega_{1-3} } \frac{1}{\omega_1 -\omega_3 -\omega_{1-3} }  

\end{document}
