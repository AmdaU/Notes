\documentclass{article}    
\usepackage[utf8]{inputenc}    
    
\title{Épisode 4}    
\author{Jean-Baptiste Bertrand}    
\date{\today}    
    
\setlength{\parskip}{1em}    
    
\usepackage{physics}    
\usepackage{graphicx}    
\usepackage{svg}    
\usepackage[utf8]{inputenc}    
\usepackage[T1]{fontenc}    
\usepackage[french]{babel}    
\usepackage{fancyhdr}    
\usepackage[total={19cm, 22cm}]{geometry}    
\usepackage{enumerate}    
\usepackage{enumitem}    
\usepackage{stmaryrd}    
\usepackage{mathtools,slashed}
%\usepackage{mathtools}
\usepackage{cancel}
    
\usepackage{pdfpages}
%packages pour faire des math    
%\usepackage{cancel} % hum... pas sur que je vais le garder mais rester que des fois c'est quand même sympatique...
\usepackage{amsmath, amsfonts, amsthm, amssymb}    
\usepackage{esint}  
\usepackage{dsfont}

\usepackage{import}
\usepackage{pdfpages}
\usepackage{transparent}
\usepackage{xcolor}
\usepackage{tcolorbox}

\usepackage{mathrsfs}
\usepackage{tensor}

\usepackage{tikz}
\usetikzlibrary{quantikz}
\usepackage{ upgreek }

\newcommand{\incfig}[2][1]{%
    \def\svgwidth{#1\columnwidth}
    \import{./figures/}{#2.pdf_tex}
}

\newcommand{\cols}[1]{
\begin{pmatrix}
	#1
\end{pmatrix}
}

\newcommand{\avg}[1]{\left\langle #1 \right\rangle}
\newcommand{\lambdabar}{{\mkern0.75mu\mathchar '26\mkern -9.75mu\lambda}}

\pdfsuppresswarningpagegroup=1


\begin{document}
2022-32-14

\section*{QCD (Quantum Chromodynamics)}


\subsection*{Couleur de quarks}

On justifie la notion de \textit{couleur} par trois observation

\begin{enumerate}
	\item $\Delta^{++}$: Des baryons produit lorsqu'on envoie un faisceau de proton sur des cibles. Cette particule se désintègre extremement rapidement. Tellement qu'on ne peut pas l'obsever directement. On l'observe seulement par \textit{résonance} dans le processus de scattering. Cette particule n'as pas l'air élémentaire (et ne l'est pas) à cause de sa charge de $2e$. Sont spin et de $J= \frac{3}{2} $ ce qui est étrange sans le modèle des quarks. Cette particules est en quelque sorte l'état fondamentale de trois quarks.    

Généralement, la fonction d'onde de l'état fondamentale d'un système est symétrique car l'antisymétrie et lié à des variation plus rapide et donc plus d'énérgie. La fonction d'onde de $\Delta^{++}$ est donc symétrique.  

La particules est donc symétrique en spin ( $\ket{\frac{3}{2} } = \ket{\uparrow\uparrow\uparrow}$  ) \textbf{et} en position. Pourtant, étant un fermion, cette particule devrait avoir une fonction d'onde antisymétrique. On déduit de cela qu'il doit y avoir un autre degré de liberté interne. Ce degré de liberté additionnel est la couleur. On peut montrer que cette antisymétrie requirt une degré de liberté à \textit{trois dimentions}

Champ de dirac:

\[ \psi_{\alpha,i}  \] 

$\alpha$ est ici l'indice de Dirac et $i \in \{ 1,2,3 \} $ est l'indice de couleur  

\[ \mathscr{L} = \sum_{i=1}^{3} \left( i \bar \psi_i \partial_{\mu} \gamma^{\mu}\psi_i - m \bar \psi_i \psi_i \right) \qq{pour 1 type de quark} \] 

au lieu d'utiliser les indice $\{ 1,2,3 \} $, on préfère $\{ 	R, G, B \} $ 

\[ \frac{1}{\sqrt{6}} \left\Big\{ \ket{RGB} + \ket{GBR} + \ket{BRG} - \ket{GRB} - \ket{RBG} - \ket{BGR} \right\}  \] 

\[ = \frac{1}{\sqrt{6}} \epsilon_{ijk} \ket{ijk} \] 


\item \[ e^{+}e^{-} \to \text{hadron}  \] 
	\[ e^{+} e^{-} \to \mu^{+}\mu^{-} \] 


  \[ R = \frac{\sigma_{e^{+}e^{-}}\to\text{hadrons} }{\sigma_{e^{+}e^{-}\to\mu^{+}\mu^{-}}} = \sum_a Q_a^{2} \] 


\begin{figure}[ht]
    \centering
    \incfig{graphique}
    \caption{Argument 2, on semble avoir 3 fois plus de type de particules que prévu}
    \label{fig:graphique}
\end{figure}


\item neutralité éléctrique des familles

On demande la neutralité éléctrique des différents types de famille, entre autre dans le but de construire la mer de Dirac.



\subsubsection*{Rotation de couleur}


\[ \psi_{\to} \psi_{i}' = \sum_{j=1}^{3}U_{ij} \psi_j  \] 

\[ U^{-1} =U ^{\dagger} \] 

\[ \mathscr{L} = \sum_{i=1}^{3} \left( i \bar \psi_i \partial_{\mu} \gamma^{\mu}\psi_i - m \bar\psi_i \psi_i \right)  \] 


\[ \ket{\psi} = \frac{1}{\sqrt{6}} \epsilon_{ijk} \ket{ijk} \] 

\[ \ket{\psi'} = \frac{1}{\sqrt{6}} \underbrace{\epsilon_{ijk} u_{il} u_{jm} u_{kn} }_{\epsilon_{lmn} \det U} \ket{lmn}   \] 

\[ =\det U \frac{1}{\sqrt{6}} \epsilon_{lmn} \ket{lnm} \] 

\[ =\underbrace{\det U}_{1} \ket{\psi}  \] 



\[ \implies U \in \text{SU}(3)  \] 

\begin{tcolorbox}[title=Parenthèse sur la théorie des groupe]
    groups définis par des matrices 

    $a,b,c \in G$  
    $a,b \in G \implies ab \in G$ 
    \[ \exists e \in G | ae =ea =a \] 
    \[ (ab)c = a(bc) \] 
    \[ \forall a \in g \exists a^{-1} | a a^{-1} = e \] 

    \rule 

    $\rm U(n)$: groupe des matrices unitaires de dimension $n$  

    $\text{SU}(n) $: U($n$) et $\det \in G =  1$ 

    $\rm O(n)$: matrice orthogonales 

    ${\rm SO}(n)$ : speciale orthogonales  
\end{tcolorbox}

État quark-antiquark ($q\bar q$)

\[ c_k \to U c_k \]  
\[  d_k^{\dagger} \to U d_k^{\dagger} \] 
\[  c_k^{\dagger} \to U ^{*} c_k^{\dagger} \] 
\[  d_k \to U ^{*} d_k \] 

\[ \ket{\psi}_{\text{meson}} = \frac{1}{\sqrt{3}} \left\Big( \ket{R\bar R} + \ket{G \bar G} + \ket{B\bar B} \right)  = \frac{1}{\sqrt{3}} \sum_{i=1}^{3} \ket{i\bar i} \] 

\[ \ket{\psi'} \to \frac{1}{\sqrt{3}} \sum_{i=1}^{3} U_{ij} ^{*} U_{iR} \ket{j\bar R} = \frac{1}{\sqrt{3}} \sum_{jk} \delta_{jk} \ket{j \bar k} = \frac{1}{\sqrt{3}} \sum_j \ket{j \bar j} = \ket{\psi} \] 



\subsection*{Théorie de Yang-Mills (1954)}
\subsection*{Gross, Wilczek, Politzer (1973)}

C'est une théorie de l'intéraction forte qui était très remise en question avant l'idée des quarks. 



\[ \psi_i \to \psi_{i}' = \underbrace{U_{ij}}_{\in \rm{SU}(3)}  (x) \psi_j  \] 


\[ \mathscr{L} = \sum_{j=1}^{3} \bar\psi_i \left( i\gamma^{\mu}\partial_{\mu} -m \right) \psi_i \] 

On construit alors la dérivé covariante 


\[  \mathscr{D}_\mu \psi = \left( \partial_{\mu} + i g A_\mu \right) \psi \] 

En éléctromagnétisme: 

\[ \psi' = \psi e^{ie \xi(r) } \psi \] 
\[  \left( \mathcal{D}_\mu \psi \right)' = e^{i\xi } \mathcal{D}_\mu \psi  \] 


En chromodynamique 

\[ \psi' = U\psi \] 
\[  \left( \mathcal{D}_\mu \psi \right)' = U \mathcal{D}_\mu \psi \] 

\[  \mathcal{D}_\mu ' \psi' =  \left( \partial_{\mu} _ i g A_{\mu} ' \right)U \psi = U \left( \partial_{\mu} _ ig A_{\mu}  \right) \psi \] 

\[  = \partial_{\mu} U \psi + \cancel{U \partial_{\mu} \spi} + ig A'_\mu U \psi = \cancel{U \partial_{\mu} \psi} + i gU A_{\mu} \psi \forall \psi \] 

\[  \implies \partial_{\mu} U + i g_{\mu} ' = i gU A_\mu \] 
\[  \implies \partial_{\mu} U U ^{\dagger} _ ig A'\mu = ig U A_{\mu} U^{\dagger}  \] 

\[ \boxed{A'_\mu = \frac{i}{g} \partial_{\mu} U U^{\dagger} + u A_{\mu} U^{\dagger} } \] 



\[ \mathscr{L} = \bar\psi \left( i\slashed{\mathcal{D}} - m \right) \psi = \bar \psi \left( i \slashed \partial - m - g \slashed A  \right) \psi \] 


\begin{tcolorbox}[title=lien entre matrice unitaire et hermitienne]
    \[ U = e^{iH}\qquad U^{-1}=  e^{-iH} = U ^{\dagger} \]  

    \[  \det U = 1 \to \tr H =0  \] 

    \[\rm{SU}(2) \to e^{\frac{1}{2} i \omega_a \sigma_{a}} \qquad \rm{SU}(3) \to e^{i\omega_a T_a} \] 
    Il y a 8 matrices $T_a$, analogues aux matrices de Pauli, ce sont les matrices de Gell-mann 
    \[ T_a = \frac{1}{2} \lambda a \] 
\end{tcolorbox}


\[\lambda_1 = \mqty(0 & 1 & 0 \\1 & 0 &0 \\ 0 & 0 &0) \quad \lambda_2 = \mqty(0 & -i & 0 \\i & 0 &0 \\ 0 & 0 &0) \quad \lambda_3 = \mqty(1 & 0 & 0 \\0 & -1 &0 \\ 0 & 0 &0)  \] 
\[\lambda_1 = \mqty(0 & 1 & 0 \\1 & 0 &0 \\ 0 & 0 &0) \quad \lambda_2 = \mqty(0 & -i & 0 \\i & 0 &0 \\ 0 & 0 &0) \quad \lambda_3 = \mqty(1 & 0 & 0 \\0 & -1 &0 \\ 0 & 0 &0)  \] 










\end{enumerate}
\end{document}
