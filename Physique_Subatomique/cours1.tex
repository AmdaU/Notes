\documentclass{article}    
\usepackage[utf8]{inputenc}    
    
\title{Épisode 4}    
\author{Jean-Baptiste Bertrand}    
\date{\today}    
    
\setlength{\parskip}{1em}    
    
\usepackage{physics}    
\usepackage{graphicx}    
\usepackage{svg}    
\usepackage[utf8]{inputenc}    
\usepackage[T1]{fontenc}    
\usepackage[french]{babel}    
\usepackage{fancyhdr}    
\usepackage[total={19cm, 22cm}]{geometry}    
\usepackage{enumerate}    
\usepackage{enumitem}    
\usepackage{stmaryrd}    
\usepackage{mathtools,slashed}
%\usepackage{mathtools}
\usepackage{cancel}
    
\usepackage{pdfpages}
%packages pour faire des math    
%\usepackage{cancel} % hum... pas sur que je vais le garder mais rester que des fois c'est quand même sympatique...
\usepackage{amsmath, amsfonts, amsthm, amssymb}    
\usepackage{esint}  
\usepackage{dsfont}

\usepackage{import}
\usepackage{pdfpages}
\usepackage{transparent}
\usepackage{xcolor}
\usepackage{tcolorbox}

\usepackage{mathrsfs}
\usepackage{tensor}

\usepackage{tikz}
\usetikzlibrary{quantikz}
\usepackage{ upgreek }

\newcommand{\incfig}[2][1]{%
    \def\svgwidth{#1\columnwidth}
    \import{./figures/}{#2.pdf_tex}
}

\newcommand{\cols}[1]{
\begin{pmatrix}
	#1
\end{pmatrix}
}

\newcommand{\avg}[1]{\left\langle #1 \right\rangle}
\newcommand{\lambdabar}{{\mkern0.75mu\mathchar '26\mkern -9.75mu\lambda}}

\pdfsuppresswarningpagegroup=1

\begin{document}

{{ \Huge Physique subatomique}}



\section*{Théorie quantique des champs}

La QFT est une théorie très réductionniste. L'idée est de comprendre le monde à partir de l'échelle la plus petite possible.

\begin{tcolorbox}[title=Remarque (Historique/Paradigme)]
	Démocrite à la première théorie "réductionniste". Il pense que les atomes  sont agencé de manière aléatoire (non divine). Il pense que tout les atomes sont différents. Un théorie qui s'y oppose est la théorie des élément qui viennent seulement en 4 types mais ou tout est continue. 
	 

Ces deux théorie on été "combinée" par Dalton qui parlait d'atome d'un nombre de type fini.

En théorie quantique des champs est très continue. Des champs émane les particules et non le contraire.

\end{tcolorbox}

Il existe une correspondance en QFT entre les types de particules (ex. éléctrons). et les champs. Il y a deux grandes catégories de champs (particlues donc): les fermions (qui ont un spin demi-entier) et les bosons (qui ont un spin entier). Les fermions sont beaucoup moins "classiques" que les bosons. 


\begin{figure}[h!]
    \centering
    \incfig{tableau-periodique}
    \caption{fermions}
    \label{fig:tableau-periodique}
\end{figure}

\begin{figure}[h!]
    \centering
    \incfig{bosons}
    \caption{bosons}
    \label{fig:bosons}
\end{figure}


Il est impossible d'isoler un quark seul. On ne peut qu'observer des combinaisons de quarks.


Tout les fermions sont décris par l'équation de Dirac. Au contraire, les bosons sont décris par des théories de Gauge. Bien que ces transformation de Gauge soient présenté comme relativement peut importante dans le cadre de l'éléctromagnétiste, c'est le fondement de la QFT.

Toutes les particules en QFT on une antiparticule qui leur est associé, bien que les bosons soient pour la plupart leur propre anti-particule (sauf $W^{+}\leftrightarrow W^{-}$)


Les champ sont toujours dans leur état fondamentale, sauf lorsqu'il a des particules. L'exception à cette règle est le champ de Higgs qui a une valeur constante non-nulle.



\subsection*{Les masses}


\begin{tcolorbox}[title=Remarque (unitées)]
    On n'utilise pas le système SI dans le cadre de la QFT. Les masses sont plutôt exprimées en MeV. On utilise également souvent les unités naturelles ( $c =1\; \hbar =1$  ).
Le fait que $c=1 \implies$ on ne fait pas de différente entre longeur et temps. $\hbar =1 \implies E = \omega$. Comme tout peut finalement s'exprimer en énergie on prend une on peut prendre une unité d'énérgie : le MeV.
     
\end{tcolorbox}


$(q,q,q) \leftrightarrow \text{baryons (sont des fermions)} $ 

$$\text{proton} \rightarrow \text{uud} \sim 238 \text{MeV} \sim 1 \text{GeV}    $$ 
$$\text{neutron} \rightarrow udd \sim 237 MeV$$ 

$\text{mesons} \rightarrow q\bar q \text{ (sont des bosons) }  $ 
$$\text{pions}\quad \pi^{0}: u\bar u \quad \pi^{+}: u\bar d \quad \pi^{-}:\bar u d $$ 

Les muons: on deux cents fois la masses de l'éléctron. 

\begin{figure}[ht]
    \centering
    \incfig{pluie}
    \caption{pluie}
    \label{fig:pluie}
\end{figure}

\section*{Rappels sur la relativité}

quadrivecteur : $$x = (t,x,y,z)$$ 
$$x^{\mu}= [t,x,y,z]$$ 
$$p^{\mu}= [E,p]$$ 
$$k^{\mu}=[\omega, k] = \frac{1}{\hbar} p^\mu$$ 
$$j^{\mu}= [p, \vec j] $$ 
$$\partial_{\mu} = [\pdv{t}, \grad] $$  

$$A'^\mu = \Lambda^\mu_{\nu} A^{\nu} \text{Contravarient} $$ 
$$A'_\mu = (\Lambda^{-1})^\nu_{\mu} A_{\nu} \text{Covarient} $$ 

$$A_{mu} = g_{\nu\mu} A^{\nu} \quad\quad A^{\mu} =g^{\nu\mu}A_\nu$$ 


Où $g$ est le tenseur métrique.

\underline{Quadrivecteur}
$$\partial_{mu} j^{\mu}(x) = \partial'_\mu j'^\mu \text{scalaire (donc invarient)}  $$ 



\end{document}
