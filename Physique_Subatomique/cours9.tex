\documentclass{article}    
\usepackage[utf8]{inputenc}    
    
\title{Épisode 4}    
\author{Jean-Baptiste Bertrand}    
\date{\today}    
    
\setlength{\parskip}{1em}    
    
\usepackage{physics}    
\usepackage{graphicx}    
\usepackage{svg}    
\usepackage[utf8]{inputenc}    
\usepackage[T1]{fontenc}    
\usepackage[french]{babel}    
\usepackage{fancyhdr}    
\usepackage[total={19cm, 22cm}]{geometry}    
\usepackage{enumerate}    
\usepackage{enumitem}    
\usepackage{stmaryrd}    
    
%packages pour faire des math    
%\usepackage{cancel} % hum... pas sur que je vais le garder mais rester que des fois c'est quand même sympatique...
\usepackage{amsmath, amsfonts, amsthm, amssymb}    
\usepackage{esint}  

\begin{document}

2022-09-28


\section*{Section efficasse avec des diagrammes de Feynman}


$$\mathcal{M} = g^{2}\left\{ \frac{1}{\left( p_1 - p_3  \right) ^2 -m_2} + \frac{1}{\left( p_1 - p_4 \right)^2 - m^2 } + \frac{1}{\left( p_1 + p_2 \right)^2 -m^2 }   \right\} $$ 


Puisque les particules en question sont indiscernables on rajoute un facteur $\frac{1}{2} $ pour enlever les états comptés en trop


$$\dv{\sigma}{\Omega} = \frac{1}{2} \frac{{\abs{\mathcal{M} }}^{2}}{\left( 8\pi \right)^2} \frac{\abs{\vb{p}_3} }{\abs{\vb{p}_1} } \frac{1}{E^2}  $$ 


\begin{figure}[ht]
    \centering
    \incfig{collsion}
    \caption{collsion}
    \label{fig:collsion}
\end{figure}


$$\vb{p}_1 = p \hat x$$ 
$$\vb{p}_2 = -p \hat x$$ 
$$\vb{p}_3 = p \hat n$$ 
$$\vb{p}_4 = -p \hat n $$ 


$$\dotsb$$ 
$$\left( p_1 + p_4 \right)^2 = (2E)^2 = 4 E^2$$ 


$$\dv{\sigma}{\Omega} = \frac{1}{128\pi^{2}} \frac{1}{\gamma^2}  \left( \frac{1}{4(\gamma^{2}-1 ) \sin^{2}\frac{\theta}{2} +1}  + \frac{1}{4(\gamma^{2}-1 ) \cos^{2}\frac{\theta}{2} +1} + \frac{1}{4\gamma^{2}-1}  \right) $$ 
limite non relativiste ($\gamma \to 1$):

$$\dv{\sigma}{\Omega} \to \frac{g^{4}}{12 8 \pi^{2}m^4} \left( \frac{5}{3}  \right) ^2 $$ 


Limite ultra relativiste ( $\gamma \gg 1$  ):

$$\dv{\sigma}{\Omega} \to \frac{g^4}{128\pi^{2}m^2} \frac{1}{\gamma?} \left( \frac{1}{\sin^2\theta} - \frac{1}{4}  \right)^2  $$ 

On remarque que la probabilité de collision dans la limite ultra relativiste est beaucoup plus faible que dans la limite classique.


\section*{Feynman rules!}

\begin{enumerate}
	\item $i \mathcal{M} $ 
	\item identifier les particles entrantes et sortantes
	\item construire les diagrammes $\to N$ vertex (ordre $N$ en théorie des perturbation) 
	\item chaque ligne $\to$ 4-impulsion 
	\item vertex $\to$ $-ig \left( 2\pi \right) ^4 \delta(k_1 + k_2 - k_3)$  
	\item ligne interne $\to \ \frac{-i}{q^2-m^{2}} $ 
	\item intégrer sur les 4-impulsion internes $\int \frac{\dd^{4}q  }{\left( 2\pi \right) ^4} $ 
	\item Amputer le facteur global $(2\pi)^4 \delta(p_1 +p_2 \dotsb -p_n)$ 
\end{enumerate}

\begin{figure}[ht]
    \centering
    \incfig{diagramme}
    \caption{Diagramme}
    \label{fig:diagramme}
\end{figure}


\begin{figure}[ht]
    \centering
    \incfig{diagrammes}
    \caption{diagrammes}
    \label{fig:diagrammes}
\end{figure}
\clearpage

\section*{Potentiel de Yukawa}

Le potentiel de Yukawa donne le potentiel généré par des particules virtuelles. Il décroit exponentiellement en fonction de la masse. Cela explique la porté limités des forces qui utilise des bosons massifs. La force électromagnétique a une portée infinie car la photon est sans masse.

$$U(r) = - \frac{g}{4\pi} \frac{e^{-m r}}{r} $$ 

$$\nabla^{2}\Phi \cancel{-\partial_?^{?}\Phi} = -e \delta (r) \qq{Potentiel retardé}$$ 

$$\nabla^{2}\Phi = -e \delta(\vb{r}) \to \Phi (\vb{r}) = \frac{e}{4\pi{}r} $$ 

\underline{spineur} 

$$\Psi =\mqty(\psi_\uparrow\\\psi_\downarrow)$$ 

rotation:
$$\vb{r} \to \vb{r}'= \mathscr{R}(\vb{r}, \theta)\vb{r}$$ 

$$\psi (\vb{r}) \to R(\hat n, \theta)\psi(\vb{r})$$ 

$$R(\hat n, \theta ) = \exp(i \frac{\theta}{2} \hat n \cdot  \vb{\sigma}) = \cos \frac{\theta}{2} + i \hat n \cdot \vb{\sigma}\sin \frac{\theta}{2} \qquad \vb{\sigma} = \mqty(\sigma_x & \sigma_y & \sigma_z)$$ 

$$\psi^{\dagger}x \to \text{scalaire} $$ 
$$\psi^{\dagger} \vb{\sigma} x \to \text{vecteur} $$ 


\end{document}
