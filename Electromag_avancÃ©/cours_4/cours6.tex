\documentclass{article}
\usepackage[utf8]{inputenc}

\title{Cours 6}
\author{Jean-Baptiste Bertrand}
\date{\today}

%\setlength{\parindent}{4em}
\setlength{\parskip}{1em}


\usepackage[utf8]{inputenc}
\usepackage[T1]{fontenc}
\usepackage[french]{babel}
\usepackage{fancyhdr}
\usepackage[total={19cm, 22cm}]{geometry}
\usepackage{enumerate}
\usepackage{enumitem}
\usepackage{svg}
\usepackage{physics}
\usepackage{mathrsfs}  
\usepackage{tcolorbox}

%packages pour faire des math
%\usepackage{cancel} % hum... pas sur que je vais le garder mais rester que des fois c'est quand même sympatique...
\usepackage{amsmath, amsfonts, amsthm, amssymb}
\usepackage{esint}

\begin{document}

\maketitle

Pour passer de $\vec j \to \vec B$ on utilise la loi de Biot-Savard

Pour faire le contraire on utilise l'équation locale $\grad \cross \vb B = \mu_0\vb j$
où sa forme intégrale le théorème d'Ampère

On a aussi les contraintes $$\grad \cdot B = 0$$ et $$\vb B = \grad \cross \vb A$$

On doit aussi raisoner par linéartité, c'est à dire que le champ causé à un point sera la somme de tout les champs causé en ce point par chaque élément de courrant.


$$\phi = \iint_2 \vb B \dd \vb S = \iint_2 \qty(\vb B_1 + \vb B_2) \cdot \dd \vb S = \phi_{1\to2} + \phi_{2\to2} = M_{21} I_1 + M_22 I_2$$

$$M_{21} = M_{12}$$

\section{Induction}

On fait plus de l'éléctrostatique Whoooop.

$$\boxed{\grad \cross \vb E = -\dv{\vb B}(t) \quad \text{Loi de Faraday}}$$

$$\oint \vb E \cdot \dd \vb l = \iint  \qty(\grad \cross \vb E) \dd \vb S = \pdv{t}\iint \vb B \cdot \dd \vb S = -\pdv{\phi}{t}$$





\end{document}