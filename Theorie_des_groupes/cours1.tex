\documentclass{article}    
\usepackage[utf8]{inputenc}    
    
\title{Épisode 4}    
\author{Jean-Baptiste Bertrand}    
\date{\today}    
    
\setlength{\parskip}{1em}    
    
\usepackage{physics}    
\usepackage{graphicx}    
\usepackage{svg}    
\usepackage[utf8]{inputenc}    
\usepackage[T1]{fontenc}    
\usepackage[french]{babel}    
\usepackage{fancyhdr}    
\usepackage[total={19cm, 22cm}]{geometry}    
\usepackage{enumerate}    
\usepackage{enumitem}    
\usepackage{stmaryrd}    
    
%packages pour faire des math    
%\usepackage{cancel} % hum... pas sur que je vais le garder mais rester que des fois c'est quand même sympatique...
\usepackage{amsmath, amsfonts, amsthm, amssymb}    
\usepackage{esint}  

\begin{document}

\section*{Introduction à la théorie des groupes et applications à la supracondicivité}


\section*{Concept Généraux}

\underline{Définition: Groupe}

Un \bf{groupe} est une ensemble $\left\{ a,b,c \right\}$ muni d'une { \it multiplication} respectant les règles suivantes

\begin{itemize}
	\item Si $a, b \in G$, alors $a*b \in G$\\
	\item $(a*b)*c = a*(b*c)$
\end{itemize}

Un ensemble de matrice peut définir un groupe, car la multiplication matriciel repecte les bonne propriétés. Les matrices doivent êtres non-nuls pour respecter la condition d'existence d'un inverse.
\[
	GL(N) \quad \det(A) \neq 0 \quad \text{corp: }\mathbb{R}
\]
est le groupe le plus grand qui existe


\subsection*{Groupes ponctuelles communs}

Il y a deux notation principales pour différentier les groupes
Schönfhes et l'autre

il y a $C_n, D_n, C_{nv}, C_{nh}, D_{nh} ... $

$a$ et $b$ sont conjugés $(a \sim b)$ s'il existe $c\in G | b  =c^{-1}ac$
$a$ et $b$ décrivent alors le {\it même genre} de transformation car relié par un changement de base

La relation de conjugasion est une relation d'équivalence:

(reflexive: $a\sim a$, symétrique $a\sim b \to b\sim a$, transitive: $a\sim b & b\sim c \implies a\sim c$)

\section*{Représentation}

$$\mathcal{R}: G \to GL(d) \qquad a\to R(a) \qquad R(ab) = R(a)R(b)$$
L'espace vectoriel sur lequel agit $R(a)$: module V de la représentation

Deux représentation sont équivalent si elle ne diffèrent que par un changement de base

Représentation unitaire \[
	\implies R^{-1}(a) = R^{\dagger}(a) \forall a \in G

\]
Toute représentation d'un groupe fini est équivalente à une  représentation unitaire

Représentation réductible: $\exists$ base $| R(A)= R^{(1)}(a) \oplus R^{(2)}(a), \forall a \in G$

On s'interresse au représentations irréductible car elle sont plus {\it fondamentale}

\begin{tcolorbox}[title=Exemple: $C_4$ et base des vecteurs x,y]

	$$g: C_4 \quad R(g):\; \cols{0 & -1 \\ 1 &0} \dotsb $$
	 
	Les fonction de degré $m$ forment une repreésentation
\end{tcolorbox}


{\bf Lemme 1}: Si $R, R'$ sont deux représentation irréductibles (R.I.) inéquivalentes, alors il n'y a pas de matrice $H$ non-nulle telle que que $HR'(a) = R(a)H \forall a \in G$


{\bf Lemme 2}: Si $R$ est une R.I et $H$ une matrice non-nulle telle que $H R(A) = R(a)H \forall a \in G$ alors $H$ est un multiple de l'identité : $H = \lambda I$

Conséquence: soit une représentation réductible $R =R_1 \osum  R_2 \dotsb$

\section*{Caractères}

Relation d'orthogonalité
\[
	\frac{n_\nu}{g} \sum_{a\in G} R_{ik}^{(\nu)*} (a) R_{jl}^{(\mu)}(a) = \delta_{\mu\nu}\delta_{ij}\delta_{kl}
\]

corollaire : $\sum_\mu^r n_\mu^2 \leq g$ où $r$ est le nimbre de représentation distincte. En fait $=$ 

Def: caractère d'une classe dans une représentation

$\chi(a) = \tr R(a)$


les vecteurs de caractères sont orthogonauxé..

\[
	\sum_i^K \frac{g_i}{g} \chi_i^{(\nu)*}\chi_i^{(\mu)} = \delta_{\mu\nu}\\
	\sum_\mu^K \frac{g_i}{g} \chi_i^{(\mu)*}\chi_i^{(\mu)} = \delta_{ij}
\]



\end{document}
