\documentclass{article}    
\usepackage[utf8]{inputenc}    
    
\title{Épisode 4}    
\author{Jean-Baptiste Bertrand}    
\date{\today}    
    
\setlength{\parskip}{1em}    
    
\usepackage{physics}    
\usepackage{graphicx}    
\usepackage{svg}    
\usepackage[utf8]{inputenc}    
\usepackage[T1]{fontenc}    
\usepackage[french]{babel}    
\usepackage{fancyhdr}    
\usepackage[total={19cm, 22cm}]{geometry}    
\usepackage{enumerate}    
\usepackage{enumitem}    
\usepackage{stmaryrd}    
\usepackage{mathtools,slashed}
%\usepackage{mathtools}
\usepackage{cancel}
    
\usepackage{pdfpages}
%packages pour faire des math    
%\usepackage{cancel} % hum... pas sur que je vais le garder mais rester que des fois c'est quand même sympatique...
\usepackage{amsmath, amsfonts, amsthm, amssymb}    
\usepackage{esint}  
\usepackage{dsfont}

\usepackage{import}
\usepackage{pdfpages}
\usepackage{transparent}
\usepackage{xcolor}
\usepackage{tcolorbox}

\usepackage{mathrsfs}
\usepackage{tensor}

\usepackage{tikz}
\usetikzlibrary{quantikz}
\usepackage{ upgreek }

\newcommand{\incfig}[2][1]{%
    \def\svgwidth{#1\columnwidth}
    \import{./figures/}{#2.pdf_tex}
}

\newcommand{\cols}[1]{
\begin{pmatrix}
	#1
\end{pmatrix}
}

\newcommand{\avg}[1]{\left\langle #1 \right\rangle}
\newcommand{\lambdabar}{{\mkern0.75mu\mathchar '26\mkern -9.75mu\lambda}}

\pdfsuppresswarningpagegroup=1


\begin{document}
2024-01-09

\section{(Rappel) Électromagnétisme classique}

On s'intéresse d'abord à développer la théorie du magnétisme en physique classique. Cette théorie échoue à décrire correctement le magnétisme. En effet il faut la mécanique quantique pour le faire. Par contre le formalisme développé en quantique reste utile.

\begin{figure}[ht]
    \centering
    \incfig{densité-de-courrant}
    \caption{densité de courrant}
    \label{fig:densité-de-courrant}
\end{figure}

On a la loi suivant pour obenir le champ mangétique

\[ \mathbf{B}(\mathbf{r})=\frac{1}{c} \int \mathrm{J}\left(\mathbf{r}^{\prime}\right) \times \frac{\mathbf{r}-\mathbf{r}^{\prime}}{\left|\mathbf{r}-\mathbf{r}^{\prime}\right|^3} d^3 r^{\prime} \quad \text{(Biot-Savard)}  \] 

Qu'on peut réécrire comme 
\[ \mathbf{B}(\mathbf{r})=\frac{1}{c} \nabla \times \int \frac{\mathrm{J}\left(\mathbf{r}^{\prime}\right)}{\left|\mathbf{r}-\mathbf{r}^{\prime}\right|} d^3 r^{\prime} \]

Puisque $\vb{B} = \grad \times \vb{A}$

\[ \vb{A} = \frac{1}{c} \int \frac{\mathrm{J}\left(\mathbf{r}^{\prime}\right)}{\left|\mathbf{r}-\mathbf{r}^{\prime}\right|} d^3 r^{\prime} \]

Pas de monopoles magétiques:
\[ \grad \cdot \vb{B} = 0 \]

\[ \grad \times \vb{B} = \frac{1}{c} \grad \times \grad \times \int \frac{\mathrm{J}\left(\mathbf{r}^{\prime}\right)}{\left|\mathbf{r}-\mathbf{r}^{\prime}\right|} d^3 r^{\prime} = \dotsb \]

On developpe $\frac{1}{\abs{\vb{r}-\vb{r}'} } $

\[ \frac{1}{\abs{\vb{r}-\vb{r}'} } = \frac{1}{\abs{\vb{r}}} + \frac{\vb{r}\cdot \vb{r}'}{\abs{\vb{r}}^{3}}   + \dotsb\]

Le premier terme non-nul est le terme dit \textit{dipolaire}

\[ A_{\text{dip}}(\mathbf{r})=\frac{1}{c} \frac{1}{|\mathbf{r}|^3} \sum_{i, j} \hat{n}_i r_j \int r_j^{\prime} J_i d^3 r^{\prime} = \dotsb = -\frac{1}{c} \frac{1}{|\mathbf{r}|^3} \sum_i \hat{n}_i \frac{1}{2}\left[\mathbf{r} \times \int\left(\mathbf{r}^{\prime} \times \mathbf{J}\right)\right]_i d^3 r^{\prime}\]

On définit alors la densité de moment dipolaire comme

\[ \mathcal{M}(\vb{r}) = \frac{1}{2c} \vb{r} \times \vb{J}(\vb{r})  \]
Ce qui permet de réécrire 

\[\vb{A}_{\text{dip}} =  -\frac{1}{|\mathbf{r}|^3} \mathbf{r} \times \int \overrightarrow{\mathscr{M}}\left(\mathbf{r}^{\prime}\right) d^3 r^{\prime} \equiv \frac{\vb{m} \times \vb{r}}{\abs{\vb{r}}^{3}}  \]

Le moment magnétique peut se réécrire de la forme


\[ \vb{m} = \sum_{i}^{} \gamma_i \mathcal{L}_i \]

avec $\mathcal{L}_i$ le moment cinétique et $\gamma_i= \frac{q_i}{2M_i c} $ le facteur gyromagnétique 





\end{document}
