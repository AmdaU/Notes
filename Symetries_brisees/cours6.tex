\documentclass{article}    
\usepackage[utf8]{inputenc}    
    
\title{Épisode 4}    
\author{Jean-Baptiste Bertrand}    
\date{\today}    
    
\setlength{\parskip}{1em}    
    
\usepackage{physics}    
\usepackage{graphicx}    
\usepackage{svg}    
\usepackage[utf8]{inputenc}    
\usepackage[T1]{fontenc}    
\usepackage[french]{babel}    
\usepackage{fancyhdr}    
\usepackage[total={19cm, 22cm}]{geometry}    
\usepackage{enumerate}    
\usepackage{enumitem}    
\usepackage{stmaryrd}    
\usepackage{mathtools,slashed}
%\usepackage{mathtools}
\usepackage{cancel}
    
\usepackage{pdfpages}
%packages pour faire des math    
%\usepackage{cancel} % hum... pas sur que je vais le garder mais rester que des fois c'est quand même sympatique...
\usepackage{amsmath, amsfonts, amsthm, amssymb}    
\usepackage{esint}  
\usepackage{dsfont}

\usepackage{import}
\usepackage{pdfpages}
\usepackage{transparent}
\usepackage{xcolor}
\usepackage{tcolorbox}

\usepackage{mathrsfs}
\usepackage{tensor}

\usepackage{tikz}
\usetikzlibrary{quantikz}
\usepackage{ upgreek }

\newcommand{\incfig}[2][1]{%
    \def\svgwidth{#1\columnwidth}
    \import{./figures/}{#2.pdf_tex}
}

\newcommand{\cols}[1]{
\begin{pmatrix}
	#1
\end{pmatrix}
}

\newcommand{\avg}[1]{\left\langle #1 \right\rangle}
\newcommand{\lambdabar}{{\mkern0.75mu\mathchar '26\mkern -9.75mu\lambda}}

\pdfsuppresswarningpagegroup=1


\begin{document}
2024-02-09


\section*{Transitions de phase}

\underline{Définition}: Transition de phase \(\implies\) émérgence d'un paramètre d'ordre (\(\expval{\hat \phi} \neq 0 \)) en dessous d'une température (pression, champ magnétique, ...) \(T_c\) suite à une brisure (spontané) de symétrie. 

\begin{figure}[ht]
		\centering
		\incfig{deux-grands-types-de-symétries}
		\caption{Deux grands types de symétries}
		\label{fig:deux-grands-types-de-symétries}
\end{figure}

\begin{table}[ht]
	\centering
	\label{tab:transitions}

	\begin{tabular}{c|c|c|c}
		Transition & \(\expval{\phi}\) & ordre & nouvelles excitations\\\hline
		gaz-liquide & \(\rho_L -\rho_G\)& 1 ou 2 & aucune \\\hline
		liquide-solide & $\rho_?$ & 1 & phonons \\\hline
		Para - ferro (magnétique) & ?? & 2 & magnons, antiferro magnons\\\hline
		Cristaux liquides & (une fonction de l'angle)&2 & oui(parce que \(\theta\) varie continuement )\\\hline
		Superfuluide (\(^4 \rm{He}^{(1)} \to ^4 \rm{He}^{(2)}\)) & fonctions d'onde macrosopique & & \textit{mode de vibrations}, vortex \\\hline
		supracondictivité & \(\Psi \sim \abs{\Psi} e^{i\varphi}\) & & Pas d'excitation sans gap, Mécanisme Anderson-Higgs
	\end{tabular}
\end{table}

\section*{Gaz-liquide/liquide-solide}

Ensemble isobar-isotherme

\[ G(T,P,N) = N g(T,P,) \]

\[ \left( \pdv{G}{N}  \right)_{T,P} = \mu = g \]

\end{document}
