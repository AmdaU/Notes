\documentclass{article}    
\usepackage[utf8]{inputenc}    
    
\title{Épisode 4}    
\author{Jean-Baptiste Bertrand}    
\date{\today}    
    
\setlength{\parskip}{1em}    
    
\usepackage{physics}    
\usepackage{graphicx}    
\usepackage{svg}    
\usepackage[utf8]{inputenc}    
\usepackage[T1]{fontenc}    
\usepackage[french]{babel}    
\usepackage{fancyhdr}    
\usepackage[total={19cm, 22cm}]{geometry}    
\usepackage{enumerate}    
\usepackage{enumitem}    
\usepackage{stmaryrd}    
\usepackage{mathtools,slashed}
%\usepackage{mathtools}
\usepackage{cancel}
    
\usepackage{pdfpages}
%packages pour faire des math    
%\usepackage{cancel} % hum... pas sur que je vais le garder mais rester que des fois c'est quand même sympatique...
\usepackage{amsmath, amsfonts, amsthm, amssymb}    
\usepackage{esint}  
\usepackage{dsfont}

\usepackage{import}
\usepackage{pdfpages}
\usepackage{transparent}
\usepackage{xcolor}
\usepackage{tcolorbox}

\usepackage{mathrsfs}
\usepackage{tensor}

\usepackage{tikz}
\usetikzlibrary{quantikz}
\usepackage{ upgreek }

\newcommand{\incfig}[2][1]{%
    \def\svgwidth{#1\columnwidth}
    \import{./figures/}{#2.pdf_tex}
}

\newcommand{\cols}[1]{
\begin{pmatrix}
	#1
\end{pmatrix}
}

\newcommand{\avg}[1]{\left\langle #1 \right\rangle}
\newcommand{\lambdabar}{{\mkern0.75mu\mathchar '26\mkern -9.75mu\lambda}}

\pdfsuppresswarningpagegroup=1


\begin{document}
2024-01-16

\section*{On poursuit sur le Magnétisme quantique }

\[ H = \frac{\vb{p}^2}{2m} - \gamma \vb{L} \cdot \vb{H} + \frac{e^{2}}{2mc^2} \vb{A}^2 \]


\section*{Comment faire apparaitre le spin 1/2}

L'équation de Shordinger n'est pas invariante de Lorentz


L'idée de Dirac, prendre un $H$ linéaire en $p$ mais dont le carré redonne $E = p^{2}c^{2}+ m^{2}c^4$


On pose la forme
\[ H = c \vec{\alpha} \cdot \vec{p} + \beta m c^{2} \]

\[ = c^{2}\sum_{ij} \alpha_i \alpha_j p_i p_j +\beta^{2} + \dotsb \]


C'est impossible de trouver des matrices 2x2 qui fonctionne, on prends donc des matrices 4x4


\[ \psi = \begin{pmatrix} \chi \\ \Phi \end{pmatrix}  \]

On rajoute le champ mangétique dans l'équation par $\vb{P} \to \vb{P} - \frac{e}{c} \vb{A}$


\section*{Magnétisme quantique}

\[ Z = \tr e^{-\beta H} = \sum e^{-\beta E}  =  \sum \expval{e^-{\beta H}} \qq{ensemble canonique} \]

On considère que les spins vivent sur un réseau.

On négligle l'intéraction avec les spins?? (Je sais pas ce que ça veut dire, spin-spin surement)

On considère que le champ magnétique externe est constant









\end{document}
