\documentclass{article}    
\usepackage[utf8]{inputenc}    
    
\title{Épisode 4}    
\author{Jean-Baptiste Bertrand}    
\date{\today}    
    
\setlength{\parskip}{1em}    
    
\usepackage{physics}    
\usepackage{graphicx}    
\usepackage{svg}    
\usepackage[utf8]{inputenc}    
\usepackage[T1]{fontenc}    
\usepackage[french]{babel}    
\usepackage{fancyhdr}    
\usepackage[total={19cm, 22cm}]{geometry}    
\usepackage{enumerate}    
\usepackage{enumitem}    
\usepackage{stmaryrd}    
\usepackage{mathtools,slashed}
%\usepackage{mathtools}
\usepackage{cancel}
    
\usepackage{pdfpages}
%packages pour faire des math    
%\usepackage{cancel} % hum... pas sur que je vais le garder mais rester que des fois c'est quand même sympatique...
\usepackage{amsmath, amsfonts, amsthm, amssymb}    
\usepackage{esint}  
\usepackage{dsfont}

\usepackage{import}
\usepackage{pdfpages}
\usepackage{transparent}
\usepackage{xcolor}
\usepackage{tcolorbox}

\usepackage{mathrsfs}
\usepackage{tensor}

\usepackage{tikz}
\usetikzlibrary{quantikz}
\usepackage{ upgreek }

\newcommand{\incfig}[2][1]{%
    \def\svgwidth{#1\columnwidth}
    \import{./figures/}{#2.pdf_tex}
}

\newcommand{\cols}[1]{
\begin{pmatrix}
	#1
\end{pmatrix}
}

\newcommand{\avg}[1]{\left\langle #1 \right\rangle}
\newcommand{\lambdabar}{{\mkern0.75mu\mathchar '26\mkern -9.75mu\lambda}}

\pdfsuppresswarningpagegroup=1


\begin{document}
\section*{Indice de rotation et Umalufsatz}

$\alpha$ une courbe planaire

On peut assigner un signe à la courbe.

On définit $T$ comme d'habitude soit $T(s) \equiv \alpha'(s)$ mais $N(s):=R_{\frac{\pi}{2} }T(s)$ 

Où $R_{\frac{\pi}{2} }$ est une rotation de $\frac{\pi}{2}$. ON a donc

$$ T(s) = (x(s), y(s)) \implies N(s) = (-y(s), x(s))$$ 

et 
$$\kappa(s) := T'(s)\cdot N(s)$$ 

Fenet-Seret dans $\mathbb{R}^2$: $$\begin{aligned} T'(s) &= \kappa(s)N(s)\\ N'(s) &=-\kappa(s)T(s)\end{aligned}$$  
\underline{autres interprétation de $\kappa(s)$}: Dans $\mathbb{R}^2$, on peut toujours écrire $T(s) = (\cos(\theta(s)), \sin(\theta(s)))$  

$$T'(s) = (-\sin(\theta(s))\theta'(s). \cos(\theta(s))\theta'(s)) = \theta'(s)M(s)$$ 

On comprend donc que $\theta'(s) = \kappa(s)$ 

On peut donc définir $\theta(s)$ comme $$\theta(s) = \int_0^s\kappa(t) \dd t + \theta(0) \implies \theta(s)-\theta(0) = \int_0^s \kappa(t)\dd t$$  

Si $\alpha$ est une courbe \underline{fermée} ( $\alpha(s+L) = \alpha(s)$ ) alors on a que $$\theta(L) =\theta(0) = 2k\pi$$ 

On appelle $\frac{1}{2\pi} \int_0^L \kappa(t) \dd t = R$ l'\underline{indice de rotaition}

Exemples: 

Le $R$ d'un cercle parcourus en sens anti-horaire est de $1$ alors que celui d'une cercle parcourus en sens anti-horaire est de $-1$. Celui d'un leminiscate est de $0$

\section*{Théorème des tangeantes tournantes}

Si $\alpha$ est une courbe \underline{fermée} \underline{simple} (sans auto-intersection) alors $R=\pm1$  


\end{document}
