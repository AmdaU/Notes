\documentclass{article}    
\usepackage[utf8]{inputenc}    
    
\title{Épisode 4}    
\author{Jean-Baptiste Bertrand}    
\date{\today}    
    
\setlength{\parskip}{1em}    
    
\usepackage{physics}    
\usepackage{graphicx}    
\usepackage{svg}    
\usepackage[utf8]{inputenc}    
\usepackage[T1]{fontenc}    
\usepackage[french]{babel}    
\usepackage{fancyhdr}    
\usepackage[total={19cm, 22cm}]{geometry}    
\usepackage{enumerate}    
\usepackage{enumitem}    
\usepackage{stmaryrd}    
    
%packages pour faire des math    
%\usepackage{cancel} % hum... pas sur que je vais le garder mais rester que des fois c'est quand même sympatique...
\usepackage{amsmath, amsfonts, amsthm, amssymb}    
\usepackage{esint}  

\begin{document}

\underline{Rappels} 

\begin{itemize}
	\item Le plan hyperbolique $$U = \left\{ (u,v) | v > 0 \right\} $$ muni de la métrique $$M_I = \cols{\frac{1}{v^2} & 0 \\ 0 & \frac{1}{v^2} }$$ Courbure gaussiene : $\kappa_g \equiv 1$ Symbol de Chrsitoffel $\Gamma_{uu}^{v}=\frac{1}{v} \;\Gamma_{vv}^{v}=\Gamma_{uv}^{u}=-\frac{1}{v} $ Les autres sont nuls   
\end{itemize}

\underline{Proposition}: Les géodésiques de $\mathbb{H}^{2} $ sont des demis-droites verticales et les demi-cercles centrées sur l'axe $u$ .

\underline{Démonstration}: $x$ constant: good (demi-droite vecticale)
Sinon: 

$$\begin{cases}
	u'' - \frac{2}{v} u'v' = 0 \\ v'' + \frac{1}{v} (u'^2 v'^2) = 0	
\end{cases}$$ 


On réécrit en termes de $\dv{v}{u},\,\dv[2]{v}{u}$ 

$$\dv{u} \left( v(t(u)) \right)  = \dv{v}{t} \dv{t}{u} = \dv{v}{t} \left( \dv{u}{t} \right)^{-1}$$ 

$$\begin{aligned}
	\dv[2]{v}{u} &= \dv{u}\left( \dv{v}{t}?\dv{t}{u} \right) \\ &= \dv{v}{u} \left( \dv{t}{u} \right)^2 + \dv{v}{t}\dv[2]{t}{u} \\ &= \dv{v}{u} \left( \dv{u}{t} \right)^{-2} + \dv{v}{t} = \dotsb = \frac{-\qty(u'^2+v'^2)}{vu'^2}  \\ 
\end{aligned}$$
$$\dv[2]{v}{u} = \frac{- \left( u'^2 + v'^2 \right) }{vu'^2} =- \frac{1}{v} - \frac{1}{v} \left( \frac{v'}{u'}  \right)^2 = - \frac{1}{v} \left(  1 + \left( dv v u \right)^2 \right)   $$ 

$$v \dv{2}{vu} + \left( \dv{v}{u}  \right)^2 =-1$$ 

$$\dv{u} \left( v \dv{v}{u}  \right)  = -1$$ 

$$v \dv{v}{u} = -u +C$$ 

$$\int v \dd v = \int -u \dd u + C \dd u$$ 

$$\frac{v^2}{2} = - \frac{u^2}{2}  + Cu + C'$$ 

$$v^2 + u^2 -2Cu = 2 C'$$ 

$$v^2 + (u-C)^2 = 2c' + c^2$$ 


Équation d'un cercle de rayon $\sqrt{2C'+C^2}$ et centré en $(C,0)$  


$$\blacksquare$$ 

Maintenant qu'on connait les lignes droite de notre nouvelles géométrie, on peut essayer de trouver des propriété analogues avec la géométries qui nous est familiaire, soit la géométrie Euclidienne.

Le plan hyperbolique satisfait aux axiômes d'Euclides saut le cinquième.

Celui-ci s'énonce: 

\begin{tcolorbox}[title=Cinquieme postuat]
	 Étant donné une droite $\ell$ et un point $p \notin \ell $ il existe une \underline{unique} droite $\ell'$ par $p$ parallèle à $\ell$ 
\end{tcolorbox}

Dans la géométrie Hyperbolique par contre, il existe une \underline{infinité} de tels droites!


\begin{figure}[ht]
    \centering
    \incfig{comparaison-entre-5ieme-postulat-euclidien-et-hyperbolique}
    \caption{Comparaison entre 5ieme postulat euclidien et hyperbolique}
    \label{fig:comparaison-entre-5ieme-postulat-euclidien-et-hyperbolique}
\end{figure}

Tout les triangles hyperboliques on un aire inferieur à $\pi$ 

Gauss-Bonnet: $$\theta_1 + \theta_2 + \theta_3  = \pi -\rm{Aire}(t)$$ 
:)
\end{document}
