\documentclass{article}    
\usepackage[utf8]{inputenc}    
    
\title{Épisode 4}    
\author{Jean-Baptiste Bertrand}    
\date{\today}    
    
\setlength{\parskip}{1em}    
    
\usepackage{physics}    
\usepackage{graphicx}    
\usepackage{svg}    
\usepackage[utf8]{inputenc}    
\usepackage[T1]{fontenc}    
\usepackage[french]{babel}    
\usepackage{fancyhdr}    
\usepackage[total={19cm, 22cm}]{geometry}    
\usepackage{enumerate}    
\usepackage{enumitem}    
\usepackage{stmaryrd}    
\usepackage{mathtools,slashed}
%\usepackage{mathtools}
\usepackage{cancel}
    
\usepackage{pdfpages}
%packages pour faire des math    
%\usepackage{cancel} % hum... pas sur que je vais le garder mais rester que des fois c'est quand même sympatique...
\usepackage{amsmath, amsfonts, amsthm, amssymb}    
\usepackage{esint}  
\usepackage{dsfont}

\usepackage{import}
\usepackage{pdfpages}
\usepackage{transparent}
\usepackage{xcolor}
\usepackage{tcolorbox}

\usepackage{mathrsfs}
\usepackage{tensor}

\usepackage{tikz}
\usetikzlibrary{quantikz}
\usepackage{ upgreek }

\newcommand{\incfig}[2][1]{%
    \def\svgwidth{#1\columnwidth}
    \import{./figures/}{#2.pdf_tex}
}

\newcommand{\cols}[1]{
\begin{pmatrix}
	#1
\end{pmatrix}
}

\newcommand{\avg}[1]{\left\langle #1 \right\rangle}
\newcommand{\lambdabar}{{\mkern0.75mu\mathchar '26\mkern -9.75mu\lambda}}

\pdfsuppresswarningpagegroup=1



\begin{document}

\underline{Rappels} 

La \underline{dérivé} d'une application $f: S \to \mathbb{R}$ est l'application linéaire $\eval{df}_{x}:T_{x} S \to \mathbb{R} $ de matrice $D(f\circ p)$ dans la base $p_{u}, p_{v} $ 



Par une application $f: S \to S^*$ 

$$\eval{df}_{x} T_{x} S \to T_{f(x)} S^*$$

est donné par la matrice $D\circ p^{-1} \circ f \circ p$ dans la base $p_{u}, p_{v} $ de $T_xS$ et $p_{u}^* p_{v}^*$ 

Application de Gauss $n: S \to S^2$ vecteur normal uniraite pour $p$ fixée, n = $\frac{p_{u}\cross p_{v}}{\norm{ p_{u}\cross p_v}}$ 



Opérateur de formce: $\mathcal{S}_x: T_{x} S \to T_{x} S$ 


$$\mathcal{S}_x(x) = -dn_{x} (X)$$ 

seconde forme fondamentale

$$II_x(X,Y) = \mathcal{S}_x(X) \cdot Y = I( \mathcal{S}_x(X), Y)$$ 


\underline{Exemple:} 


$$p(u,v) = \cols{u\\v\\u^2 + v^2}$$ 

$$p_{u} = \cols{1\\0\\2u} \qquad p_{v}  \cols{0\\1\\2v}$$ 



$$p_{u} p_{v} = \cols{-2u\\ -2v \\1 }$$ 

$$\norm{p_{u} \times p_{v}} = \sqrt{4u^2+4v^2+1}$$ 

$$n = \frac{1}{\sqrt{4u^24v^2+1}} \cols{-2u\\ -2v\\ 1}$$ 



On a montré, dans la démonstration que $II$ est symétrique,  que $$II_{x} = \cols{p_{uu} \cdot n & p_{uv} \cdot n\\ p_{vu} \cdot n & p_{vv} \cdot n} = \cols{2&0\\0&2} \frac{1}{\sqrt{4u^2+4v^2+1}}$$ 




\underline{Résumé intra} 

\renewcommand{\labelenumi}{\Roman{enumi}}


\begin{enumerate}
	\item Coubrebes dans $\mathbb{R}^{2} et \mathbb{R}^{3}$\begin{itemize}
		\item Longeur d'arc $\to l(\gamma)=\int_{a}^b \norm{\gamma'(t)} \dd t$ 
		\item Paramétrisation par longueur d'arc
			$$\Psi(t) = \int_{a}^t \norm{\gamma'(t)}\dd t$$ $\gamma(s) = (\gamma\circ\psi^{-1})(s)$ 
		\item Courbure, torsion, repère de Frenet
		\item Formules de Frenet-Serret $$\begin{matrix} T'(s)= & & \kappa(s)N(s) & \\ N'(s) = &\kappa T & & \tau B \\ B'(s) = & & -\tau N &	
		\end{matrix}$$ 
		\item Théorème fondamentale des courbes dans $\mathbb{R}^{3}$ - isométrie de $\mathbb{R}^{3} $  + isométrie discrète 
		\item courbures signées d'une coube dans $\mathbb{R}^{2} $ 
		\item Indice de rotation $$R(\gamma) = \frac{1}{2\pi} \int_{0}^L\kappa(s) \dd s$$ 
		\item Umlaufsatz
	\end{itemize} 
\item \underline{Surfaces dans $\mathbb{R}^{3}$ } \begin{itemize}
	\item Cartes de surfaces (paramétrisation)
	\item Surfaces lisses 
	\item Plan tangent $T_{p(u,v)} S = \avg{p_{u}, p_{v}}$ (engendré par) 
	\item \underline{Théorème des valeurs régulières} $$f:\mathbb{R}^{3} \to \mathbb{R}$$ $$f(x,y,x) = x$$ est une surface lisse si c est une valeur régulière $\iff$ tout les points de $f^{-1}(c)$ sont réguliers ( $df \neq 0$ )   
	\item Première forme fondamentale $$X,Y\in T_{x} S$$ $$I_{x} (X,Y) = X \cdot Y$$  donné par la matrice $$I = \cols{p_{u} \cdot p_{u} & p_{u} \cdot p_{v} \\ p_{v} \cdot p_{u} & p_{v} \cdot p_{v} }$$ dans la base $p_{u} p_{v}$ 
	\item Surfaces localement isométrique (même première forme fondamenetale)
\end{itemize} 
\end{enumerate}









	
\end{document}
