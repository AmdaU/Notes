\documentclass{article}    
\usepackage[utf8]{inputenc}    
    
\title{Épisode 4}    
\author{Jean-Baptiste Bertrand}    
\date{\today}    
    
\setlength{\parskip}{1em}    
    
\usepackage{physics}    
\usepackage{graphicx}    
\usepackage{svg}    
\usepackage[utf8]{inputenc}    
\usepackage[T1]{fontenc}    
\usepackage[french]{babel}    
\usepackage{fancyhdr}    
\usepackage[total={19cm, 22cm}]{geometry}    
\usepackage{enumerate}    
\usepackage{enumitem}    
\usepackage{stmaryrd}    
\usepackage{mathtools,slashed}
%\usepackage{mathtools}
\usepackage{cancel}
    
\usepackage{pdfpages}
%packages pour faire des math    
%\usepackage{cancel} % hum... pas sur que je vais le garder mais rester que des fois c'est quand même sympatique...
\usepackage{amsmath, amsfonts, amsthm, amssymb}    
\usepackage{esint}  
\usepackage{dsfont}

\usepackage{import}
\usepackage{pdfpages}
\usepackage{transparent}
\usepackage{xcolor}
\usepackage{tcolorbox}

\usepackage{mathrsfs}
\usepackage{tensor}

\usepackage{tikz}
\usetikzlibrary{quantikz}
\usepackage{ upgreek }

\newcommand{\incfig}[2][1]{%
    \def\svgwidth{#1\columnwidth}
    \import{./figures/}{#2.pdf_tex}
}

\newcommand{\cols}[1]{
\begin{pmatrix}
	#1
\end{pmatrix}
}

\newcommand{\avg}[1]{\left\langle #1 \right\rangle}
\newcommand{\lambdabar}{{\mkern0.75mu\mathchar '26\mkern -9.75mu\lambda}}

\pdfsuppresswarningpagegroup=1


\begin{document}
\section*{Rappels}

Pour une courbe de $\mathbb{R}^2$, la courbure à un signe

$$\kappa(s) = T'(s)\cdot N(s)$$ 
où $N(s) = R_{\frac{\pi}{2} } T(s)$" 

L'indice de rotation d<une courbe fermée (periodique) est $$\mathcal R(\alpha)=\frac{1}{2\pi} \int_0^L\kappa(s)\dd s$$ 

où $R\in \mathbb{Z}$ 

Umlaufsatz (tangeantes tournantes). Si $\alpha$ est \underline{simple} (pas d'auto-intersection) $\mathcal R(\alpha)=1$  

Si on écrit $T(s) = (\cos(\theta(s)),\sin(\theta(s)))$, alors $\kappa(s)=\theta'(s)$  


\section*{Chapitre 2: Surfaces dans $\mathbb{R}^3$}

On va maintenant parler des surfaces dans $\mathbb{R}^3$ 

\underline{Rappels}: $f: \mathbb{R}^n\to \mathbb{R}^n$  

$$\eval{Df}_p = \eval{\begin{pmatrix} \pdv{f_1}{x_1} & \pdv{f_1}{x_2}&  \dotsb & \pdv{f_1}{x_n}\\ \vdots & \vdots & \ddots & \vdots\\ \dv{f_n}{x_1} & \dv{f_n}{x_1}& \dotsb& \dv{f_n}{x_n}
\end{pmatrix}}_p$$ 


La différentiel de $f$ en $p$

$U \subset \mathbb{R}^n$ est \underline{ouvert} ssi $\forall \vec x \in U \exists \epsilon \geq 0$ t.q. $B_\epsilon(\vec x) \subseteq U$    

$S \subseteq \mathbb{R}^n$. UN sous-ensemble $U\subseteq S$ est \underline{ouvert dans $S$ } ssi $\forall \vec x \in U \exists \epsilon \ge 0$ t.q $B_\epsilon (\vec x) \bigcap S \subseteq U$    

\underline{Exemple} $S^2 = \{(x,y,z)^T \in \mathbb{R}^3 | x^2 + y^2 + z^2 =1\}$  

ON peut parametriser une partie de $S^2$ à l'aide de \underline{coordonn.es sphériques} 


\begin{figure}[h!]
    \centering
    \incfig{parametrisation-shperique}
    \caption{parametrisation shperique}
    \label{fig:parametrisation-shperique}
\end{figure}
	
$(0, \cos\varphi,\sin\varphi)^T,\quad -\frac{pi}{2} \le \varphi \ge \frac{pi}{2} $ 

Rotation autour de $\theta$ $$R_\theta = \dotsb$$ 

Les pôles ne sont pas dans notre paramétrisation

\underline{Déf} Une application $p: I \subseteq \to \mathbb{R}^3$ ( $U$ Ouvert) est une \underline{carte de surface lisse} si elle est lisse, bijective et $Df$ est de plein range $\forall p \in U$

\underline{Déf} Une surface lisse $S \subset \mathbb{R}^3$ est un sous-ensemble t.q tout point $\vec x\in S$  est contenue dans l'image d<une carte de surface lisse $p:U\to S$ t.q. $p$ est une \underline{homéomorphisme} (application bijective continue d'inverse continu) entre $u$ et une ouvert de $S$    

\begin{figure}[h!]
    \centering
    \incfig{mapping-dune-surface}
    \caption{mapping dune surface}
    \label{fig:mapping-dune-surface}
\end{figure}

UNe collection de paramétrisation $p_i: U_{\rm i}\to S$ t.q. $p_i(u_i)$ recouvrent $S$ s<appelle un \underline{atlas}    


\underline{Exemple} Pour la shpère, on peut construitre un atla avec 2 cartes de surfaces lisses 

\begin{figure}[h!]
    \centering
    \incfig{mapping-de-la-sphère}
    \caption{mapping de la sphère}
    \label{fig:mapping-de-la-sphère}
\end{figure}

On peut aussi construire un atlas de $S^2$ en utilisant des "projections inverses"

\begin{figure}[h!]
    \centering
    \incfig{projection-inverses}
    \caption{projection inverses}
    \label{fig:projection-inverses}
\end{figure}


$$p_1(x,y) = (x,y,\sqrt{1--x^2-y^2})$$ 


On doit prendre un total de 6 hemi-sphere pour couvrir toute la sphère de cette manière. Sinon il manque toujours de points sur l'équateur.

\underline{Exemple 2} le \underline{graph}  d<une fonction lisse $f: \mathbb{R}^{2} \to \mathbb{R}$ est une carte de surface lisse  
$$F: \mathbb{R}^{2} \to \mathbb{R}^{3} $$ 

$$DF = \begin{pmatrix}
	1 &0\\ 0 & 1 \\ \dv{f}{x} & \dv{f}{y}
\end{pmatrix}$$ 

toujours de premier rang

\underline{Exemple 3: l'hélicoïde} est une hélive dans $\mathbb{R}^{3} $ à laquelle on ajouter des segments horizontaux 

$$p(u,v) = (u\cos(v),u\sin(v0, bv) \quad (b \ge 0)$$ 

Domaine $U \ge 0, v\ in \mathbb{R}$ 

\begin{figure}[h!]
    \centering
    \incfig{helicoide}
    \caption{helicoide}
    \label{fig:helicoide}
\end{figure}

Une seule carte forme un atlas

$$Dp = \begin{pmatrix}
	\cos(v) & - u\sin(v)\\ \sin(v) & u\cos(v) \\ 0 & 0
\end{pmatrix}$$ 

On notes les colonnes de $Dp$ par $p_u$ et $p_v$


\underline{Exemple 4: Le toree} 

\begin{figure}[ht]
    \centering
    \incfig{parametrisation-du-tore}
    \caption{parametrisation du tore}
    \label{fig:parametrisation-du-tore}
\end{figure}


$$p(u,v) = \begin{pmatrix} 
	(a+b\cos u) \cos v \\ (a + b\cos*u)\sin v\\ b \sin s
\end{pmatrix}$$ 

Peut être couvert avec 4 cartes en changeant le domaine de $p$ de $\pm \pi$

Plus généralement, si $\alpha: [a,b] \to bb R 3$ est une courbe régulière dans le plan $y,z$ avec $y.0$. La surface de révolution associée est une surface lisse.

Si $\alpha(t) = (0, f(t), g(t))$

$$p(t, \tethe) = \begin{pmatrix} f(t) \cos\theta\\ f(t)\sin\theta\\ g(t)
	
\end{pmatrix}$$ 


\underline{Déf:} Soit $f: \mathbb{R}^{3} <to \mathbb{R}$ une fonction lisse. Un point $ \vec x \in \mathbb{R}^{3} $ t.q. $\eval{Df}_{\vec x} = 0$ est un \underline{point critique} et la valeur associée $a = f(\vec x)$ est une \underline{valeur critique} . UNe valeur $a \in \mathbb{R}$ est \underline{régulière} si elle n'est pas critique.    

\underline{Exemple:} $f(x,y,z) = x^2 + y^2 + z^2$ 

$$\eval(Df)_{(x,y,z)} = (2x, 2y, 2z)$$

le seul point critique est $(0,0,0)$. La seule valeur critique est $f(0,0,0)=0$. Toutes les valeurs dans $\mathbb{R}\backslash\{0\}$ sont des valeurs régulières.

On a $S^2 = f^{-1}(1)$

$$f^{-1}(0) = \{(0,0,0)^T\} \quad \text{pas une surface lisse}$$ 

\underline{Proposition} Si $f: \mathbb{R}^{3} \to \mathbb{R}$ est lisse et $a\in \mathbb{R}$ est une valeur régulière de $f$, alors $S = f^{-1}(a)$ est une surface lisse $= \{\vec x \in \mathbb{R}^{3} | f(\vec x) = a\}$      
\underline{Rappel} \underline{Théorème de la fonction inverse} Soit $F: \mathbb{R}^n \to \mathbb{R}^n$ différentiable $\mathcal C^K$ et $\vec x \in \mathbb{R}^n$ t.q. $\eval{Df}_{\vec x}$ est inversible. Alors il existe des ouverts $U \ni \vec x$ $V\ni F(\vec x)$ t.q. $F: U \to V$ est inversible d<inverse de classe $\mathcal C^k$         

\end{document}
