\documentclass{article}    
\usepackage[utf8]{inputenc}    
    
\title{Épisode 4}    
\author{Jean-Baptiste Bertrand}    
\date{\today}    
    
\setlength{\parskip}{1em}    
    
\usepackage{physics}    
\usepackage{graphicx}    
\usepackage{svg}    
\usepackage[utf8]{inputenc}    
\usepackage[T1]{fontenc}    
\usepackage[french]{babel}    
\usepackage{fancyhdr}    
\usepackage[total={19cm, 22cm}]{geometry}    
\usepackage{enumerate}    
\usepackage{enumitem}    
\usepackage{stmaryrd}    
\usepackage{mathtools,slashed}
%\usepackage{mathtools}
\usepackage{cancel}
    
\usepackage{pdfpages}
%packages pour faire des math    
%\usepackage{cancel} % hum... pas sur que je vais le garder mais rester que des fois c'est quand même sympatique...
\usepackage{amsmath, amsfonts, amsthm, amssymb}    
\usepackage{esint}  
\usepackage{dsfont}

\usepackage{import}
\usepackage{pdfpages}
\usepackage{transparent}
\usepackage{xcolor}
\usepackage{tcolorbox}

\usepackage{mathrsfs}
\usepackage{tensor}

\usepackage{tikz}
\usetikzlibrary{quantikz}
\usepackage{ upgreek }

\newcommand{\incfig}[2][1]{%
    \def\svgwidth{#1\columnwidth}
    \import{./figures/}{#2.pdf_tex}
}

\newcommand{\cols}[1]{
\begin{pmatrix}
	#1
\end{pmatrix}
}

\newcommand{\avg}[1]{\left\langle #1 \right\rangle}
\newcommand{\lambdabar}{{\mkern0.75mu\mathchar '26\mkern -9.75mu\lambda}}

\pdfsuppresswarningpagegroup=1


\begin{document}

\underline{Rappels} 

\begin{itemize}
	\item $M_{I} = \cols{E & F \\ G & G}$	
	\item $M_{II} =\cols{L & M\\ M &N}$
	\item $M_{\mathcal{S} = M_{I}^{-1}\cdot M_{II} } $
	\item $\mathcal{S}$ est diagonalisable
	\item $k_{1},k_{2} \to$ \underline{courbure principale} 
	\item $\det(\mathcal{S} ) \to $ courbure gaussienne
	\item $\tr(\mathcal{S} )\to$ Courbure moyennne
	\item Symbol de Christoffel
\end{itemize}
	
Tentons d'exprimer $\Gamma_{ij}^k$ en termes de $E,F,G$ 

$$p_{uu} p_{u} = \Gamma_{uu}^u E + \Gamma_{uu}^v F = \frac{E_{u}}{2} $$ 
$$p_{uu}p_{v} = \Gammma_{uu}^u F + \Gamma_{uu}^v G = F_{u} \frac{E_{v}}{2} $$ 
$$p_{uv} p_{u} = \Gamma_{uv}^u E + \Gamma_{uv}^v F = \frac{E_{v}}{2} $$ 
$$p_{uv} p_{v} = \Gamma_{uv}^u F + \Gamma_{uv}^v G= \frac{G_{u}}{2} $$ 
$$p_{vv} p_{u} \Gamma_{vv}^u E + \Gamma_{vv}^v F + F_{v} - \frac{G_{u}}{2} $$ 
$$p_{vv} p_{v} = \Gamma_{vv}^u F + \Gamma_{vv}^v G = \frac{G_{v}}{2}  $$ 

Pour obtenir les équations de Gauss-Codazzi, on compare $p_{uuv}$ et $p_{uvu} $  $$p_{uuv} = \left( \Gamma_{uu}^u \right)_v p_{u}  + \Gamma_{uu}^up_{uv} + \left( \Gamma_{uu}^v \right)_v p_{v} + \Gamma_{uu}^vp_{vv} + L_{v} n + L n_{v} $$ 

$$= \left( \Gamma_{uu}^u \right)_v p_{u} + \Gamma_{uu}^u(\Gamma_{uu}^u \left( \Gamma_{uv}^u p_{u} + \Gamma_{uu}^v + \Gamma_{uu}^v p_{v} +M_n \right)  + \left( \Gamma_{uu}^v \right)_v p_{v} + \Gamma_{uu}^v\left(\Gamma_{vv}^u p+u + \Gamma_{vv}^v p_{v} + N_{n}   \right) +L_{v} n + L (bp_{u} + dp_{v} ) $$ 


$$\boxed{D_{n} = \cols{a & | & b\\ c&| &d } = -M_{I}^{-1}M_{II}\\ \implies n_{u} = ap_{u} + cp_{v} \\ n_{v} = bp_{u}+dp_{v} }$$ 
	

$$=\left( \left( \Gamma_{uu}^u \right)_v + \Gamma_{uu}^u\Gamma_{uv}^u + \Gamma_{uu}^v\Gamma_{vv}^u + Lb \right) p_{u} + \left( \left( \Gamma_{uu}^v \right)_v + \Gamma_{uu}^u\Gamma_{uv}^v + \Gamma_{uv}^v\Gamma_{vv}^? + Ld \right) p_{v} + \left( \Gamma_{uu}^u M + \Gamma_{uu}^v N + L_v \right) n$$ 

De la même manière
$$p_{uvu} = \left( \left( \Gamma_{uv}^u \right)_u + \Gamma_{uv}^u \Gamma_{uu}^u + \Gamma_{uv}^v\Gamma_{uv}^u + Ma \right) p_{u} + \left( \left( \Gamma_{uv}^v \right)_u + \Gamma_{uv}^u\Gamma_{uu}^v + \Gamma_{uv}^v\Gamma_{uv}^v + Mc \right) p_{v} + \left( \Gamma_{uv}^u + \Gamma_{iv}^v M + M_u \right) n$$ 

$$p_{uuv} = p_{uvu} $$ 
On compare les coefficients de $p_{v} $ 
$$Mc - Ld = \left( \Gamma_{uu}^v \right)_v - \left( \Gamma_{uv}^v \right)_u = \Gamma_{uv}^u \Gamma_{uu}^v =\Gamma_{uv}^v\Gamma_{uv}^v + \Gamma+uu^u\Gamma_{uv}^v + \Gamma_{uu}^v \Gamma_{vv}^v$$ 

$$Mc -Ld = -\cols{E & F\\F & G}^{-1}\cols{L&M\\M&N} = \frac{-1}{EG-F^2} \cols{Gl-MF & GM -F? \\ EM-FL & EN - ??$}$$ 

$$MC = Ld = M \left( \frac{FL-EM}{EF-G^2}  \right) -L \left( \frac{FM-EN}{EF-G^2}  \right) = \frac{E-M^2+LN}{EF-G^2} = E\cdot k$$ 

$$E\cdot k = \Gamma_{uu}^u\Gamma_{uv}^v + \Gamma_{uu}^v\Gamma_{vv}^v-\Gamma_{uv}^u\Gamma_{uu}^v - \left( \Gamma_{uv}^v \right)^2 + \left( \Gamma_{uu}^v \right)_v - \left( \Gamma_{uv}^v \right)_u $$ 

$k$ est intrinsèque (Peut être calculé avec $E,F,G$ et leurs 2 premières dérivées

On peut aussi obtenir des équation similaire avec

$$F\cdot k\qquad F\cdot K \qquad g\cdot k$$ 


\underline{Équations de Cedazzi} 

$$L_{v} = M_{u} = L\Gamma_{uv}^u + M \left( \Gamma_{uv}^v - \Gamma_{uu}^u \right) - N \Gamma_{uu}^v$$ 
$$M_{v} - N_{u} = L \Gamma_{vv}^u + M \left( \Gamma_{vv}^v-\Gamma_{uv}^u \right) -N\Gamma_{uv}^v$$ 

Comme la courbure est une propriété intrinsèque, Si un surface possède un surface de Gauss nul (comme une pizza). On peut forcer sa courbure dans une direction à être nul en la faisant courber dans un autre direction. Si la courbe imposé est \textit{vers le haut}. Il n'y aura aucune courbure vers le bas et la aliments ne tomberons pas.


\end{document}
