\documentclass{article}
\usepackage[utf8]{inputenc}

\title{meme}
\author{Jean-Baptiste Bertrand}
\date{December 2021}

\usepackage[utf8]{inputenc}
\usepackage[T1]{fontenc}
\usepackage[french]{babel}
\usepackage{fancyhdr}
\usepackage[total={19cm, 22cm}]{geometry}
\usepackage{enumerate}
\usepackage{enumitem}

%packages pour faire des math
%\usepackage{cancel} % hum... pas sur que je vais le garder mais rester que des fois c'est quand même sympatique...
\usepackage{amsmath, amsfonts, amsthm, amssymb}
\usepackage{esint}

\begin{document}


\section{Chapitre 0}

On s'interesse à qualifier des courbes sans étudier les propriétés des fonctions. Par exemple, on veut considérer $y=x^2$ et $y^2=x$ comme identique à roatition près malgré le fait qu'elle soit définis comme deux équations assez différentes.

On va distingues les propriétés intrinsèques et extrinsèques d'une surface.

Une propriété intrinsèques pourrait être détécté par quelqu'un vivant dans la surface.

La distance de longueure d'arc est une quantité intrinsèque à la sphère tandis que la \textit{logueure cordale} est une quantitée intrinsèque.

La \underline{corubure gaussiènne} est la plus importante quantitié intrinsèque associé à une surface.

La courbure gaussienne ne change pas si on la déforme de manière rigide.

\subsection{Courbure d'un polyèdre}

Défault d'angle:

$c(s)=2 \pi-\sum_{T \text { face }} \theta_{T}(s)$


\underline{La caractéristique d'Euleur} d'un polytope $P$ est la quantité

$$\chi(P) = V - E + F$$


\subsection{Théorème de Gauss-Bonnet discret}

$$\sum_{s \in P} c(s) = 2\pi \chi(P)$$


\underline{Dém} On compte de le défaut d'andre total de deux manières différentes

\begin{itemize}
	\item Défault d'angle total $\sum_{s \in P} c(s)$
	\item Dans chanque face triangulaire de $P$, la somme des angles $=\pi$. Le défault d'angle total: $2\pi V-\pi F$
\end{itemize}

Chaque arrête à 2 faces

Chaque face à 3 arrêtes

$$2E =3F$$

On compte la carinalité des $\{(a,f)| a\in f\}$

\begin{align*}
	2\pi\chi(P) & = 2\pi(V-E+F)         \\
	            & =2\pi(V-\frac32 F +F) \\
	            & = 2\pi V-\pi F
\end{align*}

\underline{Ex:} En utilisant le théorème démontré plus haut. et le fait que toute triangulation d'une sphère satisfait $\chi(P)=2$ classifie les solides réguliers. (Les face sont des polygones r/guliers. Même nombre de faces à chaque sommet)


\section{Chapitre 2}

\underline{Définition: } UNe fonction vectorielle $f:(a,b) \to \mathbb{R}^3$ est $C^k$ si $f$ et ses $k$ premières dérivées existent et sont continues sur $(a,b)$. On dit que $f$ est lisse si c'est vrai pour tout $k>0$

Une \underline{courbe paramétré} est une application  $C^3$

$\alpha I\to \mathbb{R}^3$

Ex:
\begin{itemize}
	\item $p \neq q \in \mathbb{R}^3$, on définit $V=q-p$ et $\alpha(t) = p+tv, \; t\in \mathbb{R}$
	\item $\alpha(t) = (a\cos(t), a\sin(t))\quad 0\leq t\leq 2\pi$ (le cerlce de rayon $a$)
	\item Courbe cubique sigulière: $\alpha(t) = (t^2, t^3), \quad \alpha'(t) = (2t, 3t^2) \implies \alpha'(0)=(0,0)$
	      non-régilère en $t=0$
\end{itemize}

\end{document}