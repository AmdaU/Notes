\documentclass{article}    
\usepackage[utf8]{inputenc}    
    
\title{Épisode 4}    
\author{Jean-Baptiste Bertrand}    
\date{\today}    
    
\setlength{\parskip}{1em}    
    
\usepackage{physics}    
\usepackage{graphicx}    
\usepackage{svg}    
\usepackage[utf8]{inputenc}    
\usepackage[T1]{fontenc}    
\usepackage[french]{babel}    
\usepackage{fancyhdr}    
\usepackage[total={19cm, 22cm}]{geometry}    
\usepackage{enumerate}    
\usepackage{enumitem}    
\usepackage{stmaryrd}    
    
%packages pour faire des math    
%\usepackage{cancel} % hum... pas sur que je vais le garder mais rester que des fois c'est quand même sympatique...
\usepackage{amsmath, amsfonts, amsthm, amssymb}    
\usepackage{esint}  


\begin{document}

\section*{Rappels}

Fourmule de la courbure $\frac{\norm{\alpha'(t) \cross \alpha''(t)}}{\norm{\alpha'(t)}^3}$

$$\kappa(s)=- \iff \text{ segment de droite}$$ 

$$\tau =0 \iff \text{ la courbe est planaire}$$ 

$$\tau(s) = 0 \quad\text{et}\quad \kappa(s)\equiv c \iff \alpha \text{ portion de cercle de rayon }$$ 

Forme locale canonique (Taylor)

Isométrie $x \mapsto Ax+b \quad AA^t = 1$

La courbure et la torsion sont \underline{invarientes par isométries}

Pour $A$ une isométrie directe $A\vec u \cross A\vec v = A (\vec u \vec v)$ 
En général $A\vec u \cross A \vec v = \det(A) A(\vec u \cross \vec v)$

\section*{Théorème fondamentale des courbes dans $\mathbb{R}^3$}

Deux courbes $C$, $C^*$ dans $\mathbb{R}^3$ de courbure non-nulle diffèrent par une isométrie directe $\iff$ ellse ont la même courbure et torsion ( $\kappa = \kappa^*$ et $\tau = \tau^*$ )

\underline{Dém}
Soit $\alpha$, $\alpha^*$ des courbes paramétrées par longueures d'arc de $C,C^*$

Prenons $A$, l'unique matrice orthogonale t.q. $$ \begin{aligned}AT(0) &= T^*(0)\\AN(0) &= N^*(0)\\AB(0) &= B^*(0)\end{aligned}$$ 

\begin{tcolorbox}
	Rappel: si $A$ enovie une base orhtonormée vers une base orthonormée alors $A$ est orthogonale.\\	
	Si $A$ envoie une base positiviement orienté à une base positiviment orientée alors $\det{A} > 0$ 
\end{tcolorbox}

Soit $\vec b \in \mathbb{R}^3$ t.q. $A\cdot\alpha(0) + \vec b = \alpha^*(0)$

Définissons $I(x) = Ax +\vec b$ et $\tilde \alpha(s) = I(\alpha(s)) = A\alpha(s) + b$ 

reste à montrer que $\tilde \alpha(s) = \alpha^*(s) \forall s$ 

On a $\tilde \alpha(0) = A\alpha(0) + \vec b = \alpha^*(0)$ 

Et comme $I$ est une isométrie

$$\tilde T(0) = AT(0) = T^*(0)\\\tilde N(0)=AN(0)=N^*(0)\\\tilde B(0)=AB(0)=B^*(0)$$ 

Comme $\kappa$, $\tau$ sont ivarients par isométries directe

$$\kappa^*(s) = \kappa(s)= \tilde \kappa(s)\\\tau^*(s) = \tau(s)= \tilde \tau(s)$$ 

Définissons une fonction $f(s) = \tilde T(s)\cdot T^*(s)+\tilde N\cdot N^*+\tilde B\cdot b^*$ 

$f'(s)= \text{C'est vraiment long à écrire, fuck ça, règle de chaine mdr }= 0$ 

$$\implies f(s) \equiv C \text{ mais } f(0)=1+1+1=3 \implies f(s) = 3$$ 

Par l'inégalité de Chauchy-Swatzsdfjhh ( $|u\cdot v| \leq \norm{u}\norm{v}$ )



$\tilde T(s)\cdot T^*(s) \leq 1$

$\tilde N(s)\cdot N^*(s) \leq 1$

$\tilde B(s)\cdot B^*(s) \leq 1$

On en conclut que les vecteur du repert de frenet tilde et étoile sont les mêmes

En particulier $\tilde \alpha'(s) = \alpha^{*\prime}(s) \implies \tilde \alpha(s) = \alpha^*(s) + \vec v_{0}$ mais $\vec v_{0} = 0$ car $\tilde \alpha(0) = \alpha^*(0)$

\underline{Question:} Étant donné deux donctions $\kappa(s), \tau(s)$, existe-t-il une courbe $\alpha$ ayant $\kappa,\tau$ comme courbure et torsion?

Oui! (avec suffisement de régularité)

Pour trouver $\alpha$, on résout le système
$$
\begin{matrix}
T' &= & &\kappa T\\
N' &= &-\kappa T & &\tau B\\
B' &= & & \tau B
\end{matrix}
$$

puis on intègre $T$. On sait qu'une solution existe grace au théorème d'exsitance des solutions d'éqiation différentielles.

\section*{Courbes planaires}

\underline{Théorème} [inégalité isopérimétrique]:

Soit $C$ une courbe planaire \underline{simple} fermée de longeure $l$ et $A$ est l'aire de la région bornée par $C$. Alors $l^2-4\pi A\leq 0$
Avec $= \iff C$ est un cercle 

\begin{tcolorbox}[title=Rappel]	
	Théroème de Greene : $$\int_{\rm C} \vb F \cdot \dd \vb r = \iint_{\rm R} rot(\vb F) \dd A$$ 
	En particulier, $\text{aire}(R)=\int_{\rm C}=\frac{1}{2}\int_c(y x)\cdot \dd \vb r = \frac{1}{2}\int xy' - yx' \dd t$ 
\end{tcolorbox}


\begin{figure}[ht]
    \centering
    \incfig{parametrisation-isoperimetrique}
    \caption{parametrisation isoperimetrique}
    \label{fig:parametrisation-isoperimetrique}
\end{figure}

$\alpha$ paramétrée par longeure d'arc de $C$ 
$\bar \alpha$ paramétré du cercle

Calculons

$$A + \bar A = A + \pi r^2 = \int_0^l x(s)y'(s)\dd s + \int_0^l - \bar y(s)x'(s) \dd y$$ 

Fuck les notes; dodo. Aussi, criss que mon shéma est laid, faut vraiment que j'aprène à utiliser inkscape


\end{document}
