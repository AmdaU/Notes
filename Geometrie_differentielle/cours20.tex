\documentclass{article}    
\usepackage[utf8]{inputenc}    
    
\title{Épisode 4}    
\author{Jean-Baptiste Bertrand}    
\date{\today}    
    
\setlength{\parskip}{1em}    
    
\usepackage{physics}    
\usepackage{graphicx}    
\usepackage{svg}    
\usepackage[utf8]{inputenc}    
\usepackage[T1]{fontenc}    
\usepackage[french]{babel}    
\usepackage{fancyhdr}    
\usepackage[total={19cm, 22cm}]{geometry}    
\usepackage{enumerate}    
\usepackage{enumitem}    
\usepackage{stmaryrd}    
\usepackage{mathtools,slashed}
%\usepackage{mathtools}
\usepackage{cancel}
    
\usepackage{pdfpages}
%packages pour faire des math    
%\usepackage{cancel} % hum... pas sur que je vais le garder mais rester que des fois c'est quand même sympatique...
\usepackage{amsmath, amsfonts, amsthm, amssymb}    
\usepackage{esint}  
\usepackage{dsfont}

\usepackage{import}
\usepackage{pdfpages}
\usepackage{transparent}
\usepackage{xcolor}
\usepackage{tcolorbox}

\usepackage{mathrsfs}
\usepackage{tensor}

\usepackage{tikz}
\usetikzlibrary{quantikz}
\usepackage{ upgreek }

\newcommand{\incfig}[2][1]{%
    \def\svgwidth{#1\columnwidth}
    \import{./figures/}{#2.pdf_tex}
}

\newcommand{\cols}[1]{
\begin{pmatrix}
	#1
\end{pmatrix}
}

\newcommand{\avg}[1]{\left\langle #1 \right\rangle}
\newcommand{\lambdabar}{{\mkern0.75mu\mathchar '26\mkern -9.75mu\lambda}}

\pdfsuppresswarningpagegroup=1


\begin{document}

\underline{Rappels}

\begin{itemize}

	\item Paramétisation orthogonale: $p_u p_v =0 = F$ 
	\item Intégrale de surface $f: S \to \mathbb{R}$ $$\iint_{p(u)} f(x) \dd S := \iint_u f(u,v) \norm{p_u \times  p_v } \dd u\dd v = \iint_U f(u,v) \sqrt{EG-F^2}\dd u \dd v$$ 
	\item $$\varphi_{12} := (\grad_{\alpha'} e_{1)} \cdot e_2$$ où $e_1 = \frac{p_u}{\sqrt{E}} ,\ e_2 = \frac{p_v}{\sqrt{G}} $  "Rotation de $e_{1}\ e_2 $ le lond de $\alpha$ "  
	\item $\alpha$ courbe fermée borant $R$ $$\implies \int_{\alpha} \varphi_{12} \dd s = - \iint_R \kappa \dd S $$   
	\item Si $\alpha'(s) = T(s) = \cos \theta e_{1} + \sin \theta e_2$ alors $\underbrace{k_g}_{\text{Courbure de $\alpha$ intrinsèque à la surface} }  = \underbrace{\varphi_{12}}_{\text{Rotation du repère $e_{1},\ e_2$ } }  + \underbrace{\theta'}_{\text{rotation de $\alpha'$ dans le repère $e_1 , \ e_2$ } }  $  
	
\end{itemize}


\underline{"Umlanfsatz" sur une surface} : Si $\alpha$ est une courbe simple fermée contractible sur $S$ alors $$\int_0^L \theta' \dd s = 2\pi$$  


\underline{Contractible}P qui peut être "remplie" par/borne un disque 

\begin{figure}[ht]
    \centering
    \incfig{contractible-vs-non-contractible}
    \caption{Contractible vs non-contractible}
    \label{fig:contractible-vs-non-contractible}
\end{figure}

\underline{"Dém"}: La quantité $\int_0^L$ est toujours une mutiple de $2 \pi$ (On commence et finit par le même vecteur). C'est une fonction continue de la courbe $\alpha$    

\underline{Version "avec des coins"} 

Si $\alpha$ est lisse par morceaux $$\int_0^L = 2\pi = \sum \epsilon_k$$ avec $\epsilon_k$ les angles exterieurs de $\alpha$    

\begin{figure}[ht]
    \centering
    \incfig{umlanfsatz-sur-une-surface-avec-des-coins}
    \caption{Umlanfsatz sur une surface avec des coins}
    \label{fig:umlanfsatz-sur-une-surface-avec-des-coins}
\end{figure}
	
$\epsilon_k$: l'angle entre le vecteur tangeant entrant et le vecteur sortant au sommet de $k$

\underline{Théorème de Gauss-Bonnet local} : Soit $\alpha$ une courbe lisse par morceaux fermées, simples ???, contractible, bornant $R$ dans la surface $S$. Alors 

$$\iint_R \kappa \dd S + \int_{\alpha} \kappa_g \dd s  + \sum \epsilon_k = 2\pi$$ 

\underline{Démonstration}: On a montré $$\iint_R \kappa \dd S = -S_{\alpha} \varphi_{12} \dd s = -\int_{\alpha} \kappa_g -\theta' \dd s = -\int_{\alpha} \kappa_g + \int \theta' \dd s = -\kappa_g \dd S + 2\pi -\sum \epsilon_k \qquad \blacksquare$$  

\underline{Exemple}: Dans le plan ou sur un cylindre:

$$\implies \int_{\alpha} \kappa_g \dd s + \sum \epsilon_k = 2\pi$$ 

Si $$ \kappa_g = 0 \text{(cotées sont des géodésiques)} $$

$$\sum \epsilon_f = 2\pi$$ 


Si $\alpha$ est lisse $$\int_{\alpha} \kappa_g \dd s = 2\pi$$ : Umlasfstax  


\underline{Exemple 2}: Sur $S^2$ $\kappa = 1$

Cercle de lattitude $\varphi_0$
$$k_g = \cot \varphi_0	$$ 

$$\iint_R \kappa \dd S = \int_0^{2\pi} \int_0^{\vaprhi_0} \cos \varphi \dd \varphi \dd \theta = \dotsb = 2\pi$$ 

\begin{figure}[ht]
    \centering
    \incfig{théorème-de-gauss-bonnet-sur-la-sphère-unitée}
    \caption{Théorème de Gauss-Bonnet sur la sphère unitée}
    \label{fig:théorème-de-gauss-bonnet-sur-la-sphère-unitée}
\end{figure}

Cela doit être vrai pour n'importe quel surface qu'on a déformé continuement. ( $\alpha$ reste contractible )

\begin{figure}[ht]
    \centering
    \incfig{sphère-avec-un-pustule}
    \caption{sphère avec un pustule}
    \label{fig:sphère-avec-un-pustule}
\end{figure}

\underline{Corrolaire}: si $\theta_1 , \ \theta_2 ,\ \theta_3$ sont les angles intérieurs d'un triangle géodésique $T$ dans $S_1$ alors $$\int_T \kappa \dd s = -\pi + \sum_i^3  \theta_i$$ 

\underline{Démonstration}:

Comme les côtés sont géodésiques $k_g = 0$ sur les côtées.

$$\implies \iint_T \kappa \dd S + \epsilon_1 +\epsilon_2 +\epsilon_3 = 2\pi$$ 

$$\implies \iint_T \kappa \dd S + 3\pi -\theta_1 -\theta_2 -\theta_3 = 2\pi$$

$$\implies \iint_T \kappa \dd S = \theta_1 + \theta_2 +\theta_3 - \pi \qquad \blacksquare$$ 

Courbure positive: $$\sum_i^3 \theta_i = \pi \iint_T \kappa \dd S > \pi$$ 

Courbure négative $$\sum_i^3 \theta_i = \pi \iint_T \kappa \dd S - \pi$$ 


\begin{figure}[ht]
    \centering
    \incfig{trinagle-sur-surface-courbées}
    \caption{Trinagle sur surface courbées}
    \label{fig:trinagle-sur-surface-courbées}
\end{figure}

\underline{Caractéristique d'Euleur d'une surface}:

Une trinangulation d'une surface est formée par une collection de triangles (image d'un triangle dans le plan par $p$)

\begin{itemize}
	\item 2 triangles se rencontrent en un coté ou rien
	\item Les trinagles recouvrent la sruface
	\item Chaque triangle a au plus 1 côté sur le bord de la surface
\end{itemize}

\begin{figure}[ht]
    \centering
    \incfig{triangulation}
    \caption{Triangulation d'une surface}
    \label{fig:triangulation}
\end{figure}

La caractéristique d'Euleur d'une triangulation $T$ d'une surface $S$ est $\chi(s,\tau) = V-E+F$

Par exemple, la caracthéristique d'Euleur d'un triangle est $\chi(\Delta) = 3-3+1 =1$ 

\underline{Théorème de Gauss-Bonnet (Global)}

$$\iint_S \kappa \dd S + \int_{\partial S} \kappa_g \dd s + \sum \epsilon_k = 2\pi \chi(s)$$ 

\underline{Corollaire}: Pour une surface sans bord ( $\partial S = $\o )  

$$\iint_S \kappa \dd S = 2\pi \chi(s)$$ 

\underline{Démonstration}: Pour chaque triangle $\triangle$ de la triangulation $\tau$, on applique le théorème local $$\iint_{\triangle} \kappa \dd S = \int_{\partial S} \kappa_g \dd s+ \sum \epsilon_k^\triangle = 2\pi $$ et on fait la somme sur les $F$ triangles $$\iint_S \kappa \dd S + \underbrace{\sum_{\triangle\in\tau} \int_{\partial\tau} \kappa_g \dd s}_{\text{S'annullent en paires} }   + \sum_{\triangle\in\tau} \sum_k^3 \epsilon_k^\triangle = 2 \pi F$$  

\begin{figure}[ht]
    \centering
    \incfig{circulation-des-triangles}
    \caption{Circulation des triangles}
    \label{fig:circulation-des-triangles}
\end{figure}


$$\iint_S \kappa \dd S + \int_{\partial S} \kappa_g \dd s \sum_{\triangle} \sum_k e_k^triangle =2 \pi F $$

On calcul la somme...

Je vois pas très bien pis le cours est fini

\end{document}
