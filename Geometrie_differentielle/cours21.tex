\documentclass{article}    
\usepackage[utf8]{inputenc}    
    
\title{Épisode 4}    
\author{Jean-Baptiste Bertrand}    
\date{\today}    
    
\setlength{\parskip}{1em}    
    
\usepackage{physics}    
\usepackage{graphicx}    
\usepackage{svg}    
\usepackage[utf8]{inputenc}    
\usepackage[T1]{fontenc}    
\usepackage[french]{babel}    
\usepackage{fancyhdr}    
\usepackage[total={19cm, 22cm}]{geometry}    
\usepackage{enumerate}    
\usepackage{enumitem}    
\usepackage{stmaryrd}    
    
%packages pour faire des math    
%\usepackage{cancel} % hum... pas sur que je vais le garder mais rester que des fois c'est quand même sympatique...
\usepackage{amsmath, amsfonts, amsthm, amssymb}    
\usepackage{esint}  


\begin{document}

\underline{Rappels} 

\begin{itemize}
	\item Angle extérieur $\epsilon_k$
	\item Umlanfsatz: Si $\alpha$ est une courbe dans une surface, $\alpha'$ $\dotsb$   
	\item  Gauss-Bonnet local $$\iint_R \kappa \dd A + \int_{\partial R} \kappa_g \dd s + \sum\epsilon_k 2\pi $$ 
	\item Gauss-Bonnet Global $$\iint_S \kappa \dd S + \int_{\partial S} \kappa \dd s + \sum \epsilon_k = 2\pi \chi(S)$$ 
	\item Caractéristique d'Euleur $$\chi(S,\tau) = V -E +F$$ 
\end{itemize}

\underline{Démonstration}: $\tau$ est une triangulation de $S$. On applique Gauss-Bonnet à tous les triangle $\triangle\int\tau$ et on fait la somme    

$$\iint_S \kappa \dd S + \int_{\partial S} \kappa_g \dd s + \sum_{\triangle\in\tau} \sum_{j=i}^{3} \epsilon_k^\triangle  =2\pi   $$ 

$$\sum_{\triangle\in\tau} \sum_{j=1}^{3} \epsilon_j^\triangle  = \sum_{\triangle\in\tau} \sum_{j=1}^{3} (\pi - i_j^\triangle) = 3\pi F- \sum_{\triangle\in\tau} \sum_{j=1}^{3} i_j^\triangle = 3\piF -\left( 3\pi V - \sum \epsilon_k \right) $$ 

Donc,

$$\iint_S \kappa \ddS + \int_{\partial S} \kappa_g \dd s+ 3\pi F - 2\pi V + \sum \epsilon_k = 2\pi F $$ 

$$\iint_S \kappa \dd S + \int_{\partial S} \kappa_g \dd s + \sum \epsilon_k = -\piF + 2\piV$$ 
\begin{tcolorbox}[title=]
	Chaque face a 3 arrêtes, chaque arrête est adjacente à deux faces $$\implies 3F=2E$$ $$\begin{aligned}\implies \chi(S,\tau) &= V-E+F \\ &= V-\frac{3}{2} F +F \\ &= V- \frac{1}{2} F \\
	 \end{aligned}$$  
\end{tcolorbox}

$$= 2\pi(V - \frac{1}{2} F) = 2\pi\chi(S,\tau)$$ 

\underline{Conséquence} La caractéristique d'Euleur ne dépend pas de choix de triangulation $\tau$.

Dans un cours de topologie, on démontre que $\chi(S)$ est une invariant topologique (ne change pas pour une déformation continue de la surface)

\begin{figure}[ht]
    \centering
    \incfig{déformation-continue}
    \caption{Déformation continue}
    \label{fig:déformation-continue}
\end{figure}

La quantité $$\iint_S \kappa \dd S + \int_{\partial S} \kappa_g \dd s + \sum \epsilon_f $$ est invariante sous déformations continues de la surface $S$.


\underline{Exemple}: $\chi(S^2) = 2$

$$V-E+F =4-6+4 =2$$

\begin{figure}[ht]
    \centering
    \incfig{triangulation-d'une-shpère}
    \caption{Triangulation d'une shpère}
    \label{fig:triangulation-d'une-shpère}
\end{figure}


$\chi(\Pi^2) = 0$ (exercice)

\begin{figure}[ht]
    \centering
    \incfig{tore}
    \caption{tore}
    \label{fig:tore}
\end{figure}


\underline{Exemple}: Une surface de courbure $\kappa \leq 0$ ne contiens \underline{pas} de bigone géodésique.

\begin{figure}[ht]
    \centering
    \incfig{bigone-géodésique-sur-une-sphère}
    \caption{bigone géodésique sur une sphère}
    \label{fig:bigone-géodésique-sur-une-sphère}
\end{figure}

Gauss-Bonnet

$$\implies \iint_R \kappa \dd S + \int_{\alpha_1} \kappa_g \dd s\int_{alpha_{2}}  \kappa_g \dd s + \epsilon_1 + \epsilon_2 = 2\pi$$

mais $$\iint \kappa \dd S + \epsilon_1 \epsilon_2 \leq \epsilon_1 + \epsilon_2 \leq \pi \pi = 2\pi \quad \lightning$$ 

Donc un bigone ne peut pas exister si $\kappa \leq 0$ 

\underline{Exemple}: Si une surface topologiquement équivalente ;a une cylindre à $k < 0$ alors elle a au plus une géodésique ? fermée.  

\end{document}
