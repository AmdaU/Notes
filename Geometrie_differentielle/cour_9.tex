\documentclass{article}    
\usepackage[utf8]{inputenc}    
    
\title{Épisode 4}    
\author{Jean-Baptiste Bertrand}    
\date{\today}    
    
\setlength{\parskip}{1em}    
    
\usepackage{physics}    
\usepackage{graphicx}    
\usepackage{svg}    
\usepackage[utf8]{inputenc}    
\usepackage[T1]{fontenc}    
\usepackage[french]{babel}    
\usepackage{fancyhdr}    
\usepackage[total={19cm, 22cm}]{geometry}    
\usepackage{enumerate}    
\usepackage{enumitem}    
\usepackage{stmaryrd}    
\usepackage{mathtools,slashed}
%\usepackage{mathtools}
\usepackage{cancel}
    
\usepackage{pdfpages}
%packages pour faire des math    
%\usepackage{cancel} % hum... pas sur que je vais le garder mais rester que des fois c'est quand même sympatique...
\usepackage{amsmath, amsfonts, amsthm, amssymb}    
\usepackage{esint}  
\usepackage{dsfont}

\usepackage{import}
\usepackage{pdfpages}
\usepackage{transparent}
\usepackage{xcolor}
\usepackage{tcolorbox}

\usepackage{mathrsfs}
\usepackage{tensor}

\usepackage{tikz}
\usetikzlibrary{quantikz}
\usepackage{ upgreek }

\newcommand{\incfig}[2][1]{%
    \def\svgwidth{#1\columnwidth}
    \import{./figures/}{#2.pdf_tex}
}

\newcommand{\cols}[1]{
\begin{pmatrix}
	#1
\end{pmatrix}
}

\newcommand{\avg}[1]{\left\langle #1 \right\rangle}
\newcommand{\lambdabar}{{\mkern0.75mu\mathchar '26\mkern -9.75mu\lambda}}

\pdfsuppresswarningpagegroup=1


\begin{document}

\underline{Rappels} 

\begin{itemize}
\item Carte de Surface : $p: U \subseteq \mathbb{R}^{2} \to S \subseteq \mathbb{R}^{3}$ 

	lisse\\
	homéomorphisme entre $U$ et $p(U)$\\
	$Dp = (p_{u}|p_{V})$ rang maximal\\
	
\item Surface $S \subseteq \mathbb{R}^{3}$ 

	tout point est contenu dans la carte de surface

\item Point régulier $p$ de $f: \mathbb{R}^{3} \to \mathbb{R}: \eval{Df}_p\neq 0$  

	valeur régulière: $f(p)$\\
	valeur critique $\iff$ non-régulière 
\end{itemize}


\underline{Proposition} Si $\alpha \in \mathbb{R} est une valeur régulière. alors $f^{-1}(a)$ est une surface lisse$  

\underline{Dém:} Soit $\vec x \in \mathbb{R}^{3}$ t.q. $f(\vec x) =a$   

Comme $a$ est une valeur régulière, $\eval{df}_{\vec x} \neq 0$

$\implies$ un des dérivé partitiel est non-nulle

Sans perte de généralité, disons

$\eval{\dv{f}{z}}_{\vec x}\neq0$

Définissons $F: \mathbb{R}^{3} <to \mathbb{R}^{3} $

$$F \begin{pmatrix} x\\ y\\ z
	
\end{pmatrix} = \begin{pmatrix} x \\ y \\ f(x,y,z) 
	
\end{pmatrix}$$ 


$$\implies Df = \begin{pmatrix} 1 & 0 &0 \\ 0 &1 & 0 \\ \dv{f}{x} & \dv{f}{y} & \dv{f}{z}
	
\end{pmatrix}$$ 


$$\implies \eval{\det(DF)}_{\vec x} = \eval{\dv{f}{z}}_{\vec x} \neq 0$$ 

on peut applique le thm de la fonction inverse

$$\exists U, V \text{ ouverts }, U \ni \vec x, \quad V \ni F(\vec x) = (x_{0}, y_{0}, z_{0})^T$$ 

t.q. $F: U \to V$ est inversible et $F^{-1}$ est lisse.

Soit $W$ la projection de $V$ sir le plan $(x,y)$   

$$p: W \to S\quad (x,y,z)^T \to F^{-1}(x,y,z)^T \in f^{-1}(a)$$ 

\begin{figure}[ht]
    \centering
    \incfig{bingobong}
    \caption{bingobong}
    \label{fig:bingobong}
\end{figure}

comme $DF^{-1}\eval{}_{F(\vec x)} = (\eval{DF}_{\vec x})^{-1}$ 

$Dp = $ deux premires colonnes de $DF^{-1}$ est de range maxmial $\blacksquare$  


\underline{Exemple}

$$f(x,y,z) = x^2 + y^2 +z^2$$ 

$$Df = (2x, 2y, 2z)$$ 

Le seul point critique est $(0,0,0)$

La seule valeur critique est $f(0,0,0) =0 $ 

$$f^{-1} = \{(x,y,z)^T| x^2 + y^2 - z^2 =1\}$$ 

\section*{première forme fondamentale}

\begin{figure}[ht]
    \centering
    \incfig{exesmples-de-fonctions}
    \caption{Exesmples de fonctions}
    \label{fig:exesmples-de-fonctions}
\end{figure}


\begin{figure}[ht]
    \centering
    \incfig{forme-fondamentale}
    \caption{forme fondamentale}
    \label{fig:forme-fondamentale}
\end{figure}

	\underline{Définition} Étant donnée une carte de surface lisse $p$, la \underline{première forme fondamentale} ou \underline{métrique} est $$I_{u,v}= \eval{\begin{pmatrix} p_{u}\cdot p_{u} & p_{u}\cdot p_{v}\\ p_{v}\cdot p_{u}& p_{v}\cdot p_{v}
	
		\end{pmatrix}}_{(u,v)}  = \begin{pmatrix} E &F \\ F & G
	
\end{pmatrix}$$ 

\underline{Définition}: Deux surfaces $S, S^*$ sont localement isométrique s'il existe un ouvert $U \subseteq \mathbb{R}^{2} $ et des paramétristion $p, p^*$ t.q. $\eval{I}_{(u,v)} = \eval{I^*}_{(u,v)}$    

\underline{Exemple}: Considérons $S$ Le plan $x,y $ paramétrisé par $p_{(u,v)} = \begin{pmatrix} u\\v\\0
	
\end{pmatrix}$    

et le cylindre $S^*$ paramétrisé par $p^*(u,v) = (\cos(u),\sin(u), v)^T$

ON a $$p_{u}= (1,0,0), p_{v}=(0,1,0), p_{u}^* =(-\sin(u), \cos(u),0), p_{v}^* = (0,0,1)$$ 

			$$I = \begin{pmatrix} 1 & 0 \\ 0 & 1
	
\end{pmatrix} = I^*$$

$\implies S$ est localement isométrique à $S^*$  

\begin{figure}[ht]
    \centering
    \incfig{forme-fondamentale}
    \caption{forme fondamentale}
    \label{fig:forme-fondamentale}
\end{figure}

\end{document}
