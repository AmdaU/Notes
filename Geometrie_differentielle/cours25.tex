\documentclass{article}    
\usepackage[utf8]{inputenc}    
    
\title{Épisode 4}    
\author{Jean-Baptiste Bertrand}    
\date{\today}    
    
\setlength{\parskip}{1em}    
    
\usepackage{physics}    
\usepackage{graphicx}    
\usepackage{svg}    
\usepackage[utf8]{inputenc}    
\usepackage[T1]{fontenc}    
\usepackage[french]{babel}    
\usepackage{fancyhdr}    
\usepackage[total={19cm, 22cm}]{geometry}    
\usepackage{enumerate}    
\usepackage{enumitem}    
\usepackage{stmaryrd}    
\usepackage{mathtools,slashed}
%\usepackage{mathtools}
\usepackage{cancel}
    
\usepackage{pdfpages}
%packages pour faire des math    
%\usepackage{cancel} % hum... pas sur que je vais le garder mais rester que des fois c'est quand même sympatique...
\usepackage{amsmath, amsfonts, amsthm, amssymb}    
\usepackage{esint}  
\usepackage{dsfont}

\usepackage{import}
\usepackage{pdfpages}
\usepackage{transparent}
\usepackage{xcolor}
\usepackage{tcolorbox}

\usepackage{mathrsfs}
\usepackage{tensor}

\usepackage{tikz}
\usetikzlibrary{quantikz}
\usepackage{ upgreek }

\newcommand{\incfig}[2][1]{%
    \def\svgwidth{#1\columnwidth}
    \import{./figures/}{#2.pdf_tex}
}

\newcommand{\cols}[1]{
\begin{pmatrix}
	#1
\end{pmatrix}
}

\newcommand{\avg}[1]{\left\langle #1 \right\rangle}
\newcommand{\lambdabar}{{\mkern0.75mu\mathchar '26\mkern -9.75mu\lambda}}

\pdfsuppresswarningpagegroup=1

\begin{document}

\section*{Révision}

\underline{Différentes dérivée} 

$$f: S \to \mathbb{R}^n$$ 

Dérivé directionnelle de $f$ au point $x$ dans la direction $\vb{v}$: $D_x f(\vb{v})$    

\begin{figure}[ht]
    \centering
    \incfig{dérivée-directionnelle}
    \caption{Dérivée directionnelle}
    \label{fig:dérivée-directionnelle}
\end{figure}

$$D_x f(\vb{v}) := \eval{\dv{{}}{t}}_{t=0} f(\gamma(t)) $$ 

Indépendant du choix de $\gamma$ 

\begin{tcolorbox}[title=Raccourcit de notation, colframe=orange]
Si le chemin $\gamma(t)$ est fixé au départ, on note souvent $f(\gamma(t)) = f(t)$. Dans ce cas $$D_x f(\vb{v}) = \eval{\dv{{}}{t} }_{t=0} f(t) = f'(t) $$   
\end{tcolorbox}

Si on écrit $$f(p(u,v)) = f(u,v)$$ 

La matrice de $D_x f$ dans la base $p_u ,\, p_v $ est $$\cols{\dv{f_1}{u} & \dv{f_1}{v} \\ \dv{f_2}{u} & \dv{f_2}{v} \\\vdots & \vdots\\ \dv{f_n}{u} & \dv{f_n}{v} }$$  

C'est-à-dire $$D_x f(a p_u + b p_v ) =  \cols{\dv{f_1}{u} & \dv{f_1}{v} \\ \dv{f_2}{u} & \dv{f_2}{v} \\\vdots & \vdots\\ \dv{f_n}{u} & \dv{f_n}{v} } \cols{a \\b}$$ 

\underline{Exemple} 

$$p(\theta, \varphi) = \cols{\cos\theta\sin\varphi\\ \sin\theta\sin\varphi\\\cos\varphi}$$ 

$$f(\theta, \varphi) = 5\theta -\theta^2 + 8\varphi -\varphi^2$$ 

Pour trouver le max/min de $f$ sur $$S^2$$,

$$Df = (5-2\theta, 4-2\varphi) = 0 \implies \theta= \frac{5}{2} \quad \varphi= \frac{4}{2} = 2$$ 

\underline{Projection de Mercader} 

$$\begin{aligned}
	M: &C &\to S^2\\
	   &\cols{x\\y\\z} &\mapsto \cols{\sqrt{1-z^2}x \\ \sqrt{1-z^2} y \\z}
\end{aligned}$$ 

\begin{figure}[ht]
    \centering
    \incfig{projection-de-mercate}
    \caption{Projection de Mercate}
    \label{fig:projection-de-mercate}
\end{figure}

$$C: \cols{\cos\theta\\\sin\theta\\z}$$ 

$$M (\theta, z) = \cols{\sqrt{1-z^2}\cos\theta\\\sqrt{1-z^2} \\ z} = p(\theta, \arccos(z))$$ 

\underline{Champs de vecteur} 

Champ de vecteur $F$ sur $S$  
$$F: S \to \mathbb{R}^{3} | F(x) \in T_x S$$ 

Dérivé directionnelle: Comme pour une fonction quelconque $$D_x F(\vb{v})$$ 

% \begin{figure}[ht]
%     \centering
%     \incfig{champ-de-vecteur-sur-une-sphere}
%     \caption{Champ de vecteur sur une sphere}
%     \label{fig:champ-de-vecteur-sur-une-sphere}
% \end{figure}

Dérivée \underline{covariante} 

$$\nabla_{\vb{v}} F = D_x F(\vb{v}) - \left( D_x F(\vb{v}) \cdot \vb{n})  \right)\vb{n} $$ 

(Projection orthogonale sur $T_x S$) 


\end{document}
