\documentclass{article}    
\usepackage[utf8]{inputenc}    
    
\title{Épisode 4}    
\author{Jean-Baptiste Bertrand}    
\date{\today}    
    
\setlength{\parskip}{1em}    
    
\usepackage{physics}    
\usepackage{graphicx}    
\usepackage{svg}    
\usepackage[utf8]{inputenc}    
\usepackage[T1]{fontenc}    
\usepackage[french]{babel}    
\usepackage{fancyhdr}    
\usepackage[total={19cm, 22cm}]{geometry}    
\usepackage{enumerate}    
\usepackage{enumitem}    
\usepackage{stmaryrd}    
    
%packages pour faire des math    
%\usepackage{cancel} % hum... pas sur que je vais le garder mais rester que des fois c'est quand même sympatique...
\usepackage{amsmath, amsfonts, amsthm, amssymb}    
\usepackage{esint}  


\begin{document}

\underline{Rappels:}

Théorème fondamentale des surfaces dans $\mathbb{R}^{3}$:
$$I^* = I \wedge II = II^* \iff p^*=T\cdot p, T|T \text{ est une isométrie directe}$$ 

Champ de vecteurs sur $S$

$$x:S \to \mathbb{R}^{3} | X(x) \in T_{x}S$$ 

$x$ serait la vitesse d'un fluide sur la surface 

\underline{Dérivé couvariante} $$\grad_{v}: X = \pi_{T_{x} s} D_{v} X = D_{v} X -(D_{v} x \cdot u) u$$ 


\begin{tcolorbox}[title=Rappels sur les dérivées directionnelles]
	Soit $f: \mathbb{R}^{3} \to \mathbb{R}^{n} $ on peut définir sa dérivées directionelle dans la directiondu vecteur $v$ $$D_{v} f:= \lim_{t\to0} \frac{f(x+tv)-f(x)}{t}  $$. 

On peut l'évaluer enutilisant un chemin: $$D_{v} f = \eval{\frac{d}{dt}}_{t=0} f(\gamma(t)) \qquad \text{ avec } \gamma(0)=x, \gamma'(0)=v$$ 

Si $f$ est définie sur une surface $s$, on utilise un chemin de la forme $\gamma(t)=p(u(t),v(t))$   

Les vecteur tangants à $\gamma(t)$ sont $\frac{d}{dt} \gamma(t) = u'p_{u} + v'p_{v} $ 

Si on connaît la fonctions en coordonnées locales ($f(u,v)$), on a: $$D(u'p_{u} + v' p_{v})(f) = \eval{\frac{d}{dt} }_{t=0} f(u(t), v(t))$$ 

\end{tcolorbox}

Un champ de vecteur sur $S$ est intrinsèque. Si $X$ set un champ de vecteurs sur $S$, on peut l'écrire comme $$X - f(u,v)p_{u} + g(u,v)p_{v} $$  
$$\begin{aligned}
	Du'p_{u} v' p_{v} X &= u'(f(u,v)p_{u} + g(u,v)p_{v} )_u + v'(f(u,v) p_{u} +g(u,v)p_{v} )_v\\
			&= u'(f_{u} p_{u} + fp_{uu} +g_{u} p_{v} +g p_{uv}) +v'(f_{v} p_{u} + f p_{uv} + g_{u} p_{u} g p_uu)
\end{aligned}$$ 

Les termes $p_{uu}, p_{uv} \text{ et } p_{vv}$ ont des composantes en $u$. $\implies$ La dérivée directionnelle n'est pas un champ de vecteur sur $S$ mais la dérivée covariante oui!


\underline{Lien avec les symbols de cristoffel}: $p_{u}$ et $p_{v}$ sont des champs de vecteurs sur $S$

Exemple:

\begin{figure}[ht]
    \centering
    \incfig{champ-de-vecteur}
    \caption{Champ de vecteur}
    \label{fig:champ-de-vecteur}
\end{figure}

Sue chaque point on peut calculer la dérivé covariante de $p_{u}$ par rapport à $p_v$

$$\grad p_{u} p_{u} = \pi_{T_xS}^\perp (Dp_{u} p_{u}) = \pi_{T_xS}^\perp(\Gamma_{uu}^u p_{u} + \Gamma_{uu}^v p_{v} + Lu = \Gamma_{uu}^u p_{u} + \Gamma_{uu}^v p_v$$ 


On peut alors trouver les quantitées intrinsèques (ignornent les isométries locales)

$$\grad p_{u} p_{v} = \Gamma_{uv}^u p_{u} + \Gamma_{uv}^v p_{v}\\ \grad p_{v} p_{u} = \Gamma_{uv}^u p_{u} + \Gamma_{uv}^u p_{v}\\ \grad p_{v} p_{u} =\Gamma_{vu}^up_{u} + \Gamma_{vv}^u p_v$$ 

Propriétes de la dérivé couvariante

\begin{enumerate}
	\item Linéarité 1: $$\grad_{v} (x_{1} + x_{2}) = \grad_{v} X_{1} + \grad_{v} x_{1} + \grad_{v} x_{2} $$ 
	\item Règles de Leibnitz: $$\grad_{v} (fX) = (D_vf)X + f\grad_{v} X \qquad f: S\to \mathbb{R}$$ 
	\item linéarité 2: $$\grad_{av_1+bv_2} X = a \grad_{v_1}X + b\grad_{v_2} X $$ 
\end{enumerate}

\underline{Démonstation de la deuxième propriété:} 

$$\begin{aligned}
	\grad_{v} (fx) &= D_{v} (fx) -(D_{v} (fx)\cdot u)n\\ &=(D_{v} f)x +f(D_{v} x) - ((D_{v} f)\underbrace{x\cdot n}_0) + f(D_{u} x)\cdot n) n \\ &= D_{v} fx + f(D_{v} s = [(D_vx)\cdot n]n)\\&= D_{v} fx + f\grad_{v} x
\end{aligned}$$ 


En utilisant les trois propriétées on calculs:

$$\begin{aligned}
	\grad p_{u} (fp_{u} + g p_{v} ) = \grad p_{u} (f p_{u} ) + \grad_{p_u} (gp_v) \\ &= (Dp_{u} f)p_{u} + (DP_{u} g) p_{v} + f \grad_{p_{u}}  p_{u} + g \grad_{p_u} p_{v} \\ &= (f_{u} + f \Gamma_{uu}^u + g\Gamma_{uv}^u) p_{u} + (g_{u} + \Gamma_{uu}^v + g \Gamma_{uv}^u)p_v
\end{aligned}$$ 

On fait la même chose pour $\grad_{p_v}$ et avec la propriété 3 on peut calculer $\grad_{v} x$ pour n'importe que $v,x$. 


\end{document}
