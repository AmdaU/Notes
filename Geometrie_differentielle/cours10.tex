\documentclass{article}    
\usepackage[utf8]{inputenc}    
    
\title{Épisode 4}    
\author{Jean-Baptiste Bertrand}    
\date{\today}    
    
\setlength{\parskip}{1em}    
    
\usepackage{physics}    
\usepackage{graphicx}    
\usepackage{svg}    
\usepackage[utf8]{inputenc}    
\usepackage[T1]{fontenc}    
\usepackage[french]{babel}    
\usepackage{fancyhdr}    
\usepackage[total={19cm, 22cm}]{geometry}    
\usepackage{enumerate}    
\usepackage{enumitem}    
\usepackage{stmaryrd}    
\usepackage{mathtools,slashed}
%\usepackage{mathtools}
\usepackage{cancel}
    
\usepackage{pdfpages}
%packages pour faire des math    
%\usepackage{cancel} % hum... pas sur que je vais le garder mais rester que des fois c'est quand même sympatique...
\usepackage{amsmath, amsfonts, amsthm, amssymb}    
\usepackage{esint}  
\usepackage{dsfont}

\usepackage{import}
\usepackage{pdfpages}
\usepackage{transparent}
\usepackage{xcolor}
\usepackage{tcolorbox}

\usepackage{mathrsfs}
\usepackage{tensor}

\usepackage{tikz}
\usetikzlibrary{quantikz}
\usepackage{ upgreek }

\newcommand{\incfig}[2][1]{%
    \def\svgwidth{#1\columnwidth}
    \import{./figures/}{#2.pdf_tex}
}

\newcommand{\cols}[1]{
\begin{pmatrix}
	#1
\end{pmatrix}
}

\newcommand{\avg}[1]{\left\langle #1 \right\rangle}
\newcommand{\lambdabar}{{\mkern0.75mu\mathchar '26\mkern -9.75mu\lambda}}

\pdfsuppresswarningpagegroup=1


\begin{document}

\underline{Rappels} 

Première forme fondamentale

$$I = \cols{p_{u}\cdot p_{u} & p_{u}\cdot p_{v}\\ p_{v}\cdot p_{u}& p_{v}\cdot p_{v}}$$ 

La première forme fondamentale est une forme bilinéaire symétique définie positive (produit scalaire) sur $T_{x} S$: l'espace tangeant au point $x\in S$.

C'est $X,Y \in T_{x}S$ 

$$I_{x}(X.Y) = X\cdot Y$$ 

Dans la base $p_{u},p_{v}$ la matrice de $I$ est  
$$I = \cols{p_{u}\cdot p_{u} & p_{u}\cdot p_{v}\\ p_{v}\cdot p_{u}& p_{v}\cdot p_{v}}$$ 

Autrement dit, si  $X= ap_{u}+bp_{v}\quad Y = cp_{u}dp_{v}$ 

$$I_{x}(X,Y) =\cols{a & b} \cols{p_{u}\cdot p_{u} & p_{u}\cdot p_{v}\\ p_{v}\cdot p_{u}& p_{v}\cdot p_{v}} \cols{c\\d}$$ 

\underline{Rappels (encore)} 

$$f: \mathbb{R}^n \to \mathbb{R}^n$$ 

$$\eval{Df}_x = \cols{\dv{f_1}{x_1} & \dotsb & \dv{f_1}{x_n}\\ \vdots & \ddots & \vdots \\ \dv{f_n}{x_1} & \dotsb &\dv{f_n}{x_n}}$$ 


$$\eval{D_vf}_x = \eval{Df}_x \cdot v$$

est la \underline{dérivée directionnelle} de $f$ dams la direction $v$.

$$lim_{t\to o} \frac{f(x + tv) - f(x)}{t}$$ 


Rèlge de chaîne:

$$D(g \circ f)\eval{\;}_x= \eval{Dg}_{f(x)} \cdot \eval{Df}_x$$ 


\begin{tcolorbox}[title=Remarge]
	Soit un chemain $\gamma:(-\epsilon, \epsilon) \to \mathbb{R}^3$ t.q. $\gamma(0)=x; \quad \gamma'(0)= v$  	
$$f: \mathbb{R} \to \mathbb{R}$$ 
$$\eval{D(f\circ \gamma)}_0 = \eval{Df}_{\gamma(0)} \cdot \eval{D\gamma}_0 = \eval{Df}_{\gamma(0)} \gamma'(0) = \eval{Df}_x \cdot v = \eval{D_{v}f}_x$$ 

Dérivée directionelle de $f$ dans la direction $v$. Dépend \underline{uniqument} de $gamma(0)$ et  $\gamma'(0)$     

\end{tcolorbox}


Si $p:U\to S$ est une carte locale de surface et que $\gamma$ est un chemin dans $U$, alors $p \circ \gamma$i est un chemain dans $S$.   


\begin{figure}[ht]
    \centering
    \incfig{chemin-dans-une-surface}
    \caption{chemin dans une surface}
    \label{fig:chemin-dans-une-surface}
\end{figure}


\underline{Définition} 

Soit $f: S\to \mathbb{R}$. ON dit que $f$ est différentiable en $x\in S$ si pour une carte $p: U \to S$ t.q. $p(u_?,v_?) =x$, $f\circ p$ est différentialble en $(u_{0},v_{0})$     
Dans ce cas la \underline{dérivée} de $f$ est $x_{i}$ nortée $\eval{df}_x$ est définie par

$$\eval{df}_x: T_{x}S \to \mathbb{R}\qquad X\to D_x\eval{f}_x$$ 


La composistion $F=f\circ p$, s'appelle \underline{l'expression de $f$ en coordonnées locales}  

Sans la base $p_{u},p_{v}$ de $T_{x}S$ la dérivé $d\eval{f}_x$ a pour matrice:

$$\eval{Df}_{u_{0}, v_{0}} = \eval{\cols{\pdv{F}{u} & \pdv{F}{v}}}_{(u_{0}v_{0})}$$ 

$$\gamma(t) = (u_{0}v_{0}\circ t(a,b)$$ 
alors $$\eval{D(p\circ \gamma)}_? = D\eval{p}_?\cdot\eval{D\gamma}_? = \eval{Df}_? \cdot \cols{a\\ b} = \eval{(p_u|p_v) \cols{a\\b}}_{u_{0}, v_{0}}$$ 


Si $X =a p_{u}+ bp_{v}$ 

$$\eval{Df}_x = D\eval{\qty(f \circ p \circ\gamma)}_0 = D\eval{\qty(F\circ \gamma)}_0 = \eval{Df}_{\gamma(0)} D\eval{\gamma}_0 = D\eval{F}_{(u_{0},v_{0}) \cdot \cols{a\\b}}$$ 

Donc, la matrice de $\eval{df}_x$ dans la base $p_{u}, p_{v}$ est bien $\eval{DF}_x$.   


\underline{Exemple} 


$$p(\theta, z) = \cols{\cos\theta\\\sin\theta\\z}$$ 

$$f: S \to \mathbb{R}$$ 
$$f(x) = x\cdot x = \norm{x}^2$$ 


En coordonnées localess $f(x)$ est donnée par $F = f\circ p$   
$$I = \cols{p_{u}\cdot p_{u} & p_{u}\cdot p_{v}\\ p_{v}\cdot p_{u}& p_{v}\cdot p_{v}}$$ 


$$f(p(0,z)) = \cos^2\theta + \sin^2\theta +z^2 = 1+z^2$$ 


$$DF = \cols{\dv{F}{\theta}, dv{F}{z}} = \cols{0 & 2z}$$ 


est la matrice de $df$ en coordonnées locales


\underline{Définition}  

Soit $S$, $S^*$ deux surfaces et $f: S\to S^*$   

On dit que $f$ est dérivable/différentiable en $x \in S$ si, pour des cartes $p$ de $S$, $p^*$ de $S^*$, la composition 


\begin{figure}[ht]
    \centering
    \incfig{le-même-dessin-que-d'habithude}
    \caption{Le même dessin que d'habithude}
    \label{fig:le-même-dessin-que-d'habithude}
\end{figure}

On appelle $F = p \circ g \circ p$, l'expression en coordonées locales de $f$

La dérivée de $f$ est $\eval{df}_{x} T_{x} S \to T_{f(x)} S^*$ dont la matrice ?? les bases $p_{u} p_{v}$ de $T_{x} S$ et $p^*, p^*$ de $T_{f(x)}S^*$ est $\eval{DF}_{f(x)?} $  



\section*{Application de Gauss}


Étant donné une surface $S$ un choix ? de vecteurs unitaires normales s'appelles une \underline{orientation} sur $S$   
	
\underline{L'application de Gauss} est la fonction $$n: S \to S^2$$ qui associe à un point $x\in S$ le vecteur normal en $x$. (défini sur une surface orientée)   

par exemple, si $ S = S^2$, $n:S^2 \to S^2$ est l'identitié.

Si $S$ est un plan $n$ est constant?

Si $S$ est un cylinde, l'image de $S$ est un grand cercle

\begin{figure}[ht]
    \centering
    \incfig{grand-cercle-à-shpère}
    \caption{grand cercle à shpère}
    \label{fig:grand-cercle-à-shpère}
\end{figure}

Si on a plutôt une scelle:

\begin{figure}[ht]
    \centering
    \incfig{scelle-vers-shpère}
    \caption{Scelle vers shpère}
    \label{fig:scelle-vers-shpère}
\end{figure}

\underline{Définition} \underline{L'opérateur de forme} (shape operator) d'une surface $S$ est $\mathcal{S}_x(s) = - dn(x)$   
$$\mathcal{S} : T_{x} S \to T_{n(x)} S$$ 

\underline{"Demonstaraion"} 


\underline{Déf} \underline{La seonde forme fondamentale}  de $S$ est 

\newcommand{\rom}[1]
    {\MakeUppercase{\romannumeral #1}}


$$\rom 2_x (X, Y) = \mathcal{S}(X)\cdot Y$$ 

$$(X< y \in T_{x} S)$$ 


\underline{Prop:} $\rom 2_x$ est une forme bilinéaire symétrique

\underline{Dém:} $\rom2_x$ est bilinéaire car le produit scalaire est bilinéaire et $\mathcal{S} = - dn$ est linéaire

Calculons $\rom 2_x$ sur $p_{u,} p_{v} $

ON sait que $\eval{p_u}_{u,v} = n(?) = 0 $ 

On prend $\dv{v}$ de chaque côté

Fuck, je vais noter la conclusion quand on ferra le prochain rappel	

\end{document}

