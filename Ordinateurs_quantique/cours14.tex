\documentclass{article}    
\usepackage[utf8]{inputenc}    
    
\title{Épisode 4}    
\author{Jean-Baptiste Bertrand}    
\date{\today}    
    
\setlength{\parskip}{1em}    
    
\usepackage{physics}    
\usepackage{graphicx}    
\usepackage{svg}    
\usepackage[utf8]{inputenc}    
\usepackage[T1]{fontenc}    
\usepackage[french]{babel}    
\usepackage{fancyhdr}    
\usepackage[total={19cm, 22cm}]{geometry}    
\usepackage{enumerate}    
\usepackage{enumitem}    
\usepackage{stmaryrd}    
\usepackage{mathtools,slashed}
%\usepackage{mathtools}
\usepackage{cancel}
    
\usepackage{pdfpages}
%packages pour faire des math    
%\usepackage{cancel} % hum... pas sur que je vais le garder mais rester que des fois c'est quand même sympatique...
\usepackage{amsmath, amsfonts, amsthm, amssymb}    
\usepackage{esint}  
\usepackage{dsfont}

\usepackage{import}
\usepackage{pdfpages}
\usepackage{transparent}
\usepackage{xcolor}
\usepackage{tcolorbox}

\usepackage{mathrsfs}
\usepackage{tensor}

\usepackage{tikz}
\usetikzlibrary{quantikz}
\usepackage{ upgreek }

\newcommand{\incfig}[2][1]{%
    \def\svgwidth{#1\columnwidth}
    \import{./figures/}{#2.pdf_tex}
}

\newcommand{\cols}[1]{
\begin{pmatrix}
	#1
\end{pmatrix}
}

\newcommand{\avg}[1]{\left\langle #1 \right\rangle}
\newcommand{\lambdabar}{{\mkern0.75mu\mathchar '26\mkern -9.75mu\lambda}}

\pdfsuppresswarningpagegroup=1


\begin{document}
2023-10-31

\setcounter{section}{4}

\section{système quantiques ouverts}


\begin{figure}[ht]
    \centering
    \incfig{systeme-en-interaction-avec-un-bain}
    \caption{systeme en interaction avec un bain}
    \label{fig:systeme-en-interaction-avec-un-bain}
\end{figure}

\subsection{Matrices densité}

On s'imagine avoir préparer l'état dans $\ket{+}$


Après un mesure dans le base $z$ quel est l'état?

$\ket{\psi_{p}} = \sqrt{1-p}\ket{0} + \sqrt{p}\ket{1}?$

Évidement pas, c'est l'état avant la mesure.

Ce qu'on mesure est une valeur moyenne

$\expval{\hat O} = \bra{\psi}\hat O \ket{\psi}$

Normalement si à chaque préparation on obtiens $\ket{\psi_i}$ est obtenue avec poids $p$ quel est $\expval{\hat O}$?

\[ \expval{\hat O}  =\sum_i p_i \expval{\hat O}{\psi_{i}} = \Tr(p \hat O) \]

Si on a une connaissance parfaite pour un $i$ alors $\rho = \ketbra{\psi}$

\[ \implies \Tr(\rho) =1 \]


\subsubsection{États purs et mixte}

Si on a l'info complète sur le système $\rho = \ketbra{\psi}$

C'est un \textbf{état pur}

un état mixte

\[ \rho = \sum_i p_i \braket{\psi} \]

Les états sont tels que $\rho^{2}= \rho\implies \Tr \rho^{2}= \Tr \rho =1$

Pour les états mixtes $\Tr \rho^{2} = \Tr \sum_i p_i^{2} \ketbra{\psi_{i}} = \sum_i p_i^{2}\leq 1$

\underline{Exemples} 

\begin{enumerate}
    \item $\ket{0} \to \ketbra{0}$
    \item $\ket{+} \to \ketbra{+} = \frac{1}{2} \left( \ketbra{0} + \ketbra{0}{1} + \ketbra{1}{0} + \ketbra{1} \right) $
    \item État mixte de $\ket{+}$ et $\ket{-}$
        \[ \rho = \frac{1}{2} \ketbra{+} + \frac{1}{2} \ketbra{-} \]
\end{enumerate}

\subsubsection{État mixtes depuis état purs intriquées}

\begin{figure}[ht]
    \centering
    \incfig{intrication-rhoooo}
    \caption{Intrication rhoooo}
    \label{fig:intrication-rhoooo}
\end{figure}


Bob ne se fait donner le second qubit qu'à la fin et ne sait pas le circuit qui à été appliqué!

Bob mesure l'opérateur $\hat O_B$, sa valeur moyenne est \[ \expval{\hat O_{b}} = \bra{\psi} \mathds{1}_A \otimes \hat O_b \ket{\psi} \]


\[ \expval{\hat O_{b}} = \Tr \left( \mathds{1}_A \otimes \hat O_b \ketbra{\psi} \right) = \sum_b \bra{b} \hat O_b \rho_b \ket{b}  \]

où $\rho_b = \sum_a \ketbra{a}{\psi}\ketbra{\psi}{a}$

\[ \expval{\hat O_{B}} = \Tr_B(\rho_B \hat O_B \]


\end{document}
