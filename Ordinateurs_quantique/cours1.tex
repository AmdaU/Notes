\documentclass{article}    
\usepackage[utf8]{inputenc}    
    
\title{Épisode 4}    
\author{Jean-Baptiste Bertrand}    
\date{\today}    
    
\setlength{\parskip}{1em}    
    
\usepackage{physics}    
\usepackage{graphicx}    
\usepackage{svg}    
\usepackage[utf8]{inputenc}    
\usepackage[T1]{fontenc}    
\usepackage[french]{babel}    
\usepackage{fancyhdr}    
\usepackage[total={19cm, 22cm}]{geometry}    
\usepackage{enumerate}    
\usepackage{enumitem}    
\usepackage{stmaryrd}    
    
%packages pour faire des math    
%\usepackage{cancel} % hum... pas sur que je vais le garder mais rester que des fois c'est quand même sympatique...
\usepackage{amsmath, amsfonts, amsthm, amssymb}    
\usepackage{esint}  


\begin{document}

2023-08-29

{\section*{Principe Physique des ordinateurs quantique}\centering}


\subsection*{intro}

\begin{itemize}
	\item Architectures d'ordinateurs quantiques
	\begin{itemize}
		\item Qubit supraconducteurs
		\item ions piégés
		\item qubits de spin
		\item qubits topologiques
		\item qubits photoniques
	\end{itemize}
	\item Défi d'un ordinateur quantique
		\begin{itemize}
			\item Avoir un long temps de vie
			\item Pouvoir faire des opération à un qubit
			\item Pouvoir faire des opération à deux qubits
			\item le long temp de vie et le contrôle ont des besoin contradictoire (beaucoup d'interaction vs le moins d'interaction possible)
		\end{itemize}
	\item Circuit QED: Qubits supra (transmon) + cavité micro-onde
\end{itemize}

\subsection*{Plan}
\begin{itemize}
	\item Notion de base de l'info Q
	\item oscillateur harmoniques et circuits supra
	\item qubit supra
	\item interaction lumière-matière
	\item Dissipation
	\item info quantique
\end{itemize}

\section{Info quantique: notion de base}

\subsection{Bits et qubits}

\begin{center}
	classique: ${0,1}$ $0:\mqty(1\\0),\, 1: \mqty(0\\1)$ \qquad quantique: \{ \ket 0, \ket 1  \}, $0 \sim \mqty(1\\0),\, 1 \sim \mqty(0\\1)$

\end{center}
	Les qubits peuvent être en superposition 
	\[ \ket{\psi} = \psi_0 \ket{0} + \ket{1} = \mqty(\psi_0 \\\psi_1) \]
	\[ \bra{\psi}\ket{\psi} = 1\]
	
	Plusieurs qubits: \[ \ket{0} \otimes \ket{1} \otimes \ket{0} \dotsb \otimes \ket{0} = \ket{010\dotsb0} \]

\subsection{Opérations logiques}

\subsubsection{Opérations à 1 bit}

\begin{center}
	
bit: $\mathds{1}$, NOT

qubit: Une infinité d'opérations
\end{center}

Les opération sur des qubits sont des matrices unitaires 

\[ U^{\dagger}U = \mathds{1} \]

Les matrices de Pauli forment un base des opération unitaires.




\[ \mathds{1} = \mqty(1 & 0\\ 0& 1),  X = \mqty(0 &1 \\ 1 &0), Y= \mqty(0 & -i \\ i &0), Z = \mqty(1 & 0 \\ 0 &-1) \]


Ce sont les générateurs de rotation dans $\mathds{R}^{3}$: isomorphisme entre SO3 et SU2.

\[ R_z (\theta) = e^{-i Z \theta \over 2}\]

Plus généralement

\[ R_{\hat{n}}(\theta) = e^{i \hat n \cdot \vec{{\sigma}} \over 2} \]



\subsubsection{Opérations à 2 bits}

\verb|NAND| est une porte universelle! (On peut construire tout les portes à $n>2$ bits avec)

La \textit{version quantique} de cette porte et le \verb|CNOT| (Control not) 
\begin{table}[htpb]
	\centering
	\begin{tabular}{c|c}
		IN & OUT \\ 00 & 00 \\ 01 & 01\\ 10 & 11\\ 11 & 10
	\end{tabular}
\end{table}

\[ \verb|CNOT| = \mqty(\mathds{1} & {0}\\ {0} & X) \]


\begin{center}
\begin{quantikz}
	& \targ{} & \qw &\\
	& \ctrl{-1}& \qw &\\
\end{quantikz}
	
\end{center}



\subsection{Critère de D}
\textit{Critères minimal pour avoir un ordinateur quantique}

\begin{enumerate}
	\item Un system avec des qubits bien définis pouvant être \textit{mis à l'échelle}
		\underline{qubit}: Système à deux niveau
		\underline{mise à l'échelle}: requiert la correction d'erreur
	\item Possibilité d'initialiser un état: Ôter l'entropie du system
	Un moyen de le faire dans un system suffisement froid: attendre la relaxation: $\bra{1} \ket{\psi} \to 1$ pour $t \ll 1$
\end{enumerate}

\end{document}
