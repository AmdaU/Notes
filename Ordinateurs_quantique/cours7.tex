\documentclass{article}    
\usepackage[utf8]{inputenc}    
    
\title{Épisode 4}    
\author{Jean-Baptiste Bertrand}    
\date{\today}    
    
\setlength{\parskip}{1em}    
    
\usepackage{physics}    
\usepackage{graphicx}    
\usepackage{svg}    
\usepackage[utf8]{inputenc}    
\usepackage[T1]{fontenc}    
\usepackage[french]{babel}    
\usepackage{fancyhdr}    
\usepackage[total={19cm, 22cm}]{geometry}    
\usepackage{enumerate}    
\usepackage{enumitem}    
\usepackage{stmaryrd}    
    
%packages pour faire des math    
%\usepackage{cancel} % hum... pas sur que je vais le garder mais rester que des fois c'est quand même sympatique...
\usepackage{amsmath, amsfonts, amsthm, amssymb}    
\usepackage{esint}  


\begin{document}
2023-09-20

\[ \boxed{H \approx \hbar \omega_0 a^{\dagger}a +\hbar\omega_c b^{\dagger} b - \frac{E_c}{2} b^{\dagger}b^{\dagger}b b+ \hbar g \left( a^{\dagger} b + a b^{\dagger} \right)}   \]

En supposant que seul les niveau $\ket{0} et \ket{1}$ sont les seul niveau du trasmons auquel on accède on peut réécrire le Hamiltonien comme le Hamiltonien de Jaynes-Cumming qui est: 

\[ \boxed{H = \hbar \omega_0 a^{\dagger}a +\hbar \frac{\omega_q}{2}  \sigma_z + \hbar g \left( a^{\dagger} \sigma_-  + a \sigma_+ \right)}  \]


C'est l'hamiltonien décrivant l'échange d'une quanta entre un atome et un champ éléctromagnétique 

\vspace{1cm}
\hrule 

\section*{Charge de cours avec Othomane }

%\begin{figure}[ht]
    %\centering
    %\incfig{éléments-de-circuits}
    %\caption{test}
    %\label{fig:test}
%\end{figure}


\underline{Relation Constitutive de la JJ} 

On considère une la JJ réel comme ayant un capacitance parasite en parallèles 

\begin{align*}
    i\hbar \frac{\partial {\hat \rho}}{\partial {t}}  &= [\hat\rho, H]\\
                                                      &= [\hat\rho, 4E_c \hat n^{2}- E_J \cos \rho]\\
                                                      &= [\hat \rho, 4 E_c \hat n^2]\\
                                                      &= 4 E_c \left[ \underbrace{[\hat\rho, n]}_{i}  + n \underbrace{[\hat\rho, n]}_{i}   \right] 
\end{align*}


\[ \frac{d {\rho}}{d {t}}= 4 \frac{E_c}{\hbar} b = \frac{2\pi}{\Phi_0} \hat V\]


\begin{align*}
    [n, H] = - E_J [n, \cos\rho] 
\end{align*}

\[ \implies \frac{d {n}}{d {t}} = E_J \left[ n, \rho \right] \sin(\rho) = - \frac{E_j}{\hbar} \sin \rho 
 \]


\end{document}
