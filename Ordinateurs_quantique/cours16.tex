\documentclass{article}    
\usepackage[utf8]{inputenc}    
    
\title{Épisode 4}    
\author{Jean-Baptiste Bertrand}    
\date{\today}    
    
\setlength{\parskip}{1em}    
    
\usepackage{physics}    
\usepackage{graphicx}    
\usepackage{svg}    
\usepackage[utf8]{inputenc}    
\usepackage[T1]{fontenc}    
\usepackage[french]{babel}    
\usepackage{fancyhdr}    
\usepackage[total={19cm, 22cm}]{geometry}    
\usepackage{enumerate}    
\usepackage{enumitem}    
\usepackage{stmaryrd}    
\usepackage{mathtools,slashed}
%\usepackage{mathtools}
\usepackage{cancel}
    
\usepackage{pdfpages}
%packages pour faire des math    
%\usepackage{cancel} % hum... pas sur que je vais le garder mais rester que des fois c'est quand même sympatique...
\usepackage{amsmath, amsfonts, amsthm, amssymb}    
\usepackage{esint}  
\usepackage{dsfont}

\usepackage{import}
\usepackage{pdfpages}
\usepackage{transparent}
\usepackage{xcolor}
\usepackage{tcolorbox}

\usepackage{mathrsfs}
\usepackage{tensor}

\usepackage{tikz}
\usetikzlibrary{quantikz}
\usepackage{ upgreek }

\newcommand{\incfig}[2][1]{%
    \def\svgwidth{#1\columnwidth}
    \import{./figures/}{#2.pdf_tex}
}

\newcommand{\cols}[1]{
\begin{pmatrix}
	#1
\end{pmatrix}
}

\newcommand{\avg}[1]{\left\langle #1 \right\rangle}
\newcommand{\lambdabar}{{\mkern0.75mu\mathchar '26\mkern -9.75mu\lambda}}

\pdfsuppresswarningpagegroup=1


\begin{document}
2023-11-07

\section*{suite}

\[\dot \rho = - \frac{1}{2} [H, \rho] +\sum_i \gamma_i \mathcal{D} [L_i]\rho\]

\[ \qq{où} \mathcal{D}[L] \rho = L\rho L^{\dagger} - \frac{1}{2} \{ L^{\dagger}L, \rho \}  \]

\setcounter{section}{5}
\setcounter{subsection}{2}
\subsection{Oscillateur harmonique amortis}
\subsubsection{Équation maitraisse}
\[ H = \hbar \omega_r a^{\dagger} a + \sum_m \hbar \omega_m c_m^{\dagger}c_m  - \sum_m \lambda_m (c^{\dagger}_m + c)(a^{\dagger}+a) \]

En prenant la limite continue 


\[ H = \hbar \omega_r a^{\dagger} a + \hbar\int_0^\infty \dd\omega \omega c(\omega)^{\dagger}c(\omega)  - \int_o^\infty \lambda(\omega) (c^{\dagger}(\omega) + c(\omega))(a^{\dagger}+a) \]

Après la RWA

\[ H = \hbar \omega_r a^{\dagger} a + \hbar\int_0^\infty \dd\omega \omega c(\omega)^{\dagger}c(\omega)  - \int_o^\infty \lambda(\omega) (c^{\dagger}(\omega)a + c(\omega) a^{\dagger}) )\]

On constate que la contribution du dernier terme est beaucoup plus importante autour de $\omega = \omega_r $

\[ H = \hbar \omega_r a^{\dagger} a + \hbar\int_0^\infty \dd\omega \omega c(\omega)^{\dagger}c(\omega)  - \lambda(\omega_r)\int_0^\infty  (c^{\dagger}(\omega)a + c(\omega) a^{\dagger}) )\]

Le dernier terme deviens

\[ -\lambda(\omega_r) \left( B^{\dagger} a + B a ^{\dagger} \right) \]
À température nulle $\expval{c^{\dagger}(\omega) c(\omega)} =0 $


\[ \boxed{L = \sqrt{?}a} \]

En posant $\rho = \ketbra{n}$ on trouve \[ -\kappa a \rho a^{\dagger} = -n \kappa \rho \]

Ce terme génère donc bien une perte! Le terme de \textit{jump} à fait sont travail. Mais que font les autres termes \textinterrobang

...

\[ \rho(t) = \left( 1- e^{-\kappa t} \ketbra{0} + e^{-\kappa t} \ketbra{1} \right)  \]

(Je comprend pas trop la demarche mais on se rends compte que ça préserve la normalisation)

On essaie de trouver un interpretation \textit{moins ennuyante} des \textit{no jumps terms}.

On s'imagine un système préparé dans l'état 

\[ \frac{1}{2} \ketbra{0} + \frac{1}{2} \ketbra{1} \]

...

Les \textit{no jump termes} reflète le fait qu'on gagne de l'information qu'on a sur la cavité, on est de plus en plus sur d'être proche de 0 en ne voyant pas de photons sortir? Ish


À température finie

\[ \expval{c^{\dagger}(\omega) c(\omega')} = n_B (\omega) \delta(\omega-\omega') \]


\subsection{pour un qubit}


\[ L = \sqrt{\gamma_{1}} \sigma_-  \]

\[ \dot \rho = - \frac{i}{\hbar} [H,, \rho] + \gamma \mathcal{D}(\sigma_- )\rho \]

\subsubsection{Déphasage}


\[ H = \hbar \delta(t) \frac{\sigma_z}{2}  \]

où $\delta(t)$ est une fonction fluctuante à \textit{basse fréquence}.

On modélise cela par \[ \dot \rho \frac{\gamma_{\varphi}}{2} \mathcal{D}[\sigma_{z]} \rho \qquad L = \sqrt{\frac{\gamma_{\varphi}}{2} }\sigma_z \]

\[ \ketbra{0}{1} \to^{t} \ketbra{0}{1}e^{-\gamma_{\varphi} t} \]
En combinant avec $\gamma_1$

\[ \dot \rho = \gamma_1 \mathcal{D}[\sigma_{z]\rgo} + \frac{\gamma_{\varphi}}{2} \mathcal{D}[\sigma_{z]} \rho \]

\[ \rho(t) = \begin{pmatrix} \rho_{11} (0) e^{-\gamma_1 t} & \rhp_{10} (0) e^{-\gamma_2 t} \\ \rho_{00} (0) e^{-\gamma_2 t} & \rho_{00} (0) \left[ 1 - e^{-\gamma_1 t} \right] 	
\end{pmatrix} \]





\end{document}
