\documentclass{article}    
\usepackage[utf8]{inputenc}    
    
\title{Épisode 4}    
\author{Jean-Baptiste Bertrand}    
\date{\today}    
    
\setlength{\parskip}{1em}    
    
\usepackage{physics}    
\usepackage{graphicx}    
\usepackage{svg}    
\usepackage[utf8]{inputenc}    
\usepackage[T1]{fontenc}    
\usepackage[french]{babel}    
\usepackage{fancyhdr}    
\usepackage[total={19cm, 22cm}]{geometry}    
\usepackage{enumerate}    
\usepackage{enumitem}    
\usepackage{stmaryrd}    
    
%packages pour faire des math    
%\usepackage{cancel} % hum... pas sur que je vais le garder mais rester que des fois c'est quand même sympatique...
\usepackage{amsmath, amsfonts, amsthm, amssymb}    
\usepackage{esint}  


\begin{document}
2023-11-01

\setcounter{section}{5}
\setcounter{subsection}{1}

\setcounter{subsubsection{3}

\subsubsection{Évolution de la matrice densité}

\[ \rho = \sum_i \p_i \ketbra{\psi_{i}} \qquad i\hbar \partial_t \ket{\psi} = H \ket{\psi} \]

\[ \dot \rho = \sum_i p_i \ket{\dot\psi}\bra{\psi} + \ket{\psi}\bra{\dot\psi} \]

\[ \implies \dot \rho = - \frac{i}{\hbar} [H, \rho] \to \mathscr{L}\rho \]

\subsection{Équation maîtresse }

\subsubsection{Hamiltonien système - bain}


\begin{figure}[ht]
    \centering
    \incfig{circuit-effectif}
    \caption{circuit effectif}
    \label{fig:circuit-effectif}
\end{figure}

\[ H = \dotsb = \hbar \omega_r a ^{\dagger}a + \sum^{\infty}\hbar \omega_i c^{\dagger}_i c - \sum^{\infty} \lambda_i \left( c_a^{\dagger} +c_a \right) \left( a ^{\dagger} +a   \right)   \]

\[ H  = H_s + H_B + H_{SB}  \]

\[ \rho_S (t) = \Tr_B \rho_{SB} (t) \]

\subsubsection{Matrices densité réduite et canaux quantiques}

On cherche $\rho_S(T)$

On sait que 

$\rho_{SB} (t) = U(t) \rho_{SB} U^{\dagger}(t) = U(t) \left( \rho_0 + \ketbra{e} \right) U^{\dagger}(t)$


\[ \rho_S = \Tr_B \left( \dotsb \right) = \sum_k \bra{B_k} \dotsb \ket{B_{k}} = \sum_k E_k \rho^{0} E_k^{\dagger}  \]

$E_k = \bra{B_{k}} U_{SB} (t) \ket{B_k}$

\[ \rho(t) = \mathcal{E}(\rho(0)) \qq{Quantum map!} \]



\[ \Tr \mathcal{E}(\rho) = \Tr \sum_k E_k \rho E_k^{\dagger} = \Tr E_k^{\dagger} E_k \rho = 1 \implies \sum_k E_k^{\dagger} E_k = \mathds{1} \]

Les $\mathcal{E}$ sont compressibles 

\[ \mathcal{E}_2 \circ \mathcal{E}_1(\rho) = \mathcal{E}_2 \left( \sum_k E_k \rho E_k^{\dagger} \right) = \sum_{j,k} E_j E_k \rho E_k^{\dagger} E_j^{\dagger} = \sum_i G_i \gho G_i^{\dagger} =  \]

\subsubsection{Équation maîtrise de Linblad }


\[ \dot \rho = \frac{1}{\hbar} [H, \rho]  \]


\[ \rho(t+\dd t) =  \rho(t) - \frac{i}{\hbar} [H, \rho(t)] \dd t \]


On suppose qu'à chaque $\dd t$ on suppose que $S$ et $B$ sont factorisable.

On définit 3 échelles de temps

\underline{corse grain time}: temps de discrétisation ($t_{\text{corse}} = \Delta t \sim \delta t $)

\underline{temps de mémoire de l'environemment} $t_b$, après $t_b$, l'evironement oubli avoir interagit avec le système 


On suppose que $t_{\text{corse}} \gg t_B $


On suppose un chaine de Markov (faux mais vrai approximativement(?))

$t_c$ est le temps caractéristique du système  

\[ t_c \gg t_{\text{corse}} \gg t_B  \]



En prennat la trace sur le bain


\[ \rho(t + \delta t) = \mathcal{E}_H \rho(t) = \sum_j E_j \rho(t) E_j^{\dagger} \]


On cherche les $E_j$

\[ E_j = \mathds{1} + \hat O \dd t = \mathds{1} + \left(  \frac{-i}{\hbar} +K \right) \dd t  \]

\[ I = \sum E_j ^{\dagger} E_j = \dotsb \]

\[ E_j = L_j \sqrt{\dd t} \]








\end{document}
