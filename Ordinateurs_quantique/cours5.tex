\documentclass{article}    
\usepackage[utf8]{inputenc}    
    
\title{Épisode 4}    
\author{Jean-Baptiste Bertrand}    
\date{\today}    
    
\setlength{\parskip}{1em}    
    
\usepackage{physics}    
\usepackage{graphicx}    
\usepackage{svg}    
\usepackage[utf8]{inputenc}    
\usepackage[T1]{fontenc}    
\usepackage[french]{babel}    
\usepackage{fancyhdr}    
\usepackage[total={19cm, 22cm}]{geometry}    
\usepackage{enumerate}    
\usepackage{enumitem}    
\usepackage{stmaryrd}    
    
%packages pour faire des math    
%\usepackage{cancel} % hum... pas sur que je vais le garder mais rester que des fois c'est quand même sympatique...
\usepackage{amsmath, amsfonts, amsthm, amssymb}    
\usepackage{esint}  


\begin{document}
2023-09-12

\setcounter{section}{2}

\section{Qubits supraconducteurs}


\subsection{Jonction Josephson}

On a constaté que de piloter un circuit LC à sa fréquence de résonance génère un état cohérent (ce qui ne ressemble pas du tout à un système à deux niveau). Pour avoir un système à deux niveau on ajoute un élément non linéaire à notre circuit: la jonction josephson

\subsubsection{Hamiltonien et relation de commutation}

\begin{figure}[ht]
    \centering
    \incfig{constitution-de-jj}
    \caption{constitution de jj}
    \label{fig:constitution-de-jj}
\end{figure}


\[ n_1 + n_2 = \text{cte}  \]

\[ n = n_1 - n_2 \text{ peut changer par effet tunnel!}  \]


\underline{Description quantique} 


Base de charge : \[ \hat n \ket{n} = n \ket{n} \quad n \in ] -\infty, \infty [ \]


Dans cette base, l'hamiltonien qui décrit l'effet tunnels de paires de cooper est

\[ H_J = - \frac{E_J}{2} \sum_{n=-\infty}^{\infty} \left( \ket{n}\bra{n+1} + \ket{n+1}\bra{n} \right)  \]


$E_J =  \frac{h \Delta}{8 e^{2}R_n} $ est l'énergie de Josephson

avec $\Delta$ l'énergie de gap et $R_n$ la résitance de l'état normal


\subsubsection{Base de phase}


\[ \ket{\psi} = \sum_{n=-\infty}^{\infty} e^{in\varphi}\ket{n} \]

avec $\varphi \in [0, 2\pi[$

De la même façon \[ \ket{n} = \frac{1}{2\pi} \int_{0}^{2\pi}\dd \varphi e^{-in\varphi} \ket{\psi}  \] 

Dans cette base le Hamiltonien s'écrit

\[ H_J = - \frac{E_J}{2} \sum_{n=-\infty}^{\infty} \left( \frac{1}{(2\pi)^2} \iint_0^{2\pi} \dd \varphi \dd \varphi' e^{-in\varphi} e^{i(n+1)\varphi'} \ket{\varphi}\bra{\varphi'} + \text{H.C.}   \right)  \]

\[  = - \frac{E_J}{2} \frac{1}{2\pi} \int_{0}^{\infty}\dd \varphi \left( e^{i\varphi} + e^{-i\varphi} \right) \ketbra{\varphi}  \]


On introduit

$e^{i \hat \varphi} =  \frac{1}{2\pi} \int \dd \varphi e^{i\varphi} \ket{\varphi}\bra{\varphi}$


qui agit sur $\ket{n}$ comme

\[ e^{\pm i \hat \varphi} \ket{n} = \ket{n \mp 1}\]

\[ H_g = E_J \frac{e^{i\hat\varphi}+e^{-i \varphi}}{2} = - E_J \cos \har \varphi  \]

la variable $\varphi = \varphi_1 -\varphi_2$ s'interprète comme la différence de phase entre les deux côté de la jonction

\subsubsection{Relation de commutation et relation constiutive}


\[ [e^{\pm{}i\hat\varphi}, \hat{n}] = e^{\pm i \hat \varphi}  \]


C'est plus clair quand $\hat \varphi$ est dans une fonction periodique

En utilisant la représentation de Heisenberg on peut trouver comment les opérateurs évoluent 


\begin{align}
    \frac{d {\hat \varphi}}{d {t}} &= \frac{2e}{\hbar} \hat V\\
    \hat I &= I_c \sin \hat \varphi
\end{align}


\[I_c = \frac{2e E_{J}}{\hbar} \text{: le courant critique}   \]


Le sinus est la non linéarité qu'on cherchait!


\section{Transmons}


\begin{figure}[ht]
    \centering
    \incfig{remplacement-par-une-inductance-non-lineaire}
    \caption{remplacement par une inductance non-lineaire}
    \label{fig:remplacement-par-une-inductance-non-lineaire}
\end{figure}

On remplace l'inductance par une jonction josephson qui agit dans un certain régime comme un inducteur linéaire


\[ H = 4 E_c \left( \hat n - n_g \right)^2 - E_J \cps \hat \varphi \]


\begin{figure}[ht]
    \centering
    \incfig{energie-en-fonction-du-flux}
    \caption{Energie en fonction du flux}
    \label{fig:energie-en-fonction-du-flux}
\end{figure}



$H$ est controlé par un seul paramètre soit le ratio $\frac{E_J}{E_C} $. Quel ration donne le meilleur qubit? On veut une bonne anharmonicité et un bon temps de cohérence
L'anharmonicité est $<alpha = E_{12} - E_{01} $

anharmonicité relative: \[ \alpha_r = \frac{\alpha}{E_{01} }  \]


Temps de cohérence $T_2$:









\end{document}
