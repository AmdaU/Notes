\documentclass{article}    
\usepackage[utf8]{inputenc}    
    
\title{Épisode 4}    
\author{Jean-Baptiste Bertrand}    
\date{\today}    
    
\setlength{\parskip}{1em}    
    
\usepackage{physics}    
\usepackage{graphicx}    
\usepackage{svg}    
\usepackage[utf8]{inputenc}    
\usepackage[T1]{fontenc}    
\usepackage[french]{babel}    
\usepackage{fancyhdr}    
\usepackage[total={19cm, 22cm}]{geometry}    
\usepackage{enumerate}    
\usepackage{enumitem}    
\usepackage{stmaryrd}    
\usepackage{mathtools,slashed}
%\usepackage{mathtools}
\usepackage{cancel}
    
\usepackage{pdfpages}
%packages pour faire des math    
%\usepackage{cancel} % hum... pas sur que je vais le garder mais rester que des fois c'est quand même sympatique...
\usepackage{amsmath, amsfonts, amsthm, amssymb}    
\usepackage{esint}  
\usepackage{dsfont}

\usepackage{import}
\usepackage{pdfpages}
\usepackage{transparent}
\usepackage{xcolor}
\usepackage{tcolorbox}

\usepackage{mathrsfs}
\usepackage{tensor}

\usepackage{tikz}
\usetikzlibrary{quantikz}
\usepackage{ upgreek }

\newcommand{\incfig}[2][1]{%
    \def\svgwidth{#1\columnwidth}
    \import{./figures/}{#2.pdf_tex}
}

\newcommand{\cols}[1]{
\begin{pmatrix}
	#1
\end{pmatrix}
}

\newcommand{\avg}[1]{\left\langle #1 \right\rangle}
\newcommand{\lambdabar}{{\mkern0.75mu\mathchar '26\mkern -9.75mu\lambda}}

\pdfsuppresswarningpagegroup=1


\begin{document}
2023-09-13

Régime transmon $\frac{E_J }{E_c } $ grand:

La dissipation de charge va comme $e^{\frac{E_J }{E_c } }$

\begin{figure}[ht]
    \centering
    \incfig{graphiques}
    \caption{graphiques}
    \label{fig:graphiques}
\end{figure}


\begin{figure}[ht]
    \centering
    \incfig{anharmonicité}
    \caption{anharmonicité}
    \label{fig:anharmonicité}
\end{figure}

Pour atteindre ce régime : $C_s \gg C_J$


\setcounter{section}{3}
\setcounter{subsection}{2}

\subsubsection{Approximation Keir du Transmon}

\underline{Approx:} On laisse tomber $n_g$

\[  \hat H = 4 E_c \hat n^{2}- E_j \cos \hat \varphi = 4E_c \hat n^{2}+ \frac{\hat \varphi^2}{2L_? } - E_g \left( cos( ? ) - ? \right)  = \hat H_{l} + \hat H_{nl} = \left[ \left( \dotsb \right)  \right]   \]
\[ \hat \varphi = \left( \frac{2E_{c}}{E_j}    \right) ^{\frac{1}{4} }\left( b ^{\dagger} + b \right) \qquad \hat n = \frac{i}{2} \left( \frac{E_{J}}{2E_c}  \right)^{\frac{1}{4} } \left( b^{\dagger} - b \right)   \]
donc $H_l = \hbar \omega_p b^{\dagger} b$ avec $\omega_p = \sqrt{8 E_J E_c}/\hbar$


Puisque la \textit{particule} est massive, elle n'explore que le bas du puit. On peut donc faire une expension en série de $H_{\text{nl}} $

\[ \hat H_{\text{nl}} - \frac{1}{4!} E_J \hat \varphi^{4}= - \frac{1}{4!} E_c^{\frac{1}{4} } \left( b^{\dagger} + b \right)^4 \approx - E_c b^{\dagger} b - \frac{E_c}{2} b^{\dagger} ba^{\dagger} bb \]

On ne garder que les termes ayant le même nombre de $b$ et $b^{\dagger}$ (Ce qui reviens à l'approximation séculaire)

Pour s'en convaincre, on passe à un référentiel tournant à $\omega_p$

\[ H' \sim - \frac{E_c}{12} \left( b^{\dagger} e^{i\omega_p t}  + e^{-i\omega_p t} \right)^4 = \text{... expension...}  \]

Le terme qui tourne le moins vite, $b^{\dagger}^3b$ apprait $6$ fois

\[ 2 \omega_p \gg \frac{6E_{c}}{12} \to \sqrt{\frac{E_j}{E_c} } \gg 1 \text{: satisfait par le régime transmon} \]

De retour dans le référention du labo

\[ H \approx \hbar \omega_p b^{\dagger}b - E_c b^{\dagger} b - E_c b^{\dagger} b - \frac{E_c}{J} b^{\dagger} b^{\dagger} b b = \hbar \omega_q b^{\dagger} b - \frac{E_c}{2} b^{\dagger} b^{\dagger} bb \]

On peut le réecrire le hamiltonien pour mieux comprendre l'effet de la non-linéairité

\[ H = \left( \hbar\omega_q' - \frac{E_c}{2} b^{\dagger} b \right) b^{\dagger} b \]

Chaque niveau d'énérgie dépend néativement du nombre de niveau, on voit donc que l'énérgie entre chaque niveau diminue.


\begin{tcolorbox}[title=Remarque sur $\hat phi$]
     $\phi$ à seulement vraiment un sens lorsque dans une fonction périodique. En prenant un série de Taylor on perd la périodicité de la fonction. On pert une partie de la physique, donc. 

     \underline{Anharmonicité}: \[ \frac{E_{c}}{\hbar\omega_{q}} \sim \frac{E_c}{\sqrt{8E_J E_c }} \quad \text{: petit dans le régime transmon}    \]
     En partique $  \frac{E_c}{h} \sim 100 - 400 MHz$
\end{tcolorbox}



\subsubsection{Transmons ajustable par le flux}

\begin{figure}[ht]
    \centering
    \incfig{double-jj}
    \caption{Double jj}
    \label{fig:double-jj}
\end{figure}

\[ L = \frac{1}{2} C_3 \ddot \phi + \frac{1}{2} C_{J_1} \dot \phi^{2}+ \frac{1}{2} C_{J_2} \left( \dot \phi \do \Phi_{\text{ext}}  \right)^2 + E_{J_1} \cos \vap + E_{J_2} \cos \left( \varphi + \varphi_{\text{ext}}  \right) \]

\[ H = 4 E_x \hat n^{2}- \cancelto{0}{2 e \frac{C_J}{C_s} \dot \phi_{\text{ext}} \hat n} - E_{J_1} \cos \hat \varphi - E_j \cos \left( \hat \varphi + \varphi_{\text{ext}}  \right)  \]

avec $c_g = c_s + c_{J_1} + c_{J_2} $

Dans le cas $   E_{J_1} = E_{J_2} \equiv \frac{E_J}{2} $ alors


\[  H = \dotsb \]


Dans le régime transmon \[ \hbar\omega_q = \sqrt{8E_c \abs{E_g (\Phi_{\text{ext}} )} } -E_C\]

\end{document}
