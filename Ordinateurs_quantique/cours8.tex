\documentclass{article}    
\usepackage[utf8]{inputenc}    
    
\title{Épisode 4}    
\author{Jean-Baptiste Bertrand}    
\date{\today}    
    
\setlength{\parskip}{1em}    
    
\usepackage{physics}    
\usepackage{graphicx}    
\usepackage{svg}    
\usepackage[utf8]{inputenc}    
\usepackage[T1]{fontenc}    
\usepackage[french]{babel}    
\usepackage{fancyhdr}    
\usepackage[total={19cm, 22cm}]{geometry}    
\usepackage{enumerate}    
\usepackage{enumitem}    
\usepackage{stmaryrd}    
\usepackage{mathtools,slashed}
%\usepackage{mathtools}
\usepackage{cancel}
    
\usepackage{pdfpages}
%packages pour faire des math    
%\usepackage{cancel} % hum... pas sur que je vais le garder mais rester que des fois c'est quand même sympatique...
\usepackage{amsmath, amsfonts, amsthm, amssymb}    
\usepackage{esint}  
\usepackage{dsfont}

\usepackage{import}
\usepackage{pdfpages}
\usepackage{transparent}
\usepackage{xcolor}
\usepackage{tcolorbox}

\usepackage{mathrsfs}
\usepackage{tensor}

\usepackage{tikz}
\usetikzlibrary{quantikz}
\usepackage{ upgreek }

\newcommand{\incfig}[2][1]{%
    \def\svgwidth{#1\columnwidth}
    \import{./figures/}{#2.pdf_tex}
}

\newcommand{\cols}[1]{
\begin{pmatrix}
	#1
\end{pmatrix}
}

\newcommand{\avg}[1]{\left\langle #1 \right\rangle}
\newcommand{\lambdabar}{{\mkern0.75mu\mathchar '26\mkern -9.75mu\lambda}}

\pdfsuppresswarningpagegroup=1


\begin{document}
2023-09-26

\setcounter{section}{4}
\setcounter{subsection}{1}



\section*{Circuit QED}

\subsection{Hamiltonien de Jaynes-Cummings}

\[ H_{JC} = \hbar a ^{\dagger} a + \hbar \frac{\omega_a}{2} \sigma_z + \hbar g \left( a ^{\dagger} \sigma_- a \sigma_+  \right)  \]
Le couplage est:
\[ g = dE_0  \]

On peut faire la diagonalisation par block pour trouver les énergies propres et les états propres. Les dress states:

\[ E_{\bar{\sigma n}}, \ket{\bar{\sigma n}}  \]


Exception: l'état fondamentale

\[ E_{\bar{g0}} = E_{go} = -\hbar \omega_{q}/2\]


dans le sous espace à n quanta:

\[ E_{\bar{gn}} = \hbar n\omega_r - \sqrt{\Delta^ + 4g n} \qquad E_{\bar{gn}} = \hbar n\omega_r + \sqrt{\Delta^ + 4g n}  \]


\[ \ket{\bar{gn}} = \cos\theta_n \ket{gn} -\sin\theta_n \ket{en-1 } \qquad \ket{\bar{en-1}} = \sin\theta_n \ket{gn} +\cos\theta_n \ket{en-1 }\]

avec $\theta_n = \arctan(2 g \sqrt{n}/ \Delta)$ l'angle de mélange


\begin{figure}[ht]
    \centering
    \incfig{niveaux-d'énérgies}
    \caption{niveaux d'énérgies}
    \label{fig:niveaux-denergies}
\end{figure}

\begin{figure}[ht]
    \centering
    \incfig{circuit-avec-drive}
    \caption{circuit avec drive}
    \label{fig:circuit-avec-drive}
\end{figure}

2 premier états excités à $\Delta=0$

\[ \ket{\bar{g1}} = \frac{1}{\sqrt{2}} \left( \ket{g1} - \ket{e0}  \right) \qquad  \ket{\bar{e0}} = \ket{g1} + \ket{\bar{e0}}  \]

\begin{figure}[ht]
    \centering
    \incfig{transmission-en-fonction-de-la-fréquance}
    \caption{transmission en fonction de la fréquance}
    \label{fig:transmission-en-fonction-de-la-fréquance}
\end{figure}


\subsection{Régime dispersif}

à $\Delta=0$, le qubit est maximalement intriqué avec les photon: le qubit est essentiellement dans un état aléatoire si on a pas acces au photon


\begin{figure}[ht]
    \centering
    \incfig{delta-pas-zero}
    \caption{delta pas zero}
    \label{fig:delta-pas-zero}
\end{figure}


\subsubsection{Transformation de Shrieffer-Wolff}


En quantique, L'approche usuelle pour solutionner un problème est de diagonaliser l'hamiltonien 

\[ H_D = U H U^{\dagger}  \]

Malheureusement, ce n'est pas toujours possible, on représente alors notre hamiltonien comme

\[ H = H_D + V \]

Ou $V$ est un \textit{petit} terme qui \textit{perturbe} note Hamiltonien

La perturbation couple faiblement les sous=espaces $\mu$


On prend un trasformation unitaire qui diagonalise approximetivement le Halitonien 

\[ H' = e^{-S} H e^{S} \qq{avec $S^{\dagger} = S$ pour que $e^{S} soit unitaire$} \]


\begin{align*}
    H' &= \left( \mathds{1} -s + \frac{s^2}{2!} + \dotsb  \right) H \left( \mathds{1} + s + \frac{s^2}{2} + \dotsb  \right) \\
       &= H + [H, S] + \frac{1}{2!} [[H, S], S] + \dotsb\\
    &= \sum_{k=0}^{\infty}\frac{1}{k!} [H, S]^{(k)} 
\end{align*}

\end{document}
